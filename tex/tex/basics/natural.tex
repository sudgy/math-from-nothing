\documentclass[../../math.tex]{subfiles}
\externaldocument{../../math.tex}
\externaldocument{foundations}
\externaldocument{elementary_algebra}

\begin{document}

\setcounter{chapter}{2}

\chapter{The Natural Numbers} \label{chap_natural}

\begin{definition}[Natural Numbers]
    Define the type $\N$ inductively with two constructors: a value constructor
    $0_\N$ and a function constructor $S_\N : \N \rightarrow \N$.
\end{definition}

In this chapter, $0_\N$ will just be called $0$ and $S_\N$ will just be called
$S$.  By the properties of inductive constructions (which I admittedly didn't
develop), this definition satisfies all of the Peano axioms.

The following basic definitions will also be useful in the future:
\begin{definition}
    Let $\U$ be a type and $f$ be a function from $\U$ to $\U$.  Then we define
    the iterated function $f^n(x)$ recursively as follows:
    \begin{itemize}
        \item $f^0(x) = x$
        \item $f^{S(n)}(x) = f(f^n(x))$.
    \end{itemize}
\end{definition}
\begin{definition}
    Given a type $\U$, we call a function $\N \rightarrow \U$ a sequence of
    $\U$.
\end{definition}

\section{Addition}

\begin{instance}
    Let $a$ and $b$ be natural numbers.  Then we define addition in $\N$
    recursively on $a$:
    \begin{itemize}
        \item $0 + b = b$
        \item $S(a) + b = a + S(b)$.
    \end{itemize}
\end{instance}

\begin{theorem}
    For all $a$ and $b$,
    \[
        S(a) + b = S(a + b).
    \]
\end{theorem}
\begin{proof}
    We will do this by induction on $a$ with the statement ``For all $b$, $S(a)
    + b = S(a + b)$.''  When $a = 0$,
    \[
        S(0) + b = 0 + S(b) = S(b) = S(0 + b),
    \]
    so the base case is true.  Now assume that for all $b$, $S(a) + b = S(a +
    b)$.  Then
    \[
        S(S(a)) + b = S(a) + S(b) = S(a + S(b)) = S(S(a) + b).
    \]
    Thus, by induction, $S(a) + b = S(a + b)$ for all $a$ and $b$.
\end{proof}

\begin{theorem}
    For all $a$ and $b$,
    \[
        a + S(b) = S(a + b).
    \]
\end{theorem}
\begin{proof}
    \[
        a + S(b) = S(a) + b = S(a + b).
    \]
\end{proof}

\begin{instance}
    In the natural numbers we define $0 = 0_\N$.
\end{instance}

\begin{instance}
    Zero in $\N$ is a left additive identity.
\end{instance}
\begin{proof}
    This follows directly from the definition.
\end{proof}

\begin{theorem}
    For all $a$, $a + 0 = a$.
\end{theorem}
\begin{proof}
    The proof will be by induction on $a$.  When $a = 0$, $0 + 0 = 0$, so the
    base case is true.  Now assume that $a + 0 = a$.  Then
    \[
        S(a) + 0 = S(a + 0) = S(a),
    \]
    so the inductive case is true.  Thus, by induction, $a + 0 = a$ for all $a$.
\end{proof}
Note that once we've proved addition to be commutative, we won't need the
previous theorem any more.

\begin{instance}
    Addition in $\N$ is commutative.
\end{instance}
\begin{proof}
    Let $a$ and $b$ be natural numbers.  The proof will be by induction on $a$.
    When $a = 0$, $0 + b = b = b + 0$, so the base case is true.  Now assume
    that $a + b = b + a$.  Then
    \[
        S(a) + b = S(a + b) = S(b + a) = b + S(a),
    \]
    so the inductive case is true.  Thus, by induction, addition is commutative.
\end{proof}

\begin{instance}
    Addition in $\N$ is associative.
\end{instance}
\begin{proof}
    Let $a$, $b$, and $c$ be natural numbers.  The proof will be by induction on
    $a$.  When $a = 0$, $0 + (b + c) = b + c = (0 + b) + c$, so the base case is
    true.  Now assume that $a + (b + c) = (a + b) + c$.  Then
    \[
        S(a) + (b + c) = S(a + (b + c)) = S((a + b) + c) = S(a + b) + c
        = (S(a) + b) + c,
    \]
    so the inductive case is true.  Thus, by induction, addition is associative.
\end{proof}

\begin{instance}
    Addition in $\N$ is left cancellative.
\end{instance}
\begin{proof}
    Let $a$, $b$, and $c$ be natural numbers.  The proof will by by induction on
    $c$.  When $c = 0$, we get $0 + a = 0 + b \rightarrow a = b$, so the base
    case is true.  Now assume that $c + a = c + b \rightarrow a = b$.  Then
    \begin{align*}
        S(c) + a &= S(c) + b \\
        S(c + a) &= S(c + b) \\
        c + a &= c + b \\
        a &= c,
    \end{align*}
    so the inductive case is true.  Thus, by induction, addition is left
    cancellative.
\end{proof}

\begin{theorem}
    For all $a$ and $b$ in $\N$, if $0 = a + b$, then $0 = a$ and $0 = b$.
\end{theorem}
\begin{proof}
    If $0 \neq a$, then $a = S(a')$ for some $a' : \N$.  Then we would have $0 =
    S(a') + b = S(a' + b)$.  But it's impossible for zero to equal the successor
    of a natural number, so we must have $0 = a$.  Then we also get $0 = a + b =
    0 + b = b$, so we now have $0 = a$ and $0 = b$.
\end{proof}

\section{Multiplication}

\begin{instance}
    Let $a$ and $b$ be natural numbers.  Then we define multiplication in $\N$
    recursively on $a$:
    \begin{itemize}
        \item $0b = 0$
        \item $S(a)b = b + ab$.
    \end{itemize}
\end{instance}

\begin{instance}
    In the natural numbers we define $1 = S(0)$.
\end{instance}

\begin{instance}
    One in $\N$ is a left multiplicative identity.
\end{instance}
\begin{proof}
    \[
        1a = S(0)a = 0a + a = 0 + a = a.
    \]
\end{proof}

\begin{theorem}
    For all $a$ and $b$,
    \[
        S(a)b = b + ab.
    \]
\end{theorem}
\begin{proof}
    This follows directly from the definition.
\end{proof}

\begin{theorem}
    For all $a$ and $b$,
    \[
        aS(b) = a + ab.
    \]
\end{theorem}
\begin{proof}
    The proof will be by induction on $a$.  When $a = 0$, we have
    \[
        0S(b) = 0 = 0 + 0 = 0 + 0b,
    \]
    so the base case is true.  Now assume that $aS(b) = a + ab$.  Then
    \[
        S(a)S(b) = S(b) + aS(b) = S(b) + a + ab = b + S(a) + ab = S(a) + b + ab
        = S(a) + S(a)b,
    \]
    so the inductive case is true.  Thus, by induction, $aS(b) = a + ab$ for all
    $a$ and $b$.
\end{proof}

\begin{instance} \label{nat_ldist}
    Multiplication in $\N$ is left distributive over addition.
\end{instance}
\begin{proof}
    Let $a$, $b$, and $c$ be natural numbers.  The proof will be by induction on
    $a$.  When $a = 0$, $0(b + c) = 0 = 0 + 0 = 0b + 0c$, so the base case is
    true.  Now assume that $a(b + c) = ab + ac$.  Then
    \[
        S(a)(b + c) = b + c + a(b + c) = b + c + ab + ac = b + ab + c + ac =
        S(a)b + S(a)c,
    \]
    so the inductive case is true.  Thus, by induction, multiplication is left
    distributive over addition.
\end{proof}

\begin{instance}
    Multiplication in $\N$ is commutative.
\end{instance}
\begin{proof}
    Let $a$ and $b$ be natural numbers.  The proof will be by induction on $a$.
    When $a = 0$, $0b = 0 = b0$, so the base case is true.  Now assume that $ab
    = ba$.  Then
    \[
        S(a)b = b + ab = b + ba = bS(a),
    \]
    so the inductive case is true.  Thus, by induction, multiplication is
    commutative.
\end{proof}
A special note is in order about the base case in the previous proof.  We got
$0b = 0$ from the definition of multiplication, but what about $0 = b0$?  This
came from the fact that you can prove the annihilation property from the
distributive property.  However, we've only proven left distributivity, not
right distributivity.  We haven't proved commutativity yet so we can't use
Instance \ref{mult_lanni_ranni} or Instance \ref{ldist_rdist} either.  So is the
proof even valid?  It is, because Instance \ref{ring_mult_ranni} uses
\textit{left} distributivity to prove that zero is a \textit{right} annihilator.
Thus, $0 = b0$ is a valid step here through Instances \ref{ring_mult_ranni} and
\ref{nat_ldist}.

\begin{instance}
    Multiplication in $\N$ is associative.
\end{instance}
\begin{proof}
    Let $a$, $b$, and $c$ be natural numbers.  The proof will be by induction on
    $a$.  When $a = 0$, $0(bc) = 0 = 0c = (0b)c$, so the base case is true.  Now
    assume that $a(bc) = (ab)c$.  Then
    \[
        S(a)(bc) = bc + a(bc) = bc + (ab)c = (b + ab)c = (S(a)b)c,
    \]
    so the inductive case is true.  Thus, by induction, multiplication is
    associative.
\end{proof}

\begin{theorem} \label{nat_neq_suc_mult}
    For all $a$ and $b$ in $\N$, $0 \neq S(a)S(b)$.
\end{theorem}
\begin{proof}
    If $0 = S(a)S(b)$, we would have
    \[
        0 = S(a)S(b) = S(b) + aS(b) = S(b + aS(b)),
    \]
    so zero would equal the successor of a natural number, which is impossible.
\end{proof}

\begin{instance}
    Multiplication in $\N$ is left cancellative.
\end{instance}
\begin{proof}
    Let $a$, $b$, and $c$ be natural numbers such that $ca = cb$ with $c \neq
    0$.  Because $c \neq 0$, we can write $c = S(c')$ so that we instead have
    $S(c')a = S(c')b$.  We will now do induction on $a$ with the statement ``For
    all $b$, $S(c')a = S(c')b \rightarrow a = b$.''

    First, when $a = 0$, we must prove that $0 = S(c')b$ implies that $0 = b$.
    If $0 \neq b$, we could write $b = S(b')$ and thus $0 = S(c')S(b')$, which
    contradicts Theorem \ref{nat_neq_suc_mult}.  Thus, the base case is true.

    Now assume that for all $b$, $S(c')a = S(c')b \rightarrow a = b$, and that
    $S(c')S(a) = S(c')b$.  We must prove that $S(a) = b$.  Now if $b = 0$, we
    would have $S(c')S(a) = 0$, which again contradicts Theorem
    \ref{nat_neq_suc_mult}, so $b \neq 0$.  This means that we can write $b =
    S(b')$ for some $b'$, so we now have
    \begin{align*}
        S(c')S(a) &= S(c')S(b') \\
        S(c') + S(c')a &= S(c') + S(c')b' \\
        S(c')a &= S(c')b' \\
        a &= b' \\
        S(a) &= S(b') = b,
    \end{align*}
    so the inductive case is true.  Thus, by induction, multiplication is
    associative.
\end{proof}

\begin{instance}
    The natural numbers are not trivial.
\end{instance}
\begin{proof}
    Because $0 \neq S(a)$ for any natural number $a$, we have $0 \neq 1$, so
    $\N$ is not trivial.
\end{proof}

\section{Order}

\begin{instance}
    Let $a$ and $b$ be natural numbers.  Then we define order in $\N$
    recursively on $a$:
    \begin{itemize}
        \item $0 \leq b = \vtt{True}$
        \item $S(a) \leq 0 = \vtt{False}$
        \item $S(a) \leq S(b) = a \leq b$.
    \end{itemize}
\end{instance}

\begin{instance}
    All natural numbers are positive.
\end{instance}
\begin{proof}
    This follow directly from the definition.
\end{proof}

\begin{theorem} \label{nat_pos2}
    For all $n : \N$, $0 < S(n)$.
\end{theorem}
\begin{proof}
    By the definition of $\leq$ we have $0 \leq S(n)$, and by the definition of
    $\N$ we have $0 \neq S(n)$.
\end{proof}

\begin{theorem}
    $0 < 1$.
\end{theorem}
\begin{proof}
    This is a direct application of Theorem \ref{nat_pos2}.
\end{proof}
\noindent Note that we can't use Theorem \ref{one_pos} in this case because the
proof used additive inverses, which don't exist in the natural numbers.

\begin{theorem} \label{nat_sucs_le}
    For all $a$ and $b$, $S(a) \leq S(b) \leftrightarrow a \leq b$.
\end{theorem}
\begin{proof}
    This is true by definition.
\end{proof}

\begin{theorem} \label{nat_sucs_lt}
    For all $a$ and $b$, $S(a) < S(b) \leftrightarrow a < b$.
\end{theorem}
\begin{proof}
    This is equivalent to saying
    \[
        (S(a) \leq S(b) \wedge S(a) \neq S(b)) \leftrightarrow
        (a \leq b \wedge a \neq b).
    \]
    From Theorem \ref{nat_sucs_le} we know that $S(a) \leq S(b) \leftrightarrow
    a \leq b$, and by the properties of inductive definitions we know that $S(a)
    \neq S(b) \leftrightarrow a \neq b$.
\end{proof}

\begin{instance}
    The order on $\N$ is connex.
\end{instance}
\begin{proof}
    Let $a$ and $b$ be natural numbers.  We will do induction on $a$ with the
    statement ``For all $b$, $a \leq b \vee b \leq a$.''  When $a = 0$, we know
    that $0 \leq b$, so the base case is true.  Now assume that for all $b$, $a
    \leq b \vee b \leq a$.  We must prove that $S(a) \leq b \vee b \leq S(a)$.
    Now when $b = 0$, we would have $0 \leq S(a)$ as required.  When $b \neq 0$,
    we have $b = S(b')$ for some $b : \N$.  Thus we now must prove that $S(a)
    \leq S(b') \vee S(b') \leq S(a)$.  By Theorem \ref{nat_sucs_le}, this is
    equivalent to proving $a \leq b' \vee b' \leq a$, which follows from the
    inductive hypothesis.  Thus, by induction, the order on $\N$ is connex.
\end{proof}

\begin{instance}
    The order on $\N$ is antisymmetric.
\end{instance}
\begin{proof}
    Let $a$ and $b$ be natural numbers.  We will do induction on $a$ with the
    statement ``For all $b$, $a \leq b \rightarrow b \leq a \rightarrow a =
    b$.''  When $a = 0$, we have $b \leq 0$, so $b = 0$ as well, proving that $a
    = b$, so the base case is true.  Now assume that for all $b$, $a \leq b
    \rightarrow b \leq a \rightarrow a = b$, and also assume that $S(a) \leq b$
    and $b \leq S(a)$.  We must prove that $S(a) = b$.  Now $b \neq 0$, because
    if it was equal to zero, we would have $S(a) \leq 0$.  Thus $b = S(b')$ for
    some $b' : \N$, so we now have $S(a) \leq S(b')$ and $S(b') \leq S(a)$,
    which is equivalent to $a \leq b'$ and $b' \leq a$.  Then by the inductive
    hypothesis we have $a = b'$, so $S(a) = S(b') = b$ as required.  Thus, by
    induction, the order on $\N$ is antisymmetric.
\end{proof}

\begin{instance}
    The order on $\N$ is transitive.
\end{instance}
\begin{proof}
    Let $a$, $b$, and $c$ be natural numbers.  We will do induction on $c$ with
    the statement ``For all $a$ and $b$, $a \leq b \rightarrow b \leq c
    \rightarrow a \leq c$.''  When $c = 0$, we have $b \leq 0$, so $b = 0$.
    Then $a \leq 0$, so $a = 0$ as well.  Thus, $a = 0 \leq 0 = c$, so the base
    case is true.

    Now assume that for all $a$ and $b$, $a \leq b \rightarrow b \leq c
    \rightarrow a \leq c$, that $a \leq b$, and that $b \leq S(c)$.  We must
    prove that $a \leq S(c)$.  Now when $b = 0$, we have $a \leq 0$, so $a = 0$,
    so $a = 0 \leq S(c)$ is true.  When $b \neq 0$, we have some $b'$ such that
    $b = S(b')$.  We now have $a \leq S(b')$ and $S(b') \leq S(c)$.  Again, if
    $a = 0$, we have $a = 0 \leq S(c)$, so consider when $a \neq 0$.  Then there
    is an $a' : \N$ such that $a = S(a')$.  Thus we now have $S(a') \leq S(b')$
    and $S(b') \leq S(c)$.  By Theorem \ref{nat_sucs_le}, this is the same as
    $a' \leq b'$ and $b' \leq c$.  Then by the inductive hypothesis, we have $a'
    \leq c$, so $a = S(a') \leq S(c)$ as required.  Thus, by induction, the
    order on $\N$ is transitive.
\end{proof}

\begin{theorem} \label{nat_le_suc}
    For all $a : \N$, $a \leq S(a)$.
\end{theorem}
\begin{proof}
    The proof will be by induction on $a$.  When $a = 0$, then we have $0 \leq
    1$, which is true.  When $a \leq S(a)$, we have $S(a) \leq S(S(a))
    \leftrightarrow a \leq S(a)$, so the theorem is true by induction.
\end{proof}

\begin{theorem} \label{nat_lt_suc}
    For all $a : \N$, $a < S(a)$.
\end{theorem}
\begin{proof}
    This proof is identical in form to the previous one.
\end{proof}

\begin{instance}
    The order on $\N$ is left additive.
\end{instance}
\begin{proof}
    Let $a$, $b$, and $c$ be natural numbers such that $a \leq b$.  We must
    prove that $c + a \leq c + b$.  The proof will be by induction on $c$.  When
    $c = 0$, we have $0 + a = a \leq c = 0 + c$, so the base case is true.  Now
    assume that $c + a \leq c + b$.  Then
    \begin{align*}
        c + a &\leq c + b \\
        S(c + a) &\leq S(c + b) \\
        S(c) + a &\leq S(c) + b,
    \end{align*}
    so the inductive step is true.  Thus, by induction, the order on $\N$ is
    left additive.
\end{proof}

\begin{instance}
    The order on $\N$ is left addition cancellative.
\end{instance}
\begin{proof}
    Let $a$, $b$, and $c$ be such that $c + a \leq c + b$.  The proof will be by
    induction on $c$.  When $c = 0$, we have $a = 0 + a \leq 0 + b = b$, so the
    base case is true.  Now assume that $c + a \leq c + b \rightarrow a \leq b$.
    We must prove that $S(c) + a \leq S(c) + b \rightarrow a \leq b$.  Assuming
    $S(c) + a \leq S(c) + b$, we get
    \begin{align*}
        S(c) + a &\leq S(c) + b \\
        S(c + a) &\leq S(c + b) \\
        c + a &\leq c + b \\
        a &\leq b,
    \end{align*}
    so the inductive case is true.  Thus, by induction, the order on $\N$ is
    left addition cancellative.
\end{proof}

\begin{instance}
    The order on $\N$ is multiplicative.
\end{instance}
\begin{proof}
    We always have $0 \leq ab$ in $\N$.
\end{proof}

\begin{instance}
    The order on $\N$ is left multiplicative.
\end{instance}
\begin{proof}
    Let $a$, $b$, and $c$ be natural numbers such that $a \leq b$.  We must
    prove that $ca \leq cb$.  The proof will be by induction on $c$.  When
    $c = 0$, we have $0a = 0 \leq 0 = 0c$, so the base case is true.  Now
    assume that $ca \leq cb$.  Then
    \begin{align*}
        S(c)a &\leq S(c)b \\
        a + ca &\leq b + cb,
    \end{align*}
    and because $a \leq b$ and $ca \leq cb$, this is true, so the inductive step
    is true.  Thus, by induction, the order on $\N$ is left multiplicative.
\end{proof}

\begin{theorem} \label{nat_lt_suc_le}
    For all $a$ and $b$ in $\N$, $a < S(b) \leftrightarrow a \leq b$.
\end{theorem}
\begin{proof}
    To prove the forward implication, we will prove by induction on $a$ the
    statement ``For all $b$, $a < S(b) \rightarrow a \leq b$.''  When $a = 0$,
    we have $0 \leq b$, so the base case is true.  Now assume that for all $b$,
    $a < S(b) \rightarrow a \leq b$, and assume that $S(a) < S(b)$, which also
    means that $a < b$.  We must prove that $S(a) \leq b$.  Now because $a < b$,
    $b \neq 0$, so there is some $b'$ such that $b = S(b')$.  We thus have $a <
    S(b')$, so by the inductive hypothesis, we have $a \leq b'$, so $S(a) \leq
    S(b') = b$, so the inductive case is true.  Thus, by induction, the forward
    implication is true.

    Now assume that $a \leq b$.  By Theorem \ref{nat_lt_suc} we have $b < S(b)$,
    so by transitivity we have $a < S(b)$.
\end{proof}

\begin{theorem} \label{nat_le_suc_lt}
    For all $a$ and $b$ in $\N$, $S(a) \leq b \leftrightarrow a < b$.
\end{theorem}
\begin{proof}
    If $b = 0$, both directions cause a contradiction because $S(a) \nleq 0$ and
    $a \nless 0$, so $b \neq 0$.  Thus there exists a $b'$ with $b = S(b')$, so
    we need to prove
    \[
        S(a) \leq S(b) \leftrightarrow a < S(b).
    \]
    By Theorem \ref{nat_sucs_le} this is the same as
    \[
        a \leq b \leftrightarrow a < S(b),
    \]
    which is precisely Theorem \ref{nat_lt_suc_le}.
\end{proof}

\begin{theorem} \label{nat_le_self_lplus}
    For all $a$ and $b$ in $\N$, $a \leq a + b$.
\end{theorem}
\begin{proof}
    We know that $0 \leq b$, and adding $a$ to both sides yields $a \leq a + b$.
\end{proof}

\begin{theorem} \label{nat_le_self_lmult}
    For all $a$ and $b$ in $\N$, if $0 \neq b$, $a \leq ba$.
\end{theorem}
\begin{proof}
    We use induction on $a$.  When $a = 0$, $0 \leq 0 = b0$, so the base case is
    true.  Now assume that $a \leq ba$.  Because $0 \neq b$, $1 \leq b$.  By
    Theorem \ref{le_lrplus} we get
    \begin{align*}
        1 + a &\leq b + ba \\
        S(a) &\leq bS(a),
    \end{align*}
    so the inductive case is true.  Thus, by induction, the theorem is true.
\end{proof}

\begin{proof} \label{nat_neq0_leq1}
    For all natural numbers $a$, if $0 \neq a$, then $1 \leq a$.
\end{proof}
\begin{proof}
    By Theorem \ref{nat_le_suc_lt}, then is the same as proving that $0 < a$,
    which is true by the definition of $\leq$ and $0 \neq a$.
\end{proof}

\begin{theorem}[Strong Induction]
    For all sets $S$ of the natural numbers, if for all $n : \N$, $m \in S$ for
    all $m < n$ implies that $n \in S$, then $n \in S$ for all $n : \N$.
\end{theorem}
\begin{proof}
    Define a new set $T = \{n : \N \mid \forall m, m < n \rightarrow m \in S\}$.
    We will prove by induction that for all $n$, $n \in T$.

    First, for $n = 0$, there is no $m < 0$, so $0 \in T$ vacuously.

    Now assume that $n \in T$, that is, that for all $m < n$, $m \in S$.  We
    must prove that for all $m < S(n)$, $m \in S$.  By the conditions of the
    theorem, we can prove that $m \in S$ by proving that for all $m' < m$, $m'
    \in S$.  Because $m < S(n)$, by Theorem \ref{nat_lt_suc_le}, we have $m \leq
    n$, so by transitivity we have $m' < n$.  Thus, by the inductive hypothesis,
    $m' \in S$, meaning that $m \in S$, so the inductive step is true.

    We now know by induction that for all $n$, $n \in T$.  So now let $n$ be any
    natural number.  We want to prove that $n \in S$.  From the conditions of
    the theorem, we can prove this if for all $m < n$, $m \in S$.  This is true
    because $n \in T$.  Thus, $n \in S$.
\end{proof}

\begin{instance}[Well-Ordering Principle]
    The natural numbers are well-ordered.
\end{instance}
\begin{proof}
    Let $S$ be a nonempty set of natural numbers.  Assume that there was no
    least element, that is, that for all $a \in S$, there exists a $b \in S$
    such that $b < a$.  We will now prove that $S$ is empty.  Let $x$ be a
    natural number.  We will prove by strong induction on $x$ that $x \notin S$.
    The inductive hypothesis is ``For all $m < x$, $m \notin S$.''  Now if $x
    \in S$, because we assumed that there is no least element, there exists such
    an $m < x$ that is in $S$, which contradicts the inductive hypothesis.
    Thus, $x \notin S$, so $S$ is empty by strong induction.  However, $S$ is
    nonempty by hypothesis, so the original assumption that there is no least
    element in $S$ was incorrect.
\end{proof}

\section{Subtraction}

While a normal definition of subtraction is impossible in the natural numbers,
there are many situations where a suitable substitute is useful.  The most
useful of these substitutes are probably the following two theorems.

\begin{theorem} \label{nat_le_ex}
    For all $a$ and $b$ in $\N$, if $a \leq b$, then there exists a $c : \N$
    such that $a + c = b$.
\end{theorem}
\begin{proof}
    We will do induction on $a$ with the statement ``for all $b$ with $a \leq
    b$, there exists a $c : \N$ such that $a + c = b$.''  When $a = 0$, then
    setting $c = b$ is the answer because $0 + b = b$, so the base case is true.
    Now assume that for all $b$ with $a \leq b$, there exists a $c : \N$ such
    that $a + c = b$.  We must prove that if $S(a) \leq b$, there exists a $d :
    \N$ such that $S(a) + d = b$.  Now $b \neq 0$ because if it was equal to
    zero, we would have $S(a) \leq 0$.  So there exists a $b' : \N$ such that $b
    = S(b')$.  Then $S(a) \leq S(b') \rightarrow a \leq b'$, so by the inductive
    hypothesis there exists a $c$ such that $a + c = b'$.  Setting $d = c$
    works, because
    \begin{align*}
        a + c &= b' \\
        S(a + c) &= S(b') \\
        S(a) + c &= b,
    \end{align*}
    so the inductive case in true.  Thus, by induction, the theorem is true.
\end{proof}

\begin{theorem} \label{nat_lt_ex}
    For all $a$ and $b$ in $\N$, if $a < b$, then there exists a $c : \N$
    such that $a + S(c) = b$.
\end{theorem}
\begin{proof}
    By the previous theorem, we already have a $c$ with $a + c = b$.  Now if $c
    = 0$, we would have $a = b$, which contradicts $a < b$.  Thus, $c \neq 0$,
    so there exists a $c'$ such that $c = S(c')$.  We now have $a + S(c') = b$,
    so $c'$ is the number required for the theorem.
\end{proof}

Another approach to defining subtraction is to only define it for cases where
where the difference is still positive.  We can do this with optional types,
which were defined in Section \ref{sec_types}.

\begin{definition}
    Let $a$ and $b$ be natural numbers.  Then we can define a function $\ominus
    : \N \rightarrow \N \rightarrow \O(\N)$ recursively on $a$ as
    follows:
    \begin{itemize}
        \item $a \ominus 0 = \val a$
        \item $0 \ominus S(b) = \nil$
        \item $S(a) \ominus S(b) = a \ominus b$.
    \end{itemize}
\end{definition}

\begin{theorem} \label{nat_minus_eq}
    For all $a : \N$, $a \ominus a = \val 0$.
\end{theorem}
\begin{proof}
    We will use induction on $a$.  When $a = 0$, $0 \ominus 0 = \val 0$,
    so the base case is true.  Now assume that $a \ominus a = \val 0$.  Then
    $S(a) \ominus S(a) = a \ominus a = \val 0$, so the inductive case is true.
    Thus, by induction, $a \ominus a = \val 0$.
\end{proof}

\begin{theorem} \label{nat_minus_zero}
    For all $a : \N$, $a \ominus 0 = \val a$.
\end{theorem}
\begin{proof}
    This is true by definition.
\end{proof}

\begin{theorem} \label{nat_minus_lt}
    If $a < b$, then $a \ominus b = \nil$.
\end{theorem}
\begin{proof}
    By Theorem \ref{nat_lt_ex}, there is some $c$ such that $a + S(c) = b$.  We
    now need to prove that $a \ominus (a + S(c)) = a \ominus S(a + c) = \nil$.
    We will do so by induction on $a$.  When $a = 0$, $0 \ominus S(c) = \nil$,
    so the base case is true.  Now assume that $a \ominus S(a + c) = \nil$.
    Then
    \[
        S(a) \ominus S(S(a) + c) = S(a) \ominus S(S(a + c)) = a \ominus S(a + c)
        = \nil,
    \]
    so the inductive case is true.  Thus, by induction, the theorem is true.
\end{proof}

\begin{theorem} \label{nat_minus_plus}
    For all $a$ and $b$, $(a + b) \ominus a = \val b$.
\end{theorem}
\begin{proof}
    The proof will be by induction on $a$.  When $a = 0$, $b \ominus 0 = \val
    b$, so the base case is true.  Now assume that $(a + b) \ominus a = \val b$.
    Then
    \[
        (S(a) + b) \ominus S(a) = S(a + b) \ominus S(a) = (a + b) \ominus a =
        \val b,
    \]
    so the inductive case is true.  Thus, by induction, the theorem is true.
\end{proof}

Yet another way of doing subtraction in the natural numbers is to consider the
absolute value of the difference of two natural numbers.

\begin{definition}
    Let $a$ and $b$ be natural numbers.  Then we define $|a - b|$ recursively as
    follows:
    \begin{itemize}
        \item $|a - 0| = a$
        \item $|0 - b| = b$
        \item $|S(a) - S(b)| = |a - b|$.
    \end{itemize}
    Note that the notation $|a - b|$ is used here as the notation for a new
    function, not as the composition of subtraction and an absolute value.
\end{definition}

\begin{theorem}
    For all $a : \N$, $|a - a| = 0$.
\end{theorem}
\begin{proof}
    The proof will be by induction on $a$.  When $a = 0$, $|0 - 0| = 0$, so the
    base case is true.  Now when $|a - a| = 0$, $|S(a) - S(a)| = |a - a| = 0$,
    so the inductive case is true as well.  Thus, by induction, the theorem is
    true.
\end{proof}

\begin{theorem}
    For all $a$ and $b$ in $\N$, $|a - b| = |b - a|$.
\end{theorem}
\begin{proof}
    The proof will by induction on $a$ with the statement ``For all $b$, $|a -
    b| = |b - a|$.  When $a = 0$, $|0 - b| = b = |b - 0|$, so the base case is
    true.  Now assume that for all $b$, $|a - b| = |b - a|$.  We must prove that
    $|S(a) - b| = |b - S(a)|$.  Now if $b = 0$, the equation will be true for
    the same reason as before.  If $b \neq 0$, there exists a $b'$ such that $b
    = S(b')$.  Then $|S(a) - S(b')| = |a - b'|$, so by the inductive hypothesis
    $|a - b'| = |b' - a| = |S(b') - S(a)|$, so the inductive case is true.
    Thus, by induction, the theorem is true.
\end{proof}

\begin{theorem}
    For all $a$ and $b$ in $\N$, $|(a + b) - a| = b$.
\end{theorem}
\begin{proof}
    The proof will be by induction on $a$.  When $a = 0$, $|b - 0| = b$, so the
    base case is true.  Now assume that $|(a + b) - a| = b$.  Then
    \[
        |(S(a) + b) - S(a)| = |S(a + b) - S(a)| = |(a + b) - a| = b,
    \]
    so the inductive case is true.  Thus, by induction, the theorem is true.
\end{proof}

\begin{theorem} \label{nat_abs_minus_min}
    For all $a$ and $b$ in $\N$, $|a - b| + \min(a, b) = \max(a, b)$.
\end{theorem}
\begin{proof}
    There are two cases: $a \leq b$, and $b \leq a$.
    \begin{case} $a \leq b$. \\
        The equation to be proved becomes $|a - b| + a = b$.  By Theorem
        \ref{nat_le_ex}, there is some $c$ such that $a + c = b$.  Then
        \[
            |a - b| + a = |a - (a + c)| + a = c + a = b.
        \]
    \end{case}
    \begin{case} $b \leq a$. \\
        The equation to be proved becomes $|a - b| + b = a$.  By Theorem
        \ref{nat_le_ex}, there is some $c$ such that $b + c = a$.  Then
        \[
            |a - b| + b = |(b + c) - b| + b = c + b = a.
        \]
    \end{case}
\end{proof}

\begin{theorem} \label{nat_abs_minus_zero}
    For all $a$ and $b$ in $\N$, If $|a - b| = 0$, then $a = b$.
\end{theorem}
\begin{proof}
    The proof will be by induction on $a$ with the statement ``For all $b$, if
    $|a - b| = 0$, then $a = b$.''  First, for $a = 0$, we get $|0 - b| = b =
    0$, so $a = b$ as required.  Thus, the base case is true.  Now assume that
    for all $b$, if $|a - b| = 0$, then $a = b$.  Assume that $|S(a) - b| = 0$.
    Then if $b = 0$, we would have $|S(a) - 0| = S(a) = 0$, which is impossible,
    so $b \neq 0$.  Thus, there exists a $b'$ such that $b = S(b')$.  Then
    $|S(a) - S(b')| = |a - b'| = 0$, so by the inductive hypothesis, $a = b'$.
    Then we have $S(a) = S(b') = b$, so the inductive case is true.  Thus, the
    theorem is true by induction.
\end{proof}

\section{Division}

While division is impossible in the natural numbers, we can do something similar
with Euclidean division.

\begin{theorem}[Euclidean Division]
    For all natural numbers $a$ and $b$ with $b \neq 0$, there exist $q$ and $r$
    such that $a = bq + r$ and $r < b$.
\end{theorem}
\begin{proof}
    Define a set $S$ by $S = \{n : \N \mid a < bn\}$.  Now because $b \neq 0$,
    we have $0 < b$, and adding $a$ to both sides we get $a < b + a$.
    Furthermore, by Theorem \ref{nat_le_self_lmult}, $a \leq ba$, and adding $b$
    to both sides we get $b + a \leq b + ba = b(1 + a)$.  Thus, by transitivity,
    $a < b(1 + a)$, showing that $1 + a \in S$.

    By the well-ordering principle, because $S$ is nonempty, there exists a
    minimum element $q' \in S$, with $a < bq'$.  Now if $q' = 0$, we would have
    $a < 0$, which is impossible, so $q' \neq 0$.  Thus there exists a $q$ such
    that $q' = S(q)$, so we have $a < bS(q) = b + bq$.

    Now if $a < bq$, we would have $q \in S$, which contradicts the minimality
    of $q'$ in $S$.  Thus, we have $bq \leq a$.  Then by Theorem
    \ref{nat_le_ex}, we have some $r$ such that $bq + r = a$.  We thus have our
    required $q$ and $r$, and we only need to show that $r < b$.  From $a < q +
    bq$, we get $bq + r < b + bq$, and canceling $bq$ we get $r < b$ as
    required.
\end{proof}

While there is much more we can describe about division and divisibility in the
natural numbers, we will wait to do this until we have described everything in a
more abstract setting in ring theory.  The reason for including this proof here
is that it is useful in a proof involving the cardinals.

\section{Factorials}

\begin{definition}
    Given an $a : \N$, we define the factorial $a!$ recursively:
    \begin{itemize}
        \item $0! = 1$
        \item $S(n)! = S(n) n!$
    \end{itemize}
\end{definition}

\begin{definition}
    Let $n$ and $k$ be natural numbers.  Then we define the binomial coefficient
    $\binom{n}{k}$ recursively:
    \begin{itemize}
        \item $\displaystyle \binom{n}{0} = 1$
        \item $\displaystyle \binom{0}{k} = 0$
        \item $\displaystyle \binom{S(n)}{S(k)} = \binom{n}{k} +
            \binom{n}{S(k)}$.
    \end{itemize}
\end{definition}

\begin{theorem} \label{factorial_nz}
    For all $n : \N$, $n! \neq 0$.
\end{theorem}
\begin{proof}
    The proof will be by induction on $n$.  When $n = 0$, we have $1 \neq 0$, so
    the base case is true.  Now assume that $n! \neq 0$.  Then $S(n)! = S(n)
    n!$.  Now $S(n) \neq 0$ by definition, and $n! \neq 0$ by the inductive
    hypothesis, so their product is not equal to zero either, so the inductive
    case is true.  Thus, by induction, the theorem is true.
\end{proof}

\begin{theorem}
    For all $n$, $\displaystyle \binom{n}{1} = n$.
\end{theorem}
\begin{proof}
    The proof will be by induction.  When $n = 0$, $\binom{0}{1} = 0$, so the
    base case is true.  Now assume that $\binom{n}{1} = n$.  Then
    \[
        \binom{S(n)}{1} = \binom{n}{0} + \binom{n}{1} = 1 + n = S(n),
    \]
    so the inductive case is true.  Thus, by induction, the theorem is true.
\end{proof}

\begin{theorem} \label{binom_greater}
    For all $n$ and $k$ with $n < k$, $\displaystyle \binom{n}{k} = 0$.
\end{theorem}
\begin{proof}
    The proof will by induction on $n$ with the statement ``For all $k$ with $n
    < k$, $\binom{n}{k} = 0$.''  When $n = 0$, because $n < k$, $k \neq 0$, so
    we have $\binom{0}{k} = 0$ by definition, so the base case is true.  Now
    assume that for all $k$ with $n < k$, $\binom{n}{k} = 0$.  We must prove
    that for all $k$ with $S(n) < k$, $\binom{S(n)}{k} = 0$.  Now because $S(n)
    < k$, $k \neq 0$, so there exists a $k'$ such that $k = S(k')$.  We now have
    $S(n) < S(k') \rightarrow n < k'$, so by the inductive hypothesis we have
    $\binom{n}{k'} = 0$.  We also have $n < S(k')$, so again by the inductive
    hypothesis we have $\binom{n}{S(k')} = 0$.  Then
    \[
        \binom{S(n)}{k} = \binom{S(n)}{S(k')} = \binom{n}{k'} + \binom{n}{S(k')}
        = 0 + 0 = 0,
    \]
    so the inductive case is true.  Thus, by induction, the theorem is true.
\end{proof}

\begin{theorem}
    For all $n : \N$, $\displaystyle \binom{n}{n} = 1$.
\end{theorem}
\begin{proof}
    The proof will be by induction on $n$.  First, when $n = 0$, $\binom{0}{0} =
    1$ by definition, so the base case is true.  Now assume that $\binom{n}{n} =
    1$.  Then
    \[
        \binom{S(n)}{S(n)} = \binom{n}{n} + \binom{n}{S(n)} = 1 + 0 = 1,
    \]
    where in the second-to-last step we used the inductive hypothesis and
    Theorem \ref{binom_greater}
\end{proof}

\begin{theorem}
    For all $m$ and $n$ in $\N$,
    \[
        \binom{n + m}{n} n! m! = (n + m)!
    \]
\end{theorem}
\begin{proof}
    The proof will be by double induction.  The first induction will be on $m$
    with the statement ``For all $n$,
    \[
        \binom{n + m}{n} n! m! = (n + m)!.\text{''}
    \]
    First, when $m = 0$,
    \[
        \binom{n}{n} n! 0! = 1 n! 1 = n!,
    \]
    so the base case is true.  Now assume that for all $n$,
    \[
        \binom{n + m}{n} n! m! = (n + m)!.
    \]
    We must prove that
    \[
        \binom{n + S(m)}{n} n! S(m)! = (n + S(m))!.
    \]
    We will do this by another induction on $n$.  When $n = 0$, by the argument
    above, the equation holds, so the base case is true.  Now assume that
    \[
        \binom{n + S(m)}{n} n! S(m)! = (n + S(m))!.
    \]
    We must now prove that
    \[
        \binom{S(n) + S(m)}{S(n)} S(n)! S(m)! = (S(n) + S(m))!.
    \]
    We first see that
    \begin{align*}
        \binom{S(n) + S(m)}{S(n)} S(n)! S(m)! &=
            \binom{S(n + S(m))}{S(n)} S(n)! S(m)! \\
        &= \left( \binom{n + S(m)}{n} + \binom{n + S(m)}{S(n)} \right)
            S(n)! S(m)! \\
        &= \left( \binom{n + S(m)}{n} + \binom{S(n) + m}{S(n)} \right)
            S(n)! S(m)! \\
        &= \binom{n + S(m)}{n} S(n)! S(m)! +
            \binom{S(n) + m}{S(n)} S(n)! S(m)!  \\
        &= \binom{n + S(m)}{n} n! S(m)! S(n) +
            \binom{S(n) + m}{S(n)} S(n)! m! S(m). \\
    \intertext{Now using both inductive hypotheses,}
        &= (n + S(m))! S(n) + (S(n) + m)! S(m) \\
        &= S(n + m)! S(n) + S(n + m)! S(m) \\
        &= S(n + m)! (S(n) + S(m)) \\
        &= S(n + m)! S(S(n + m)) \\
        &= S(S(n + m))! \\
        &= (S(n) + S(m))!.
    \end{align*}
    This means that the inductive case for our induction on $n$ is true, which
    means that the inductive case for our induction on $m$ is true as well.
    Thus, the theorem is true by double induction.
\end{proof}

\begin{theorem}
    For all $m$ and $n$ in $\N$,
    \[
        \binom{m + n}{m} = \binom{m + n}{n}.
    \]
\end{theorem}
\begin{proof}
    We know that $(m + n)! = (n + m)!$, and using the previous theorem, we get
    \[
        \binom{m + n}{m} m! n! = \binom{n + m}{n} n! m!.
    \]
    By Theorem \ref{factorial_nz}, we can cancel $m!$ and $n!$ to get the
    desired result.
\end{proof}

\section{The Relation Between Natural Numbers And Other Types}

In this whole section, let $\U$ be a type with whatever algebraic operations are
needed.  We will be exploring several ways to connect $\U$ with $\N$.

\subsection{Sums}

\begin{definition}
    Let $f$ be a function from $\N$ to $\U$, and let $m$ and $n$ be natural
    numbers.  Then we define the sum $\sum_m^{\downarrow n} f$ recursively on
    $n$:
    \begin{align*}
        \sum_m^{\downarrow 0}    f &= 0 \\
        \sum_m^{\downarrow S(n)} f &= \sum_m^{\downarrow n} f + f(m + n).
    \end{align*}
\end{definition}
If we want to write an expression in the place of $f$, we will use the notation
\[
    \sum_{i = m}^{\downarrow n} f(i).
\]
\textbf{Note that this is very different from the usual definition of sums!} The
usual notation for sums means that $\sum_m^n$ adds the result of the function on
all values from $m$ to $n$, inclusive, whereas this definition adds all values
from $m$ to $m + n - 1$.  Working with this definition is simpler than the usual
definition when working in Coq, so we'll use it here too.  To be honest, the
down arrow in the notation is just there to make it look so weird that you are
forced to remember that the definition is not the traditional definition.

\begin{theorem}
    For all functions $f$ and natural numbers $m$, $\sum_m^{\downarrow 0} f =
    0$.
\end{theorem}
\begin{proof}
    This is true by definition.
\end{proof}

\begin{theorem}
    For all functions $f$ and natural numbers $m$, $\sum_m^{\downarrow S(n)} f =
    \sum_m^{\downarrow n} f + f(m + n)$.
\end{theorem}
\begin{proof}
    This is true by definition.
\end{proof}

\begin{theorem}
    For all functions $f$ and $g$ and natural numbers $m$ and $n$ such that for
    all $a < n$, $f(m + a) = g(m + a)$, we have
    \[
        \sum_m^{\downarrow n} f = \sum_m^{\downarrow n} g.
    \]
\end{theorem}
\begin{proof}
    The proof will be by induction on $n$ on the statement ``If $f(m + a) = g(m
    + a)$ for all $a < n$, then $\sum_m^{\downarrow m} f = \sum_m^{\downarrow n}
    g$''.  The base case is trivial.  Now assume that $f(m + a) = g(m + a)$ for
    all $a < n$ implies that $\sum_m^{\downarrow n} f = \sum_m^{\downarrow n}
    g$, and that for all $a < S(n)$, we have $f(m + a) = g(m + a)$.  Now for $a
    < n$, we have $a < S(n)$, so $f(m + a) = g(m + a)$ for all $a < n$, so the
    inductive hypothesis states that $\sum_m^{\downarrow n} f =
    \sum_m^{\downarrow n} g$.  Then
    \begin{align*}
           & \sum_m^{\downarrow S(n)} f \\
        ={}& \sum_m^{\downarrow n} f + f(m + n) \\
        ={}& \sum_m^{\downarrow n} g + f(m + n) \\
    \intertext{by the inductive hypothesis.  Because $n < S(n)$, we have $f(m +
    n) = g(m + n)$, so}
        ={}& \sum_m^{\downarrow n} g + g(m + n) \\
        ={}& \sum_m^{\downarrow S(n)} g,
    \end{align*}
    showing that the inductive case is true.  Thus, the theorem is true by
    induction.
\end{proof}

\begin{theorem}
    For all natural numbers $m$ and $n$, $\sum_m^{\downarrow n} 0 = 0$.
\end{theorem}
\begin{proof}
    The proof will be by induction on $n$.  The base case is trivial.  Now
    assume that $\sum_m^{\downarrow n} 0 = 0$.  Then
    \begin{align*}
           & \sum_m^{\downarrow S(n)} 0 \\
        ={}& \sum_m^{\downarrow n} + 0 \\
        ={}& 0 + 0 \\
        ={}& 0,
    \end{align*}
    so the inductive case is true.  Thus, the theorem is true by induction.
\end{proof}

\begin{theorem}
    For all functions $f$ and natural numbers $m$ and $n$, if for all $a < n$,
    $f(m + a) = 0$, then $\sum_m^{\downarrow n} f = 0$.
\end{theorem}
\begin{proof}
    This follows directly from the previous two theorems.
\end{proof}

\begin{theorem}
    For all functions $f$ and natural numbers $a$, $b$, and $c$,
    \[
        \sum_a^{\downarrow b} f + \sum_{a + b}^{\downarrow c} f =
        \sum_a^{\downarrow b + c} f.
    \]
\end{theorem}
\begin{proof}
    The proof will be by induction on $c$.  When $c = 0$,
    \[
        \sum_a^{\downarrow b} f + \sum_{a + b}^{\downarrow 0} f =
        \sum_a^{\downarrow b} f + 0 =
        \sum_a^{\downarrow b + 0} f,
    \]
    so the base case is true.  Now assume that
    \[
        \sum_a^{\downarrow b} f + \sum_{a + b}^{\downarrow c} f =
        \sum_a^{\downarrow b + c} f.
    \]
    Then
    \begin{align*}
           & \sum_a^{\downarrow b} f + \sum_{a + b}^{\downarrow S(c)} f \\
        ={}& \sum_a^{\downarrow b} f + \sum_{a + b}^{\downarrow c} f
            + f(a + b + c) \\
        ={}& \sum_a^{\downarrow b + c} f + f(a + (b + c)) \\
        ={}& \sum_a^{\downarrow S(b + c)} f \\
        ={}& \sum_a^{\downarrow b + S(c)} f,
    \end{align*}
    so the inductive case is true.  Thus, by induction, the theorem is true.
\end{proof}

\begin{theorem}
    For all functions $f$ and natural numbers $a$, $b$, and $c$,
    \[
        \sum_{n = a}^{\downarrow b} f(n + c) = \sum_{a + c}^{\downarrow b} f.
    \]
\end{theorem}
\begin{proof}
    The proof will by induction on $b$.  When $b = 0$,
    \[
        \sum_{n = a}^{\downarrow 0} f(n + c) = 0 =
        \sum_{a + c}^{\downarrow 0} f,
    \]
    so the base case is true.  Now assume that
    \[
        \sum_{n = a}^{\downarrow b} f(n + c) = \sum_{a + c}^{\downarrow b} f.
    \]
    Then
    \begin{align*}
           & \sum_{n = a}^{\downarrow S(b)} f(n + c) \\
        ={}& \sum_{n = a}^{\downarrow b} f(n + c) + f (a + c + b) \\
        ={}& \sum_{a + c}^{\downarrow b} f + f((a + c) + b) \\
        ={}& \sum_{a + c}^{\downarrow S(b)} f,
    \end{align*}
    so the inductive case is true.  Thus, by induction, the theorem is true.
\end{proof}

\begin{theorem}
    For all functions $f$ and $g$ and natural numbers $a$ and $b$,
    \[
        \sum_{n = a}^{\downarrow b} (f(n) + g(n)) =
        \sum_a^{\downarrow b} f + \sum_a^{\downarrow b} g.
    \]
\end{theorem}
\begin{proof}
    The proof will be by induction on $b$.  When $b = 0$,
    \[
        \sum_{n = a}^{\downarrow 0} (f(n) + g(n)) =
        0 = 0 + 0 =
        \sum_a^{\downarrow 0} f + \sum_a^{\downarrow 0} g,
    \]
    so the base case is true.  Now assume that
    \[
        \sum_{n = a}^{\downarrow b} (f(n) + g(n)) =
        \sum_a^{\downarrow b} f + \sum_a^{\downarrow b} g.
    \]
    Then
    \begin{align*}
           & \sum_{n = a}^{\downarrow S(b)} (f(n) + g(n)) \\
        ={}& \sum_{n = a}^{\downarrow b} (f(n) + g(n)) + f(a + b) + g(a + b) \\
        ={}& \sum_a^{\downarrow b} f + \sum_a^{\downarrow b} g
            + f(a + b) + g(a + b) \\
        ={}& \sum_a^{\downarrow b} f + f(a + b) +
             \sum_a^{\downarrow b} g + g(a + b) \\
        ={}& \sum_a^{\downarrow S(b)} f + \sum_a^{\downarrow S(b)} g,
    \end{align*}
    so the inductive case is true.  Thus, by induction, the theorem is true.
\end{proof}

\begin{theorem}
    For all functions $f$ and natural numbers $a$ and $b$,
    \[
        \sum_{n = a}^{\downarrow b} (-f(n)) = -\sum_a^{\downarrow b} f.
    \]
\end{theorem}
\begin{proof}
    The proof will be by induction on $b$.  When $b = 0$,
    \[
        \sum_{n = a}^{\downarrow 0} (-f(n)) =
        0 = -0 =
        -\sum_a^{\downarrow 0} f,
    \]
    so the base case is true.  Now assume that
    \[
        \sum_{n = a}^{\downarrow b} (-f(n)) = -\sum_a^{\downarrow b} f.
    \]
    Then
    \begin{align*}
           & \sum_{n = a}^{\downarrow S(b)} (-f(n)) \\
        ={}& \sum_{n = a}^{\downarrow b} (-f(n)) - f(a + b) \\
        ={}& -\sum_a^{\downarrow b} f - f(a + b) \\
        ={}& -\left( \sum_a^{\downarrow b} f + f(a + b) \right) \\
        ={}& -\sum_a^{\downarrow S(b)} f,
    \end{align*}
    so the inductive case is true.  Thus, by induction, the theorem is true.
\end{proof}

\subsection{Multiplying by Natural Numbers}

\begin{definition}
    Given an $n : \N$ and an $a : \U$, we can define the product $na$
    recursively on $n$:
    \begin{itemize}
        \item $0a = 0$
        \item $S(n)a = a + na$.
    \end{itemize}
    Note that while we are using the same notation as multiplication, this is
    not an instance of Class \ref{mult}.
\end{definition}

\begin{theorem}
    For all $a : \N$, $a0 = 0$, where $0 : \U$.
\end{theorem}
\begin{proof}
    The proof will by induction on $a$.  When $a = 0$, $00 = 0$ by definition,
    so the base case is true.  Now assume that $a0 = 0$.  Then $S(a)0 = 0 + a0 =
    a0 = 0$, so the inductive case is true.  Thus, the theorem is true by
    induction.
\end{proof}

\begin{theorem}
    For all $a : \U$, $1a = a$, where $1 : \N$.
\end{theorem}
\begin{proof}
    \[
        1a = S(0)a = a + 0a = a + 0 = a.
    \]
\end{proof}

\begin{theorem} \label{nat_mult_commute_single}
    For all $a : \N$ and $b$ and $c$ in $\U$, if $b + c = c + b$, then $ab + c =
    c + ab$.
\end{theorem}
\begin{proof}
    The proof will be by induction on $a$.  When $a = 0$, $0b + c = c = c + 0b$,
    so the base case is true.  Now assume that $ab + c = c + ab$.  Then
    \[
        S(a)b + c = b + ab + c = b + c + ab = c + b + ab = c + S(a)b,
    \]
    so the inductive case is true.  Thus, the theorem is true by induction.
\end{proof}

\begin{theorem}
    For all $a : \N$ and $b$ and $c$ in $\U$, if $b + c = c + b$, then $a(b + c)
    = ab + ac$.
\end{theorem}
\begin{proof}
    The proof will be by induction on $a$.  When $a = 0$, then $0(b + c) = 0 = 0
    + 0 = 0b + 0c$, so the base case is true.  Now assume that $a(b + c) = ab +
    ac$.  Then
    \[
        S(a)(b + c) = b + c + a(b + c) = b + c + ab + ac = (b + ab) + (c + ac) =
        S(a)b + S(a)c,
    \]
    so the inductive case is true.  Thus, the theorem is true by induction.
\end{proof}

\begin{theorem}
    For all $a$ and $b$ in $\N$ and $c$ in $\U$, $(a + b)c = ac + bc$.
\end{theorem}
\begin{proof}
    The proof will be by induction on $a$.  When $a = 0$, then $(0 + b)c = bc =
    0a + bc$, so the base case is true.  Now assume that $(a + b)c = ac + bc$.
    Then
    \[
        (S(a) + b)c = S(a + b)c = c + (a + b)c = c + ac + bc = S(a)c + bc,
    \]
    so the inductive case is true.  Thus, the theorem is true by induction.
\end{proof}

\begin{theorem}
    For all $a$ and $b$ in $\N$ and $c$ in $\U$, $(ab)c = a(bc)$.
\end{theorem}
\begin{proof}
    The proof will be by induction on $a$.  When $a = 0$, then $(0b)c = 0c = 0 =
    0(bc)$, so the base case is true.  Now assume that $(ab)c = a(bc)$.
    Then
    \[
        (S(a)b)c = (b + ab)c = bc + (ab)c = bc + a(bc) = S(a)(bc),
    \]
    so the inductive case is true.  Thus, the theorem is true by induction.
\end{proof}

\begin{theorem}
    For all $a$ in $\N$ and $b$ and $c$ in $\U$, $a(bc) = (ab)c$.
\end{theorem}
\begin{proof}
    The proof will be by induction on $a$.  When $a = 0$, $0(bc) = 0 = (0b)c$,
    so the base case is true.  Now assume that $a(bc) = (ab)c$.  Then
    \[
        S(a)(bc) = bc + a(bc) = bc + ab(c) = (b + ab)c = (S(a)b)c,
    \]
    so the inductive case is true.  Thus, the theorem is true by induction.
\end{proof}

\begin{theorem}
    For all $a : \N$ and $b : \U$, $-(ab) = a(-b)$.
\end{theorem}
\begin{proof}
    The proof will be by induction on $a$.  When $a = 0$, $-(0b) = 0 = 0(-b)$,
    so the base case is true.  Now assume that $-(ab) = a(-b)$.  Then
    \[
        -(S(a)b) = -(b + ab) = -(ab) - b = a(-b) - b = S(a)(-b),
    \]
    so the inductive case is true.  Thus, the theorem is true by induction.
\end{proof}

\begin{theorem} \label{nat_mult_commute_double}
    For all $a$ and $b$ in $\N$ and $c$ and $d$ in $\U$, if $c + d = d + c$,
    then $ac + bd = bd + ac$.
\end{theorem}
\begin{proof}
    The proof will be by induction on $a$.  When $a = 0$, $0c + bd = bd = bd +
    0c$, so the base case is true.  Now assume that $ac + bd = bd + ac$.  Then
    \[
        S(a)c + bd = c + ac + bd = c + bd + ac = bd + c + ac = bd + S(a)c,
    \]
    so the inductive case is true.  Thus, the theorem is true by induction.
\end{proof}

\begin{theorem} \label{nat_mult_commute}
    For all natural numbers $a$ and $b$ and $c : \U$, $ac + bc = bc + ac$ even
    when addition in $\U$ is not commutative.
\end{theorem}
\begin{proof}
    This follows directly from the previous theorem.
\end{proof}

\begin{theorem} \label{nat_mult_commute_neg}
    For all natural numbers $a$ and $b$ and $c : \U$, $ac - bc = -bc + ac$ even
    when addition in $\U$ is not commutative.
\end{theorem}
\begin{proof}
    Because $c$ commutes with $-c$, this follows directly from Theorem
    \ref{nat_mult_commute_double}.
\end{proof}

\subsection{Exponentiating by Natural Numbers}

\begin{definition}
    Given an $a : \U$ and an $n : \N$, we can define the $n$th power of $a$
    recursively on $n$:
    \begin{itemize}
        \item $a^0 = 1$
        \item $a^{S(n)} = a^na$.
    \end{itemize}
    Note that with this definition, $0^0 = 1$.
\end{definition}

\begin{theorem}
    For all $n : \N$, $0^{S(n)} = 0$.
\end{theorem}
\begin{proof}
    \[
        0^{S(n)} = 0^n0 = 0.
    \]
\end{proof}

\begin{theorem}
    For all $a : \U$, $a^1 = a$.
\end{theorem}
\begin{proof}
    \[
        a^1 = a^{S(0)} = a^0a = 1a = a.
    \]
\end{proof}

\begin{theorem}
    For all $n : \N$, $1^n = 1$.
\end{theorem}
\begin{proof}
    The proof will be by induction on $n$.  When $n = 0$, $1^0 = 1$, so the base
    case is true.  Now assume that $1^n = 1$.  Then
    \[
        1^{S(n)} = 1^n1 = 1^n = 1,
    \]
    so the inductive case is true.  Thus, the theorem is true by induction.
\end{proof}

\begin{theorem}
    For all $a : \U$ and all $b$ and $c$ in $\N$,
    \[
        a^{b + c} = a^b a^c.
    \]
\end{theorem}
\begin{proof}
    The proof will be by induction on $c$.  When $c = 0$,
    \[
        a^{b + 0} = a^b = a^b1 = a^ba^0,
    \]
    so the base case is true.  Now assume that $a^{b + c} = a^b a^c$.  Then
    \[
        a^{b + S(c)} = a^{S(b + c)} = a^{b + c}a = a^b a^c a = a^b a^{S(c)},
    \]
    so the inductive case is true.  Thus, by induction, the theorem is true.
\end{proof}

\begin{theorem}
    For all $a$ and $b$ in $\U$ and $n : \N$,
    \[
        (ab)^n = a^nb^n.
    \]
\end{theorem}
\begin{proof}
    The proof will be by induction on $n$.  When $n = 0$,
    \[
        (ab)^0 = 1 = 1(1) = a^0b^0,
    \]
    so the base case is true.  Now assume that $(ab)^n = a^nb^n$.  Then
    \[
        (ab)^{S(n)} = (ab)^n ab = a^n b^n a b = a^n a b^n b = a^{S(n)} b^{S(n)},
    \]
    so the inductive case is true.  Thus, by induction, the theorem is true.
\end{proof}

\begin{theorem}
    For all $a$ in $\U$ and $b$ and $c$ in $\N$,
    \[
        (a^m)^n = a^{mn}.
    \]
\end{theorem}
\begin{proof}
    The proof will be by induction on $n$.  When $n = 0$,
    \[
        (a^m)^0 = 1 = a^0 = a^{m0},
    \]
    so the base case is true.  Now assume that $(a^m)^n = a^{mn}$.  Then
    \[
        (a^m)^{S(n)} = (a^m)^n a^m = a^{mn} a^m = a^{mn + m} = a^{mS(n)},
    \]
    so the inductive case is true.  Thus, by induction, the theorem is true.
\end{proof}

\begin{theorem} \label{nat_pow_not_zero}
    For all $a : \U$ and $n : \N$, if $a \neq 0$, then $a^n \neq 0$.
\end{theorem}
\begin{proof}
    The proof will be by induction on $n$.  When $n = 0$, we have $a^0 = 1 \neq
    0$, so the base case is true.  Now assume that $a^n \neq 0$.  Then $a^{S(n)}
    = a^na.$  Because $a^n \neq 0$ and $a \neq 0$, we have $a^na \neq 0$, so the
    inductive case is true.  Thus, by induction, the theorem is true.
\end{proof}

\begin{theorem} \label{nat_pow_neg_even}
    For all $n : \N$, $(-1)^{2n} = 1$, where $1 : \U$.
\end{theorem}
\begin{proof}
    The proof will be by induction on $n$.  When $n = 0$, $(-1)^0 = 1$, so the
    base case is true.  Now assume that $(-1)^{2n} = 1$.  Then
    \[
        (-1)^{2S(n)} = (-1)^{2n + 2} = (-1)^{2n} (-1)^2 = 1 (-1)^2 = (-1)^2 =
        (-1)^1 (-1) = (-1) (-1) = 1,
    \]
    so the inductive case is true.  Thus, by induction, the theorem is true.
\end{proof}

\begin{theorem} \label{nat_pow_neg_odd}
    For all $n : \N$, $(-1)^{2n + 1} = -1$, where $-1 : \U$.
\end{theorem}
\begin{proof}
    \[
        (-1)^{2n + 1} = (-1)^{2n} (-1)^1 = 1 (-1) = -1.
    \]
\end{proof}

\begin{theorem} \label{nat_pow_neg_binom2}
    For all $n : \N$,
    \[
        (-1)^{\binom{n + 2}{2}} = -(-1)^{\binom{n}{2}},
    \]
    where $-1 : \U$.
\end{theorem}
\begin{proof}
    \begin{align*}
        (-1)^{\binom{n + 2}{2}}
        &= (-1)^{\binom{n + 1}{1} + \binom{n + 1}{2}} \\
        &= (-1)^{\binom{n}{0} + \binom{n}{1} + \binom{n}{1} + \binom{n}{2}} \\
        &= (-1)^{1 + n + \binom{n}{2} + n} \\
        &= (-1)^{1} (-1)^{2n} (-1)^{\binom{n}{2}} \\
        &= (-1) (1) (-1)^{\binom{n}{2}} \\
        &= -(-1)^{\binom{n}{2}}.
    \end{align*}
\end{proof}

\begin{theorem} \label{nat_pow_pos}
    For all $a : \U$ with $0 \leq a$, $0 \leq a^n$ for all $n : \N$.
\end{theorem}
\begin{proof}
    The proof will be by induction on $n$.  When $n = 0$, $0 \leq 1 = a^0$, so
    the base case is true.  Now assume that $0 \leq a^n$.  Then because $0 \leq
    a$ and $0 \leq a^n$, we have $0 \leq a^na = a^{S(n)}$, so the inductive case
    is true.  Thus, by induction, the theorem is true.
\end{proof}

\begin{theorem} \label{nat_pow_pos2}
    For all $a : \U$ with $0 < a$, $0 < a^n$ for all $n : \N$.
\end{theorem}
\begin{proof}
    This proof is identical in form to the previous one.
\end{proof}

\begin{theorem}
    For all $a : \U$ and $m$ and $n$ in $\N$, if $1 \leq a$ and $m \leq n$, then
    $a^m \leq a^n$.
\end{theorem}
\begin{proof}
    By Theorem \ref{nat_le_ex}, there exists some $c : \N$ such that $m + c =
    n$.  We now need to prove that $a^m \leq a^{m + c}$, which we will do by
    induction on $c$.  When $c = 0$, we have $a^m \leq a^m$, so the base case is
    true.  Now assume that $a^m \leq a^{m + c}$.  Because $1 \leq a$, we have $0
    < a$, so by Theorem \ref{nat_pow_pos2} we have $0 < a^{m + c}$.  We can thus
    multiply the inequality $1 \leq a$ by $a^{m + c}$ to get $a^{m + c} \leq
    a^{m + c}a = a^{m + S(c)}$.  Because $a^m \leq a^{m + c}$ and $a^{m + c}
    \leq a^{m + S(c)}$, by transitivity we have $a^m \leq a^{m + S(c)}$, so the
    inductive case is true.  Thus, by induction, the theorem is true.
\end{proof}

\begin{theorem}
    For all $a : \U$ and $m$ and $n$ in $\N$, if $1 < a$ and $m < n$, then $a^m
    < a^n$.
\end{theorem}
\begin{proof}
    Before proving the theorem the following lemma is useful: For all $n : \N$,
    $a^n < a^na$.  Because $1 < a$, $0 < a$, so by Theorem \ref{nat_pow_pos2},
    $0 < a^n$.  Thus, we can multiply the inequality $1 < a$ by $a^n$ to get
    $a^n < a^na$.

    Now let $m$ and $n$ be natural numbers such that $m < n$.  By Theorem
    \ref{nat_lt_ex}, there exists some $c : \N$ such that $m + S(c) = n$.  We
    now need to prove that $a^m < a^{m + S(c)}$, which we will do by induction
    on $c$.  When $c = 0$, we have $a^m < a^ma$, which follows from the lemma,
    so the base case is true.  Now assume that $a^m < a^{m + S(c)}$.  Then by
    the lemma, $a^{m + S(c)} < a^{m + S(c)}a = a^{m + S(S(c))}$, so by
    transitivity we have $a^m < a^{m + S(S(c))}$, so inductive case is true.
    Thus, by induction, the theorem is true.
\end{proof}

\begin{theorem} \label{nat_pow_le_one}
    For all $a : \U$ with $1 \leq a$, $1 \leq a^n$ for all $n : \N$.
\end{theorem}
\begin{proof}
    The proof will be by induction on $n$.  When $n = 0$, $1 \leq 1 = a^0$, so
    the base case is true.  Now assume that $1 \leq a^n$.  Then because
    everything involved is positive, we can multiply the inequalities $1 \leq
    a^n$ and $1 \leq a$ to get $1 \leq a^na = a^{S(n)}$, so the inductive case
    is true.  Thus, the theorem is true by induction.
\end{proof}

\begin{theorem} \label{nat_pow_lt_one}
    For all $a : \U$ with $1 < a$, $1 < a^{S(n)}$ for all $n : \N$.
\end{theorem}
\begin{proof}
    The proof will be by induction on $n$.  When $n = 0$, $1 < a = a^1$, so the
    base case is true.  Now assume that $1 < a^{S(n)}$.  Then because everything
    involved is positive, we can multiply the inequalities $1 < a^{S(n)}$ and $1
    < a$ to get $1 < a^{S(n)}a = a^{S(S(n))}$, so the inductive case is true.
    Thus, the theorem is true by induction.
\end{proof}

\subsection{The Natural Numbers in Other Types}

\begin{definition} \label{from_nat}
    We can define an inclusion $\iN : \N \rightarrow \U$ recursively on an
    argument $n$:
    \begin{itemize}
        \item $\iN(0_\N) = 0_\U$
        \item $\iN(S(n)) = 1_\U + \iN(n)$.
    \end{itemize}
    After this chapter, the function $\iN$ will be not be written and will
    be assumed to be used any time a natural number appears in an expression
    involving other types, unless the notation is necessary to avoid ambiguity.
\end{definition}

From this function, we can define the characteristic of a type.

\begin{class}
    We say that the characteristic of a type $\U$ is $n$ if $n$ is the least
    nonzero natural number such that $\iN(n) = 0$.  If no such $n$ exists,
    we say that the characteristic is zero.
\end{class}

It can also be useful to just keep track of what values are zero or nonzero.
Using this name for this concept is original to me.

\begin{class}
    A type $\U$ has nullity $n$ if $\iN(n) = 0$.
\end{class}
\begin{class}
    A type $\U$ does not have nullity $n$ if $\iN(n) \neq 0$.
\end{class}

We easily see that a type has a characteristic of zero if and only if it does
not have nullity $n$ for all $n \neq 0$.

\begin{theorem}
    For all $n : \N$, $\iN(n) = n$.
\end{theorem}
\begin{proof}
    The proof will be by induction on $n$.  When $n = 0$, $\iN(0) = 0$, so
    the base case is true.  Now assume that $\iN(n) = n$.  Then
    \[
        \iN(S(n)) = 1 + \iN(n) = 1 + n = S(n),
    \]
    so the inductive case is true.  Thus, by induction, the theorem is true.
\end{proof}

\begin{instance}
    If $\U$ has characteristic zero, then $\iN$ is injective.
\end{instance}
\begin{proof}
    Assume that $a$ and $b$ are natural numbers such that $\iN(a) =
    \iN(b)$.  We must prove that $a = b$.  We will do this by induction on
    $a$ with the statement ``For all $b$, $\iN(a) = \iN(b) \rightarrow
    a = b$.''  When $a = 0$, we have $0 = \iN(b)$.  Because $\U$ has
    characteristic zero, we must have $b = 0$, so the base case is true.  Now
    assume that for all $b$, $\iN(a) = \iN(b) \rightarrow a = b$, and
    that $\iN(S(a)) = \iN(b)$.  We must prove that $S(a) = b$.  Now if
    $b = 0$, we would have $\iN(S(a)) = 0$, which contradicts $\U$ having
    characteristic zero.  Thus, $b \neq 0$, so there exists a $b'$ such that $b
    = S(b')$.  We now have
    \begin{align*}
        \iN(S(a)) &= \iN(b) \\
        \iN(S(a)) &= \iN(S(b')) \\
        1 + \iN(a) &= 1 + \iN(b') \\
        \iN(a) &= \iN(b') \\
        a &= b' \\
        S(a) &= S(b') = b,
    \end{align*}
    so the inductive case is true.  Thus, by induction, the theorem is true.
\end{proof}

\begin{theorem}
    For all $a : \N$, $a1_\U = \iN(a)$.
\end{theorem}
\begin{proof}
    The proof will be by induction an $a$.  When $a = 0$, $0(1_\U) = 0 =
    \iN(0)$, so the base case is true.  Now assume that $a1_\U = \iN(a)$.  Then
    \[
        S(a)1_\U = 1_\U + a1_\U = 1_\U + \iN(a) = \iN(S(a)),
    \]
    so the inductive case is true.  Thus, by induction, the theorem is true.
\end{proof}

\begin{instance}
    $\iN$ is additive.
\end{instance}
\begin{proof}
    The proof will be by induction on $a$.  When $a = 0$,
    \[
        \iN(0 + b) = \iN(b) = 0 + \iN(b) = \iN(0) + \iN(b),
    \]
    so the base case is true.  Now assume that $\iN(a + b) = \iN(a) + \iN(b)$.
    Then
    \[
        \iN(S(a) + b) = \iN(S(a + b)) = 1 + \iN(a + b) = 1 + \iN(a) + \iN(b) =
        \iN(S(a)) + \iN(b),
    \]
    so the inductive case is true.  Thus, by induction, the theorem is true.
\end{proof}

\begin{instance}
    $\iN$ is multiplicative.
\end{instance}
\begin{proof}
    The proof will be by induction on $a$.  When $a = 0$,
    \[
        \iN(0b) = \iN(0) = 0 = 0\iN(b) = \iN(0)\iN(b),
    \]
    so the base case is true.  Now assume that $\iN(ab) = \iN(a)\iN(b)$.
    Then
    \[
        \iN(S(a)b) = \iN(b + ab) = \iN(b) + \iN(ab) = \iN(b) + \iN(a)\iN(b) =
        (1 + \iN(a)) \iN(b) = \iN(S(a))\iN(b),
    \]
    so the inductive case is true.  Thus, by induction, the theorem is true.
\end{proof}

\begin{theorem}
    For all $a : \N$ and $b : \U$, $ab = \iN(a)b$.
\end{theorem}
\begin{proof}
    The proof will be by induction on $a$.  When $a = 0$, $0 = \iN(0)$, so $0b =
    \iN(0)b$, so the base case is true.  Now assume that $ab = \iN(a)b$.  Then
    \[
        S(a)b = b + ab = b + \iN(a)b = (1 + \iN(a))b = \iN(S(a))b,
    \]
    so the inductive case is true.  Thus, by induction, the theorem is true.
\end{proof}

\begin{theorem} \label{from_nat_pos2}
    For all $n : \N$, $0_\U \leq \iN(n)$.
\end{theorem}
\begin{proof}
    The proof will be by induction on $n$.  When $n = 0$, $0 \leq 0 = \iN(0)$,
    so the base case is true.  Now assume that $0 \leq \iN(n)$.  Then because $0
    \leq 1$, we can add the two inequalities to get $0 \leq 1 + \iN(n) =
    \iN(S(n))$, so the inductive case is true.  Thus, by induction, the theorem
    is true.
\end{proof}

\begin{theorem} \label{from_nat_pos}
    For all $n : \N$, $0_\U < \iN(S(n))$.
\end{theorem}
\begin{proof}
    The proof will be by induction on $n$.  When $n = 0$, $0 < 1 = \iN(1)$, so
    the base case is true.  Now assume that $0 < \iN(S(n))$.  Then because $0 <
    1$, we can add the two inequalities to get $0 < 1 + \iN(S(n)) =
    \iN(S(S(n)))$, so the inductive case is true.  Thus, by induction, the
    theorem is true.
\end{proof}

\begin{instance}
    An ordered ring has characteristic zero.
\end{instance}
\begin{proof}
    This follows directly from the previous theorem.
\end{proof}

\begin{instance}
    $\iN$ is orderly.
\end{instance}
\begin{proof}
    Let $a$ and $b$ be natural numbers.  We must prove that $a \leq b
    \rightarrow \iN(a) \leq \iN(b)$.  We will use induction on $a$ with the
    statement ``For all $b : \N$, $a \leq b \rightarrow \iN(a) \leq \iN(b)$.''
    When $a = 0$ we need to prove $\iN(0) = 0 \leq \iN(b)$, which follows from
    Theorem \ref{from_nat_pos2}, so the base case is true.  Now assume that for
    all $b : \N$, $a \leq b \rightarrow \iN(a) \leq \iN(b)$, and that $S(a) \leq
    b$.  Then if $b = 0$, we would have $S(a) \leq 0$, which is impossible.
    Thus, $b \neq 0$, so there exists a $b'$ such that $b = S(b')$.  Then
    \begin{align*}
        S(a) &\leq b \\
        S(a) &\leq S(b') \\
        a &\leq b' \\
        \iN(a) &\leq \iN(b') \\
        1 + \iN(a) &\leq 1 + \iN(b') \\
        \iN(S(a)) &\leq \iN(S(b')) \\
        \iN(S(a)) &\leq \iN(b),
    \end{align*}
    so the inductive case is true.  Thus, by induction, the theorem is true.
\end{proof}

Notice that by Instances \ref{homo_le_lt}, \ref{homo_le_le2}, and
\ref{homo_lt_lt2}, $\iN$ is both equivalently orderly and strictly equivalently
orderly.

\begin{theorem} \label{from_nat_pos1}
    For all $a$, $1 \leq \iN(S(a))$.
\end{theorem}
\begin{proof}
    Because $0 \neq S(a)$, by Theorem \ref{nat_neq0_leq1} we have $1 \leq S(a)$.
    Thus, we have $\iN(1) = 1 \leq \iN(S(a))$.
\end{proof}

\begin{instance}
    A ring of characteristic zero is not trivial.
\end{instance}
\begin{proof}
    We instantly have $0 \neq 1$ from the definition of characteristic zero.
\end{proof}
\noindent Note that this doesn't allow us to remove non-triviality from the
conditions for an ordered ring/field because proving that an ordered ring has
characteristic zero uses non-triviality.

\subsection{The Archimedean Property}

The most general definition of the Archimedean property is the following:

\begin{class}
    Let $\U$ be an ordered group.  Then $\U$ is Archimedean if for all $x > 0$
    and $y > 0$, there exists some $n : \N$ such that $x < ny$.
\end{class}

Two other useful characterizations (which are not always equivalent to the
original definition!) are as follows:

\begin{definition}
    $\U$ is Archimedean$_1$ if for all $x > 0$ in $\U$, there exists an $n : \N$
    such that $x < \iN(n)$.
\end{definition}

\begin{definition}
    $\U$ is Archimedean$_2$ if for all $\varepsilon > 0$ in $\U$, there exists
    an $n : \N$ such that $\iN(S(n))^{-1} < \varepsilon$.
\end{definition}

While these characterizations are not always equivalent, they are equivalent in
ordered fields.

\begin{theorem}
    In an ordered field, being Archimedean is the same thing as being
    Archimedean$_1$.
\end{theorem}
\begin{proof}
    We will first prove that Archimedean$_1$ $\rightarrow$ Archimedean.  Let $x$
    and $y$ both be positive values.  Then $\frac{x}{y}$ is positive, so because
    $\U$ is Archimedean$_1$, there exists an $n$ such that $\frac{x}{y} < n$.
    Then multiplying both sides by $y$ we get $x < ny$ as required.

    Now we will prove that Archimedean $\rightarrow$ Archimedean$_1$.  Let $x$
    be a value in $\U$ with $0 < x$.  Then because $0 < 1$ and $\U$ is
    Archimedean, we have some $n$ such that $x < n1 = n$ as required.
\end{proof}

\begin{theorem}
    In an ordered field, being Archimedean is the same thing as being
    Archimedean$_2$.
\end{theorem}
\begin{proof}
    We will prove this by showing that being Archimedean$_1$ is equivalent to
    being \linebreak Archimedean$_2$.

    We will first prove that Archimedean$_2$ $\rightarrow$ Archimedean$_1$.  Let
    $x > 0$.  Then $x^{-1} > 0$ as well, so because $\U$ is Archimedean$_2$, we
    have an $n : \N$ such that $S(n)^{-1} < x^{-1}$.  Because $0 < S(n)$ and $0
    < x$ we can invert this to $x < S(n)$ as required.

    Now we will prove that Archimedean$_1$ $\rightarrow$ Archimedean$_2$.  Let
    $\varepsilon$ be a value in $\U$ with $0 < \varepsilon$.  Then
    $0 < \varepsilon^{-1}$ as well, so because $\U$ is Archimedean$_1$ we have
    an $n : \N$ such that $\varepsilon^{-1} < n$.  Now if $n$ was zero we would
    have $\varepsilon^{-1} < 0$ which contradicts $0 < \varepsilon^{-1}$, so $n
    \neq 0$.  We thus have an $n'$ such that $n = S(n')$, so we now have
    $\varepsilon^{-1} < S(n)$.  Because both $\varepsilon^{-1}$ and $S(n)$ are
    greater than zero, we can invert this inequality to get $S(n)^{-1} <
    \varepsilon$ as required.
\end{proof}

\begin{theorem}
    If an ordered ring is Archimedean$_1$, then for all $x : \U$, there exists
    an $n$ with $x < \iN(n)$.
\end{theorem}
\begin{proof}
    If $x > 0$, then this follows from $\U$ being Archimedean$_1$.  If $x \leq
    0$, then $n = 1$ works.
\end{proof}

The following theorem can be useful sometimes in analysis.

\begin{theorem} \label{arch_pow2}
    In an Archimedean ordered field, for all $\varepsilon > 0$, there exists an
    $n$ with $(2^n)^{-1} < \varepsilon$.
\end{theorem}
\begin{proof}
    By the Archimedean property, there exists an $n$ such that $S(n)^{-1} <
    \varepsilon$.  We will show that $(2^{S(n)})^{-1} < \varepsilon$.  Because
    $S(n)^{-1} < \varepsilon$, the result will follow from transitivity if we
    can prove that $(2^{S(n)})^{-1} < S(n)^{-1}$.  Both of these are positive,
    so we can invert the inequality to get $S(n) < 2^{S(n)}$.  Let $n' = S(n)$.
    We will prove this new inequality by induction on $n'$, so we need to prove
    that $n' < 2^{n'}$.  First, when $n' = 0$, we have $0 < 1 = 2^0$, so the
    base case is true.  Now assume that $n' < 2^{n'}$.  Then we have $S(n') < 1
    + 2^{n'}$.  By Theorem \ref{nat_pow_le_one}, we have $1 \leq 2^{n'}$, and by
    adding $2^{n'}$ to both sides we get $1 + 2^{n'} \leq 2^{n'} + 2^{n'} =
    2^{n'}2 = 2^{S(n')}$.  Thus, by transitivity, we have $S(n') < 2^{S(n')}$,
    so the inductive case is true.  Thus, the theorem is true by induction.
\end{proof}

\end{document}
