\documentclass[../../math.tex]{subfiles}
\externaldocument{../../math.tex}
\externaldocument{foundations}

\begin{document}

\setcounter{chapter}{1}

\chapter{Elementary Algebra} \label{chap_ealgebra}

In this chapter we will develop an extensive system of classes that encode basic
algebraic manipulations.  Note that the classes we define will not be things
like groups, rings, fields, etc.  Instead, we will define classes for each
individual property that we might want.  This ``unbundled'' approach, as it's
called, has a much wider range of applicability than the bundled approach.
There are many types that don't satisfy all of the axioms for a ring or a field
but which still have enough to do basic algebra.  (A notable example is the
ordinals, which have addition and multiplication defined, but neither is
commutative.)  We will still define the traditional algebraic concepts, but they
will not be used that much in this chapter.

In addition to considering single algebraic types, we will also be describing
maps between two algebraic types, similar to homomorphisms.  Similar to the
unbundled approach taken to algebraic structures, the properties of these
functions will be unbundled as well.  More traditional homomorphisms will be
defined later.

In the Coq code, there are notably many more theorems than what are said here.
Those are for convenience when doing algebra in Coq, but the theorems are so
simple that here they can just be considered as ``skipping'' a few trivial
steps.

Throughout this chapter, let $\U$ be a type.  All classes defined are
parameterized by this type.

\section{Addition and Subtraction}

\begin{class}
    A binary operation $\U \rightarrow \U \rightarrow \U$ which is called
    addition, denoted with +.
\end{class}

\begin{class}
    Addition is associative if for all $a$, $b$, and $c$ in $\U$, $a + (b + c) =
    (a + b) + c$.
\end{class}

\begin{class}
    Addition is commutative if for all $a$ and $b$ in $\U$, $a + b = b + a$.
\end{class}

\begin{class}
    A special value $0 : \U$.
\end{class}

\begin{class}
    Zero is a left identity if for all $a$ in $\U$, $0 + a = a$.
\end{class}

\begin{class}
    Zero is a right identity if for all $a$ in $\U$, $a + 0 = a$.
\end{class}

\begin{class}
    Addition is left cancellative if for all $a$, $b$, and $c$ in $\U$, $c + a =
    c + b \rightarrow a = b$.
\end{class}

\begin{class}
    Addition is right cancellative if for all $a$, $b$, and $c$ in $\U$, $a + c
    = b + c \rightarrow a = b$.
\end{class}

\begin{class}
    An unary operation $\U \rightarrow \U$ which is called negation, denoted
    with $-$.  We also use the shorthand $a - b = a + (-b)$.
\end{class}

\begin{class}
    Negation is a left inverse if for all $a$ in $\U$, $-a + a = 0$.
\end{class}

\begin{class}
    Negation is a right inverse if for all $a$ in $\U$, $a - a = 0$.
\end{class}

\begin{class}[Group]
    A type $\U$ is a Group if it has instances of addition, zero, and negation
    such that addition is associative, zero is a left and right identity, and
    negation is a left and right inverse.
\end{class}

\begin{class}[Abelian Group]
    A type $\U$ is an Abelian Group if it is a Group and addition is
    commutative.
\end{class}

\begin{class}
    If $f : \U \to \V$ is a function such that $f(a + b) = f(a) + f(b)$ for all
    $a$ and $b$ in $\U$, then we say that $f$ is additive.
\end{class}

\begin{class}
    If $f : \U \to \V$ is a function such that $f(0) = 0$, we say that $f$ is
    nullitive.
\end{class}

\begin{class}
    If $f : \U \to \V$ is a function such that $f(-a) = -f(a)$ for all $a$ in
    $\U$, then we say that $f$ is negatory.
\end{class}

Several of these classes together provide instances of some of the others.

\begin{instance}
    If addition is commutative and zero is a left identity, then zero is a right
    identity.
\end{instance}
\begin{proof}
    Trivial.
\end{proof}

\begin{instance}
    If addition is commutative and left cancellative, then addition is right
    cancellative.
\end{instance}
\begin{proof}
    Trivial.
\end{proof}

\begin{instance}
    If addition is commutative and negation is a left inverse, then negation is
    a right inverse.
\end{instance}
\begin{proof}
    Trivial.
\end{proof}

\begin{instance}
    If addition is cancellative and $f$ is additive, then $f$ is nullitive.
\end{instance}
\begin{proof}
    Because $0 + 0 = 0$, we also have $f(0 + 0) = f(0)$.  Using additivity,
    $f(0) + f(0) = f(0)$, and by canceling $f(0)$ we get $f(0) = 0$.
\end{proof}

\begin{instance}
    If $f$ is additive, then $f$ is negatory.
\end{instance}
\begin{proof}
    We know that $f(a - a) = f(0) = 0$, and using additivity we have $f(a) +
    f(-a) = 0$, and rearranging we get $f(-a) = -f(a)$.
\end{proof}

The previous instances allows us to only check for one-sided addition laws when
addition is commutative.

\begin{instance} \label{plus_linv_lcancel}
    If addition is associative, zero is a left identity, and negation is a left
    inverse, then addition is left cancellative.
\end{instance}
\begin{proof}
    Let $a$, $b$, and $c$ be such that $c + a = c + b$.  Then
    \begin{align*}
        c + a &= c + b \\
        -c + (c + a) &= -c + (c + b) \\
        (-c + c) + a &= (-c + c) + b \\
        0 + a &= 0 + b \\
        a &= b.
    \end{align*}
\end{proof}

\begin{instance} \label{plus_rinv_rcancel}
    If addition is associative, zero is a right identity, and negation is a
    right inverse, then addition is right cancellative.
\end{instance}
\begin{proof}
    This proof is the same in form as the previous one.
\end{proof}

\begin{instance}
    The composition of two additive functions is additive.
\end{instance}
\begin{proof}
    Let $f$ and $g$ be two additive functions.  Then for all $a$ and $b$,
    \[
        f(g(a + b)) = f(g(a) + g(b)) = f(g(a)) + f(g(b)).
    \]
\end{proof}

\begin{instance}
    The composition of two nullitive functions is nullitive.
\end{instance}
\begin{proof}
    Let $f$ and $g$ be two nullitive functions.  Then
    \[
        f(g(0)) = f(0) = 0.
    \]
\end{proof}

\begin{instance}
    The composition of two negatory functions is negatory.
\end{instance}
\begin{proof}
    Let $f$ and $g$ be two negatory functions.  Then for all $x$,
    \[
        f(g(-x)) = f(-g(x)) = -f(g(x)).
    \]
\end{proof}

Throughout the rest of this section, the classes required will no longer be
specified and will have to be determined from the proofs.  As said previously,
Coq does this automatically.

\begin{theorem}
    $-0 = 0$.
\end{theorem}
\begin{proof}
    We have both $0 - 0 = 0$ and $0 - 0 = -0$, so $-0 = 0$.
\end{proof}

\begin{theorem}
    For all $a : \U$, $-{-a} = a$.
\end{theorem}
\begin{proof}
    \begin{align*}
        -{-a} &= a \\
        -{-a} - a &= a - a \\
        -(-a) + (-a) &= a - a \\
        0 &= 0.
    \end{align*}
\end{proof}

\begin{theorem} \label{neg_plus}
    For all $a$ and $b$ in $\U$, $-(a + b) = -b + -a$.  Furthermore, if addition
    is commutative, $-(a + b) = -a + -b$.
\end{theorem}
\begin{proof}
    \begin{align*}
        (a + b) - (a + b) &= 0 \\
        a + (b - (a + b)) &= 0 \\
        b - (a + b) &= -a \\
        -(a + b) &= -b + -a.
    \end{align*}
\end{proof}

\begin{theorem} \label{not_trivial_zero}
    If $\U$ is nontrivial, then there exists an $A : \U$ such that $a \neq 0$.
\end{theorem}
\begin{proof}
    This is a direct application of Theorem \ref{not_trivial2}.
\end{proof}

\begin{theorem} \label{homo_zero_inj}
    If $f$ is additive and negatory, if $0 = f(a)$ implies that $0 = a$ for all
    $a$ in $\U$, then $f$ is injective.
\end{theorem}
\begin{proof}
    Let $b$ and $c$ be values in $\U$ such that $f(b) = f(c)$.  We must prove
    that $b = c$.  Using the properties of $f$, we know that $0 = f(b) - f(c) =
    f(b - c)$.  We can also get away with proving that $0 = b - c$.  Thus, by
    setting $b - c = a$, the theorem is true by hypothesis.
\end{proof}

\begin{theorem} \label{homo_inj_zero}
    If $f$ is nullitive and injective, then $0 \neq a$ implies that $0 \neq
    f(a)$ for all $a : \U$.
\end{theorem}
\begin{proof}
    Assume that $0 = f(a)$.  Then $f(0) = f(a)$, and because $f$ is injective,
    $0 = a$, contradicting $0 \neq a$.
\end{proof}

\section{Multiplication}

\begin{class} \label{mult}
    A binary operation $\U \rightarrow \U \rightarrow \U$ which is called
    multiplication, denoted by concatenation.
\end{class}

\begin{class}
    Multiplication is left distributive if for all $a$, $b$, and $c$ in $\U$,
    $a(b + c) = ab + ac$.
\end{class}

\begin{class}
    Multiplication is right distributive if for all $a$, $b$, and $c$ in $\U$,
    $(a + b)c = ac + bc$.
\end{class}

\begin{class}
    Multiplication is associative if for all $a$, $b$, and $c$ in $\U$, $a(bc) =
    (ab)c$.
\end{class}

\begin{class}
    Multiplication is commutative if for all $a$ and $b$ in $\U$, $ab = ba$.
\end{class}

\begin{class}
    Zero is a left annihilator if for all $a$ in $\U$, $0a = 0$.
\end{class}

\begin{class}
    Zero is a right annihilator if for all $a$ in $\U$, $a0 = 0$.
\end{class}

\begin{class}
    A special value $1 : \U$.
\end{class}

\begin{class}
    One is a left identity if for all $a$ in $\U$, $1a = a$.
\end{class}

\begin{class}
    One is a right identity if for all $a$ in $\U$, $a1 = a$.
\end{class}

\begin{class}
    Multiplication is left cancellative if for all $a$, $b$, and $c$ in $\U$
    with $c \neq 0$, $ca = cb \rightarrow a = b$.
\end{class}

\begin{class}
    Multiplication is right cancellative if for all $a$, $b$, and $c$ in $\U$
    with $c \neq 0$, $ac = bc \rightarrow a = b$.
\end{class}

\begin{class}
    Multiplication has the zero property if for all $a$ and $b$, if $0 = ab$,
    then $0 = a$ or $0 = b$.
\end{class}

\begin{class}[Rng]
    A type $\U$ is a Rng if addition forms an Abelian Group and if there is an
    instance of multiplication such that multiplication is left distributive,
    right distributive, and associative.
\end{class}

\begin{class}[Ring]
    A type $\U$ is a Ring if it is a Rng and there is an instance of one such
    that one is a left and right identity.
\end{class}

\begin{class}[Commutative Ring]
    A type $\U$ is a Commutative Ring if it is a Ring and multiplication is
    commutative.
\end{class}

\begin{class}[Integral Domain]
    A type $\U$ is an Integral Domain if it is a Commutative Ring and
    multiplication is left and right cancellative.
\end{class}

\begin{definition}
    Given a type $\A$ with addition and a one, we define $2 = 1 + 1$, $3 = 1 +
    1 + 1$, etc.
\end{definition}

\begin{class}
    If $f : \U \to \V$ is a function such that $f(ab) = f(a)f(b)$ for all $a$
    and $b$ in $\U$, then we say that $f$ is multiplicative.
\end{class}

\begin{class}
    If $f : \U \to \V$ is a function such that $f(1) = 1$, we say that $f$ is
    unital.
\end{class}
Notice that from this and additivity we get $f(2) = 2$, $f(3) = 3$, etc.

Like with addition, several of these classes together provide instances of some
of the others.

\begin{instance}
    If multiplication is commutative and one is a left identity, then one is a
    right identity.
\end{instance}
\begin{proof}
    Trivial.
\end{proof}

\begin{instance} \label{ldist_rdist}
    If multiplication is commutative and left distributive, then multiplication
    is right distributive.
\end{instance}
\begin{proof}
    Trivial.
\end{proof}

\begin{instance} \label{mult_lanni_ranni}
    If multiplication is commutative and zero is a left annihilator, then zero
    is a right annihilator.
\end{instance}
\begin{proof}
    Trivial.
\end{proof}

\begin{instance}
    If multiplication is commutative and left cancellative, then multiplication
    is right cancellative.
\end{instance}
\begin{proof}
    Trivial.
\end{proof}

Like with addition, the previous instances allows us to only check for one-sided
multiplication laws when multiplication is commutative.

\begin{instance} \label{ring_mult_lanni}
    If addition is cancellative, zero is an identity, and multiplication is
    right distributive, then zero is a left annihilator.
\end{instance}
\begin{proof}
    \begin{align*}
        0a + 0a &= 0a + 0a \\
        0a + 0a &= (0 + 0)a \\
        0a + 0a &= 0a \\
        0a + 0a &= 0a + 0 \\
        0a &= 0 \\
    \end{align*}
\end{proof}

\begin{instance} \label{ring_mult_ranni}
    If addition is cancellative, zero is an identity, and multiplication is
    left distributive, then zero is a right annihilator.
\end{instance}
\begin{proof}
    The proof is the same in form to the proof of Instance
    \ref{ring_mult_lanni}.
\end{proof}

\begin{instance}
    If multiplication is cancellative, then it has the zero property.
\end{instance}
\begin{proof}
    Assume that $0 \neq a$.  Then
    \begin{align*}
        0 &= ab \\
        a0 &= ab \\
        0 &= b,
    \end{align*}
    so we have either $0 = a$ or $0 = b$.
\end{proof}

\begin{instance}
    The composition of two multiplicative functions is multiplicative.
\end{instance}
\begin{proof}
    Let $f$ and $g$ be two multiplicative functions.  Then for all $a$ and $b$,
    \[
        f(g(ab)) = f(g(a)g(b)) = f(g(a))f(g(b)).
    \]
\end{proof}

\begin{instance}
    The composition of two unital functions is unital.
\end{instance}
\begin{proof}
    Let $f$ and $g$ be two unital functions.  Then
    \[
        f(g(1)) = f(1) = 1.
    \]
\end{proof}

\begin{theorem} \label{not_trivial_one}
    If $\U$ is not trivial, then $0 \neq 1$.
\end{theorem}
\begin{proof}
    By Theorem \ref{not_trivial_zero}, there exists an $a : \U$ such that $a
    \neq 0$.  But if $0 = 1$, we could multiply both sides by $a$ to get $0 =
    a$, which is a contradiction, so we must have $0 \neq 1$.
\end{proof}

\begin{theorem}
    For all $a$ and $b$ in $\U$, $(-a)b = -(ab)$.
\end{theorem}
\begin{proof}
    \begin{align*}
        (-a)b &= -(ab) \\
        ab + (-a)b &= ab - (ab) \\
        (a - a)b &= 0 \\
        0b &= 0 \\
        0 &= 0.
    \end{align*}
\end{proof}

\begin{theorem}
    For all $a$ and $b$ in $\U$, $a(-b) = -(ab)$.
\end{theorem}
\begin{proof}
    This proof is identical in form to the last proof.
\end{proof}

\begin{theorem}
    For all $a$ in $\U$, $(-1)a = -a$.
\end{theorem}
\begin{proof}
    \[
        (-1)a = -(1a) = -a.
    \]
\end{proof}

Even though these next few theorems aren't really needed, they're included in
the Coq code and it is interesting to see how these basic ideas follow from the
axioms.

\begin{theorem}
    $2 + 2 = 4$.
\end{theorem}
\begin{proof}
    \[
        2 + 2 = (1 + 1) + (1 + 1) = 1 + 1 + 1 + 1 = 4.
    \]
\end{proof}

\begin{theorem}
    For all $a$ in $\U$, $2a = a + a$.
\end{theorem}
\begin{proof}
    \[
        2a = (1 + 1)a = 1a + 1a = a + a.
    \]
\end{proof}

\begin{theorem}
    $(2)(2) = 4$.
\end{theorem}
\begin{proof}
    \[
        (2)(2) = 2 + 2 = 4.
    \]
\end{proof}

\begin{theorem}
    For all $a$ and $b$ in $\U$, $aa - bb = (a + b) (a - b)$.
\end{theorem}
\begin{proof}
    \[
        (a + b)(a - b) = a(a - b) + b(a - b) = aa - ab + ba - bb = aa - bb + ab
        - ab = aa - bb.
    \]
\end{proof}

\begin{theorem} \label{mult_nz}
    If multiplication has the zero property, then for all $a \neq 0$ and $b \neq
    0$, $ab \neq 0$.
\end{theorem}
\begin{proof}
    This is the contrapositive of the zero property.
\end{proof}

\begin{theorem}
    If $a \neq 0$, then $-a \neq 0$.
\end{theorem}
\begin{proof}
    Assume that $-a = 0$.  Because $0 = -0$, we have $-a = -0$, so $a = 0$,
    which contradicts $a \neq 0$.
\end{proof}

\section{Division}

\begin{class}
    An unary operation $\U \rightarrow \U$ which is called the reciprocal.  We
    denote the reciprocal of $a$ as $a^{-1}$ or $\frac{1}{a}$, and use the
    shorthand $\frac{a}{b} = a b^{-1}$.
\end{class}

\begin{class}
    The reciprocal is a left inverse if for all $a$ in $\U$ with $a \neq 0$,
    $a^{-1}a = 1$.
\end{class}

\begin{class}
    The reciprocal is a right inverse if for all $a$ in $\U$ with $a \neq 0$,
    $aa^{-1} = 1$.
\end{class}

Notice that division is handled slightly differently than usual.  Most people do
not consider the reciprocal of zero to be defined.  Instead, we define the
reciprocal of zero, but do not put any constraints on what its value is.

\begin{class}[Field]
    A type $\U$ is a Field if it is a Commutative Ring, it has a reciprocal that
    is an inverse, and it is nontrivial.
\end{class}

\begin{instance}
    If multiplication is commutative and the reciprocal is a left inverse, then
    the reciprocal is right a right inverse.
\end{instance}
\begin{proof}
    Trivial.
\end{proof}

\begin{class}
    If $f : \U \to \V$ is a function such that $f(a^{-1}) = f(a)^{-1}$ for all
    $a \neq 0$ in $\U$, then we say that $f$ is reciprocative.
\end{class}

\begin{instance}
    Multiplication is cancellative in fields.
\end{instance}
\begin{proof}
    The proof is the same in form to the proofs of Instances
    \ref{plus_linv_lcancel} and \ref{plus_rinv_rcancel}.
\end{proof}

\begin{theorem}
    If $a \neq 0$, then $a^{-1} \neq 0$.
\end{theorem}
\begin{proof}
    Because $a \neq 0$, $\U$ is not trivial.  Now assume that $a^{-1} = 0$.
    Then we could multiply by $a$ to get $aa^{-1} = 0$.  Because $a \neq 0$, we
    get $1 = 0$, which contradicts nontriviality.
\end{proof}

\begin{theorem}
    If $a \neq 0$, then $(a^{-1})^{-1} = a$.
\end{theorem}
\begin{proof}
    \begin{align*}
        (a^{-1})^{-1} &= a \\
        a^{-1} (a^{-1})^{-1} &= a^{-1} a \\
        1 &= 1.
    \end{align*}
\end{proof}

\begin{theorem}
    If $a \neq 0$, then $(-a)^{-1} = -a^{-1}$.
\end{theorem}
\begin{proof}
    \begin{align*}
        (-a)^{-1} &= -a^{-1} \\
        (-a)(-a)^{-1} &= (-a)(-a^{-1}) \\
        1 &= -{-(aa^{-1})} \\
        1 &= aa^{-1} \\
        1 &= 1.
    \end{align*}
\end{proof}

\begin{theorem}
    If $a \neq 0$ and $b \neq 0$, $(ab)^{-1} = b^{-1}a^{-1}$.  In particular, if
    multiplication is commutative, $(ab)^{-1} = a^{-1}b^{-1}$.
\end{theorem}
\begin{proof}
    This proof is the same in form to the proof of Theorem \ref{neg_plus}.
\end{proof}

\begin{theorem}
    $1^{-1} = 1$.
\end{theorem}
\begin{proof}
    If $\U$ was trivial, the theorem would be trivial, so assume that $0 \neq
    1$.  Then we have $1^{-1} = 1 1^{-1} = 1$.
\end{proof}

\begin{instance}
    If $\U$ is a field and $f$ is multiplicative and unital, then $f$ is
    injective.
\end{instance}
\begin{proof}
    We will use Theorem \ref{homo_zero_inj}.  Let $a$ be such that $0 = f(a)$
    and assume for a contradiction that $0 \neq a$.  Then by multiplying by
    $f(a^{-1})$ on both sides of $0 = f(a)$ we get
    \[
        0 = f(a) f(a^{-1}) = f(a a^{-1}) = f(1) = 1,
    \]
    which contradicts non-triviality.
\end{proof}

\begin{instance}
    If $f$ is multiplicative and unital, then $f$ is reciprocative.
\end{instance}
\begin{proof}
    Let $a \neq 0$.  Then by Theorem \ref{homo_inj_zero}, $f(a) \neq 0$ as well.
    Then we know that \[
        f(a) f(a^{-1}) = f(a a^{-1}) = f(1) = 1 = f(a) f(a)^{-1},
    \]
    and by canceling $f(a)$ we get $f(a^{-1}) = f(a)^{-1}$.
\end{proof}

\begin{instance}
    The composition of two reciprocative functions is reciprocative, if the
    first function is injective.
\end{instance}
\begin{proof}
    Let $f$ and $g$ be two reciprocative functions and assume that $g$ is
    injective.  Then for all $x \neq 0$,
    \[
        f(g(x^{-1})) = f(g(x)^{-1}).
    \]
    Because $g$ is injective, we know that $g(x) \neq 0$, so
    \[
        f(g(x)^{-1}) = f(g(x))^{-1}.
    \]
\end{proof}

\section{Relations}

\begin{definition}
    Given a type $\U$, we use the term relation to refer to functions $\U
    \rightarrow \U \rightarrow \vtt{Prop}$.  Given a relation $R$, we often use
    the notation $aRb$ to mean that $R(a, b)$ is true.
\end{definition}

\begin{definition}
    Given a relation $R$, we can define the strict relation $R'$ given by $aR'b
    = aRb \wedge a \neq b$.
\end{definition}

\begin{definition}
    Given a relation $R$, we can define the dual relation $R'$ given by $aR'b =
    bRa$.
\end{definition}

\begin{class}
    A relation is reflexive if $aRa$ for all $a$.
\end{class}

\begin{class}
    A relation is irreflexive if $\neg aRa$ for all $a$.
\end{class}

\begin{class}
    A relation is symmetric if for all $a$ and $b$, $aRb \rightarrow bRa$.
\end{class}

\begin{class}
    A relation is antisymmetric if for all $a$ and $b$, $aRb \rightarrow bRa
    \rightarrow a = b$.
\end{class}

\begin{class}
    A relation is asymmetric if for all $a$ and $b$, $aRb \rightarrow \neg bRa$.
\end{class}

\begin{class}
    A relation is transitive if for all $a$, $b$, and $c$, $aRb \rightarrow bRc
    \rightarrow aRc$.
\end{class}

\begin{class}
    A relation is dense if for all $a$ and $b$ with $aRb$, there exists a $c$
    such that $aRc$ and $cRb$.
\end{class}

\begin{class}
    A relation is connex if for all $a$ and $b$, $\{aRb\} + \{bRa\}$.
\end{class}

\begin{class}
    A relation is trichotomous if for all $a$ and $b$, $\{aRb\} + \{a = b\} +
    \{bRa\}$.
\end{class}

\begin{class}
    A relation is well ordered if for every nonempty set $S$, there exists an
    element $a \in S$ such that $a \leq b$ for all $b \in S$.
\end{class}

\begin{instance}
    A relation that is connex is reflexive.
\end{instance}
\begin{proof}
    Let $R$ be a connex relation and let $a : \U$.  Then by connexivity, we have
    either $aRa$ or $aRa$.  Either way, $aRa$, so $R$ is reflexive.
\end{proof}

For the rest of this section, let $\leq$ be a relation on $\U$ that is possibly
reflexive, antisymmetric, transitive, and connex.  Let $<$ be $\leq$'s strict
relation and $\geq$ be $\leq$'s dual relation.

\begin{instance}
    $<$ is irreflexive.
\end{instance}
\begin{proof}
    $a \neq a$ for all $a$, so this is true by definition.
\end{proof}

\begin{instance}
    $<$ is asymmetric.
\end{instance}
\begin{proof}
    Assume that $a < b$ and $b < a$.  Then because $a \leq b$ and $b \leq a$, we
    have $a = b$, which contradicts $a < b$.
\end{proof}

\begin{instance}
    $<$ is transitive.
\end{instance}
\begin{proof}
    Assume that $a < b$ and $b < c$.  Because $a \leq b$ and $b \leq c$, we have
    $a \leq c$, so we only need to prove that $a \neq c$.  If $a = c$, we would
    have $a < b$ and $b < a$, which is a contradiction, so we must have $a \neq
    c$.  Thus, $a < c$.
\end{proof}

\begin{instance}
    $<$ is trichotomous.
\end{instance}
\begin{proof}
    Let $a$ and $b$ be values in $\U$.  If $a = b$ we're done, so assume that $a
    \neq b$.  Then because $\leq$ is connex we have $a \leq b$ or $b \leq a$.
    If $a \leq b$, we have $a \neq b$ so $a < b$.  If $b \leq a$, we have $b
    \neq a$ so $b < a$.  Thus, we either have $a < b$, $a = b$, or $b < a$, so
    $<$ is trichotomous.
\end{proof}

\begin{theorem}
    For all $a$ and $b$, $a \nleq b \leftrightarrow b < a$.
\end{theorem}
\begin{proof}
    Assume that $a \nleq b$.  Because $\leq$ is connex, we must have $b \leq a$.
    Also, if $a = b$, we would have $a \leq b$, which contradicts $a \nleq b$.
    Thus, $b \leq a$ and $b \neq a$, so $b < a$.

    Now assume that $b < a$.  Then if $a \leq b$, we would have $a \leq b$ and
    $b \leq a$, so $b = a$, which contradicts $b < a$.  Thus we must have $a
    \nleq b$.
\end{proof}

\begin{theorem}
    For all $a$ and $b$, $a \nless b \leftrightarrow b \leq a$.
\end{theorem}
\begin{proof}
    \[
        a \nless b \leftrightarrow \neg (a < b) \leftrightarrow \neg\neg (b \leq
        a) \leftrightarrow b \leq a.
    \]
\end{proof}

\begin{theorem}
    For all $a$, $b$, and $c$, if $a \leq b$ and $b < c$, $a < c$.
\end{theorem}
\begin{proof}
    We get $a \leq c$ through the transitivity of $\leq$, so all we need to
    prove is that $a \neq c$.  If $a = c$, we would have $a \leq b$ and $b \leq
    a$, so $b = a$, which contradicts $b < a$.  Thus $a \neq c$, so $a < c$.
\end{proof}

\begin{theorem}
    For all $a$, $b$, and $c$, if $a < b$ and $b \leq c$, $a < c$.
\end{theorem}
\begin{proof}
    We get $a \leq c$ through the transitivity of $\leq$, so all we need to
    prove is that $a \neq c$.  If $a = c$, we would have $a \leq b$ and $b \leq
    a$, so $a = b$, which contradicts $a < b$.  Thus $a \neq c$, so $a < c$.
\end{proof}

\section{Order}

\begin{class}
    An order on a type $\U$ is a particular relation on $\U$.  It is written
    $\leq$.  Its strict relation is written $<$, its dual relation is written
    $\geq$, and its strict dual relation is written $>$.
\end{class}

Note that in the Coq code, the dual relation is rarely used, to cut down on the
number of theorems that need to be proved.

\begin{class}
    An order is said to be a partial order if it is reflexive, antisymmetric,
    and transitive.
\end{class}

\begin{class}
    An order is said to be a total order if it is connex, antisymmetric, and
    transitive.
\end{class}

\begin{class}
    An order is said to be a well order if it is antisymmetric and well-ordered.
\end{class}

\begin{class}
    If $f : \U \to \V$ is a function such that $a \leq b$ implies that $f(a)
    \leq f(b)$ for all $a$ and $b$ in $\U$, then we say that $f$ is orderly.
\end{class}

\begin{class}
    If $f : \U \to \V$ is a function such that $a < b$ implies that $f(a) <
    f(b)$ for all $a$ and $b$ in $\U$, then we say that $f$ is strictly orderly.
\end{class}

\begin{class}
    If $f : \U \to \V$ is a function such that $a \leq b \leftrightarrow f(a)
    \leq f(b)$ for all $a$ and $b$ in $\U$, then we say that $f$ is equivalently
    orderly.
\end{class}

\begin{class}
    If $f : \U \to \V$ is a function such that $a < b \leftrightarrow f(a) <
    f(b)$ for all $a$ and $b$ in $\U$, then we say that $f$ is strictly
    equivalently orderly.
\end{class}

\begin{instance} \label{wo_connex}
    A relation that is well ordered is connex.
\end{instance}
\begin{proof}
    Let $a$ and $b$ be two values.  Then consider the set $\{a, b\}$.  Because
    the relation is well ordered, the set has a least element.  If $a$ is the
    least element, then $a \leq b$, and if $b$ is the least element, then $b
    \leq a$.  Thus, the relation is connex.
\end{proof}

\begin{instance} \label{wo_trans}
    A relation that is well ordered and antisymmetric is transitive.
\end{instance}
\begin{proof}
    Let $a$, $b$, and $c$ be such that $a \leq b$ and $b \leq c$.  Consider the
    set $\{a, b, c\}$.  Because the relation is well ordered, the set has a
    least element.  If $a$ is the least element, then $a \leq c$.  If $b$ is the
    least element, then $a \leq b$ and $b \leq a$, so $a = b$ by antisymmetry,
    meaning that $a = b \leq c$.  If $c$ is the least element, then $b \leq c$
    and $c \leq b$, so $b = c$ by antisymmetry, meaning that $a \leq b = c$.  In
    all cases $a \leq c$, so the relation is transitive.
\end{proof}

Notice now that to prove that a set is well-ordered, all we need to do is prove
that it is antisymmetric and that every set has a minimum.

\begin{instance} \label{homo_le_lt}
    If $f$ is injective and orderly, then it is strictly orderly.
\end{instance}
\begin{proof}
    Since we already know that $a \leq b \rightarrow f(a) \leq f(b)$, we only
    need to prove that $a \neq b \rightarrow f(a) \neq f(b)$, which is the
    contrapositive of the definition of injectivity.
\end{proof}

\begin{instance} \label{homo_le_le2}
    If $f$ is injective and orderly, then it is equivalently orderly.
\end{instance}
\begin{proof}
    Because $f$ is orderly already, we only need to prove that $f(a) \leq f(b)
    \implies a \leq b$.  For a contradiction, assume that $b < a$.  Then because
    $b \leq a$, we have $f(b) \leq f(a)$.  Because $f(a) \leq f(b)$ and $f(b)
    \leq f(a)$, by antisymmetry we have $f(a) = f(b)$.  By injectivity we have
    $a = b$, contradicting $b < a$.
\end{proof}

\begin{instance} \label{homo_lt_lt2}
    If $f$ is strictly orderly, then it is strictly equivalently orderly.
\end{instance}
\begin{proof}
    Because $f$ is strictly orderly already, we only need to prove that $f(a) <
    f(b) \implies a < b$.  For a contradiction, assume that $b \leq a$.  Because
    $f(a) < f(b)$, we can't have $b = a$, so $b < a$.  Then because $f$ is
    strictly orderly we have $f(b) < f(a)$.  We can't have both $f(a) < f(b)$
    and $f(b) < f(a)$, so it must be the case that $a < b$.
\end{proof}

\begin{instance} \label{homo_le_compose}
    The composition of two orderly functions is orderly.
\end{instance}
\begin{proof}
    Let $f$ and $g$ be two orderly functions.  Then for all $a$ and $b$,
    \[
        a \leq b \to g(a) \leq g(b) \to f(g(a)) \leq f(g(b)).
    \]
\end{proof}

\begin{instance}
    The composition of two strictly orderly functions is strictly orderly.
\end{instance}
\begin{proof}
    Let $f$ and $g$ be two strictly orderly functions.  Then for all $a$ and
    $b$,
    \[
        a < b \to g(a) < g(b) \to f(g(a)) < f(g(b)).
    \]
\end{proof}

\begin{instance}
    The composition of two equivalently orderly functions is equivalently
    orderly.
\end{instance}
\begin{proof}
    Let $f$ and $g$ be two equivalently orderly functions.  Then for all $a$ and
    $b$,
    \[
        a \leq b \leftrightarrow g(a) \leq g(b) \leftrightarrow f(g(a)) \leq
        f(g(b)).
    \]
\end{proof}

\begin{instance}
    The composition of two strictly equivalently orderly functions is strictly
    equivalently orderly.
\end{instance}
\begin{proof}
    Let $f$ and $g$ be two strictly equivalently orderly functions.  Then for
    all $a$ and $b$,
    \[
        a < b \leftrightarrow g(a) < g(b) \leftrightarrow f(g(a)) < f(g(b)).
    \]
\end{proof}

\begin{instance}
    The empty type is well-ordered by the relation that ``takes'' $a$ and $b$ in
    $\E$ to \verb|True|.
\end{instance}
\begin{proof}
    All of the required properties are vacuously true.
\end{proof}

\begin{instance}
    The singleton type is well-ordered by the relation that takes $a$ and $b$ in
    $\S$ to \verb|True|.
\end{instance}
\begin{proof}
    Antisymmetry follows from all values in the singleton type being equal, and
    $I$ is the minimal element of the one nonempty set.
\end{proof}

\subsection{Order and Addition}

Addition and order often interact in a consistent way, as formalized in the
following classes and theorems.

\begin{class}
    An order is said to be left additive if for all $a$, $b$, and $c$, if $a
    \leq b$, then $c + a \leq c + b$.
\end{class}

\begin{class}
    An order is said to be right additive if for all $a$, $b$, and $c$, if $a
    \leq b$, then $a + c \leq b + c$.
\end{class}

\begin{class}
    An order is said to be left addition cancellative if for all $a$, $b$, and
    $c$, if $c + a \leq c + b$, then $a \leq b$.
\end{class}

\begin{class}
    An order is said to be right addition cancellative if for all $a$, $b$, and
    $c$, if $a + c \leq b + c$, then $a \leq b$.
\end{class}

\begin{class}
    An order is said to be strictly left additive if for all $a$, $b$, and $c$,
    if $a < b$, then $c + a < c + b$.
\end{class}

\begin{class}
    An order is said to be strictly right additive if for all $a$, $b$, and $c$,
    if $a < b$, then $a + c < b + c$.
\end{class}

\begin{class}
    An order is said to be strictly left addition cancellative if for all $a$,
    $b$, and $c$, if $c + a < c + b$, then $a < b$.
\end{class}

\begin{class}
    An order is said to be strictly right addition cancellative if for all $a$,
    $b$, and $c$, if $a + c < b + c$, then $a < b$.
\end{class}

\begin{class}[Ordered Group]
    A type $\U$ is an Ordered Group if it is a Group and has an order that is
    left and right additive.
\end{class}

When addition is commutative, each left class implies the right version.  To
save on writing, all theorems and instances of this subsection will only be
stated for addition on the left.  All theorems work with addition on the right
too.

\begin{instance}
    Under the conditions of Instance \ref{plus_linv_lcancel}, if $\leq$ is left
    additive, then $\leq$ is left addition cancellative.
\end{instance}
\begin{proof}
    The proof is identical in form to the proof of Instance
    \ref{plus_linv_lcancel}.
\end{proof}

Of course, in types where additive inverses don't exist, the previous instance
can't be used.  In general, there are a surprising number of ways to prove some
of these properties from the others.  Here is a full list of the ones I could
figure out:

\begin{instance} \label{le_lplus1}
    If an order is strictly left addition cancellative, then it is left
    additive.
\end{instance}
\begin{proof}
    Let $a \leq b$.  If we had $c + b < c + a$, we could cancel $c$ to get $b <
    a$, contradicting $a \leq b$.  Thus, $c + a \leq c + b$.
\end{proof}

\begin{instance} \label{le_lplus2}
    If an order is strictly left additive, it is left additive.
\end{instance}
\begin{proof}
    Let $a \leq b$.  Then if $a = b$, we would have $c + a \leq c + b$ by
    reflexivity.  If $a \neq b$, we would have $a < b$, so we could add $c$ on
    the left to get $c + a < c + b$, which implies that $c + a \leq c + b$.
\end{proof}

\begin{instance} \label{lt_lplus1}
    If an order is left additive and addition is left cancellative, then it is
    strictly left additive.
\end{instance}
\begin{proof}
    Let $a < b$.  Because the order is left additive, we can get that $c + a
    \leq c + b$.  So it only remains to prove that $c + a \neq c + b$.  To do
    so, assume that $c + a = c + b$.  Then cancelling $c$, we get $a = b$,
    contradicting $a < b$.  Thus, $c + a \neq c + b$, so $c + a < c + b$.
\end{proof}

\begin{instance} \label{lt_lplus2}
    If an order is left addition cancellative, then it is strictly left
    additive.
\end{instance}
\begin{proof}
    Let $a < b$.  If $c + b \leq c + a$, we could cancel to get $b \leq a$,
    contradicting $a < b$.  Thus, $c + a < c + b$.
\end{proof}

\begin{instance} \label{le_plus_lcancel1}
    If an order is strictly left addition cancellative and addition is left
    cancellative, then the order is left cancellative.
\end{instance}
\begin{proof}
    Let $c + a \leq c + b$.  If $c + a = c + b$, we could cancel $c$ to get $a =
    b$, showing that $a \leq b$.  If $c + a \neq c + b$, we have $c + a < c +
    b$, so we could cancel $c$ to get $a < b$, showing that $a \leq b$.
\end{proof}

\begin{instance} \label{le_plus_lcancel2}
    If an order is strictly left additive, then it is left cancellative.
\end{instance}
\begin{proof}
    Let $c + a \leq c + b$.  If $b < a$, we could add $c$ on the left to get $c
    + b < c + a$, contradicting $c + a \leq c + b$.  Thus, $a \leq b$.
\end{proof}

\begin{instance} \label{lt_plus_lcancel1}
    If an order is left addition cancellative, then it is strictly left addition
    cancellative.
\end{instance}
\begin{proof}
    Let $c + a < c + b$.  By assumption, we can get $a \leq b$, so it only
    remains to prove that $a \neq b$.  If $a = b$, we would have $c + a = c +
    b$, contradicting $c + a < c + b$.  Thus, we must have $a \neq b$, so $a <
    b$.
\end{proof}

\begin{instance} \label{lt_plus_lcancel2}
    If an order is left additive, then it is strictly left addition
    cancellative.
\end{instance}
\begin{proof}
    Let $c + a < c + b$.  Then if $b \leq a$, we can add $c$ to get $c + b \leq
    c + a$, contradicting $c + a < c + b$.  Thus, $a < b$.
\end{proof}

\begin{instance} \label{plus_lcancel1}
    If an order is strictly left additive, then addition is left cancellative.
\end{instance}
\begin{proof}
    Let $c + a = c + b$.  If $a < b$, we would have $c + a < c + b$,
    contradicting $c + a = c + b$.  By symmetry, $b < a$ is impossible too.  The
    only case left is $a = b$, which is what we need.
\end{proof}

\begin{instance} \label{plus_lcancel2}
    If an order is left cancellative, then addition is left cancellative.
\end{instance}
\begin{proof}
    Let $c + a = c + b$.  Then we have $c + a \leq c + b$ and $c + b \leq c +
    a$.  Cancelling $c$ in each equation, we get $a \leq b$ and $b \leq a$.
    Then $a = b$ follows by antisymmetry.
\end{proof}

Given all of these implications, which should be used?  The standard route for
most types is to prove that addition is left cancellative and that the order is
left additive.  Along with commutativity, this will provide all of the
properties.  However, in cases where this is not possible, much more care needs
to be taken.  Currently, the only type this is needed in in this document is the
ordinals, so I will just explicitly say which instances I am using then.

\begin{theorem} \label{le_lrplus}
    If $a \leq b$ and $c \leq d$, then $a + c \leq b + d$.
\end{theorem}
\begin{proof}
    From $a \leq b$ we get $a + c \leq b + c$ and from $c \leq d$ we get $b + c
    \leq b + d$, and by transitivity we get $a + c \leq b + d$.
\end{proof}

\begin{theorem} \label{lt_le_lrplus}
    If $a < b$ and $c \leq d$, then $a + c < b + d$.
\end{theorem}
\begin{proof}
    This proof is identical in form to the previous proof.
\end{proof}

\begin{theorem} \label{le_lt_lrplus}
    If $a \leq b$ and $c < d$, then $a + c < b + d$.
\end{theorem}
\begin{proof}
    This proof is identical in form to the previous proofs.
\end{proof}

\begin{theorem} \label{lt_lrplus}
    If $a < b$ and $c < d$, then $a + c < b + d$.
\end{theorem}
\begin{proof}
    This proof is identical in form to the previous proofs.
\end{proof}

\subsection{Order and Multiplication}

Multiplication and order often interact in a consistent way as well, as
formalized in the following classes and theorems.

\begin{class}
    An order is said to be multiplicative if $0 \leq a$ and $0 \leq b$ implies
    $0 \leq ab$.
\end{class}

\begin{class}
    An order is said to be strictly multiplicative if $0 < a$ and $0 < b$
    implies $0 < ab$.
\end{class}

\begin{class}
    An order is said to be left multiplicative if for all $a$, $b$, and $c$ with
    $0 \leq c$, if $a \leq b$, then $ca \leq cb$.
\end{class}

\begin{class}
    An order is said to be right multiplicative if for all $a$, $b$, and $c$
    with $0 \leq c$, if $a \leq b$, then $ac \leq bc$.
\end{class}

\begin{class}
    An order is said to be left multiplication cancellative if for all $a$, $b$,
    and $c$ with $0 < c$, if $ca \leq cb$, then $a \leq b$.
\end{class}

\begin{class}
    An order is said to be right multiplication cancellative if for all $a$,
    $b$, and $c$ with $0 < c$, if $ac \leq bc$, then $a \leq b$.
\end{class}

\begin{class}
    An order is said to be strictly left multiplicative if for all $a$, $b$, and
    $c$ with $0 < c$, if $a < b$, then $ca < cb$.
\end{class}

\begin{class}
    An order is said to be strictly right multiplicative if for all $a$, $b$,
    and $c$ with $0 < c$, if $a < b$, then $ac < bc$.
\end{class}

\begin{class}
    An order is said to be strictly left multiplication cancellative if for all
    $a$, $b$, and $c$ with $0 < c$, if $ca < cb$, then $a < b$.
\end{class}

\begin{class}
    An order is said to be strictly right multiplication cancellative if for all
    $a$, $b$, and $c$ with $0 < c$, if $ac < bc$, then $a < b$.
\end{class}

\begin{class}[Ordered Ring]
    An Ordered Ring is a Ring that is also an Ordered Group and where the order
    is multiplicative.
\end{class}

\begin{class}[Ordered Integral Domain]
    An Ordered Integral Domain is an Ordered Ring that is an Integral Domain and
    where the order is left and right multiplication cancellative.
\end{class}

\begin{class}[Ordered Field]
    An Ordered Field is an Ordered Ring that is also a field.
\end{class}

When multiplication is commutative, each left class implies the right version.
To save on writing, all theorems and instances of this subsection will only be
stated for multiplication on the left.  All theorems work with multiplication on
the right too.

\begin{instance}
    In an Ordered Ring, $\leq$ is left multiplicative.
\end{instance}
\begin{proof}
    Assume that $a \leq b$ and $0 \leq c$.  Then $0 \lq b - a$, and because
    $\leq$ is multiplicative, we have $0 \leq c(b - a)$.  Then $0 \leq cb - ca$,
    so $ca \leq cb$.
\end{proof}

\begin{theorem}
    If $0 < a$, then $0 < a^{-1}$.
\end{theorem}
\begin{proof}
    Assume that $a^{-1} \leq 0$. Then multiplying by $a$ twice we get $aaa^{-1}
    \leq 0$.  Because $0 < a$, $0 \neq a$, so $aaa^{-1} = a \leq 0$.  $a \leq 0$
    contradicts $0 < a$, so we must have $0 < a^{-1}$.
\end{proof}

\begin{theorem}
    If $a < 0$, then $a^{-1} < 0$.
\end{theorem}
\begin{proof}
    Because $a < 0$, $0 < -a$, so by the previous theorem $0 < (-a)^{-1}$.
    Thus, $0 < -a^{-1}$, so $a^{-1} < 0$.
\end{proof}

\begin{instance}
    In an ordered field, $\leq$ is left multiplication cancellative.
\end{instance}
\begin{proof}
    Let $a$, $b$, and $c$ be such that $0 < c$ and $ca \leq cb$.  Because $0 <
    c$, $0 < c^{-1}$, so we can multiply by $c^{-1}$ on the left to get
    $c^{-1}ca \leq c^{-1}cb$, so $a \leq b$.
\end{proof}

Of course, in types where neither addition or multiplication are invertible, the
previous two instances can't be used.  Like with addition, there are a
surprising number of ways to prove some of these properties from the others.  In
fact, most of the implications in the last section work here too, although the
proofs often need a couple more details, so all the proofs will be stated again
here.

\begin{instance} \label{le_lt_mult}
    If an order is multiplicative and multiplication has the zero property, then
    the order is strictly multiplicative.
\end{instance}
\begin{proof}
    Let $0 < a$ and $0 < b$.  We have $0 \leq ab$ already, and $0 \neq ab$
    follows from Theorem \ref{mult_nz}.
\end{proof}

\begin{instance} \label{le_lmult1}
    If an order is strictly left multiplication cancellative, then it is left
    multiplicative.
\end{instance}
\begin{proof}
    Let $a \leq b$ and $0 \leq c$.  If $c = 0$, then $0 + a \leq 0 + b$.  If $c
    \neq 0$, we have $0 < c$.  If $cb < ca$, we could cancel $c$ to get $b < a$,
    contradicting $a \leq b$.  Thus, $ca \leq cb$.
\end{proof}

\begin{instance} \label{le_lmult2}
    If an order is strictly left multiplicative, then it is left multiplicative.
\end{instance}
\begin{proof}
    Let $a \leq b$ and $0 \leq c$.  If $c = 0$, then $0 + a \leq 0 + b$.  If $c
    \neq 0$, we have $0 < c$.  If $a = b$, then we're done.  If $a \neq b$, we
    have $a < b$, so we can multiply by $c > 0$ to get $ca < cb$, so $ca \leq
    cb$.
\end{proof}

\begin{instance} \label{lt_lmult1}
    If an order is left multiplicative and multiplication is left cancellative,
    then the order is strictly left multiplicative.
\end{instance}
\begin{proof}
    Let $a < b$ and $0 < c$.  Then we already have $ca \leq cb$ by assumption.
    If $ca = cb$, we could cancel $c$ to get $a = b$, contradicting $a < b$.
    Thus, $ca \neq cb$, so $ca < cb$.
\end{proof}

\begin{instance} \label{lt_lmult2}
    If an order is left multiplication cancellative, it is strictly left
    multiplicative.
\end{instance}
\begin{proof}
    Let $a < b$ and $0 < c$.  If $cb \leq ca$, we could cancel $c$ to get $b
    \leq a$, contradicting $a < b$.  Thus, $ca < cb$.
\end{proof}

\begin{instance} \label{le_mult_lcancel1}
    If an order is strictly left multiplication cancellative and multiplication
    is left cancellative, then the order is left multiplication cancellative.
\end{instance}
\begin{proof}
    Let $ca \leq cb$ and $0 < c$.  If $ca = cb$, we can cancel $c$ to get $a =
    b$, so $a \leq b$.  If $ca \neq cb$, we have $ca < cb$, and by cancelling
    $c$ we get $a < b$, so $a \leq b$.
\end{proof}

\begin{instance} \label{le_mult_lcancel2}
    If an order is strictly left multiplicative, it is left multiplication
    cancellative.
\end{instance}
\begin{proof}
    Let $ca \leq cb$ and $0 < c$.  If $b < a$, we could multiply by $c$ to get
    $cb < ca$, contradicting $ca \leq cb$.  Thus, $a \leq b$.
\end{proof}

\begin{instance} \label{lt_mult_lcancel1}
    If an order is left multiplication cancellative, it is strictly left
    multiplication cancellative.
\end{instance}
\begin{proof}
    Let $ca < cb$ and $0 < c$.  We already have $a \leq b$ by assumption.  If
    $a = b$, $ca < cb$ would be impossible, so $a \neq b$, meaning that $a < b$.
\end{proof}

\begin{instance} \label{lt_mult_lcancel2}
    If an order is left multiplicative, it is strictly left multiplication
    cancellative.
\end{instance}
\begin{proof}
    Let $ca < cb$ and $0 < c$.  If $b \leq a$, we could multiply by $c$ to get
    $cb \leq ca$, contradicting $ca < cb$.  Thus, $a < b$.
\end{proof}

Given all of these implications, which should be used?  The standard route for
most types is to prove that the order is multiplicative and left multiplication
cancellative.  Along with commutativity, this will provide all of the
properties.  However, in cases where this is not possible, much more care needs
to be taken.  Currently, the only type this is needed in in this document is the
ordinals, so I will just explicitly say which instances I am using then.

In some types, the conditions $0 \leq a$ in the previous classes is unnecessary
because in those types, all values are positive.  This leads to a new class:

\begin{class}
    An order is said to be positive if for all $x$, $0 \leq x$.
\end{class}

A few simple theorems follow from this.

\begin{theorem} \label{all_neg_eq}
    In a positive order, if $x \leq 0$, then $0 = x$.
\end{theorem}
\begin{proof}
    This follows directly by antisymmetry.
\end{proof}

\begin{theorem} \label{all_pos2}
    In a positive order, for all $x \neq 0$, $0 < x$.
\end{theorem}
\begin{proof}
    We have $0 \leq x$ by the order being positive and $0 \neq x$ by assumption.
\end{proof}

\begin{theorem} \label{not_neg}
    In a positive order, for all $x$, $\neg(x < 0)$.
\end{theorem}
\begin{proof}
    This is just the negation of $0 \leq x$.
\end{proof}

In a positive order, all left/right multiplications and cancellations can be
done without checking for positivity, although $0 \neq c$ still needs to be
checked in many cases.

There are two more implications that positive orders satisfy:

\begin{instance} \label{mult_lcancel1}
    If an order is positive and strictly left multiplicative, multiplication is
    left cancellative.
\end{instance}
\begin{proof}
    Let $ca = cb$ and $c \neq 0$.  Assume for a contradiction that $a \neq b$,
    and without loss of generality assume that $a < b$.  Then we can multiply by
    $c$ to get $ca < cb$, contradicting $ca = cb$.  Thus, $a = b$.
\end{proof}

\begin{instance} \label{mult_lcancel2}
    If an order is positive and left multiplication cancellative, multiplication
    is left cancellative.
\end{instance}
\begin{proof}
    Let $ca = cb$ and $c \neq 0$.  Then by reflexivity we have $ca \leq cb$ and
    $cb \leq ca$.  Cancelling $c$, we get $a \leq b$ and $b \leq a$, and by
    antisymmetry we get $a = b$.
\end{proof}

We will now consider orders that can have negative values again.

\begin{theorem} \label{le_lmult_neg}
    If $a \leq b$ and $c \leq 0$, then $cb \leq ca$.
\end{theorem}
\begin{proof}
    Because $c \leq 0$, $0 \leq -c$, so we can multiply $a \leq b$ by $-c$ to
    get $-ca \leq -cb$, so $cb \leq ca$.
\end{proof}

\begin{theorem} \label{le_mult_lcancel_neg}
    If $ca \leq cb$ and $c < 0$, then $b \leq a$.
\end{theorem}
\begin{proof}
    From $ca \leq cb$, we have $(-c)(-a) \leq (-c)(-b)$.  Because $c < 0$, $0 <
    -c$, so we can cancel $-c$ to get $-a \leq -b$, so $b \leq a$.
\end{proof}

\begin{theorem} \label{le_lrmult_pos}
    If $a \leq b$, $c \leq d$, $0 \leq a$, and $0 \leq c$, then $ac \leq bd$.
\end{theorem}
\begin{proof}
    From $a \leq b$ and $c \leq 0$, we get $ac \leq bc$.  Because $0 \leq a$ and
    $a \leq b$, we get $0 \leq b$, and with $c \leq d$ we get $bc \leq bd$.
    Thus, by transitivity, we get $ac \leq bd$.
\end{proof}

In the next several theorems, we will be proving that several of the properties
of $\leq$ hold for $<$ as well.  Because the $\leq$ properties are already
known, the only interesting part of the proofs are proving that two values are
not equal, and the rest of the proofs will be skipped.

\begin{theorem}
    If $a < b$ and $c < 0$, then $cb < ca$.
\end{theorem}
\begin{proof}
    This proof is identical in form to Theorem \ref{le_lmult_neg}.
\end{proof}

\begin{theorem}
    If $ca < cb$ and $c < 0$, then $b < a$.
\end{theorem}
\begin{proof}
    This proof is identical in form to Theorem \ref{le_mult_lcancel_neg}.
\end{proof}

\begin{theorem}
    If $a < b$, $c < d$, $0 \leq a$, and $0 \leq c$, then $ac < bd$.
\end{theorem}
\begin{proof}
    First, because $0 \leq a$ and $a < b$, we have $0 < b$.  We now have two
    cases: when $c = 0$, and when $c \neq 0$.

    When $c = 0$, we must prove that $0 < bd$.  We already know that $0 < b$,
    and because $0 \leq c$ and $c < d$, we know that $0 < d$ as well.  Because
    $<$ is multiplicative, we know that $0 < bd$.

    When $c \neq 0$, we now know that $0 < c$.  Then we can multiply $a < b$ by
    $c$ to get $ac < bc$, and we can multiply $c < d$ by $b$ to get $bc < bd$.
    Then by transitivity we get $ac < bd$.
\end{proof}

We can now move on to slightly more interesting theorems.

\begin{theorem} \label{square_pos}
    For all $a$, $0 \leq aa$.
\end{theorem}
\begin{proof}
    We know that either $0 \leq a$ or $0 \leq -a$.  When $0 \leq a$, the result
    follows from $\leq$ being multiplicative.  When $0 \leq -a$, we can write
    $aa = (-a)(-a)$, and again the result follows from $\leq$ being
    multiplicative.
\end{proof}

\begin{theorem}
    If $0 \leq a$ and $0 \leq b$, then $a \leq b \leftrightarrow aa \leq bb$.
\end{theorem}
\begin{proof}
    Assume that $a \leq b$.  Then the result follows directly from Theorem
    \ref{le_lrmult_pos}.

    Now assume that $aa \leq bb$.  If $a = 0$, the result would follow from
    Theorem \ref{square_pos}, so assume that $a \neq 0$.  From $aa \leq bb$ we
    get $0 \leq bb - aa = (b + a)(b - a)$.  Because $0 < a$ and $0 \leq b$, we
    know that $0 < b + a$, so we can cancel $b + a$ from $0 \leq (b + a)(b - a)$
    to get $0 \leq b - a$.  Thus, $a \leq b$ as required.
\end{proof}

\begin{theorem}
    If $0 \leq a$ and $0 \leq b$, then $a < b \leftrightarrow aa < bb$.
\end{theorem}
\begin{proof}
    This proof is identical in form to the previous proof.
\end{proof}

\begin{theorem} \label{one_pos}
    $0 < 1$.
\end{theorem}
\begin{proof}
    We know that $0 \neq 1$ from Theorem \ref{not_trivial_one}, so we only need
    to prove that $0 \leq 1$.  Asume that $1 < 0$.  Then $0 < -1$, and because
    $<$ is multiplicative, we get $0 < (-1)(-1) = 1$.  But this contradicts $1 <
    0$, so it must be the case that $0 \leq 1$.
\end{proof}

\begin{theorem}
    For all $x$, $x < x + 1$.
\end{theorem}
\begin{proof}
    We can cancel x to get $0 < 1$ which is true by the previous theorem.
\end{proof}

Several simple statements follow from some of the previous theorems, such as $0
< 2$, $1 < 2$, etc.

\begin{theorem} \label{inv_ge_one}
    If $1 \leq a$, then $a^{-1} \leq 1$.
\end{theorem}
\begin{proof}
    Because $1 \leq a$, we know that $0 < a$.  Thus, $0 < a^{-1}$, so we can
    multiply both sides of $1 \leq a$ by $a^{-1}$ to get $a^{-1} \leq 1$.
\end{proof}

\begin{theorem} \label{inv_le_one}
    If $0 < a \leq 1$, then $1 \leq a^{-1}$.
\end{theorem}
\begin{proof}
    This proof is similar to the previous proof.
\end{proof}

\begin{theorem}
    If $1 < a$, then $a^{-1} < 1$.
\end{theorem}
\begin{proof}
    This proof is identical in form to Theorem \ref{inv_ge_one}.
\end{proof}

\begin{theorem}
    If $0 < a < 1$, then $1 < a^{-1}$.
\end{theorem}
\begin{proof}
    This proof is identical in form to Theorem \ref{inv_le_one}.
\end{proof}

\begin{theorem}
    If $0 < a$ and $aa = 1$, then $a = 1$.
\end{theorem}
\begin{proof}
    Because $aa = 1 = 1(1)$, we have $0 = aa - 1(1) = (a + 1)(a - 1)$.  Now if
    $a \neq 1$, we would have $a - 1 \neq 0$, so we could cancel $a - 1$ and get
    $0 = a + 1$.  This implies that $a = -1$.  However, $0 < a$, so this is
    impossible.  Thus, it must be the case that $a = 1$.
\end{proof}

\begin{theorem}
    If $aa = 1$, then $a = 1$ or $a = -1$.
\end{theorem}
\begin{proof}
    By trichotomy, we have either $a < 0$, $a = 0$, or $0 < a$.  The case $a =
    0$ because this would imply $0 = 1$, and the case $0 < a$ is solved by the
    previous theorem.  Now when $a < 0$, we have $0 < -a$, and because $(-a)(-a)
    = aa = 1$, by the previous theorem we have $-a = 1$, so $a = -1$.
\end{proof}

\begin{theorem}
    \[
        \frac{a}{2} + \frac{a}{2} = a.
    \]
\end{theorem}
\begin{proof}
    \[
        \frac{a}{2} + \frac{a}{2} = \frac{2a}{2} = a.
    \]
\end{proof}

\begin{theorem} \label{le_div_pos}
    If $0 < a$ and $a \leq b$, then $b^{-1} \leq a^{-1}$.
\end{theorem}
\begin{proof}
    By transitivity we also have $0 < b$.  Thus, we can multiply both sides
    of $a \leq b$ by $a^{-1}$ and by $b^{-1}$ to get $b^{-1} \leq a^{-1}$.
\end{proof}

\begin{theorem} \label{le_div_neg}
    If $b < 0$ and $a \leq b$, then $b^{-1} \leq a^{-1}$.
\end{theorem}
\begin{proof}
    By transitivity we also have $a^{-1} < 0$.  Thus, we can multiply both sides
    of $a \leq b$ by $a^{-1}$ and by $b^{-1}$ to get $b^{-1} \leq a^{-1}$, where
    both multiplications cause the inequality to flip.
\end{proof}

\begin{theorem}
    If $0 < a$ and $a < b$, then $b^{-1} < a^{-1}$.
\end{theorem}
\begin{proof}
    The proof is identical in form to the proof of Theorem \ref{le_div_pos}.
\end{proof}

\begin{theorem}
    If $b < 0$ and $a < b$, then $b^{-1} < a^{-1}$.
\end{theorem}
\begin{proof}
    The proof is identical in form to the proof of Theorem \ref{le_div_neg}.
\end{proof}

\begin{theorem}
    If $0 < a$, then $0 < \frac{a}{2}$.
\end{theorem}
\begin{proof}
    Because $0 < 2$, we have $0 < 2^{-1}$.  Because $0 < a$ and $0 < 2^{-1}$, $0
    < a2^{-1}$.
\end{proof}

\begin{theorem}
    If $a < 0$, then $\frac{a}{2} < 0$.
\end{theorem}
\begin{proof}
    Because $0 < 2$, we have $0 < 2^{-1}$.  We can thus multiply both sides of
    $a < 0$ by $2^{-1}$ to get $a2^{-1} < 0$.
\end{proof}

\begin{theorem}
    If $0 < a$, then $0 < 2a$.
\end{theorem}
\begin{proof}
    This follows directly from $0 < a$ and $0 < 2$.
\end{proof}

\begin{theorem}
    If $a < 0$, then $2a < 0$.
\end{theorem}
\begin{proof}
    Because $0 < 2$, we can multiply both sides of $a < 0$ by $2$ to get $2a <
    0$.
\end{proof}

\begin{theorem}
    If $a < b$, then
    \[
        a < \frac{a + b}{2}.
    \]
\end{theorem}
\begin{proof}
    \begin{align*}
        a &< \frac{a + b}{2} \\
        \frac{a}{2} + \frac{a}{2} &< \frac{a}{2} + \frac{b}{2} \\
        \frac{a}{2} &< \frac{b}{2} \\
        a &< b.
    \end{align*}
\end{proof}

\begin{theorem}
    If $a < b$, then
    \[
        \frac{a + b}{2} < b.
    \]
\end{theorem}
\begin{proof}
    \begin{align*}
        \frac{a + b}{2} &< b \\
        \frac{a}{2} + \frac{b}{2} &< \frac{b}{2} + \frac{b}{2} \\
        \frac{a}{2} &< \frac{b}{2} \\
        a &< b.
    \end{align*}
\end{proof}

\begin{instance}
    Ordered fields are dense.
\end{instance}
\begin{proof}
    Given $a$ and $b$ such that $a < b$, we have
    \[
        a < \frac{a + b}{2} < b
    \]
    by the previous two theorems.
\end{proof}

\subsection{Cones}

An alternative characterization of ordered fields is using cones, where instead
of specifying an order on a field $\U$, we instead identify a certain subset
that we consider to be the positive numbers.  In this context we will call this
subset a cone.

\begin{class}
    Given a field $\U$, a cone is a subset $C$ such that:
    \begin{itemize}
        \item For all $a \in C$ and $b \in C$, $a + b \in C$.
        \item For all $a \in C$ and $b \in C$, $ab \in C$.
        \item For all $a : \U$, $aa \in C$.
        \item $-1 \notin C$.
        \item For all $a : \U$, either $a \in C$ or $-A \in C$.
    \end{itemize}
\end{class}

In the rest of this section, we will prove that using a cone on a field $\U$,
$\U$ is an ordered field.  Thus, for the rest of this section, assume that $\U$
is a field with a cone $C$.

\begin{theorem} \label{cone_one}
    $1 \in C$.
\end{theorem}
\begin{proof}
    Because $1 = (-1)(-1)$ and because all squares are in $C$, $1 \in C$.
\end{proof}

\begin{theorem} \label{cone_div}
    If $a \neq 0$ and $a \in C$, then $a^{-1} \in C$.
\end{theorem}
\begin{proof}
    Because all squares are in $C$, $a^{-1}a^{-1} \in C$.  Then because all
    products are in $C$, $a^{-1}a^{-1}a = a^{-1} \in C$.
\end{proof}

\begin{instance}
    Define an order on $\U$ by defining $a \leq b$ to mean that $b - a \in S$.
\end{instance}

\begin{theorem} \label{cone_pos}
    For all $a : \U$, $a \in S$ if and only if $0 \leq a$.
\end{theorem}
\begin{proof}
    The condition $0 \leq a$ is equivalent to $a = a - 0 \in S$.
\end{proof}

\begin{instance}
    The order in $\U$ is antisymmetric.
\end{instance}
\begin{proof}
    We have $a - b \in C$ and $b - a \in C$, and we must prove that $a = b$.  We
    can define $c = a - b$ to simplify this to proving that if $c \in C$ and $-c
    \in C$, then $c = 0$.  Assume that $c \neq 0$.  Then by Theorem
    \ref{cone_div}, $c^{-1} \in C$, so $(-c)c^{-1} = -1 \in C$, which is a
    contradiction.  Thus, $c = 0$.
\end{proof}

\begin{instance}
    The order in $\U$ is transitive.
\end{instance}
\begin{proof}
    We must prove that if $b - a \in C$ and $c - b \in C$, then $c - a \in C$.
    By adding we get $b - a + c - b = c - a \in C$.
\end{proof}

\begin{instance}
    The order in $\U$ is connex.
\end{instance}
\begin{proof}
    We must prove that for all $a$ and $b$, either $a - b \in C$ or $b - a \in
    C$.  The second one can be rewritten as $-(a - b) \in C$, so by the last
    condition for being a cone the result is true.
\end{proof}

\begin{instance}
    The order in $\U$ is left additive.
\end{instance}
\begin{proof}
    We must prove that if $b - a \in C$, then $(c + b) - (c + a) \in C$.  This
    follows directly by $c - c = 0$.
\end{proof}

\begin{instance}
    The order in $\U$ is multiplicative.
\end{instance}
\begin{proof}
    By Theorem \ref{cone_pos}, this follows directly from the multiplication
    requirement for cones.
\end{proof}

\subsection{Bounds}

In all of the following definitions, let $\leq$ be an arbitrary relation and let
$S$ be a set on $\U$.

\begin{definition}
    A value $x$ is a lower bound of $S$ if for all $y \in S$, $x \leq y$.
\end{definition}

\begin{definition}
    A value $x$ is an upper bound of $S$ if for all $y \in S$, $y \leq x$.
\end{definition}

\begin{definition}
    A value $x$ is the infimum of $S$ if it is a lower bound of $S$ and for all
    other lower bounds $y$, $y \leq x$.
\end{definition}

\begin{definition}
    A value $x$ is the supremum of $S$ if it is an upper bound of $S$ and for
    all other upper bounds $y$, $x \leq y$.
\end{definition}

\begin{class}
    A relation is supremum complete if every nonempty set that has an upper
    bound has a supremum.
\end{class}

\begin{theorem} \label{upper_bound_leq}
    If $a$ is not an upper bound of $S$ and $b$ is an upper bound of $S$, then
    $a < b$.
\end{theorem}
\begin{proof}
    Because $a$ is not an upper bound of $S$, there exists some $c \in S$ such
    that $a < c$.  Because $b$ is an upper bound of $S$, we have $c \leq b$, so
    by transitivity we get $a < b$.
\end{proof}

\begin{theorem} \label{lower_bound_leq}
    If $a$ is not a lower bound of $S$ and $b$ is a lower bound of $S$, then $a
    < b$.
\end{theorem}
\begin{proof}
    This is just the dual version of the previous theorem.
\end{proof}

\begin{theorem}
    An ordered group that is supremum complete is also ``infimum complete'',
    that is, every nonempty set with a lower bound has an infimum.
\end{theorem}
\begin{proof}
    Let $S$ be a nonempty set with a lower bound.  Then let $S' = \{x : \U \mid
    -x \in S$.  Because $S$ is nonempty, there exists some $x \in S$.  Then $-x
    \in S'$, so $S'$ is nonempty.  We also know that $S$ has a lower bound $l$.
    Then for any $y \in S'$, $-y \in S$, so $l \leq -y$, which means that $y
    \leq -l$, so $-l$ is an upper bound for $S'$.

    Because $S'$ is nonempty and has an upper bound, it has a supremum $\alpha$.
    Now consider any $x \in S$.  Then $-x \in S'$, so $-x \leq \alpha$.  Thus,
    $-\alpha \leq x$, so $-\alpha$ is a lower bound of $S$.  Furthermore, for
    any other lower bound $y$ of $S$, by the above argument we have that $-y$ is
    an upper bound of $S'$.  Thus, $\alpha \leq -y$, so $y \leq -\alpha$.  This
    means that $-\alpha$ is the infimum of $S$.
\end{proof}

\begin{theorem}
    Let $\U$ be dense.  Let $S$ be a set on $\U$, and define a new set $C = \{x
    : \U \mid \text{$x$ is not an upper bound of $S$}\}$.  Then for all
    $\alpha$, $\alpha$ is a supremum of $S$ if and only if it is a supremum of
    $C$.
\end{theorem}
\begin{proof}
    Assume that $\alpha$ is the supremum of $S$.  Let $x$ be an element of $C$.
    Because $x \in C$, $x$ is not an upper bound of $S$.  Thus, by Theorem
    \ref{upper_bound_leq}, $x < \alpha$, so $\alpha$ is an upper bound of $C$.

    Now let $y$ be another upper bound of $C$.  Assume for a contradiction that
    $y < \alpha$.  Because $\U$ is dense, there exists a $z$ such that $y < z <
    \alpha$.  Now if $z$ was an upper bound of $S$, we would have $\alpha \leq
    z$, which contradicts $z < \alpha$, so $z \in C$.  But $y$ is an upper bound
    of $C$, so $z \leq y$, which contradicts $y < z$.  Thus the original
    assumption $y < \alpha$ is false, so we must have $\alpha \leq y$.  Thus,
    $\alpha$ is the least upper bound of $C$.

    Now assume that $\alpha$ is the supremum of $C$.  Let $x$ be an element of
    $S$.  Assume for a contradiction that $\alpha < x$.  Because $\U$ is dense,
    there exists a $z$ such that $\alpha < z < x$.  Now if $z$ was an upper
    bound of $S$, we would have $x \leq z$, which contradicts $z < x$, so $z \in
    C$.  But $\alpha$ is an upper bound of $C$, so $z \leq \alpha$, which
    contradicts $\alpha < z$.  Thus the original assumption $\alpha < x$ is
    false, so we must have $x \leq \alpha$, so $\alpha$ is an upper bound of
    $S$.

    Now let $y$ be another upper bound of $S$.  By the first paragraph of the
    proof, $y$ is an upper bound of $C$ as well, so $\alpha \leq y$.  Thus,
    $\alpha$ is the least upper bound of $S$.
\end{proof}

\subsection{Absolute Values I}

There are two main notions of absolute value: the one inherent in any ordered
group, and norms from linear algebra.  We will look at the first one here.

\begin{definition}
    Let $a : \U$.  Then we define the absolute value $|a|$ piecewise:
    \[
        |a| = \begin{cases}
            a & \text{if $0 \leq a$} \\
            -a & \text{if $a < 0$.}
        \end{cases}
    \]
\end{definition}

\begin{theorem} \label{abs_pos_eq}
    If $0 \leq x$, then $|x| = x$.
\end{theorem}
\begin{proof}
    This follow directly from the definition.
\end{proof}

\begin{theorem}
    $|0| = 0$.
\end{theorem}
\begin{proof}
    This follows from $0 \leq 0$.
\end{proof}

\begin{theorem}
    $|1| = 1$.
\end{theorem}
\begin{proof}
    This follows from $0 \leq 1$.
\end{proof}

\begin{theorem}
    $0 = |x|$ iff $0 = x$.
\end{theorem}
\begin{proof}
    If $0 \leq x$, then the statement is $0 = x \leftrightarrow 0 = x$, which is
    trivially true.  So assume that $x < 0$.  Then the equation $0 = -x$ is
    equivalent to $0 = x$.
\end{proof}

\begin{theorem}
    $0 \neq |x|$ iff $0 \neq x$.
\end{theorem}
\begin{proof}
    This follows directly from the previous theorem.
\end{proof}

\begin{theorem}
    $0 \leq |x|$.
\end{theorem}
\begin{proof}
    If $0 \leq x$, then we're done, so now consider the case where $x < 0$.
    This implies that $0 < -x$, so $0 \leq -x = |x|$.
\end{proof}

\begin{theorem}
    $|{-x}| = |x|$.
\end{theorem}
\begin{proof}
    There are four cases.
    \begin{case} $0 \leq -x$, $0 \leq x$. \\
        From $0 \leq -x$ we get $x \leq 0$, and with $0 \leq x$ by antisymmetry
        we get $0 = x$.  Because $-0 = 0$, this case is true.
    \end{case}
    \begin{case} $0 \leq -x$, $x < 0$. \\
        The result becomes the reflexive equality $-x = -x$.
    \end{case}
    \begin{case} $-x < 0$, $0 \leq x$. \\
        The result becomes $-{-x} = x$, which is true.
    \end{case}
    \begin{case} $-x < 0$, $x < 0$. \\
        From $-x < 0$ we get $0 < x$, which contradicts $x < 0$, so this case is
        impossible.
    \end{case}
\end{proof}

\begin{theorem}
    If $x \leq 0$, then $|x| = -x$.
\end{theorem}
\begin{proof}
    Because $x \leq 0$ we have $0 \leq -x$.  By the previous theorem we have
    $|x| = |{-x}|$, so by Theorem \ref{abs_pos_eq} $|x| = -x$.
\end{proof}

\begin{theorem} \label{abs_minus}
    $|a - b| = |b - a|$.
\end{theorem}
\begin{proof}
    \[
        |a - b| = |{-(a - b)}| = |{-{-b}} - a| = |b - a|.
    \]
\end{proof}

\begin{theorem}
    $|ab| = |a|\,|b|$.
\end{theorem}
\begin{proof}
    We will first prove the following lemma: for all $a \geq 0$, $|ab| = a|b|$.
    Consider the case when $0 \leq b$.  Then $0 \leq ab$, so the equation to be
    proved becomes the reflexive equality $ab = ab$.  Now when $b \leq 0$, $ab
    \leq 0$, so $|ab| = -ab$ and $|b| = -b$, so the equation to be proved once
    becomes another reflexive equality $-ab = -ab$.

    Armed with this lemma, let's now attack the main theorem.  Then the case $0
    \leq a$ follows from the lemma.  Now when $a \leq 0$, we can write the
    equation to be proved as $|-ab| = -a|b|$.  Because $0 \leq -a$, this again
    follows from the lemma.
\end{proof}

\begin{theorem} \label{abs_le}
    For all $a$ and $b$, $|a| \leq b \leftrightarrow -b \leq a \leq b$.
\end{theorem}
\begin{proof}
    First assume that $|a| \leq b$ and consider the case $0 \leq a$.  Then we
    have $a \leq b$, so $-b \leq -a$ as well.  Because $0 \leq a$, we have $-a
    \leq 0$, so by transitivity we have $-b \leq a$.  Now for the case $a \leq
    0$, we have $-a \leq b$, so we directly get $-b \leq a$.  Because $a \leq
    0$, we have $0 \leq -a$, so by transitivity we get $-b \leq -a$.  We thus
    have $a \leq b$.

    Now assume that $-b \leq a \leq b$.  Then if $0 \leq a$ we have $|a| = a
    \leq b$, and if $a \leq 0$ we have $|a| = -a \leq b$.
\end{proof}

\begin{theorem} \label{abs_lt}
    For all $a$ and $b$, $|a| < b \leftrightarrow -b < a < b$.
\end{theorem}
\begin{proof}
    This proof is identical in form to the previous proof.
\end{proof}

\begin{theorem} \label{abs_le_pos}
    For all $x$, $x \leq |x|$.
\end{theorem}
\begin{proof}
    If $0 \leq x$, the proof follows by reflexivity.  If $x \leq 0$, we have $0
    \leq -x = |x|$, so by transitivity we get $x \leq |x|$.
\end{proof}

\begin{theorem} \label{abs_le_neg}
    For all $x$, $-x \leq |x|$.
\end{theorem}
\begin{proof}
    If $0 \leq x$, then $-x \leq 0$, so by transitivity we have $-x \leq x =
    |x|$.  When $x \leq 0$, the proof follows by reflexivity.
\end{proof}

\begin{theorem} \label{abs_tri}
    For all $a$ and $b$, $|a + b| \leq |a| + |b|$.
\end{theorem}
\begin{proof}
    By Theorem \ref{abs_le}, we only need to prove $-|a| - |b| \leq a + b \leq
    |a| + |b|$.  The first of these can be rewritten as $-a - b \leq |a| + |b|$.
    By Theorem \ref{abs_le_neg}, we have $-a \leq |a|$ and $-b \leq |b|$, and
    adding these gets the required result.  The second of these, $a + b \leq |a|
    + |b|$, is similar.  By Theorem \ref{abs_le_pos}, we have $a \leq |a|$ and
    $b \leq |b|$, and adding these again gets the required result.
\end{proof}

\begin{theorem}
    For all $a$ and $b$, $|a| - |b| \leq |a - b|$.
\end{theorem}
\begin{proof}
    By the triangle inequality, we have $|a| = |(a - b) + b| \leq |a - b| +
    |b|$.  Rearranging, we get $|a| - |b| \leq |a - b|$.
\end{proof}

\begin{theorem}
    For all $a$ and $b$, $||a| - |b|| \leq |a - b|$.
\end{theorem}
\begin{proof}
    We can use Theorem \ref{abs_le}, and one of the cases was already covered by
    the previous theorem.  So we just need to prove $-|a - b| \leq |a| - |b|$.
    Rearranging, we get $|b| - |a| \leq |a - b| = |b - a|$, so the result again
    follows by the previous theorem.
\end{proof}

\begin{theorem}
    For all $x$, $||x|| = |x|$.
\end{theorem}
\begin{proof}
    We know that $0 \leq |x|$, so the result follows from Theorem
    \ref{abs_pos_eq}.
\end{proof}

\begin{theorem}
    If $x \neq 0$, then $|x|^{-1} = |x^{-1}|$.
\end{theorem}
\begin{proof}
    \begin{align*}
        |x|^{-1} &= |x^{-1}| \\
        |x|^{-1} |x| &= |x^{-1}| |x| \\
        |x|^{-1} |x| &= |x^{-1}x| \\
        1 &= |1| \\
        1 &= 1 \\
    \end{align*}
\end{proof}

\subsection{Minima and Maxima}

\begin{definition}
    Let $a$ and $b$ be values in $\U$.  Then we define the minimum of $a$ and
    $b$ piecewise:
    \[
        \min(a, b) = \begin{cases}
            a & \text{if $a \leq b$} \\
            b & \text{if $b < a$.}
        \end{cases}
    \]
    We also define the maximum of $a$ and $b$ in the same way:
    \[
        \min(a, b) = \begin{cases}
            b & \text{if $a \leq b$} \\
            a & \text{if $b < a$.}
        \end{cases}
    \]
\end{definition}

\begin{theorem}
    For all $a$ and $b$, if $a \leq b$, then $\min(a, b) = a$.
\end{theorem}
\begin{proof}
    $\min(a, b)$ has two cases: $a \leq b$, and $b < a$.  When $a \leq b$, then
    $\min(a, b) = a$ as required.  The case $b < a$ contradicts the $a \leq b$
    from the theorem's hypotheses so this case is impossible.
\end{proof}

\begin{theorem}
    For all $a$ and $b$, if $b \leq a$, then $\min(a, b) = b$.
\end{theorem}
\begin{proof}
    $\min(a, b)$ has two cases: $a \leq b$, and $b < a$.  When $a \leq b$, then
    we must prove $a = b$.  Because we have $a \leq b$ and $b \leq a$, the
    result follows from antisymmetry.  When $b < a$, $\min(a, b) = b$ as
    required.
\end{proof}

\begin{theorem}
    For all $a$ and $b$, if $a \leq b$, then $\max(a, b) = b$.
\end{theorem}
\begin{proof}
    $\max(a, b)$ has two cases: $a \leq b$, and $b < a$.  When $a \leq b$, then
    $\max(a, b) = b$ as required.  The case $b < a$ contradicts the $a \leq b$
    from the theorem's hypotheses so this case is impossible.
\end{proof}

\begin{theorem}
    For all $a$ and $b$, if $b \leq a$, then $\max(a, b) = a$.
\end{theorem}
\begin{proof}
    $\max(a, b)$ has two cases: $a \leq b$, and $b < a$.  When $a \leq b$, then
    we must prove $b = a$.  Because we have $a \leq b$ and $b \leq a$, the
    result follows from antisymmetry.  When $b < a$, $\max(a, b) = a$ as
    required.
\end{proof}

When working with $\min$ and $\max$, the previous theorems allow us to consider
the cases $a \leq b$ and $b \leq a$ rather than $a \leq b$ and $b < a$, which
can make some proofs a bit simpler.

\begin{theorem} \label{min_comm}
    For all $a$ and $b$, $\min(a, b) = \min(b, a)$.
\end{theorem}
\begin{proof}
    When $a \leq b$, we get $\min(a, b) = a = \min(b, a)$, and when $b \leq a$,
    we get $\min(a, b) = b = \min(b, a)$.
\end{proof}

\begin{theorem} \label{max_comm}
    For all $a$ and $b$, $\max(a, b) = \max(b, a)$.
\end{theorem}
\begin{proof}
    When $a \leq b$, we get $\max(a, b) = b = \max(b, a)$, and when $b \leq a$,
    we get $\max(a, b) = a = \max(b, a)$.
\end{proof}

\begin{theorem}
    For all $a$, $b$, and $c$, $\min(a, \min(b, c)) = \min(\min(a, b), c)$.
\end{theorem}
\begin{proof}
    There are four cases.
    \begin{case} $a \leq b$, $b \leq c$. \\
        By transitivity we also have $a \leq c$.  Then
        \[
            \min(a, \min(b, c)) = \min(a, b) =
            a
            = \min(a, c) = \min(\min(a, b), c).
        \]
    \end{case}
    \begin{case} $a \leq b$, $c \leq b$.
        \[
            \min(a, \min(b, c)) =
            \min(a, c)
            = \min(\min(a, b), c).
        \]
    \end{case}
    \begin{case} $b \leq a$, $b \leq c$.
        \[
            \min(a, \min(b, c)) = \min(a, b) =
            b
            = \min(b, c) = \min(\min(a, b), c).
        \]
    \end{case}
    \begin{case} $b \leq a$, $c \leq b$. \\
        By transitivity we also have $c \leq a$.  Then
        \[
            \min(a, \min(b, c)) = \min(a, c) =
            c
            = \min(b, c) = \min(\min(a, b), c).
        \]
    \end{case}
\end{proof}

\begin{theorem}
    For all $a$, $b$, and $c$, $\max(a, \max(b, c)) = \max(\max(a, b), c)$.
\end{theorem}
\begin{proof}
    There are four cases.
    \begin{case} $a \leq b$, $b \leq c$. \\
        By transitivity we also have $a \leq c$.  Then
        \[
            \max(a, \max(b, c)) = \max(a, c) =
            c
            = \max(b, c) = \max(\max(a, b), c).
        \]
    \end{case}
    \begin{case} $a \leq b$, $c \leq b$.
        \[
            \max(a, \max(b, c)) = \max(a, b) =
            b
            = \max(b, c) = \max(\max(a, b), c).
        \]
    \end{case}
    \begin{case} $b \leq a$, $b \leq c$.
        \[
            \max(a, \max(b, c)) =
            \max(a, c)
            = \max(\max(a, b), c).
        \]
    \end{case}
    \begin{case} $b \leq a$, $c \leq b$. \\
        By transitivity we also have $c \leq a$.  Then
        \[
            \max(a, \max(b, c)) = \max(a, b) =
            a
            = \max(a, c) = \max(\max(a, b), c).
        \]
    \end{case}
\end{proof}

The previous few theorems allow us to to find the minimum/maximum of several
numbers with no ambiguity.

\begin{theorem} \label{lmin}
    For all $a$ and $b$, $\min(a, b) \leq a$.
\end{theorem}
\begin{proof}
    When $a \leq b$, $\min(a, b) = a \leq a$.  When $b \leq a$, $\min(a, b) = b
    \leq a$.
\end{proof}

\begin{theorem} \label{rmin}
    For all $a$ and $b$, $\min(a, b) \leq b$.
\end{theorem}
\begin{proof}
    This follows from Theorems \ref{lmin} and \ref{min_comm}.
\end{proof}

\begin{theorem} \label{lmax}
    For all $a$ and $b$, $a \leq \max(a, b)$.
\end{theorem}
\begin{proof}
    When $a \leq b$, $\max(a, b) = b \geq a$.  When $b \leq a$, $\max(a, b) = a
    \geq a$.
\end{proof}

\begin{theorem} \label{rmax}
    For all $a$ and $b$, $b \leq \max(a, b)$.
\end{theorem}
\begin{proof}
    This follows from Theorems \ref{lmax} and \ref{max_comm}.
\end{proof}

\begin{theorem}
    Given two types $\A$ and $\B$ and a function $f : \A \rightarrow \A
    \rightarrow \B$ such that $f(a, b) = f(b, a)$ for all $a$ and $b$ in $\A$
    (such as addition), then $f(\min(a, b), \max(a, b)) = f(a, b)$.
\end{theorem}
\begin{proof}
    When $a \leq b$, $f(\min(a, b), \max(a, b)) = f(a, b)$.  When $b \leq a$,
    $f(\min(a, b), \max(a, b)) = f(b, a) = f(a, b)$.
\end{proof}

\begin{theorem} \label{homo-min}
    If $f$ is a function that is equivalently orderly, then for all $a$ and $b$,
    \[
        f(\min(a, b)) = \min(f(a), f(b)).
    \]
\end{theorem}
\begin{proof}
    Either $a \leq b$ or $b \leq a$.  If $a \leq b$, then $f(a) \leq f(b)$, so
    $\min(a, b) = a$ and $\min(f(a), f(b)) = f(a)$, meaning that we just need to
    prove $f(a) = f(a)$, which is trivial.  The case $b \leq a$ is the same in
    form.
\end{proof}

\begin{theorem} \label{homo-max}
    If $f$ is a function that is equivalently orderly, then for all $a$ and $b$,
    \[
        f(\max(a, b)) = \max(f(a), f(b)).
    \]
\end{theorem}
\begin{proof}
    Either $a \leq b$ or $b \leq a$.  If $a \leq b$, then $f(a) \leq f(b)$, so
    $\max(a, b) = b$ and $\max(f(a), f(b)) = f(b)$, meaning that we just need to
    prove $f(b) = f(b)$, which is trivial.  The case $b \leq a$ is the same in
    form.
\end{proof}

\subsection{Dictionary Order}

\begin{definition}
    Given two ordered types $\A$ and $\B$, we can define an order on $\A \times
    \B$ called the dictionary order, defined by
    \[
        (a_1, b_1) \leq (a_2, b_2) = b_1 < b_2 \vee (a_1 \leq a_2 \wedge b_1 =
        b_2).
    \]
\end{definition}

\begin{theorem}
    If $\A$ and $\B$ are partially ordered, then $\A \times \B$ is partially
    ordered.
\end{theorem}
\begin{proof}
    \textit{Reflexivity.}  Because $a \leq a$ and $b = b$, we have $(a, b) \leq
    (a, b)$.

    \textit{Transitivity.}  Let $(a_1, b_1)$, $(a_2, b_2)$, and $(a_3, b_3)$ be
    values of $\A \times \B$ with $(a_1, b_1) \leq (a_2, b_2)$ and $(a_2, b_2)
    \leq (a_3, b_3)$.  We have four cases.
    \begin{case}
        $b_1 < b_2$, $b_2 < b_3$.  We then have $b_1 < b_3$ by transitivity.
    \end{case}
    \begin{case}
        $b_1 < b_2$, $a_2 \leq a_3 \wedge b_2 = b_3$.  Because $b_2 = b_3$, we
        have $b_1 < b_3$.
    \end{case}
    \begin{case}
        $a_1 \leq a_2 \wedge b_1 = b_2$, $b_2 < b_3$.  Because $b_1 = b_2$, we
        have $b_1 < b_3$.
    \end{case}
    \begin{case}
        $a_1 \leq a_2 \wedge b_1 = b_2$, $a_2 \leq a_3 \wedge b_2 = b_3$.  We
        now $a_1 \leq a_3$ by transitivity.  We also have $b_1 = b_3$, so $(a_1,
        a_3) \leq (b_1, b_3)$.
    \end{case}

    \setcounter{case}{0}
    \textit{Antisymmetry.}  Let $(a_1, b_1)$ and $(a_2, b_2$ be values of $\A
    \times \B$ such that $(a_1, b_1) \leq (a_2, b_2)$ and $(a_2, b_2) \leq (a_1,
    b_1)$.  We have four cases.
    \begin{case}
        $b_1 < b_2$, $b_2 < b_1$.  This case is impossible.
    \end{case}
    \begin{case}
        $b_1 < b_2$, $a_1 \leq a_2 \wedge b_1 = b_2$.  Because $b_1 = b_2$, we
        have $b_1 < b_1$, so this case is also impossible.
    \end{case}
    \begin{case}
        $a_1 \leq a_2 \wedge b_1 = b_2$, $b_1 < b_2$.  Because $b_1 = b_2$, we
        have $b_1 < b_1$, so this case is also impossible.
    \end{case}
    \begin{case}
        $a_1 \leq a_2 \wedge b_1 = b_2$, $a_1 \leq a_2 \wedge b_1 = b_2$.  We
        have $a_1 \leq a_2$ and $a_2 \leq a_1$, so $a_1 = a_2$.  Thus, we have
        $a_1 = a_2$ and $b_1 = b_2$, so $(a_1, b_1) = (a_2, b_2)$.
    \end{case}
\end{proof}

\begin{theorem}
    If $\A$ and $\B$ are totally ordered, then $\A \times \B$ is totally
    ordered.
\end{theorem}
\begin{proof}
    We only need to check that $\A \times \B$ is connex.  Let $(a_1, b_1)$ and
    $(a_2, b_2)$ be values of $\A \times \B$.  If $b_1 < b_2$, then $(a_1, b_1)
    \leq (a_2, b_2)$, and if $b_2 < b_1$, then $(a_2, b_2) \leq (a_1, b_1)$.
    Thus, $b_1 = b_2$ is the only case left to check.  Now if $a_1 \leq a_2$, we
    would have $(a_1, b_1) \leq (a_2, b_2)$, and if $a_2 \leq a_1$ we would have
    $(a_2, b_2) \leq (a_1, b_1)$.  All possible cases have now been checked, so
    $\A \times \B$ is connex.
\end{proof}

\begin{theorem}
    If $\A$ and $\B$ are well ordered, then $\A \times \B$ is well ordered.
\end{theorem}
\begin{proof}
    Let $S$ be a nonempty set on $\A \times \B$.  Then there exists some $(a, b)
    \in S$.  Now define a set $S_B = \{y : \B \mid \exists x, (x, y) \in
    S\}$.  Then $b \in S_B$ because $(a, b) \in S$.  Thus, $S_B$ is nonempty, so
    there exists some minimum value $b' \in S_B$, along with a corresponding
    $a_b$ such that $(a_b, b') \in S$.  Now define a set $S_A = \{x : \A \mid
    (x, b') \in S\}$.  Then $a_b \in S_A$ because $(a_b, b') \in S$, so $S_A$ is
    nonempty.  Thus, there exists some minimum value $a' \in S_A$.

    We will now prove that $(a', b')$ is the minimum of $S$.  First, $(a', b')
    \in S$ by definition.  Now let $(x, y)$ be another element of $S$.  The
    inequality $y < b'$ would contradict the minimality of $b'$, so we must have
    $b' \leq y$.  Also when $b' < y$ we have $(a', b') \leq (x, y)$, so we're
    done in this case.  The only case left is $b' = y$.  Now $x < a'$ would
    contradict the minimality of $a'$, so we must have $a' \leq x$.  We now have
    $a' \leq x$ and $b' = y$, so $(a', b') \leq (x, y)$ as required.  Thus,
    $(a', b')$ is the minimum element of $S$.
\end{proof}


\end{document}
