\documentclass[../../math.tex]{subfiles}
\externaldocument{../../math.tex}
\externaldocument{../basics/set}

\begin{document}

\setcounter{chapter}{12}

\chapter{Basic Ring Theory}

While groups have less structure than rings making studying groups before rings
seem natural, some of the content in this chapter will be useful in the chapter
about basic group theory, so we will look at ring theory first.

\section{Ideals}

\begin{definition}
    Let $\U$ be a ring.  Then a set $I : \U \to \Prop$ is called an ideal if all
    of the following are true:
    \begin{itemize}
        \item $I$ is not empty.
        \item For all $a \in I$ and $b \in I$, we have $a + b \in I$,
        \item For all $a \in I$ and $b : \U$, $ab \in I$.
        \item For all $a : \U$ and $b \in I$, $ab \in I$.
    \end{itemize}
    Note that in a commutative ring, only one of the laft two conditions needs
    to be checked.
\end{definition}

Throughout this section, let $I$ be an ideal in some ring $\U$.

\begin{theorem} \label{ideal_neg}
    If $a \in I$, then $-a \in I$.
\end{theorem}
\begin{proof}
    Since $a \in I$, we have $(-1)a \in I$, so $-a \in I$.
\end{proof}

\begin{theorem} \label{ideal_zero}
    $0 \in I$.
\end{theorem}
\begin{proof}
    By definition, $I$ is not empty.  Let $a \in I$.  Then $a0 = 0 \in I$.
\end{proof}

\begin{definition}
    Define a relation $\sim : \U \to \U \to \Prop$ where $a \sim b$ is defined
    to mean $a - b \in I$.
\end{definition}

\begin{lemma}
    $\sim$ is an equivalence relation.
\end{lemma}
\begin{proof}
    \textit{Reflexivity.} For all $x : \U$, we have $x - x = 0 \in I$.

    \textit{Symmetry.} For all $a$ and $b$ in $\U$, if $a - b \in I$, by Theorem
    \ref{ideal_neg} we have $b - a \in I$.

    \textit{Transitivity.}  If $a - b \in I$ and $b - c \in I$, adding the two
    together we get $a - b + b - c = a - c \in I$.
\end{proof}

\begin{lemma}
    Addition is well-defined under the equivalence relation $\sim$.
\end{lemma}
\begin{proof}
    Let $a \sim b$ and $c \sim d$.  We must prove that $a + c \sim b + d$.  From
    $a \sim b$ and $c \sim d$, we have $a - b \in I$ and $c - d \in I$.  Adding
    them together, we get $a + c - (b + d) \in I$, showing that $a + c \sim b +
    d$.
\end{proof}

\begin{instance}
    Define addition in $\U/{\sim}$ to be addition in $\U$ brought to the
    quotient, which is well-defined by the previous lemma.
\end{instance}

\begin{instance}
    Define zero in $\U/{\sim}$ to be $[0]$, the equivalence class of $0$ in
    $\U$.
\end{instance}

\begin{lemma}
    Negation is well-defined under the equivalence relation $\sim$.
\end{lemma}
\begin{proof}
    Let $a \sim b$.  Then $a - b \in I$, meaning that $-a - -b \in I$, showing
    that $-a \sim -b$.
\end{proof}

\begin{instance}
    Define negation in $\U/{\sim}$ to be negation in $\U$ brought to the
    quotient, which is well-defined by the previous lemma.
\end{instance}

\begin{lemma}
    Multiplication is well-defined under the equivalence relation $\sim$.
\end{lemma}
\begin{proof}
    Let $a \sim b$ and $c \sim d$.  We must prove that $ac \sim bd$.  From $a
    \sim b$ and $c \sim d$, we have $a - b \in I$ and $c - d \in I$.  Then $(a -
    b)c = ac - bc \in I$, and $b(c - d) bc - bd \in I$.  Adding them together,
    we get $ac - bc + bc - bd = ac - bd \in I$, so $ac \sim bd$.
\end{proof}

\begin{instance}
    Define multiplication in $\U/{\sim}$ to be multiplication in $\U$ brought to
    the quotient, which is well-defined by the previous lemma.
\end{instance}

\begin{instance}
    Define one in $\U/{\sim}$ to be $[1]$, the equivalence class of $1$ in
    $\U$.
\end{instance}

Trivially, all of these operations have the same algebraic properties that the
base ring does.  Thus, $\U/{\sim}$ is a ring, and if $\U$ is a commutative ring,
then $\U/{\sim}$ is as well.  Furthermore, the function $\phi : \U \to
\U/{\sim}$ given by $\phi(x) = [x]$ is a ring homomorphism.  This leads to the
following definition:

\begin{definition}
    Given a ring $\U$ and an ideal $I$, we define $\U/I$ to be the ring
    $\U/{\sim}$ developed above.  Unlike the previous quotients we have used
    which act on types, this quotient acts on rings.  It takes in a ring and an
    ideal and produces another ring, or it takes in a commutative ring and an
    ideal and produces another commutative ring.  We will use the letter $\phi$
    to denote the canonical homorphism from $\U$ to $\U/I$ given by $\phi(x) =
    [x]$.
\end{definition}

\begin{theorem} \label{to_qring_eq}
    For all $x$ and $y$ in $\U$, if $x - y \in I$, then $\phi(x) = \phi(y)$.
\end{theorem}
\begin{proof}
    This is true by definition.
\end{proof}

\begin{theorem} \label{to_qring_zero}
    For all $x : \U$, if $x \in I$, then $0 = \phi(x)$.
\end{theorem}
\begin{proof}
    Since $x \in I$, we have $x - 0 \in I$, showing that $\phi(x) = \phi(0) =
    0$.
\end{proof}

\begin{theorem} \label{qring_f_ex}
    For all rings $\V$ and ring homomorphisms $f : \U \to \V$, if $x \in I$
    implies that $0 = f(x)$ for all $x$, then there exists a ring homorphism $g
    : \U/I \to V$ such that for all $x$, $f(x) = g(\phi(x))$.
\end{theorem}
\begin{proof}
    First, given $a$ and $b$ such that $a \sim b$, we have $a - b \in I$, which
    implies that $0 = f(a - b) = f(a) - f(b)$, so $f(a) = f(b)$.  This means
    that $f$ is well-defined under the equivalence relation $\sim$, so by
    Theorem \ref{unary_op_ex} we can get a function $g : \U/I \to \V$.  This
    function is easily a ring homorphism from $f$ being a ring homorphism, and
    it satisfies $f(x) = g(\phi(x))$ by definition.
\end{proof}

Whenever possible, the previous theorem will be used when working with ring
quotients instead of the general theory of quotients developed previously.

\begin{lemma}
    Let $S$ be a set in $\U$.  Define a function $f : (\U \times S \times \U)
    \to \U$ given by $f(a, b, c) = abc$.  Define a set $I$ by saying that $x \in
    I$ if there exists an unordered list $l : \L_u(\U \times S \times \U)$ such
    that $x = \sum f(l)$.  Then $I$ is an ideal.
\end{lemma}
\begin{proof}
    \textit{$I$ is nonempty.} We have $0 = \sum f([])$, so $0 \in I$.

    \textit{If $a \in I$ and $b \in I$, then $a + b \in I$.}  We have lists $x$
    and $y$ such that $a = \sum f(x)$ and $b = \sum f(y)$.  Then $a + b = \sum
    f(x) + \sum f(x) = \sum f(x + y)$, so $a + b \in I$.

    \textit{If $b \in I$, then $ab \in I$.}  We have a list $l$ such that $b =
    \sum f(l)$.  Let $g : \U \times S \times \U \to \U \times S \times \U$ be
    given by $g(x, y, z) = (ax, y, z)$.  Then $ab = a\sum f(b) = \sum af(b) =
    f(g(b))$, showing that $ab \in I$.

    \textit{If $a \in I$, then $ab \in I$.}  We have a list $l$ such that $a =
    \sum f(l)$.  Let $g : \U \times S \times \U \to \U \times S \times \U$ be
    given by $g(x, y, z) = (x, y, zb)$.  Then $ab = \left( \sum f(a) \right) b =
    \sum f(a)b = f(g(a))$, showing that $ab \in I$.
\end{proof}

\begin{definition}
    Given a set $S$, let the ideal $I$ given by the previous lemma be called the
    ideal generated by $S$.  Note that if $\U$ is a commutative ring, the ideal
    can be simplified to only having factors on one side.
\end{definition}

\end{document}
