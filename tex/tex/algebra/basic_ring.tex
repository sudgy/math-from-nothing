\documentclass[../../math.tex]{subfiles}
\externaldocument{../../math.tex}
\externaldocument{../basics/set}

\begin{document}

\setcounter{chapter}{12}

\chapter{Basic Ring Theory}

While groups have less structure than rings making studying groups before rings
seem natural, some of the content in this chapter will be useful in the chapter
about basic group theory, so we will look at ring theory first.

\section{Ideals}

\begin{definition}
    Let $\U$ be a ring.  Then a set $I : \U \to \Prop$ is called an ideal if all
    of the following are true:
    \begin{itemize}
        \item $I$ is not empty.
        \item For all $a \in I$ and $b \in I$, we have $a + b \in I$,
        \item For all $a \in I$ and $b : \U$, $ab \in I$.
        \item For all $a : \U$ and $b \in I$, $ab \in I$.
    \end{itemize}
    Note that in a commutative ring, only one of the laft two conditions needs
    to be checked.
\end{definition}

Throughout this section, let $I$ be an ideal in some ring $\U$.

\begin{theorem} \label{ideal_neg}
    If $a \in I$, then $-a \in I$.
\end{theorem}
\begin{proof}
    Since $a \in I$, we have $(-1)a \in I$, so $-a \in I$.
\end{proof}

\begin{theorem} \label{ideal_zero}
    $0 \in I$.
\end{theorem}
\begin{proof}
    By definition, $I$ is not empty.  Let $a \in I$.  Then $a0 = 0 \in I$.
\end{proof}

\begin{definition}
    Define a relation $\sim : \U \to \U \to \Prop$ where $a \sim b$ is defined
    to mean $a - b \in I$.
\end{definition}

\begin{lemma}
    $\sim$ is an equivalence relation.
\end{lemma}
\begin{proof}
    \textit{Reflexivity.} For all $x : \U$, we have $x - x = 0 \in I$.

    \textit{Symmetry.} For all $a$ and $b$ in $\U$, if $a - b \in I$, by Theorem
    \ref{ideal_neg} we have $b - a \in I$.

    \textit{Transitivity.}  If $a - b \in I$ and $b - c \in I$, adding the two
    together we get $a - b + b - c = a - c \in I$.
\end{proof}

\begin{lemma}
    Addition is well-defined under the equivalence relation $\sim$.
\end{lemma}
\begin{proof}
    Let $a \sim b$ and $c \sim d$.  We must prove that $a + c \sim b + d$.  From
    $a \sim b$ and $c \sim d$, we have $a - b \in I$ and $c - d \in I$.  Adding
    them together, we get $a + c - (b + d) \in I$, showing that $a + c \sim b +
    d$.
\end{proof}

\begin{instance}
    Define addition in $\U/{\sim}$ to be addition in $\U$ brought to the
    quotient, which is well-defined by the previous lemma.
\end{instance}

\begin{instance}
    Define zero in $\U/{\sim}$ to be $[0]$, the equivalence class of $0$ in
    $\U$.
\end{instance}

\begin{lemma}
    Negation is well-defined under the equivalence relation $\sim$.
\end{lemma}
\begin{proof}
    Let $a \sim b$.  Then $a - b \in I$, meaning that $-a - -b \in I$, showing
    that $-a \sim -b$.
\end{proof}

\begin{instance}
    Define negation in $\U/{\sim}$ to be negation in $\U$ brought to the
    quotient, which is well-defined by the previous lemma.
\end{instance}

\begin{lemma}
    Multiplication is well-defined under the equivalence relation $\sim$.
\end{lemma}
\begin{proof}
    Let $a \sim b$ and $c \sim d$.  We must prove that $ac \sim bd$.  From $a
    \sim b$ and $c \sim d$, we have $a - b \in I$ and $c - d \in I$.  Then $(a -
    b)c = ac - bc \in I$, and $b(c - d) bc - bd \in I$.  Adding them together,
    we get $ac - bc + bc - bd = ac - bd \in I$, so $ac \sim bd$.
\end{proof}

\begin{instance}
    Define multiplication in $\U/{\sim}$ to be multiplication in $\U$ brought to
    the quotient, which is well-defined by the previous lemma.
\end{instance}

\begin{instance}
    Define one in $\U/{\sim}$ to be $[1]$, the equivalence class of $1$ in
    $\U$.
\end{instance}

Trivially, all of these operations have the same algebraic properties that the
base ring does.  Thus, $\U/{\sim}$ is a ring, and if $\U$ is a commutative ring,
then $\U/{\sim}$ is as well.  Furthermore, the function $\phi : \U \to
\U/{\sim}$ given by $\phi(x) = [x]$ is a ring homomorphism.  This leads to the
following definition:

\begin{definition}
    Given a ring $\U$ and an ideal $I$, we define $\U/I$ to be the ring
    $\U/{\sim}$ developed above.  Unlike the previous quotients we have used
    which act on types, this quotient acts on rings.  It takes in a ring and an
    ideal and produces another ring, or it takes in a commutative ring and an
    ideal and produces another commutative ring.  We will use the letter $\phi$
    to denote the canonical homorphism from $\U$ to $\U/I$ given by $\phi(x) =
    [x]$.
\end{definition}

\begin{theorem} \label{to_qring_eq}
    For all $x$ and $y$ in $\U$, if $x - y \in I$, then $\phi(x) = \phi(y)$.
\end{theorem}
\begin{proof}
    This is true by definition.
\end{proof}

\begin{theorem} \label{to_qring_zero}
    For all $x : \U$, if $x \in I$, then $0 = \phi(x)$.
\end{theorem}
\begin{proof}
    Since $x \in I$, we have $x - 0 \in I$, showing that $\phi(x) = \phi(0) =
    0$.
\end{proof}

\begin{theorem} \label{qring_f_ex}
    For all rings $\V$ and ring homomorphisms $f : \U \to \V$, if $x \in I$
    implies that $0 = f(x)$ for all $x$, then there exists a ring homorphism $g
    : \U/I \to V$ such that for all $x$, $f(x) = g(\phi(x))$.
\end{theorem}
\begin{proof}
    First, given $a$ and $b$ such that $a \sim b$, we have $a - b \in I$, which
    implies that $0 = f(a - b) = f(a) - f(b)$, so $f(a) = f(b)$.  This means
    that $f$ is well-defined under the equivalence relation $\sim$, so by
    Theorem \ref{unary_op_ex} we can get a function $g : \U/I \to \V$.  This
    function is easily a ring homorphism from $f$ being a ring homorphism, and
    it satisfies $f(x) = g(\phi(x))$ by definition.
\end{proof}

Whenever possible, the previous theorem will be used when working with ring
quotients instead of the general theory of quotients developed previously.

\begin{lemma}
    Let $S$ be a set in $\U$.  Define a function $f : (\U \times S \times \U)
    \to \U$ given by $f(a, b, c) = abc$.  Define a set $I$ by saying that $x \in
    I$ if there exists an unordered list $l : \L_u(\U \times S \times \U)$ such
    that $x = \sum f(l)$.  Then $I$ is an ideal.
\end{lemma}
\begin{proof}
    \textit{$I$ is nonempty.} We have $0 = \sum f([])$, so $0 \in I$.

    \textit{If $a \in I$ and $b \in I$, then $a + b \in I$.}  We have lists $x$
    and $y$ such that $a = \sum f(x)$ and $b = \sum f(y)$.  Then $a + b = \sum
    f(x) + \sum f(x) = \sum f(x + y)$, so $a + b \in I$.

    \textit{If $b \in I$, then $ab \in I$.}  We have a list $l$ such that $b =
    \sum f(l)$.  Let $g : \U \times S \times \U \to \U \times S \times \U$ be
    given by $g(x, y, z) = (ax, y, z)$.  Then $ab = a\sum f(b) = \sum af(b) =
    f(g(b))$, showing that $ab \in I$.

    \textit{If $a \in I$, then $ab \in I$.}  We have a list $l$ such that $a =
    \sum f(l)$.  Let $g : \U \times S \times \U \to \U \times S \times \U$ be
    given by $g(x, y, z) = (x, y, zb)$.  Then $ab = \left( \sum f(a) \right) b =
    \sum f(a)b = f(g(a))$, showing that $ab \in I$.
\end{proof}

\begin{definition}
    Given a set $S$, let the ideal $I$ given by the previous lemma be called the
    ideal generated by $S$.  Note that if $\U$ is a commutative ring, the ideal
    can be simplified to only having factors on one side.
\end{definition}

\section{Divisibility}

Throughout this section, let $\U$ be an integral domain, or some type that has
the requisite typeclasses for the definitions.

\begin{definition}
    Define a relation $\mathord{\mid} : \U \to \U \to \Prop$ where $a \mid b$
    means that there exists a $c$ with $ca = b$.
\end{definition}

\begin{definition}
    We say that a value $a : \U$ is a unit if $a \mid 1$.
\end{definition}

\begin{definition}
    We say that two values $a$ and $b$ in $\U$ associate if $a \mid b$ and $b
    \mid a$.
\end{definition}

\begin{definition}
    We say that a value $p : \U$ is irreducible if it is not zero, not a unit,
    and if for all non-unit values $a$ and $b$, $p \neq ab$.
\end{definition}

\begin{definition}
    We say that a value $p : \U$ is prime if it is not zero, not a unit, and if
    for all $a$ and $b$ such that $p \mid ab$, then $p \mid a$ or $p \mid b$.
\end{definition}

\begin{definition}
    We say that a value $a : \U$ is even if $2 \mid a$, and it is odd otherwise.
\end{definition}

\begin{instance}
    $\mid$ is reflexive.
\end{instance}
\begin{proof}
    We have $1a = a$, so $a \mid a$.
\end{proof}

\begin{instance}
    $\mid$ is transitive.
\end{instance}
\begin{proof}
    Let $a \mid b$ and $b \mid c$.  Then we have values $d$ and $e$ such that
    $da = b$ and $eb = c$.  Then $eda = eb = c$, showing that $a \mid c$.
\end{proof}

\begin{theorem} \label{one_divides}
    For all $a$, we have $1 \mid a$.
\end{theorem}
\begin{proof}
    We have $a1 = a$.
\end{proof}

\begin{theorem} \label{divides_zero}
    For all $a$, we have $a \mid 0$.
\end{theorem}
\begin{proof}
    We have $0a = 0$.
\end{proof}

\begin{theorem} \label{divides_neg}
    For all $a$ and $b$, if $a \mid b$, then $a \mid -b$.
\end{theorem}
\begin{proof}
    Since $a \mid b$, we have a $c$ such that $ca = b$.  Then $(-c)a = -ca =
    -b$, showing that $a \mid -b$.
\end{proof}

\begin{theorem} \label{plus_stays_divides}
    For all $p$, $a$, and $b$, if $p \mid a$ and $p \mid b$, then $p \mid a +
    b$.
\end{theorem}
\begin{proof}
    We have values $c$ and $d$ such that $cp = a$ and $dp = b$.  Then $(c + d)p
    = cp + dp = a + b$, showing that $p \mid a + b$.
\end{proof}

\begin{theorem} \label{plus_divides_back}
    For all $p$, $a$, and $b$, if $p \mid a$ and $p \mid a + b$, then $p \mid
    b$.
\end{theorem}
\begin{proof}
    We have values $c$ and $d$ such that $cp = a$ and $dp = a + b$.  Then $(-c +
    d)p = -cp + dp = -a + a + b = b$, showing that $p \mid b$.
\end{proof}

\begin{theorem} \label{mult_factors_extend}
    For all $p$, $a$, and $b$, if $p \mid a$, then $p \mid ab$.
\end{theorem}
\begin{proof}
    We have a $c$ such that $cp = a$.  Then $(bc)p = ba$, showing that $p \mid
    ab$.
\end{proof}

\begin{theorem} \label{mult_div_lself}
    For all $a$ and $b$, we have $a \mid ab$.
\end{theorem}
\begin{proof}
    We directly have $ba = ab$.
\end{proof}

\begin{theorem} \label{mult_div_rself}
    For all $a$ and $b$, we have $a \mid ba$.
\end{theorem}
\begin{proof}
    We directly have $ba = ba$.
\end{proof}

\begin{theorem} \label{div_rcancel}
    For all $a$, $b$, and $c$, if $0 \neq c$ and $ac \mid bc$, then $a \mid b$.
\end{theorem}
\begin{proof}
    We have some $x$ such that $xac = bc$.  Cancelling $c$, we get $xa = b$,
    showing that $a \mid b$.
\end{proof}

\begin{theorem} \label{div_lcancel}
    For all $a$, $b$, and $c$, if $0 \neq a$ and $ab \mid ac$, then $b \mid c$.
\end{theorem}
\begin{proof}
    Follows from commutativity and the previous theorem.
\end{proof}

\begin{theorem} \label{unit_div}
    For all $a$ and $b$, if $a$ is a unit, then $a \mid b$.
\end{theorem}
\begin{proof}
    Because $a$ is a unit, we have some $c$ such that $ca = 1$.  Then $(bc)a =
    b1 = b$, so $a \mid b$.
\end{proof}

\begin{theorem} \label{div_zero}
    For all $a$, if $0 \mid a$, then $0 = a$.
\end{theorem}
\begin{proof}
    Because $0 \mid a$, we have a $b$ such that $b0 = 0 = a$.
\end{proof}

\begin{theorem} \label{one_unit}
    $1$ is a unit.
\end{theorem}
\begin{proof}
    We have $(1)1 = 1$.
\end{proof}

\begin{theorem} \label{zero_not_unit}
    If $\U$ is not trivial, $0$ is not a unit.
\end{theorem}
\begin{proof}
    If $0$ was a unit, we would have some $a$ with $a0 = 0 = 1$, contradicting
    $\U$ being not trivial.
\end{proof}

\begin{theorem} \label{unit_mult}
    If $a$ and $b$ are units, then $ab$ is a unit.
\end{theorem}
\begin{proof}
    Because $a$ and $b$ are units, we have values $c$ and $d$ such that $ca = 1$
    and $db = 1$.  Then $(dc)(ab) = dcab = db = 1$, so $ab$ is a unit.
\end{proof}

\begin{theorem} \label{div_mult_unit}
    For all $a$ and $b$, if $0 \neq a$ and $ab \mid a$, then $b$ is a unit.
\end{theorem}
\begin{proof}
    Because $ab \mid a$, we have some $c$ such that $cab = a$.  Cancelling $a$,
    we get $cb = 1$, showing that $b$ is a unit.
\end{proof}

\begin{theorem} \label{prime_irreducible}
    All primes are irreducible.
\end{theorem}
\begin{proof}
    Let $p$ be prime, and let $a$ and $b$ be non-units.  For a contradiction,
    assume that $p = ab$.  Because primes are nonzero, $0 \neq p = ab$, so $a$
    and $b$ are nonzero as well.  Now $p \mid p = ab$, so by the definition of
    being prime, either $p \mid a$ or $p \mid b$.  This means that $ab \mid a$
    or $ab \mid b$.  In each case, either $a$ or $b$ is a unit by Theorem
    \ref{div_mult_unit}, which is a contradiction.  Thus we must have $p \neq
    ab$, showing that $p$ is irreducible.
\end{proof}

\begin{instance}
    Values associating with each other is reflexive.
\end{instance}
\begin{proof}
    This follows directly from $\mid$ being reflexive.
\end{proof}

\begin{instance}
    Values associating with each other is symmetric.
\end{instance}
\begin{proof}
    If $a$ is associated with $b$, then $a \mid b$ and $b \mid a$, showing that
    $b$ is associated with $a$.
\end{proof}

\begin{instance}
    Values associating with each other is transitive.
\end{instance}
\begin{proof}
    Let $a$ be associated with $b$, and $b$ with $c$.  Then we have $a \mid b$,
    $b \mid a$, $b \mid c$, and $c \mid b$.  By the transitivity of $\mid$, we
    have $a \mid c$ and $c \mid a$, showing that $a$ is associated with $c$.
\end{proof}

\begin{theorem} \label{unit_associates}
    All units associate with each other.
\end{theorem}
\begin{proof}
    This follows directly from Theorem \ref{unit_div}.
\end{proof}

\begin{theorem} \label{associates_zero}
    If $a$ is associated with $0$, then $0 = a$.
\end{theorem}
\begin{proof}
    Because $a$ is associated with $0$, we have $0 \mid a$, so the result
    follows from Theorem \ref{div_zero}.
\end{proof}

\begin{theorem} \label{associates_unit}
    If $a$ is associated with $b$, there exists a unit $c$ such that $ca = b$.
\end{theorem}
\begin{proof}
    We have $a \mid b$ and $b \mid a$, so there exist values $c$ and $d$ such
    that $ca = b$ and $db = a$.  If $0 = b$, we have $a = 0$ as well, so $(1)0 =
    0$ works since $1$ is a unit.  If $0 \neq b$, then we already have a $c$
    such that $ca = b$, and all that remains is to prove that $c$ is a unit.
    From $ca = b$ and $db = a$, we have $cdb = b$, and cancelling $b$ we get $cd
    = 1$, showing that $c$ is a unit.
\end{proof}

\section{GCDs}

\begin{definition}
    Let $a$ and $b$ be values in an integral domain $\U$.  Then $d$ is a common
    divisor of $a$ and $b$ if $d \mid a$ and $d \mid b$.
\end{definition}

\begin{definition}
    Let $a$ and $b$ be values in an integral domain $\U$.  Then $d$ is the
    greatest common divisor of $a$ and $b$ if it is a common divisor, and if for
    all other common divisors $d'$, we have $d' \mid d$.
\end{definition}

\begin{class}
    Let $\U$ be an integral domain.  Then we call $\U$ a GCD Domain if there
    exists a function $\gcd : \U \to \U \to \U$ such that for all $a$ and $b$
    with at least one of them nonzero, $\gcd(a, b)$ is a greatest common divisor
    of $a$ and $b$.
\end{class}

\begin{theorem} \label{gcd_associates}
    For all $a$ and $b$ with two greatest common divisors $d_1$ and $d_2$, $d_1$
    and $d_2$ are associated.
\end{theorem}
\begin{proof}
    Since $d_1$ and $d_2$ are both greatest common divisors, we have $d_1 \mid
    d_2$ and $d_2 \mid d_1$, so $d_1$ and $d_2$ are associated.
\end{proof}

\begin{theorem} \label{gcd_div_comm}
    In a GCD domain, for all $a$ and $b$ with at least one of them nonzero,
    $\gcd(a, b) \mid \gcd(b, a)$.
\end{theorem}
\begin{proof}
    The definition of common divisor is symmetric, so $\gcd(a, b)$ is a common
    divisor of $b$ and $a$, meaning that $\gcd(a, b) \div \gcd(b, a)$.
\end{proof}

\begin{theorem} \label{gcd_comm}
    In a GCD domain, for all $a$ and $b$ with at least one of them nonzero,
    $\gcd(a, b)$ is associated with $\gcd(b, a)$.
\end{theorem}
\begin{proof}
    Just apply the previous theorem forwards and backwards.
\end{proof}

\begin{theorem} \label{irreducible_prime}
    In a GCD Domain, all irreducible elements are prime.
\end{theorem}
\begin{proof}
    Let $p$ be irreducible, and let $p \mid ab$.  We must prove that either $p
    \mid a$ or $p \mid b$.  If $b = 0$, then $p \mid b$, so consider the case
    when $b \neq 0$.  If $ab = 0$, then $a = 0$ since $b \neq 0$, meaning that
    $p \mid a$, so consider the case when $ab \neq 0$.  Let $d = \gcd(pb, ab)$.
    Then $d \neq 0$, because if $d = 0$, we would have $0 \mid ab$, implying
    that $0 = ab$, contradicting $ab \neq 0$.  By Theorems \ref{mult_div_lself}
    and \ref{mult_div_rself}, we have $p \mid pb$, $b \mid pb$, and $b \mid ab$.
    Because of this and $p \mid ab$, both $p$ and $b$ are common divisors of
    $ab$ and $pb$, so $p \mid d$ and $b \mid d$.  Thus, we have values $u$ and
    $v$ such that $up = d$ and $vb = d$.  There will now be two cases: when $v$
    is a unit, and when $v$ is not a unit.

    When $v$ is a unit, we have a $v'$ such that $v'v = 1$.  We have $up = vb$,
    so $v'up = v'vb = b$, showing that $p \mid b$.

    When $v$ is not a unit, Because $d \mid pb$, we have $vb \mid pb$, so by
    Theorem \ref{div_rcancel}, we have $v \mid p$.  Thus, we have a $c$ such
    that $cv = p$.  Because $v$ is not a unit and $p$ is irreducible, $c$ must
    be a unit.  Then we have a $c'$ such that $c'c = 1$.  Since $cv = p$, we
    have $c'cv = c'p$, so $v = c'p$, showing that $p \mid v$.  Now from $d \mid
    ab$, we have $vb \mid ab$, and by Theorem \ref{div_rcancel} we have $v \mid
    a$.  Since $p \mid v$ and $v \mid a$, we have $p \mid a$.
\end{proof}

\end{document}
