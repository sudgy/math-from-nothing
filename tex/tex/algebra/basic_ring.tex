\documentclass[../../math.tex]{subfiles}
\externaldocument{../../math.tex}
\externaldocument{../basics/set}

\begin{document}

\setcounter{chapter}{12}

\chapter{Basic Ring Theory}

While groups have less structure than rings making studying groups before rings
seem natural, some of the content in this chapter will be useful in the chapter
about basic group theory, so we will look at ring theory first.

\section{Ideals}

\begin{definition}
    Let $\U$ be a ring.  Then a set $I : \U \to \Prop$ is called an ideal if all
    of the following are true:
    \begin{itemize}
        \item $I$ is not empty.
        \item For all $a \in I$ and $b \in I$, we have $a + b \in I$,
        \item For all $a \in I$ and $b : \U$, $ab \in I$.
        \item For all $a : \U$ and $b \in I$, $ab \in I$.
    \end{itemize}
    Note that in a commutative ring, only one of the laft two conditions needs
    to be checked.
\end{definition}

Throughout this section, let $I$ be an ideal in some ring $\U$.

\begin{theorem} \label{ideal_neg}
    If $a \in I$, then $-a \in I$.
\end{theorem}
\begin{proof}
    Since $a \in I$, we have $(-1)a \in I$, so $-a \in I$.
\end{proof}

\begin{theorem} \label{ideal_zero}
    $0 \in I$.
\end{theorem}
\begin{proof}
    By definition, $I$ is not empty.  Let $a \in I$.  Then $a0 = 0 \in I$.
\end{proof}

\begin{definition}
    Define a relation $\sim : \U \to \U \to \Prop$ where $a \sim b$ is defined
    to mean $a - b \in I$.
\end{definition}

\begin{lemma}
    $\sim$ is an equivalence relation.
\end{lemma}
\begin{proof}
    \textit{Reflexivity.} For all $x : \U$, we have $x - x = 0 \in I$.

    \textit{Symmetry.} For all $a$ and $b$ in $\U$, if $a - b \in I$, by Theorem
    \ref{ideal_neg} we have $b - a \in I$.

    \textit{Transitivity.}  If $a - b \in I$ and $b - c \in I$, adding the two
    together we get $a - b + b - c = a - c \in I$.
\end{proof}

\begin{lemma}
    Addition is well-defined under the equivalence relation $\sim$.
\end{lemma}
\begin{proof}
    Let $a \sim b$ and $c \sim d$.  We must prove that $a + c \sim b + d$.  From
    $a \sim b$ and $c \sim d$, we have $a - b \in I$ and $c - d \in I$.  Adding
    them together, we get $a + c - (b + d) \in I$, showing that $a + c \sim b +
    d$.
\end{proof}

\begin{instance}
    Define addition in $\U/{\sim}$ to be addition in $\U$ brought to the
    quotient, which is well-defined by the previous lemma.
\end{instance}

\begin{instance}
    Define zero in $\U/{\sim}$ to be $[0]$, the equivalence class of $0$ in
    $\U$.
\end{instance}

\begin{lemma}
    Negation is well-defined under the equivalence relation $\sim$.
\end{lemma}
\begin{proof}
    Let $a \sim b$.  Then $a - b \in I$, meaning that $-a - -b \in I$, showing
    that $-a \sim -b$.
\end{proof}

\begin{instance}
    Define negation in $\U/{\sim}$ to be negation in $\U$ brought to the
    quotient, which is well-defined by the previous lemma.
\end{instance}

\begin{lemma}
    Multiplication is well-defined under the equivalence relation $\sim$.
\end{lemma}
\begin{proof}
    Let $a \sim b$ and $c \sim d$.  We must prove that $ac \sim bd$.  From $a
    \sim b$ and $c \sim d$, we have $a - b \in I$ and $c - d \in I$.  Then $(a -
    b)c = ac - bc \in I$, and $b(c - d) bc - bd \in I$.  Adding them together,
    we get $ac - bc + bc - bd = ac - bd \in I$, so $ac \sim bd$.
\end{proof}

\begin{instance}
    Define multiplication in $\U/{\sim}$ to be multiplication in $\U$ brought to
    the quotient, which is well-defined by the previous lemma.
\end{instance}

\begin{instance}
    Define one in $\U/{\sim}$ to be $[1]$, the equivalence class of $1$ in
    $\U$.
\end{instance}

Trivially, all of these operations have the same algebraic properties that the
base ring does.  Thus, $\U/{\sim}$ is a ring, and if $\U$ is a commutative ring,
then $\U/{\sim}$ is as well.  Furthermore, the function $\phi : \U \to
\U/{\sim}$ given by $\phi(x) = [x]$ is a ring homomorphism.  This leads to the
following definition:

\begin{definition}
    Given a ring $\U$ and an ideal $I$, we define $\U/I$ to be the ring
    $\U/{\sim}$ developed above.  Unlike the previous quotients we have used
    which act on types, this quotient acts on rings.  It takes in a ring and an
    ideal and produces another ring, or it takes in a commutative ring and an
    ideal and produces another commutative ring.  We will use the letter $\phi$
    to denote the canonical homorphism from $\U$ to $\U/I$ given by $\phi(x) =
    [x]$.
\end{definition}

\begin{theorem} \label{to_qring_eq}
    For all $x$ and $y$ in $\U$, if $x - y \in I$, then $\phi(x) = \phi(y)$.
\end{theorem}
\begin{proof}
    This is true by definition.
\end{proof}

\begin{theorem} \label{to_qring_zero}
    For all $x : \U$, if $x \in I$, then $0 = \phi(x)$.
\end{theorem}
\begin{proof}
    Since $x \in I$, we have $x - 0 \in I$, showing that $\phi(x) = \phi(0) =
    0$.
\end{proof}

\begin{theorem} \label{qring_f_ex}
    For all rings $\V$ and ring homomorphisms $f : \U \to \V$, if $x \in I$
    implies that $0 = f(x)$ for all $x$, then there exists a ring homorphism $g
    : \U/I \to V$ such that for all $x$, $f(x) = g(\phi(x))$.
\end{theorem}
\begin{proof}
    First, given $a$ and $b$ such that $a \sim b$, we have $a - b \in I$, which
    implies that $0 = f(a - b) = f(a) - f(b)$, so $f(a) = f(b)$.  This means
    that $f$ is well-defined under the equivalence relation $\sim$, so by
    Theorem \ref{unary_op_ex} we can get a function $g : \U/I \to \V$.  This
    function is easily a ring homorphism from $f$ being a ring homorphism, and
    it satisfies $f(x) = g(\phi(x))$ by definition.
\end{proof}

Whenever possible, the previous theorem will be used when working with ring
quotients instead of the general theory of quotients developed previously.

\begin{lemma}
    Let $S$ be a set in $\U$.  Define a function $f : (\U \times S \times \U)
    \to \U$ given by $f(a, b, c) = abc$.  Define a set $I$ by saying that $x \in
    I$ if there exists an unordered list $l : \L_u(\U \times S \times \U)$ such
    that $x = \sum f(l)$.  Then $I$ is an ideal.
\end{lemma}
\begin{proof}
    \textit{$I$ is nonempty.} We have $0 = \sum f([])$, so $0 \in I$.

    \textit{If $a \in I$ and $b \in I$, then $a + b \in I$.}  We have lists $x$
    and $y$ such that $a = \sum f(x)$ and $b = \sum f(y)$.  Then $a + b = \sum
    f(x) + \sum f(x) = \sum f(x + y)$, so $a + b \in I$.

    \textit{If $b \in I$, then $ab \in I$.}  We have a list $l$ such that $b =
    \sum f(l)$.  Let $g : \U \times S \times \U \to \U \times S \times \U$ be
    given by $g(x, y, z) = (ax, y, z)$.  Then $ab = a\sum f(b) = \sum af(b) =
    f(g(b))$, showing that $ab \in I$.

    \textit{If $a \in I$, then $ab \in I$.}  We have a list $l$ such that $a =
    \sum f(l)$.  Let $g : \U \times S \times \U \to \U \times S \times \U$ be
    given by $g(x, y, z) = (x, y, zb)$.  Then $ab = \left( \sum f(a) \right) b =
    \sum f(a)b = f(g(a))$, showing that $ab \in I$.
\end{proof}

\begin{definition}
    Given a set $S$, let the ideal $I$ given by the previous lemma be called the
    ideal generated by $S$.  Note that if $\U$ is a commutative ring, the ideal
    can be simplified to only having factors on one side.
\end{definition}

\section{Divisibility}

Throughout this section, let $\U$ be an integral domain, or some type that has
the requisite typeclasses for the definitions.

\begin{definition}
    Define a relation $\mathord{\mid} : \U \to \U \to \Prop$ where $a \mid b$
    means that there exists a $c$ with $ca = b$.
\end{definition}

\begin{definition}
    We say that a value $a : \U$ is a unit if $a \mid 1$.
\end{definition}

\begin{definition}
    We say that two values $a$ and $b$ in $\U$ associate if $a \mid b$ and $b
    \mid a$.
\end{definition}

\begin{definition}
    We say that a value $p : \U$ is irreducible if it is not zero, not a unit,
    and if for all non-unit values $a$ and $b$, $p \neq ab$.
\end{definition}

\begin{definition}
    We say that a value $p : \U$ is prime if it is not zero, not a unit, and if
    for all $a$ and $b$ such that $p \mid ab$, then $p \mid a$ or $p \mid b$.
\end{definition}

\begin{definition}
    We say that a value $a : \U$ is even if $2 \mid a$, and it is odd otherwise.
\end{definition}

\begin{instance}
    $\mid$ is reflexive.
\end{instance}
\begin{proof}
    We have $1a = a$, so $a \mid a$.
\end{proof}

\begin{instance}
    $\mid$ is transitive.
\end{instance}
\begin{proof}
    Let $a \mid b$ and $b \mid c$.  Then we have values $d$ and $e$ such that
    $da = b$ and $eb = c$.  Then $eda = eb = c$, showing that $a \mid c$.
\end{proof}

\begin{theorem} \label{one_divides}
    For all $a$, we have $1 \mid a$.
\end{theorem}
\begin{proof}
    We have $a1 = a$.
\end{proof}

\begin{theorem} \label{divides_zero}
    For all $a$, we have $a \mid 0$.
\end{theorem}
\begin{proof}
    We have $0a = 0$.
\end{proof}

\begin{theorem} \label{divides_neg}
    For all $a$ and $b$, if $a \mid b$, then $a \mid -b$.
\end{theorem}
\begin{proof}
    Since $a \mid b$, we have a $c$ such that $ca = b$.  Then $(-c)a = -ca =
    -b$, showing that $a \mid -b$.
\end{proof}

\begin{theorem} \label{plus_stays_divides}
    For all $p$, $a$, and $b$, if $p \mid a$ and $p \mid b$, then $p \mid a +
    b$.
\end{theorem}
\begin{proof}
    We have values $c$ and $d$ such that $cp = a$ and $dp = b$.  Then $(c + d)p
    = cp + dp = a + b$, showing that $p \mid a + b$.
\end{proof}

\begin{theorem} \label{plus_divides_back}
    For all $p$, $a$, and $b$, if $p \mid a$ and $p \mid a + b$, then $p \mid
    b$.
\end{theorem}
\begin{proof}
    We have values $c$ and $d$ such that $cp = a$ and $dp = a + b$.  Then $(-c +
    d)p = -cp + dp = -a + a + b = b$, showing that $p \mid b$.
\end{proof}

\begin{theorem} \label{mult_factors_extend}
    For all $p$, $a$, and $b$, if $p \mid a$, then $p \mid ab$.
\end{theorem}
\begin{proof}
    We have a $c$ such that $cp = a$.  Then $(bc)p = ba$, showing that $p \mid
    ab$.
\end{proof}

\begin{theorem} \label{mult_div_lself}
    For all $a$ and $b$, we have $a \mid ab$.
\end{theorem}
\begin{proof}
    We directly have $ba = ab$.
\end{proof}

\begin{theorem} \label{mult_div_rself}
    For all $a$ and $b$, we have $a \mid ba$.
\end{theorem}
\begin{proof}
    We directly have $ba = ba$.
\end{proof}

\begin{theorem} \label{div_lmult} \label{div_rmult}
    For all $a$, $b$, and $c$, if $a \mid b$, then $ca \mid cb$.
\end{theorem}
\begin{proof}
    Because $a \mid b$, we have some $d$ with $da = b$.  Then $dca = cb$,
    showing that $ca \mid cb$.
\end{proof}

\begin{theorem} \label{div_rcancel} \label{div_lcancel}
    For all $a$, $b$, and $c$, if $0 \neq c$ and $ac \mid bc$, then $a \mid b$.
\end{theorem}
\begin{proof}
    We have some $x$ such that $xac = bc$.  Cancelling $c$, we get $xa = b$,
    showing that $a \mid b$.
\end{proof}

\begin{theorem} \label{unit_div}
    For all $a$ and $b$, if $a$ is a unit, then $a \mid b$.
\end{theorem}
\begin{proof}
    Because $a$ is a unit, we have some $c$ such that $ca = 1$.  Then $(bc)a =
    b1 = b$, so $a \mid b$.
\end{proof}

\begin{theorem} \label{div_zero}
    For all $a$, if $0 \mid a$, then $0 = a$.
\end{theorem}
\begin{proof}
    Because $0 \mid a$, we have a $b$ such that $b0 = 0 = a$.
\end{proof}

\begin{theorem} \label{one_unit}
    $1$ is a unit.
\end{theorem}
\begin{proof}
    We have $(1)1 = 1$.
\end{proof}

\begin{theorem} \label{zero_not_unit}
    If $\U$ is not trivial, $0$ is not a unit.
\end{theorem}
\begin{proof}
    If $0$ was a unit, we would have some $a$ with $a0 = 0 = 1$, contradicting
    $\U$ being not trivial.
\end{proof}

\begin{theorem} \label{unit_nz}
    If $\U$ is not trivial, then if $a$ is a unit, then $0 \neq a$.
\end{theorem}
\begin{proof}
    This is basically the contrapositive of the previous theorem.
\end{proof}

\begin{theorem} \label{unit_mult}
    If $a$ and $b$ are units, then $ab$ is a unit.
\end{theorem}
\begin{proof}
    Because $a$ and $b$ are units, we have values $c$ and $d$ such that $ca = 1$
    and $db = 1$.  Then $(dc)(ab) = dcab = db = 1$, so $ab$ is a unit.
\end{proof}

\begin{theorem} \label{lmult_unit} \label{rmult_unit}
    If $ab$ is a unit, then $a$ and $b$ are units.
\end{theorem}
\begin{proof}
    By commutativity, it suffices to prove that $a$ is a unit.  Because $ab$ is
    a unit, we have some $c$ with $cab = 1$.  Then $(cb)a = 1$, showing that $a$
    is a unit.
\end{proof}

\begin{theorem} \label{div_unit_mult}
    For all $a$, $b$, and $c$, if $a$ is a unit and $b \mid c$, then $ab \mid
    c$.
\end{theorem}
\begin{proof}
    Because $a$ is a unit, we have some $a'$ such that $a'a = 1$, and because $b
    \mid c$, we have some $d$ such that $db = c$.  Then $(da')(ab) = db = c$,
    showing that $ab \mid c$.
\end{proof}

\begin{theorem} \label{div_mult_unit}
    For all $a$ and $b$, if $0 \neq a$ and $ab \mid a$, then $b$ is a unit.
\end{theorem}
\begin{proof}
    Because $ab \mid a$, we have some $c$ such that $cab = a$.  Cancelling $a$,
    we get $cb = 1$, showing that $b$ is a unit.
\end{proof}

\begin{theorem} \label{unit_ex}
    If $a$ is a unit, then there exists a unit $b$ with $ba = 1$.
\end{theorem}
\begin{proof}
    This is basically just the definition of $a$ being a unit, but being
    explicit about the fact that $b$ is a unit.
\end{proof}

\begin{theorem} \label{prime_irreducible}
    All primes are irreducible.
\end{theorem}
\begin{proof}
    Let $p$ be prime, and let $a$ and $b$ be non-units.  For a contradiction,
    assume that $p = ab$.  Because primes are nonzero, $0 \neq p = ab$, so $a$
    and $b$ are nonzero as well.  Now $p \mid p = ab$, so by the definition of
    being prime, either $p \mid a$ or $p \mid b$.  This means that $ab \mid a$
    or $ab \mid b$.  In each case, either $a$ or $b$ is a unit by Theorem
    \ref{div_mult_unit}, which is a contradiction.  Thus we must have $p \neq
    ab$, showing that $p$ is irreducible.
\end{proof}

\begin{theorem} \label{nz_unit}
    If $a$ is a unit and $b \neq 0$, then $ab \neq 0$.
\end{theorem}
\begin{proof}
    Because $a$ is a unit, $a \neq 0$, and since $b \neq 0$, we have $ab \neq
    0$.
\end{proof}

\begin{theorem} \label{not_unit_mult}
    For all $a$ and $b$, if $b$ is not a unit, then $ab$ is not a unit.
\end{theorem}
\begin{proof}
    If $ab$ was a unit, $b$ would be a unit by Theorem \ref{rmult_unit}, which
    is a contradiction.
\end{proof}

\begin{theorem} \label{irreducible_unit}
    For all units $a$ and irreducible elements $b$, $ab$ is irreducible.
\end{theorem}
\begin{proof}
    That $ab$ is not zero and not a unit follows from Theorems \ref{nz_unit} and
    \ref{not_unit_mult}.  For the final condition, let $x$ and $y$ be non-units.
    We must prove that $ab \neq xy$.  Because $a$ is a unit, we have some unit
    $c$ such that $ca = 1$.  Then $cx$ is not a unit by Theorem
    \ref{not_unit_mult}, so by $b$ being irreducible, we know that $b \neq cxy$.
    Multiplying by $a$, we get $ab \neq acxy = xy$, showing that $ab$ is
    irreducible.
\end{proof}

\begin{theorem} \label{prime_unit}
    For all units $a$ and prime elements $b$, $ab$ is prime.
\end{theorem}
\begin{proof}
    That $ab$ is not zero and not a unit follows from Theorems \ref{nz_unit} and
    \ref{not_unit_mult}.  For the final condition, let $x$ and $y$ be values
    such that $ab \mid xy$.  We must prove that either $ab \mid x$ or $ab \mid
    y$.  Because $a$ is a unit, there exists a unit $c$ such that $ca = 1$.  By
    Theroem \ref{div_lmult}, we have $cab \mid cxy$, so $b \mid (cx)y$.  Because
    $b$ is prime, we must have either $b \mid cx$ or $b \mid y$.  If $b \mid
    cx$, by Theorem \ref{div_lmult} we have $ab \mid acx = x$.  If $b \mid y$,
    then $ab \mid y$ by Theorem \ref{div_unit_mult}.  Either way, we have either
    $ab \mid x$ or $ab \mid y$, so $ab$ is prime.
\end{proof}

\begin{instance}
    Values associating with each other is reflexive.
\end{instance}
\begin{proof}
    This follows directly from $\mid$ being reflexive.
\end{proof}

\begin{instance}
    Values associating with each other is symmetric.
\end{instance}
\begin{proof}
    If $a$ is associated with $b$, then $a \mid b$ and $b \mid a$, showing that
    $b$ is associated with $a$.
\end{proof}

\begin{instance}
    Values associating with each other is transitive.
\end{instance}
\begin{proof}
    Let $a$ be associated with $b$, and $b$ with $c$.  Then we have $a \mid b$,
    $b \mid a$, $b \mid c$, and $c \mid b$.  By the transitivity of $\mid$, we
    have $a \mid c$ and $c \mid a$, showing that $a$ is associated with $c$.
\end{proof}

\begin{theorem} \label{unit_associates}
    All units associate with each other.
\end{theorem}
\begin{proof}
    This follows directly from Theorem \ref{unit_div}.
\end{proof}

\begin{theorem} \label{associates_zero}
    If $a$ is associated with $0$, then $0 = a$.
\end{theorem}
\begin{proof}
    Because $a$ is associated with $0$, we have $0 \mid a$, so the result
    follows from Theorem \ref{div_zero}.
\end{proof}

\begin{theorem} \label{associates_one}
    For all $a$, $a$ associates with one if and only if $a$ is a unit.
\end{theorem}
\begin{proof}
    The forward direction is trivial.  For the reverse direction, $a$ being a
    unit is already one of the conditions for $a$ to associate with one.  The
    other condition is $1 \mid a$, which follows from Theorem \ref{one_divides}.
\end{proof}

\begin{theorem} \label{associates_unit}
    If $a$ is associated with $b$, there exists a unit $c$ such that $ca = b$.
\end{theorem}
\begin{proof}
    We have $a \mid b$ and $b \mid a$, so there exist values $c$ and $d$ such
    that $ca = b$ and $db = a$.  If $0 = b$, we have $a = 0$ as well, so $(1)0 =
    0$ works since $1$ is a unit.  If $0 \neq b$, then we already have a $c$
    such that $ca = b$, and all that remains is to prove that $c$ is a unit.
    From $ca = b$ and $db = a$, we have $cdb = b$, and cancelling $b$ we get $cd
    = 1$, showing that $c$ is a unit.
\end{proof}

The fact that units are not always equal to one can be annoying when working
with divisibility.  It is possible to work around this issue by passing to a
particular quotient.

\begin{definition}
    Define $\D(\U)$ to be the quotient of $\U$ by the equivalence relation of
    values associating with each other.  Given some $a : \U$, we will use
    $\D(a)$ to mean the equivalence class of $a$ in $\D(\U)$.
\end{definition}

Note that this is not a quotient by an ideal.  In fact, this quotient is not a
ring at all since addition is not well-defined.  (For example, in the integers,
$2 + (-2) = 0$, but in the quotient $2$ and $-2$ go to the same value and their
sum would not equal zero.)  However, multiplication is still well-defined, which
is what will allow this quotient to be useful.

\begin{lemma}
    Multiplication is well-defined in $\D(\U)$.
\end{lemma}
\begin{proof}
    We must prove that if $a$ and $b$ associate and if $c$ and $d$ associate,
    then $ac$ and $bd$ associate.  We have $a \mid b$, $b \mid a$, $c \mid d$,
    and $d \mid c$.  We can multiply $a \mid b$ by $c$ on the right and $c \mid
    d$ by $b$ on the left to get $ac \mid bc \mid bd$.  We can multiply $b \mid
    a$ on the right by $d$ and $d \mid c$ on the left by $a$ to get $bd \mid ad
    \mid ac$.  Thus, $ac \mid bd$ and $bd \mid ac$, so $ac$ and $bd$ associate.
\end{proof}

\begin{instance}
    Define multiplication in $\D(\U)$ to be multiplication in $\U$ passed to the
    quotient, which is well-defined by the previous lemma.
\end{instance}

Even if addition isn't well-defined in $\D(\U)$, zero itself is.

\begin{instance}
    Define zero in $\D(\U)$ to be zero in $\U$ passed to the quotient.
\end{instance}

\begin{instance}
    Define one in $\D(\U)$ to be one in $\U$ passed to the quotient.
\end{instance}

Trivially, multiplication in $\D(\U)$ is commutative and associative, one is an
identity, and zero is an annihilator.  Multiplication being cancellative is not
quite as trivial.

\begin{instance}
    Multiplication in $\D(\U)$ is cancellative.
\end{instance}
\begin{proof}
    Let $c \neq 0$ and $ca = cb$.  Bringing this back down to $\U$, we have $c$
    is not associated with $0$, and $ca$ and $cb$ associate.  Because $c$ is not
    associated with $0$, $c$ itself is not zero.  Because $ca$ and $cb$
    associate, we have $ca \mid cb$ and $cb \mid ca$.  Because $c \neq 0$, by
    Theorem \ref{div_lcancel}, we have $a \mid b$ and $b \mid a$, showing that
    $a$ and $b$ associate, so they are equal in the quotient.
\end{proof}

\begin{theorem} \label{div_equiv_div}
    For all $a$ and $b$ in $\U$, $a \mid b$ if and only if $\D(a) \mid \D(b)$.
\end{theorem}
\begin{proof}
    Assume that $a \mid b$.  Then there exists some $c$ such that $ca = b$.
    Then $\D(c) \D(a) = \D(ca) = \D(b)$, showing that $\D(a) \mid \D(b)$.

    Assume that $\D(a) \mid \D(b)$.  Then there exists some $c$ such that $\D(c)
    \D(a) = \D(ca) = \D(b)$.  Thus, $ca$ and $b$ associate.  By Theorem
    \ref{associates_unit}, there is a unit $d$ such that $dca = b$.  Since
    $(dc)a = b$ we have $a \mid b$.
\end{proof}

\begin{theorem} \label{div_equiv_unit}
    For all $a : \D(\U)$, $a$ is a unit if and only if $a = 1$.
\end{theorem}
\begin{proof}
    The reverse is true by Theorem \ref{one_unit}, so we only need to check that
    if $\D(a)$ is a unit, then $\D(a) = 1$.  If $\D(a)$ is a unit, we have
    $\D(a) \mid 1 = \D(1)$.  Thus, by Theorem \ref{div_equiv_div}, we have $a
    \mid 1$, so $a$ is a unit.  Thus, by Theorem \ref{associates_one}, $a$
    associates with $1$, meaning that $\D(a) = 1$.
\end{proof}

\begin{theorem} \label{div_equiv_unit2}
    For all $a : \U$, $a$ is a unit if and only if $\D(a)$ is a unit.
\end{theorem}
\begin{proof}
    This is just a special case of Theorem \ref{div_equiv_div}.
\end{proof}

\begin{theorem} \label{div_equiv_zero}
    For all $a : \U$, $a = 0$ if and only if $\D(a) = 0$.
\end{theorem}
\begin{proof}
    The forward direction is trivial.  For the reverse direction, assume that
    $\D(a) = 0$.  Then $a$ is associated with $0$, so $0 \mid a$, meaning that
    by Theorem \ref{div_zero}, we have $a = 0$.
\end{proof}

\begin{theorem} \label{div_equiv_irreducible}
    For all $a : \U$, $a$ is irreducible if and only if $\D(a)$ is irreducible.
\end{theorem}
\begin{proof}
    $a$ being nonzero and not a unit is equivalent to $\D(a)$ being nonzero and
    not a unit by Theorems \ref{div_equiv_zero} and \ref{div_equiv_unit2}, so we
    only need to prove that the main irreducibility condition goes back and
    forth.

    Assume that $a$ is irreducible.  Let $c$ and $d$ be such that $\D(c)$ and
    $\D(d)$ are not units.  Then by Theorem \ref{div_equiv_unit2}, $c$ and $d$
    are not units.  For a contradiction, assume that $\D(a) = \D(c) \D(d)$.
    This means that $a$ and $cd$ are associated.  Then by Theorem
    \ref{associates_unit}, we have a unit $u$ such that $a = ucd$.  By Theorem
    \ref{not_unit_mult}, $uc$ is not a unit, which means that $a$ is the product
    of two non-units $uc$ and $d$, which contradicts $a$ being irreducible.
    Thus, $\D(a) \neq \D(c) \D(d)$, showing that $\D(a)$ is irreducible.

    Now assume that $\D(a)$ is irreducible.  Let $c$ and $d$ be non-units in
    $\U$.  Then by Theorem \ref{div_equiv_unit2}, $\D(c)$ and $\D(d)$ are not
    units, so by $\D(a)$ being irreducible, we have $\D(a) \neq \D(c) \D(d) =
    \D(cd)$.  Thus, $a \neq cd$, showing that $a$ is irreducible.
\end{proof}

\begin{theorem} \label{div_equiv_prime}
    For all $a : \U$, $a$ is prime if and only if $\D(a)$ is prime.
\end{theorem}
\begin{proof}
    $a$ being nonzero and not a unit is equivalent to $\D(a)$ being nonzero and
    not a unit by Theorems \ref{div_equiv_zero} and \ref{div_equiv_unit2}, so we
    only need to prove that the main primality condition goes back and forth.

    Assume that $a$ is prime.  Let $c$ and $d$ be such that $\D(a) \mid \D(c)
    \D(d) = \D(cd)$.  By Theorem \ref{div_equiv_div}, we have $a \mid cd$, so by
    $a$ being prime, we have either $a \mid c$ or $a \mid d$.  By Theorem
    \ref{div_equiv_div} again, we have either $\D(a) \mid \D(c)$ or $\D(a) \mid
    \D(d)$, proving that $\D(a)$ is prime.

    Assume that $\D(a)$ is prime.  Let $c$ and $d$ be such that $a \mid cd$.
    Then by Theorem \ref{div_equiv_div}, we have $\D(a) \mid \D(cd) = \D(c)
    \D(d)$, so by $\D(a)$ being prime, we have $\D(a) \mid \D(c)$ or $\D(a) \mid
    \D(d)$.  By Theorem \ref{div_equiv_div} again, we have $a \mid c$ or $a \mid
    d$, proving that $a$ is prime.
\end{proof}

\begin{theorem}
    For all units $a$ and values $x$, $\D(ax) = \D(x)$.
\end{theorem}
\begin{proof}
    Because $a$ is a unit, there exists a $b$ with $ba = 1$.  Then $bax = x$,
    showing that $ax \mid x$, and $x \mid ax$ by Theorem \ref{mult_div_rself},
    so $ax$ and $x$ associate.  Thus, $\D(ax) = \D(x)$.
\end{proof}

\section{GCDs}

\begin{definition}
    Let $a$ and $b$ be values in an integral domain $\U$.  Then $d$ is a common
    divisor of $a$ and $b$ if $d \mid a$ and $d \mid b$.
\end{definition}

\begin{definition}
    Let $a$ and $b$ be values in an integral domain $\U$.  Then $d$ is the
    greatest common divisor of $a$ and $b$ if it is a common divisor, and if for
    all other common divisors $d'$, we have $d' \mid d$.
\end{definition}

\begin{class}
    Let $\U$ be an integral domain.  Then we call $\U$ a GCD Domain if there
    exists a function $\gcd : \U \to \U \to \U$ such that for all $a$ and $b$
    with at least one of them nonzero, $\gcd(a, b)$ is a greatest common divisor
    of $a$ and $b$.
\end{class}

\begin{theorem} \label{gcd_associates}
    For all $a$ and $b$ with two greatest common divisors $d_1$ and $d_2$, $d_1$
    and $d_2$ are associated.
\end{theorem}
\begin{proof}
    Since $d_1$ and $d_2$ are both greatest common divisors, we have $d_1 \mid
    d_2$ and $d_2 \mid d_1$, so $d_1$ and $d_2$ are associated.
\end{proof}

\begin{theorem} \label{gcd_div_comm}
    In a GCD domain, for all $a$ and $b$ with at least one of them nonzero,
    $\gcd(a, b) \mid \gcd(b, a)$.
\end{theorem}
\begin{proof}
    The definition of common divisor is symmetric, so $\gcd(a, b)$ is a common
    divisor of $b$ and $a$, meaning that $\gcd(a, b) \div \gcd(b, a)$.
\end{proof}

\begin{theorem} \label{gcd_comm}
    In a GCD domain, for all $a$ and $b$ with at least one of them nonzero,
    $\gcd(a, b)$ is associated with $\gcd(b, a)$.
\end{theorem}
\begin{proof}
    Just apply the previous theorem forwards and backwards.
\end{proof}

\begin{theorem} \label{irreducible_prime}
    In a GCD Domain, all irreducible elements are prime.
\end{theorem}
\begin{proof}
    Let $p$ be irreducible, and let $p \mid ab$.  We must prove that either $p
    \mid a$ or $p \mid b$.  If $b = 0$, then $p \mid b$, so consider the case
    when $b \neq 0$.  If $ab = 0$, then $a = 0$ since $b \neq 0$, meaning that
    $p \mid a$, so consider the case when $ab \neq 0$.  Let $d = \gcd(pb, ab)$.
    Then $d \neq 0$, because if $d = 0$, we would have $0 \mid ab$, implying
    that $0 = ab$, contradicting $ab \neq 0$.  By Theorems \ref{mult_div_lself}
    and \ref{mult_div_rself}, we have $p \mid pb$, $b \mid pb$, and $b \mid ab$.
    Because of this and $p \mid ab$, both $p$ and $b$ are common divisors of
    $ab$ and $pb$, so $p \mid d$ and $b \mid d$.  Thus, we have values $u$ and
    $v$ such that $up = d$ and $vb = d$.  There will now be two cases: when $v$
    is a unit, and when $v$ is not a unit.

    When $v$ is a unit, we have a $v'$ such that $v'v = 1$.  We have $up = vb$,
    so $v'up = v'vb = b$, showing that $p \mid b$.

    When $v$ is not a unit, Because $d \mid pb$, we have $vb \mid pb$, so by
    Theorem \ref{div_rcancel}, we have $v \mid p$.  Thus, we have a $c$ such
    that $cv = p$.  Because $v$ is not a unit and $p$ is irreducible, $c$ must
    be a unit.  Then we have a $c'$ such that $c'c = 1$.  Since $cv = p$, we
    have $c'cv = c'p$, so $v = c'p$, showing that $p \mid v$.  Now from $d \mid
    ab$, we have $vb \mid ab$, and by Theorem \ref{div_rcancel} we have $v \mid
    a$.  Since $p \mid v$ and $v \mid a$, we have $p \mid a$.
\end{proof}

\section{Factorization}

\begin{class}
    Let $\U$ be an integral domain.  Then we call $\U$ a unique factorization
    domain (UFD) if for all $x \neq 0$ that is not a unit, there exists a list
    of primes $a$ such that $x = \prod a$.  This list $a$ will be called the
    factorization of $x$.
\end{class}

There are a couple notable differences between this definition and the usual
definition.  Traditionally, the definition also requires uniqueness.  Also, the
list traditionally contains irreducible elements, not prime elements.  These two
differences from the traditional definition are related: if you use prime
elements instead of irreducible elements, uniqueness can be proved just from
existence.  Thus, to make it easier to prove that integral domains are UFDs,
this alternative definition has been used so that uniqueness never has to be
proved.  It turns out that UFDs are GCD domains, as will be proven below, so the
distinction between prime and irreducible elements isn't really that important.

\end{document}
