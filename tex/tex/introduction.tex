\documentclass[../math.tex]{subfiles}
\externaldocument{foundations}

\begin{document}

\section*{Introduction}
\addcontentsline{toc}{chapter}{Introduction}

This document is intended to be a version of my Coq project ``Math From
Nothing'' which is written in a more traditional format.  Feel free to look
around in here if you want to understand an argument used in the Coq code.
While I am trying to be rigorous in this document, I am not trying to be formal,
and there are probably many typos throughout it.  Also, I am not writing to be
understood by others, so the proofs given here may be a bit more obtuse than
they should be.  I'm mainly writing this to make it easier for me to understand
what my Coq code is doing.

This document uses a system based off of Coq's typeclass system.  Several
``definitions'' are given as classes instead.  These are definitions or
conditions that don't care about what thing they are attached to.  Later on,
instances of these classes can be made for particular things.  Theorems can be
proven about all things that are instances of those classes, allowing for proofs
to be generalized to all things that share certain properties.  Classes will be
defined in a similar manner to definitions, but with the name ``Class''.
Instances of a class on a particular thing will be proved similarly to theorems,
but with the name ``Instance''.  Then, at various times, certain theorems or
sections will require certain classes.  Note that not all theorems will actually
require all listed classes, or list all required classes.  Coq automatically
determines which are required itself and I don't want to do the work of figuring
this out and then listing every single required class.

\end{document}
