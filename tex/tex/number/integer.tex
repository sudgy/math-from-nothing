\documentclass[../../math.tex]{subfiles}
\externaldocument{../../math.tex}
\externaldocument{../basics/set}
\externaldocument{../basics/natural}

\begin{document}

\setcounter{chapter}{6}

\chapter{The Integers} \label{chap_integer}

\begin{definition}
    Define a relation $\sim$ on $\N \times \N$ where $(a_1, a_2) \sim (b_1,
    b_2)$ is defined to mean $a_1 + b_2 = b_1 + a_2$.
\end{definition}

\begin{lemma}
    The relation $\sim$ is an equivalence relation on $\N \times \N$.
\end{lemma}
\begin{proof}
    \textit{Reflexivitity.}  We must check $a_1 + a_2 = a_1 + a_2$, which is a
    reflexive equality.

    \textit{Symmetry.}  We must prove that if $a_1 + b_2 = b_1 + a_2$, then $b_1
    + a_2 = a_1 + b_2$.  This is true by the symmetry of equality.

    \textit{Transitivity.}  We must prove that if $a_1 + b_2 = b_1 + a_2$ and
    $b_1 + c_2 = c_1 + b_2$, then $a_1 + c_2 = c_1 + a_2$.  We can add the first
    two equalities to get
    \[
        a_1 + b_2 + b_1 + c_2 = c_1 + b_2 + b_1 + a_2.
    \]
    We can cancel $b_2$ and $b_1$ to get the result.
\end{proof}

\begin{definition}
    Define the type of integers $\Z$ to be the type $(\N \times \N)/\usim$.
\end{definition}

\section{Addition}

\begin{definition}
    Define an operation $\oplus : \N \times \N \to \N \times \N \to \N \times
    \N$ given by
    \[
        (a_1, a_2) \oplus (b_1, b_2) = (a_1 + b_1, a_2 + b_2).
    \]
\end{definition}

\begin{lemma}
    The operation $\oplus$ is well-defined under the equivalence relation
    $\sim$.
\end{lemma}
\begin{proof}
    We must prove that for all $a$, $b$, $c$, and $d$, if $a \sim b$ and $c \sim
    d$, we have $a \oplus c \sim b \oplus d$.  Thus, we have
    \[
        a_1 + b_2 = b_1 + a_2
    \]
    and
    \[
        c_1 + d_2 = d_1 + c_2
    \]
    and must prove that
    \[
        a_1 + c_1 + b_2 + d_2 = b_1 + d_1 + a_2 + c_2.
    \]
    This follows directly by rearranging and applying the previous two
    equalities.
\end{proof}

\begin{instance}
    Definition addition in the integers as the binary operation given by Theorem
    \ref{binary_op_ex} and the previous lemma.
\end{instance}

\begin{instance}
    Addition of integers is commutative.
\end{instance}
\begin{proof}
    We must prove that
    \[
        (a_1, a_2) \oplus (b_1, b_2) \sim (b_1, b_2) \oplus (a_1, a_2).
    \]
    This reduces to
    \[
        a_1 + b_1 + b_2 + a_2 = b_1 + a_1 + a_2 + b_2,
    \]
    which follows by the commutativity of natural number addition.
\end{proof}

\begin{instance}
    Addition of integers is associative.
\end{instance}
\begin{proof}
    We must prove that
    \[
        (a_1, a_2) \oplus ((b_1, b_2) \oplus (c_1, c_2)) \sim
        ((a_1, a_2) \oplus (b_1, b_2)) \oplus (c_1, c_2).
    \]
    This reduces to
    \[
        (a_1 + (b_1 + c_1)) + ((a_2 + b_2) + c_2) =
        ((a_1 + b_1) + c_1) + (a_2 + (b_2 + c_2)),
    \]
    which follows by the associativity of natural number addition.
\end{proof}

\begin{instance}
    Define zero in the integers to be $[(0, 0)]$.
\end{instance}

\begin{instance}
    Zero is an additive identity in the integers.
\end{instance}
\begin{proof}
    We must prove that
    \[
        (0, 0) \oplus (a_1, a_2) \sim (a_1, a_2).
    \]
    This reduces to
    \[
        0 + a_1 + a_2 = a_1 + 0 + a_2,
    \]
    which follows by 0 being an identity in the natural numbers.
\end{proof}

\begin{definition}
    Given $(a_1, a_2) : \N$, define
    \[
        \ominus (a_1, a_2) = (a_2, a_1).
    \]
\end{definition}

\begin{lemma}
    $\ominus$ is well-defined under $\sim$.
\end{lemma}
\begin{proof}
    We must prove that if $(a_1, a_2) \sim (b_1, b_2)$, then $\ominus (a_1, a_2)
    \sim \ominus (b_1, b_2)$.  Simplified, we have $a_1 + b_2 = b_1 + a_2$ and
    must prove that $a_2 + b_1 = b_2 + a_1$, which follows from commutativity.
\end{proof}

\begin{instance}
    Definition additive inverses in the integers as the unary operation given by
    Theorem \ref{unary_op_ex} and the previous lemma.
\end{instance}

\begin{instance}
    Negation is a left inverse in the integers.
\end{instance}
\begin{proof}
    We must prove that
    \[
        \ominus (a_1, a_2) \oplus (a_1, a_2) \sim (0, 0).
    \]
    This reduces to
    \[
        a_2 + a_1 + 0 = 0 + a_1 + a_2,
    \]
    which follows by the properties of natural number addition.
\end{proof}

\section{Multiplication}

\begin{definition}
    Define an operation $\otimes : \N \times \N \to \N \times \N \to \N \times
    \N$ given by
    \[
        (a_1, a_2) \otimes (b_1, b_2) = (a_1b_1 + a_2b_2, a_1b_2 + a_2b_1).
    \]
\end{definition}

\begin{lemma}
    The operation $\otimes$ is well-defined under the equivalence relation
    $\sim$.
\end{lemma}
\begin{proof}
    We must prove that for all $a$, $b$, $c$, and $d$, if $a \sim b$ and $c \sim
    d$, we have $a \otimes c \sim b \otimes d$.  Thus, we have
    \begin{equation} \label{int_mult_wd_ab}
        a_1 + b_2 = b_1 + a_2
    \end{equation}
    and
    \begin{equation} \label{int_mult_wd_cd}
        c_1 + d_2 = d_1 + c_2
    \end{equation}
    and must prove that
    \[
        a_1c_1 + a_2c_2 + b_1d_2 + b_2d_1 =
        b_1d_1 + b_2d_2 + a_1c_2 + a_2c_1.
    \]
    We can multiply equation \ref{int_mult_wd_ab} by $c_1$ and $c_2$ to get
    \begin{equation} \label{int_mult_wd_ab1}
        a_1c_1 + b_1c_1 = b_1c_1 + a_2c_1
    \end{equation}
    and
    \begin{equation} \label{int_mult_wd_ab2}
        a_1c_2 + b_1c_2 = b_1c_2 + a_2c_2,
    \end{equation}
    and we can multiply equation \ref{int_mult_wd_cd} by $b_1$ and $b_2$ to get
    \begin{equation} \label{int_mult_wd_cd1}
        b_1c_1 + b_1d_2 = b_1d_1 + b_1c_2
    \end{equation}
    and
    \begin{equation} \label{int_mult_wd_cd2}
        b_2c_1 + b_2d_2 = b_2d_1 + b_2c_2.
    \end{equation}
    We can add equations \ref{int_mult_wd_ab1} and \ref{int_mult_wd_cd1} to get
    \begin{align}
        a_1c_1 + b_2c_1 + b_1c_1 + b_1d_2 &=
        b_1c_1 + a_2c_1 + b_1d_1 + b_1c_2 \nonumber \\
        a_1c_1 + b_2c_1 + b_1d_2 &= a_2c_1 + b_1d_1 + b_1c_2,
        \label{int_mult_wd_eq5}
    \end{align}
    and we can add equations \ref{int_mult_wd_ab2} and \ref{int_mult_wd_cd2} to
    get
    \begin{align}
        b_1c_2 + a_2c_2 + b_2d_1 + b_2c_2 &=
        a_1c_2 + b_2c_2 + b_2c_1 + b_2d_2 \nonumber \\
        b_1c_2 + a_2c_2 + b_2d_1 &= a_1c_2 + b_2c_1 + b_2d_2.
        \label{int_mult_wd_eq6}
    \end{align}
    We can now add equations \ref{int_mult_wd_eq5} and \ref{int_mult_wd_eq6} to
    get
    \begin{align*}
        a_1c_1 + b_2c_1 + b_1d_2 + b_1c_2 + a_2c_2 + b_2d_1 &=
        a_2c_1 + b_1d_1 + b_1c_2 + a_1c_2 + b_2c_1 + b_2d_2 \\
        a_1c_1 + b_1d_2 + a_2c_2 + b_2d_1 &=
        a_2c_1 + b_1d_1 + a_1c_2 + b_2d_2 \\
        a_1c_1 + a_2c_2 + b_1d_2 + b_2d_1 &=
        b_1d_1 + b_2d_2 + a_1c_2 + a_2c_1.
    \end{align*}
\end{proof}

\begin{instance}
    Definition multiplication in the integers as the binary operation given by
    Theorem \ref{binary_op_ex} and the previous lemma.
\end{instance}

\begin{instance}
    Multiplication of integers is commutative.
\end{instance}
\begin{proof}
    We must prove that
    \[
        (a_1, a_2) \otimes (b_1, b_2) \sim (b_1, b_2) \otimes (a_1, a_2).
    \]
    This reduces to
    \[
        a_1b_1 + a_2b_2 + b_1a_2 + b_2a_1 =
        b_1a_1 + b_2a_2 + a_1b_2 + a_2b_1
    \]
    which follows by the commutativity of natural number multiplication and
    addition.
\end{proof}

\begin{instance}
    Multiplication of integers is associative.
\end{instance}
\begin{proof}
    We must prove that
    \[
        (a_1, a_2) \otimes ((b_1, b_2) \otimes (c_1, c_2)) \sim
        ((a_1, a_2) \otimes (b_1, b_2)) \otimes (c_1, c_2).
    \]
    This reduces to
    \begin{align*}
        &{}a_1(b_1c_1 + b_2c_2) + a_2(b_1c_2 + b_2c_1) +
        (a_1b_1 + a_2b_2)c_2 + (a_1b_2 + a_2b_1)c_1 \\
        {}={}&
        (a_1b_1 + a_2b_2)c_1 + (a_1b_2 + a_2b_1)c_2 +
        a_1(b_1c_2 + b_2c_1) + a_2(b_1c_1 + b_2c_2),
    \end{align*}
    which expands to
    \begin{align*}
        &{}a_1b_1c_1 + a_1b_2c_2 + a_2b_1c_2 + a_2b_2c_1 +
        a_1b_1c_2 + a_2b_2c_2 + a_1b_2c_1 + a_2b_1c_1 \\
        {}={}&
        a_1b_1c_1 + a_2b_2c_1 + a_1b_2c_2 + a_2b_1c_2 +
        a_1b_1c_2 + a_1b_2c_1 + a_2b_1c_1 + a_2b_2c_2,
    \end{align*}
    which has the same terms on both sides.
\end{proof}

\begin{instance}
    Multiplication of integers distributes over addition.
\end{instance}
\begin{proof}
    We must prove that
    \[
        (a_1, a_2) \otimes ((b_1, b_2) \oplus (c_1, c_2)) \sim
        (a_1, a_2) \otimes (b_1, b_2) \oplus (a_1, a_2) \otimes (c_1, c_2).
    \]
    This reduces to
    \begin{align*}
        &{}a_1(b_1 + c_1) + a_2(b_2 + c_2) + a_1b_2 + a_2b_1 + a_1c_2 + a_2c_1\\
        {}={}&
        a_1b_1 + a_2b_2 + a_1c_1 + a_2c_2 + a_1(b_2 + c_2) + a_2(b_1 + c_1),
    \end{align*}
    which expands to
    \begin{align*}
        &{}a_1b_1 + a_1c_1 + a_2b_2 + a_2c_2 + a_1b_2 + a_2b_1 + a_1c_2 + a_2c_1
        \\ {}={}&
        a_1b_1 + a_2b_2 + a_1c_1 + a_2c_2 + a_1b_2 + a_1c_2 + a_2b_1 + a_2c_1,
    \end{align*}
    which has the same terms on both sides.
\end{proof}

\begin{instance}
    Define one in the integers to be $[(1, 0)]$.
\end{instance}

\begin{instance}
    One is a multiplicative identity in the integers.
\end{instance}
\begin{proof}
    We must prove that
    \[
        (1, 0) \otimes (a_1, a_2) \sim (a_1, a_2).
    \]
    This reduces to
    \[
        1a1 + 0a2 + a2 = a1 + 1a2 + 0a1,
    \]
    which simplifies to
    \[
        a_1 + a_2 = a_1 + a_2.
    \]
\end{proof}

\begin{lemma}
    For all $a$ and $b$ in $\Z$, if $0 = ab$, then either $0 = a$ or $0 = b$.
\end{lemma}
\begin{proof}
    After a bit of simplification, the hypothesis $0 = ab$ becomes
    \[
        a_1b_2 + a_2b_1 = a_1b_1 + a_2b_2,
    \]
    and the conclusion becomes that either $a_1 = a_2$ or that $b_1 = b_2$.  If
    $a_1 = a_2$, we're done, so assume that $a_1 \neq a_2$.  Without loss of
    generality, assume that $a_1 < a_2$.  Then by Theorem \ref{nat_lt_ex}, there
    exists a $c : \N$ such that $a_1 + S(c) = a_2$.  Then
    \begin{align*}
        a_1b_2 + (a_1 + S(c))b_1 &= a_1b_1 + (a_1 + S(c))b_2 \\
        a_1b_2 + a_1b_1 + S(c)b_1 &= a_1b_1 + a_1b_2 + S(c)b_2 \\
        S(c)b_1 &= S(c)b_2,
    \end{align*}
    and we can cancel $S(c)$ because $S(c) \neq 0$, giving $b_1 = b_2$ as
    required.
\end{proof}

\begin{instance}
    Multiplication of integers is cancellative.
\end{instance}
\begin{proof}
    Let $a$, $b$, and $c$ be integers such that $0 \neq c$ and $ca = cb$.
    Rearranging we get $0 = c(a - b)$, so by the previous lemma, either $c = 0$
    and $a - b = 0$.  The first case is impossible becaues $0 \neq c$, and from
    the second case we get $a = b$.
\end{proof}

\begin{instance}
    The integers are not trivial.
\end{instance}
\begin{proof}
    We will prove that $0 \neq 1$.  If $0 = 1$, then we would have $(0, 0) \sim
    (1, 0)$, which means that $0 = 1$ in the natural numbers, which is
    impossible because the natural numbers are not trivial.
\end{proof}

Thus, we have now seen that the integers are an integral domain, so they are
also a commutative ring, a ring, etc.

\begin{theorem}
    For all natural numbers $n$, $\iota(n) = [(n, 0)]$.
\end{theorem}
\begin{proof}
    The proof will be by induction on $n$.  When $n = 0$,
    \[
        \iota(0) = 0 = [(0, 0)],
    \]
    so the base case is true.  Now assume that $\iota(n) = [(n, 0)]$.  Then
    \[
        \iota(S(n)) = 1 + \iota(n) = 1 + [(n, 0)].
    \]
    To check if this is equal to $[(S(n), 0)]$, we must check if
    \[
        (1, 0) \oplus [(n, 0)] \sim [(S(n), 0)],
    \]
    which simplifies to
    \[
        1 + n + 0 = S(n) + 0 + 0,
    \]
    which is true.
\end{proof}

\section{Order}

\begin{definition}
    Given $(a_1, a_2)$ and $(b_1, b_2)$ in $\N \times \N$, define $a \leqq b$ to
    mean $a_1 + b_2 \leq a_2 + b_1$.
\end{definition}

\begin{theorem}
    The relation $\leqq$ is well-defined under the relation $\sim$.
\end{theorem}
\begin{proof}
    By propositional extensionality and symmetry, it suffices to prove that for
    all pairs $a$, $b$, $c$, and $d$, if $a \sim b$, $c \sim d$, and $a \leqq
    c$, then $b \leqq d$.  Simplified, we have
    \begin{align*}
        b_1 + a_2 &= a_1 + b_2, \\
        c_1 + d_2 &= d_1 + c_2, \\
    \intertext{and}
        a_1 + c_2 \leq a_2 + c_1.
    \end{align*}
    Adding the two equations to the inequality gives
    \[
       a_1 + c_2 + b_1 + a_2 + c_1 + d_2 \leq a_2 + c_1 + a_1 + b_2 + d_1 + c_2,
    \]
    and cancelling gives
    \[
        b_1 + d_2 \leq b_2 + d_1
    \]
    meaning that $(b_1, b_2) \leqq (d_1, d_2)$ as required.
\end{proof}

\begin{instance}
    Define order in the integers as the operation obtained by applying Theorem
    \ref{binary_op_ex} to the previous lemma.
\end{instance}

\begin{instance}
    The order in $\Z$ is connex.
\end{instance}
\begin{proof}
    We must prove that for all $a_1, a_2, b_1, b_2$, either $a_1 + b_2 \leq a_2
    + b_1$ or $b_1 + a_2 \leq b_2 + a_1$.  This is true because the order on
    $\N$ is connex.
\end{proof}

\begin{instance}
    The order in $\Z$ is antisymmetric.
\end{instance}
\begin{proof}
    We must prove that if
    \[
        a_1 + b_2 \leq a_2 + b_1
    \]
    and
    \[
        b_1 + a_2 \leq b_2 + a_1,
    \]
    then
    \[
        a_1 + b_2 = b_1 + a_2.
    \]
    This follows from the antisymmetry of the order in $\N$ after using
    commutativity.
\end{proof}

\begin{instance}
    The order in $\Z$ is transitive.
\end{instance}
\begin{proof}
    We must prove that if
    \[
        a_1 + b_2 \leq a_2 + b_1
    \]
    and
    \[
        b_1 + c_2 \leq b_2 + c_1,
    \]
    then
    \[
        a_1 + c_2 \leq a_2 + c_1.
    \]
    We can add $c_2$ to the first inequality to get
    \[
        a_1 + b_2 + c_2 \leq a_2 + b_1 + c_2
    \]
    and we can add $a_2$ to the second inequality to get
    \[
        a_2 + b_1 + c_2 \leq a_2 + b_2 + c_1.
    \]
    By transitivity of the order in $\N$,
    \[
        a_1 + b_2 + c_2 \leq a_2 + b_2 + c_1,
    \]
    and by cancelling $b_2$ we get the desired result.
\end{proof}

\begin{instance}
    The order in $\Z$ is left additive.
\end{instance}
\begin{proof}
    We must prove that if $a_1 + b_2 \leq a_2 + b_1$, then
    \[
        c_1 + a_1 + c_2 + b_2 \leq c_2 + a_2 + c_1 + b_1,
    \]
    which follows directly from the additivity of the order in $\N$.
\end{proof}

\begin{theorem} \label{int_pos_nat_ex}
    For all integers $a$, if $0 \leq a$, then there exists a natural number $n$
    such that $a = n$.
\end{theorem}
\begin{proof}
    Because $0 \leq a$, we have $a_2 \leq a_1$.  Then by Theorem
    \ref{nat_le_ex}, there exists some $c$ such that $a_2 + c = a_1$.  This
    equation is equivalent to the equation
    \[
        [(c, 0)] = [(a_1, a_2)],
    \]
    so $c$ is the required natural number.
\end{proof}

\begin{theorem} \label{int_neg_nat_ex}
    For all integers $a$, if $a \leq 0$, then there exists a natural number $n$
    such that $a = -n$.
\end{theorem}
\begin{proof}
    Because $a \leq 0$, we have $0 \leq -a$, so by the previous theorem there
    exists a natural number $n$ such that $-a = n$, so $a = -n$.
\end{proof}

\begin{instance}
    The order in $\Z$ is multiplicative.
\end{instance}
\begin{proof}
    Let $0 \leq a$ and $0 \leq b$.  Then by Theorem \ref{int_pos_nat_ex}, there
    exist natural numbers $m$ and $n$ such that $a = m$ and $b = n$.  Then the
    result simplifies to
    \[
        0 + m0 + 0n \leq 0 + mn + (0)(0).
    \]
    The left-hand side simplifies to zero, and because all natural numbers are
    positive, the result is true.
\end{proof}

\begin{theorem}
    For all $a_1$, $a_2$, $b_1$, and $b_2$, $[(a_1, a_2)] < [(b_1, b_2)]$ if and
    only if $a_1 + b_2 < a_2 + b_1$.
\end{theorem}
\begin{proof}
    Unfold the definitions of $<$ and $\leq$, after using commutativity the two
    expressions are identical.
\end{proof}

\begin{theorem} \label{int_lt_suc_le} \label{int_le_suc_lt}
        \label{int_le_pre_lt} \label{int_lt_pre_le}
    For all $a$ and $b$, $a < b + 1$ if and only if $a \leq b$.
\end{theorem}
\begin{proof}
    Simplified, we must prove that $a_1 + b_2 < a_2 + b_1 + 1$ if and only if
    $a_1 + b_2 \leq a_2 + b_1$.  This follows directly from Theorem
    \ref{nat_lt_suc_le}
\end{proof}

Note that the previous theorem also gives other similar results involving
relating inequalities and strict inequalities with a $+ 1$ or $-1$ somewhere.

\begin{theorem} \label{int_pos_nat1_ex}
    For all $a : \Z$, if $0 < a$, then there exists an $n : \N$ such than $a =
    S(n)$.
\end{theorem}
\begin{proof}
    By Theorem \ref{int_pos_nat_ex}, there exists an $n : \N$ such that $a = n$.
    If $n = 0$, because $0 < a$, we would have $0 < 0$, which is a
    contradiction.  Thus, $n \neq 0$, so it is the successor of another natural
    number $m$, meaning that $a = S(m)$.
\end{proof}

\begin{theorem} \label{int_neg_nat1_ex}
    For all $a : \Z$, if $0 < a$, then there exists an $n : \N$ such than $a =
    -S(n)$.
\end{theorem}
\begin{proof}
    The proof is similar in form to Theorem \ref{int_neg_nat_ex}.
\end{proof}

\begin{instance}
    The integers are Archimedean.
\end{instance}
\begin{proof}
    We must prove that if $0 < x$ and $0 < y$, then there exists an $n : \N$
    such that $x < ny$.  Because $0 < y$, $1 \leq y$, and we can multiply by $x$
    to get $x \leq xy$.  Because $0 < y$, we can add it to the right-hand side
    to get $x < y + xy = (x + 1)y$.  By Theorem \ref{int_pos_nat1_ex}, there
    exists some natural number $n$ such that $x = n + 1$, so $x < (n + 2)y$,
    showing that $n + 2$ suffices.
\end{proof}

\section{The Relation Between Integers And Other Types}

In this whole section, let $\U$ be a type with whatever algebraic operations are
needed.  We will be exploring several ways to connect $\U$ with $\Z$.

\subsection{Multiplying by Integers}

\begin{definition}
    Given an $(n_1, n_2) : \N \times \N$ and an $a : \U$, define the product
    \[
        (n_1, n_2)a = n_1a - n_2a.
    \]
\end{definition}

\begin{lemma}
    The product in the preceding definition is well-defined under the
    equivalence relation $\sim$.
\end{lemma}
\begin{proof}
    We must prove that if $a_1 + b_2 = b_1 + a_2$, then
    \[
        a_1c - a_2c = b_1c - b_2c.
    \]
    We can multiply $a_1 + b_2 = b_1 + a_2$ by $c$ to get
    \[
        a_1c + b_2c = b_1c + a_2c,
    \]
    and rearranging we get the desired result.
\end{proof}

\begin{definition}
    Given a value $a : \U$, define the product of an integer $n$ with $a$ to be
    the operation given by Theorem \ref{unary_op_ex} and the previous lemma.
\end{definition}

\begin{theorem}
    For all $a : \U$, $0_\Z a = 0_\U$.
\end{theorem}
\begin{proof}
    \[
        0_\Z a = 0_\N a - 0_\N a = 0_\U - 0_\U = 0_\U.
    \]
\end{proof}

\begin{theorem}
    For all $a : \Z$, $a0_\U = 0_\U$.
\end{theorem}
\begin{proof}
    \[
        a0_\U = a_10_\U - a_20_\U = 0_\U - 0_\U = 0_\U.
    \]
\end{proof}

\begin{theorem}
    For al $a : \U$, $1_\Z a = a$.
\end{theorem}
\begin{proof}
    \[
        1_\Z a = 1_\N a - 0_\N a = a.
    \]
\end{proof}

\begin{theorem}
    For all $a : \Z$ and $b$ and $c$ in $\U$, if $b + c = c + b$, then
    \[
        a(b + c) = ab + ac.
    \]
\end{theorem}
\begin{proof}
    We must prove that for all natural numbers $a_1$ and $a_2$,
    \[
        a_1(b + c) - a_2(b + c) = a_1b - a_2b + a_1c - a_2c.
    \]
    Because $b$ and $c$ commute, we can expand and rearrange by the theorems for
    the product of natural numbers with $\U$.
\end{proof}

\begin{theorem}
    For all $a$ and $b$ in $\Z$ and $c : \U$,
    \[
        (a + b)c = ac + bc.
    \]
\end{theorem}
\begin{proof}
    We must prove that for all natural numbers $a_1$, $a_2$, $b_1$, and $b_2$,
    \[
        (a_1 + b_1)c - (a_2 + b_2)c = a_1c - a_2 c + b_1c - b_2c.
    \]
    Again, this follows directly by the theorems for the product of natural
    numbers with $\U$.
\end{proof}

\begin{theorem}
    For all $a$ and $b$ in $\Z$ and $c : \U$
    \[
        a(bc) = (ab)c.
    \]
\end{theorem}
\begin{proof}
    We must prove that for all natural numbers $a_1$, $a_2$, $b_1$, and $b_2$,
    \[
        a_1(b_1c - b_2c) - a_2(b_1c - b_2c) =
        (a_1b_1 + a_2b_2)c - (a_1b_2 + a_2b_1)c.
    \]
    Once again, this follows directly by the theorems for the product of natural
    numbers with $\U$.
\end{proof}

\begin{theorem}
    For all integers $a$ and $b$ and $c$ in $\U$,
    \[
        a(bc) = (ab)c.
    \]
\end{theorem}
\begin{proof}
    We must prove that for all natural numbers $a_1$ and $a_2$,
    \[
        a_1(bc) - a_2(bc) =
        (a_1b - a_2b)c.
    \]
    After distributing, this follows directly by the theorems for the product of
    natural numbers with $\U$.
\end{proof}

\begin{theorem}
    For all $a : \Z$ and $b : \U$,
    \[
        -(ab) = (-a)b.
    \]
\end{theorem}
\begin{proof}
    We must prove that for all natural numbers $a_1$ and $a_2$,
    \[
        -(a_1b - a_2b) = a_2b - a_1b.
    \]
    This follows directly from the properties of addition.
\end{proof}

\begin{theorem}
    For all $a : \Z$ and $b : \U$,
    \[
        -(ab) = a(-b).
    \]
\end{theorem}
\begin{proof}
    We must prove that for all natural numbers $a_1$ and $a_2$,
    \[
        -(a_1b - a_2b) = a_1(-b) - a_2(-b).
    \]
    After simplifying, this follows directly by the theorems for the product of
    natural numbers with $\U$.
\end{proof}

\begin{theorem}
    For all $a : \Z$ and $b$ and $c$ in $\U$, if $b + c = c + b$,
    \[
        ab + c = c + ab.
    \]
\end{theorem}
\begin{proof}
    We must prove that for all natural numbers $a_1$ and $a_2$,
    \[
        a_1b - a_2b + c = c + a_1b - a_2b.
    \]
    Because $b$ and $c$ commute, by the properties of multiplying natural
    numbers with values in $\U$, all these terms commute.
\end{proof}

\begin{theorem}
    For all $a$ and $b$ in $\Z$ and all $c : \U$,
    \[
        ac + bc = bc + ac.
    \]
\end{theorem}
\begin{proof}
    \[
        ac + bc = (a + b)c = (b + a)c = bc + ac.
    \]
\end{proof}

\subsection{The Integers in Other Types}

\end{document}
