\documentclass[../../math.tex]{subfiles}
\externaldocument{../../math.tex}
\externaldocument{../basics/elementary_algebra}
\externaldocument{integer}

\begin{document}

\setcounter{chapter}{11}

\chapter{The Complex Numbers}

I haven't really done much with the complex numbers.  I wasn't even going to
include them at first, but then once I wanted to try something with them so I
quickly did the basic construction, but that's it.  So here's the basic
construction.

\begin{definition}
    Define the complex numbers $\mathbb C$ to be the type $\mathbb R \times
    \mathbb R$.  We will write a complex number $(a, b)$ as $a + bi$, with $i$
    in this case being a purely formal notational device.  Define a canonical
    injection $\iR : \mathbb R \to \mathbb C$ given by $\iR(x) = x + 0i$.
\end{definition}

Like all canonical injections, $\iR$ will be dropped unless it is needed.

\section{Addition}

\begin{instance}
    Define addition of complex numbers by
    \[
        (a_1 + a_2i) + (b_1 + b_2i) = (a_1 + b_1) + (a_2 + b_2)i.
    \]
\end{instance}

\begin{instance}
    Addition of complex numbers is commutative.
\end{instance}
\begin{proof}
    \begin{align*}
        (a_1 + a_2i) + (b_1 + b_2i)
        &= (a_1 + b_1) + (a_2 + b_2)i \\
        &= (b_1 + a_1) + (b_2 + a_2)i \\
        &= (b_1 + b_2i) + (a_1 + a_2i).
    \end{align*}
\end{proof}

\begin{instance}
    Addition of complex numbers is associative.
\end{instance}
\begin{proof}
    \begin{align*}
        (a_1 + a_2i) + ((b_1 + b_2i) + (c_1 + c_2i))
        &= (a_1 + a_2i) + ((b_1 + c_1) + (b_2 + c_2)i) \\
        &= (a_1 + (b_1 + c_1)) + (a_2 + (b_2 + c_2))i \\
        &= ((a_1 + b_1) + c_1) + ((a_2 + b_2) + c_2)i \\
        &= ((a_1 + b_1) + (a_2 + b_2)i) + (c_1 + c_2i) \\
        &= ((a_1 + a_2i) + (b_1 + b_2)i) + (c_1 + c_2i).
    \end{align*}
\end{proof}

\begin{instance}
    Define zero in the complex numbers to be $0 + 0i$.
\end{instance}

\begin{instance}
    Zero is an additive identity in the complex numbers.
\end{instance}
\begin{proof}
    \[
        0 + a
        = (0 + 0i) + (a_1 + a_2i)
        = (0 + a_1) + (0 + a_2)i
        = a_1 + a_2i = a.
    \]
\end{proof}

\begin{instance}
    Define negation in the complex numbers as $-(a_1 + a_2i) = (-a_1) +
    (-a_2)i$.
\end{instance}

\begin{instance}
    Negation of complex numbers is an additive inverse.
\end{instance}
\begin{proof}
    \[
        -a + a
        = ((-a_1) + (-a_2)i) + (a_1 + a_2i)
        = (-a_1 + a_1) + (-a_2 + a_2)i
        = 0 + 0i
        = 0.
    \]
\end{proof}

\begin{instance}
    $\iR$ is additive.
\end{instance}
\begin{proof}
    \[
        \iR(a + b) = (a + b) + 0i = (a + 0i) + (b + 0i) = \iR(a) + \iR(b).
    \]
\end{proof}

\section{Multiplication}

\begin{instance}
    Define multiplication of complex numbers as
    \[
        (a_1 + a_2i)(b_1 + b_2i) = (a_1b_1 - a_2b_2) + (a_1b_2 + a_2 b_1)i.
    \]
\end{instance}

\begin{instance}
    Multiplication of complex numbers is commutative.
\end{instance}
\begin{proof}
    \begin{align*}
        (a_1 + a_2i)(b_1 + b_2i)
        &= (a_1b_1 - a_2b_2) + (a_1b_2 + a_2 b_1)i \\
        &= (b_1a_1 - b_2a_2) + (b_1a_2 + b_2 a_1)i \\
        &= (b_1a_1 - b_2a_2) + (b_2 a_1 + b_1a_2)i \\
        &= (b_1 + b_2i)(a_1 + a_2i).
    \end{align*}
\end{proof}

\begin{instance}
    Multiplication of complex numbers is associative.
\end{instance}
\begin{proof}
    \begin{align*}
        &\phantom{{}={}} a(bc) \\
        &= (a_1 + a_2i)((b_1 + b_2i)(c_1 + c_2i)) \\
        &= (a_1 + a_2i)((b_1 c_1 - b_2 c_2) + (b_1 c_2 + b_2 c_1)i) \\
        &= (a_1 b_1 c_1 - a_1 b_2 c_2 - a_2 b_1 c_2 - a_2 b_2 c_1) +
            (a_1 b_1 c_2 + a_1 b_2 c_1 + a_2 b_1 c_1 - a_2 b_2 c_2)i \\
        &= (a_1 b_1 c_1 - a_2 b_2 c_1 - a_1 b_2 c_2 - a_2 b_1 c_2) +
            (a_1 b_1 c_2 - a_2 b_2 c_2 + a_1 b_2 c_1 + a_2 b_1 c_1)i \\
        &= ((a_1 b_1 - a_2 b_2) + (a_1 b_2 + a_2 b_1)i)(c_1 + c_2i) \\
        &= ((a_1 + a_2i)(b_1 + b_2i))(c_1 + c_2i) \\
        &= (ab)c.
    \end{align*}
\end{proof}

\begin{instance}
    Multiplication of complex numbers is distributive.
\end{instance}
\begin{proof}
    \begin{align*}
    &\phantom{{}={}} a(b + c) \\
    &= (a_1 + a_2i)((b_1 + b_2i) + (c_1 + c_2i)) \\
    &= (a_1 + a_2i)((b_1 + c_1) + (b_2 + c_2)i) \\
    &= (a_1 b_1 + a_1 c_1 - a_2 b_2 - a_2 c_2) +
        (a_1 b_2 + a_1 c_2 + a_2 b_1 + a_2 c_1)i \\
    &= (a_1 b_1 - a_2 b_2 + a_1 c_1 - a_2 c_2) +
        (a_1 b_2 + a_2 b_1 + a_1 c_2 + a_2 c_1)i \\
    &= ((a_1 b_1 - a_2 b_2) + (a_1 b_2 + a_2 b_1)i) +
        ((a_1 c_2 - a_2 c_2) + (a_1 c_1 + a_2 c_1)i) \\
    &= (a_1 + a_2i)(b_1 + b_2i) + (a_1 + a_2i)(c_1 + c_2i) \\
    &= ab + ac.
    \end{align*}
\end{proof}

\begin{instance}
    Define one in the complex numbers to be $1 + 0i$.
\end{instance}

\begin{instance}
    The complex numbers are not trivial.
\end{instance}
\begin{proof}
    If $0 + 0i = 1 + 0i$, we would have $0 = 1$ in the real numbers, which is
    impossible.
\end{proof}

\begin{instance}
    One is a multiplicative identity in the complex numbers.
\end{instance}
\begin{proof}
    \[
        1a = (1 + 0i)(a_1 + a_2i) = (1 a_1 - 0a_2) + (1 a_2 + 0 a_1)i
        = a_1 + a_2i = a.
    \]
\end{proof}

\begin{instance}
    Define multiplicative inverses in the complex numbers using
    \[
        (a_1 + a_2i)^{-1}
        = \frac{a_1}{a_1^2 + a_2^2} - \frac{a_2}{a_1^2 + a_2^2} i.
    \]
\end{instance}

\begin{instance}
    Reciprocals are multiplicative inverses in the complex numbers.
\end{instance}
\begin{proof}
    Let $a \neq 0$.  Before we move on to the calculation, we will prove that $0
    \neq a_1^2 + a_2^2$.  If $0 = a_1^2 + a_2^2$, we would have $a_1^2 =
    -a_2^2$, and since $a_1^2 \geq 0$, we would have $-a_2^2 \geq 0$.  This
    means that $a_2^2 \leq 0$, and since $0 \leq a_2^2$, we have $a_2^2 = 0$,
    so by Theorem \ref{square_nz} we have $a_2 = 0$.  The same argument shows
    that $a_1 = 0$.  This would mean that $a = 0$, contradicting $a \neq 0$.
    Thus, we must have $a_1^2 + a_2^2 \neq 0$.  Then
    \begin{align*}
        a^{-1}a
        &= \left( \frac{a_1}{a_1^2 + a_2^2} - \frac{a_2}{a_1^2 + a_2^2}i \right)
            (a_1 + a_2i) \\
        &= \left( \frac{a_1^2}{a_1^2 + a_2^2} +
                \frac{a_2^2}{a_1^2 + a_2^2} \right) +
            \left( -\frac{a_1a_2}{a_1^2 + a_2^2} +
                \frac{a_1a_2}{a_1^2 + a_2^2} \right)i \\
        &= \frac{a_1^2 + a_2^2}{a_1^2 + a_2^2} + 0i \\
        &= 1 + 0i \\
        &= 1
    \end{align*}
\end{proof}

\begin{instance}
    $\iR$ is multiplicative.
\end{instance}
\begin{proof}
    \[
        \iR(ab)
        = ab + 0i
        = (ab - 0^2) + (a0 + 0b)i
        = (a + 0i)(b + 0i)
        = \iR(a) \iR(b).
    \]
\end{proof}

\begin{instance}
    $\iR$ is unital.
\end{instance}
\begin{proof}
    \[
        \iR(1) = 1 + 0i = 1.
    \]
\end{proof}

Thus, $\iR$ is a field homomorphism, and in particular, it is injective.

\begin{theorem}
    For all natural numbers $n$, $\iR(\iN(n)) = \iN(n)$.
\end{theorem}
\begin{proof}
    We will prove the result by induction.  When $n = 0$, the result follows
    from $\iR$ and $\iN$ being nullitive.  When $\iR(\iN(n)) = \iN(n)$,
    \[
        \iR(\iN(n + 1)) = \iR(\iN(n)) + \iR(\iN(1)) = \iN(n) + 1 = \iN(n + 1).
    \]
\end{proof}

\begin{instance}
    The complex numbers are characteristic zero.
\end{instance}
\begin{proof}
    Because the real numbers are characteristic zero, we have $0 \neq \iN(n+1)$
    in the real numbers.  Because $\iR$ is injective, we can apply it to both
    sides of the inequality to get $\iR(0) \neq \iR(\iN(n + 1))$, producing $0
    \neq \iN(n + 1)$ in the complex numbers.
\end{proof}

\begin{theorem}
    For all integers $n$, $\iR(\iZ(n)) = \iZ(n)$.
\end{theorem}
\begin{proof}
    By Theorem \ref{int_nat_ex}, we have natural numbers $a$ and $b$ such that
    $n = \iN(a) - \iN(b)$.  Then
    \begin{align*}
        \iR(\iZ(n))
        &= \iR(\iZ(\iN(a) - \iN(b))) \\
        &= \iR(\iZ(\iN(a)) - \iZ(\iN(b))) \\
        &= \iR(\iN(a) - \iN(b)) \\
        &= \iR(\iN(a)) - \iR(\iN(b)) \\
        &= \iN(a) - \iN(b) \\
        &= \iZ(n).
    \end{align*}
\end{proof}

\begin{theorem}
    For all rational numbers $n$, $\iR(\iQ(n)) = \iQ(n)$.
\end{theorem}
\begin{proof}
    We have $n = \iZ(a)/\iZ(b)$ with $b \neq 0$.  Then
    \begin{align*}
        \iR(\iQ(n))
        &= \iR\left( \iQ\left( \frac{\iZ(a)}{\iZ(b)} \right) \right) \\
        &= \iR\left( \frac{\iQ(\iZ(a))}{\iQ(\iZ(b))} \right) \\
        &= \iR\left( \frac{\iZ(a)}{\iZ(b)} \right) \\
        &= \frac{\iR(\iZ(a))}{\iR(\iZ(b))} \\
        &= \frac{\iZ(a)}{\iZ(b)} \\
        &= \iQ(n).
    \end{align*}
\end{proof}

\end{document}
