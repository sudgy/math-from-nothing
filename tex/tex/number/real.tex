\documentclass[../../math.tex]{subfiles}
\externaldocument{../../math.tex}
\externaldocument{../basics/elementary_algebra}
\externaldocument{../basics/natural}
\externaldocument{../basics/set}

\begin{document}

\setcounter{chapter}{10}

\chapter{The Real Numbers}

Constructing the real numbers is much more complicated than it initially seems.
I have gone through several constructions in the Coq code trying to find the
simplest one.  Of all of the constructions I have tried, the simplest one was
the one using Cauchy sequences of rationals, so that is the one I will be
presenting in this chapter.  The other constructions I have done will be covered
in a future chapter.  Because I have swapped constructions before and am open to
doing it again, after this chapter the particular construction used will not be
mentioned outside of this chapter.  The only important things will be developed
in the last section.

One may wonder, why are we using Cauchy sequences when we haven't done any
analysis yet?  Shouldn't we do analysis first to get the definition of Cauchy
sequences?  The issue here is that the definition of a metric space depends on
the real numbers.  One could define metric spaces where the metric has values in
an arbitrary ordered field, but this approach isn't really that useful and ends
up introducing a great deal of complexity.  Thus, all of the ideas in analysis
that we will need for this chapter will be introduced here using the rational
numbers and then never used again outside of this chapter.  Until the last
section, none of definitions or theorems in this chapter will be used anywhere
else, and some future definitions will even have the same names as in this
chapter despite being distinct.

Throughout this chapter, we will be using sequences of rational numbers
extensively.  Recall that a sequence of rational numbers is function $a$ from
$\N$ to $\Q$.  Instead of writing $a(i)$ to denote the $i$th element of the
sequence, we will use $a_i$.  We will often refer to the sequence as $a_n$
rather than just $a$, to distinguish real numbers from sequences.

\section{Basic Construction}

\begin{definition}
    Let $a_n$ be a sequence of rational numbers.  Then we say that $a_n$ is a
    Cauchy sequence if for all $\varepsilon > 0$, there exists a natural number
    $N$ such that for all $i$ and $j$ with $N \leq i$ and $N \leq j$, we have
    $|a_i - a_j| < \varepsilon$.
\end{definition}

\begin{definition}
    Define a relation $\sim$ on Cauchy sequences by saying that $a_n \sim b_n$
    if for all $\varepsilon > 0$, there exists a natural number $N$ such that
    for all $i \geq N$, we have $|a_i - b_i| < \varepsilon$.
\end{definition}

\begin{lemma} \label{real_eq_zero}
    For all sequences $a_n$ and $b_n$, if $a_i - b_i = 0$ for all $i : \N$, then
    $a_n \sim b_n$.
\end{lemma}
\begin{proof}
    Let $\varepsilon > 0$.  Then picking $N = 0$, for all $i \geq N$, we have
    \[
        |a_i - b_i| = 0 < \varepsilon.
    \]
\end{proof}

\begin{lemma}
    The relation $\sim$ is an equivalence relation on Cauchy sequences.
\end{lemma}
\begin{proof}
    \textit{Reflexivity}.  Let $a_n$ be a Cauchy sequence.  Then $a_i - a_i = 0$
    for all $i$, so $\sim$ is reflexive by Lemma \ref{real_eq_zero}.

    \textit{Symmetry}.  Let $a_n$ and $b_n$ be Cauchy sequences with $a_n \sim
    b_n$.  Let $\varepsilon > 0$.  Because $a_n \sim b_n$, we have an $N$ such
    that for $i \geq N$, we have $|a_i - b_i| < \varepsilon$.  By Theorem
    \ref{abs_minus}, this means that for all $i \geq N$, we have $|b_i - a_i| <
    \varepsilon$, showing that $b_n \sim a_n$.

    \textit{Transitivity}.  Let $a_n$, $b_n$, and $c_n$ be Cauchy sequences with
    $a_n \sim b_n$ and $b_n \sim c_n$.  Let $\varepsilon > 0$.  Then
    $\frac{\varepsilon}{2} > 0$, so by $a_n \sim b_n$ and $b_n \sim c_n$ we have
    a natural number $N_1$ such that for all $i \geq N_1$ we have $|a_i - b_i| <
    \frac{\varepsilon}{2}$ and a natural number $N_2$ such that for all $i \geq
    N_2$ we have $|b_i - c_i| < \frac{\varepsilon}{2}$.  Setting $N = \max(N_1,
    N_2)$, for all $i \geq N$, we have both inequalities.  Then we have
    \begin{align*}
        |a_i - c_i|
        &= |(a_i - b_i) + (b_i - c_i)| && \\
        &\leq |a_i - b_i| + |b_i - c_i| && \text{by Theorem \ref{abs_tri}} \\
        &< \frac{\varepsilon}{2} + \frac{\varepsilon}{2} && \\
        &= \varepsilon. &&
    \end{align*}
\end{proof}

\begin{definition}
    Define the type $\R$ of real numbers to be the quotient of Cauchy sequences
    by $\sim$.  Given a Cauchy sequence $a_n$, we will often use just $a$ to
    represent the equivalence class of $a_n$ under this quotient.
\end{definition}

\begin{lemma} \label{rat_to_real_cauchy}
    Constant sequences are Cauchy.
\end{lemma}
\begin{proof}
    For some constant rational number $q$, let $q_n = q$ for all $n : \N$.  Then
    for all $\varepsilon > 0$, setting $N = 0$, for all $i \geq 0$ and $j \geq
    0$ we have $|q_i - q_j| = |q - q| = 0 < \varepsilon$, so $a_n$ is Cauchy.
\end{proof}

\begin{definition}
    Given a rational number $q$, we can consider it to be a real number by
    turning it into the constant sequence $q_n$ (which we know is Cauchy by the
    previous lemma) and then taking the equivalence class of it.  We will often
    refer to the real number as $q$ as well.
\end{definition}

This induces a function from the rationals to the reals that is in fact equal to
$\iQ$.  However, by the time we can apply $\iQ$ to $\R$, we will be so far along
in the construction that we will never need to prove this.  Again, after this
chapter, this construction won't be used anymore, and with it this way of
turning rational numbers to real numbers.

\section{Addition}

\begin{definition}
    Define a binary operation $\oplus$ on Cauchy sequences that takes two Cauchy
    sequences $a_n$ and $b_n$ and produces a new sequence given by $a_i + b_i$.
\end{definition}

\begin{lemma}
    If $a_n$ and $b_n$ are Cauchy sequences, then $a_i \oplus b_i$ is a Cauchy
    sequence.
\end{lemma}
\begin{proof}
    Let $\varepsilon > 0$.  Then $\frac{\varepsilon}{2} > 0$, and by $a_n$ being
    Cauchy we have a natural number $N_1$ such that $|a_i - a_j| <
    \frac{\varepsilon}{2}$ for all $i \geq N_1$ and $j \geq N_1$, and by $b_n$
    being Cauchy we have a natural number $N_2$ such that $|b_i - b_j| <
    \frac{\varepsilon}{2}$ for all $i \geq N_2$ and $j \geq N_2$.  Set $N =
    \max(N_1, N_2)$.  Then for all $i \geq N$ and $j \geq N$, we have both
    inequalities.  Thus,
    \begin{align*}
        |(a_n \oplus b_n)_i - (a_n \oplus b_n)_j|
        &= |(a_i + b_i) - (a_j + b_j)| && \\
        &= |(a_i - a_j) + (b_i - b_j)| && \\
        &\leq |a_i - a_j| + |b_i - b_j| && \text{by Theorem \ref{abs_tri}} \\
        &< \frac{\varepsilon}{2} + \frac{\varepsilon}{2} && \\
        &= \varepsilon.
    \end{align*}
\end{proof}

\begin{lemma}
    The operation $\oplus$ is well-defined under the equivalence relation
    $\sim$.
\end{lemma}
\begin{proof}
    Let $a_n$, $b_n$, $c_n$, and $d_n$ be Cauchy sequences such that $a_n \sim
    b_n$ and $c_n \sim d_n$.  We must prove that $a_n \oplus c \sim b \oplus d$.
    Let $\varepsilon > 0$.  Then $\frac{\varepsilon}{2} > 0$, so by $a_n \sim
    b_n$ we have a natural number $N_1$ such that $|a_i - b_i| <
    \frac{\varepsilon}{2}$ for all $i \geq N_1$, and by $c_n \sim d_n$ we have a
    natural number $N_2$ such that $|c_i - d_i| < \frac{\varepsilon}{2}$ for all
    $i \geq N_2$.  Setting $N = \max(N_1, N_2)$, we get both inequalities for
    all $i \geq N$.  Then
    \begin{align*}
        |(a_n \oplus c_n)_i - (b_n \oplus d_n)_i|
        &= |(a_i + c_i) - (b_i + d_i)| \\
        &= |(a_i - b_i) + (c_i - d_i)| \\
        &\leq |a_i - b_i| + |c_i - d_i| \\
        &< \frac{\varepsilon}{2} + \frac{\varepsilon}{2} \\
        &= \varepsilon.
    \end{align*}
\end{proof}

\begin{instance}
    Define addition of real numbers as the binary operation given by Theorem
    \ref{binary_op_ex} and the previous lemma.
\end{instance}

\begin{instance}
    Addition of real numbers is associative.
\end{instance}
\begin{proof}
    Let $a_n$, $b_n$, and $c_n$ be Cauchy sequences.  Using Lemma
    \ref{real_eq_zero},
    \[
        (a_i + (b_i + c_i)) - ((a_i + b_i) + c_i)
        = (a_i + (b_i + c_i)) - (a_i + (b_i + c_i))
        = 0.
    \]
\end{proof}

\begin{instance}
    Addition of real numbers is commutative.
\end{instance}
\begin{proof}
    Let $a_n$ and $b_n$ be Cauchy sequences.  Using Lemma \ref{real_eq_zero},
    \[
        (a_i + b_i) - (b_i + a_i)
        = (a_i + b_i) - (a_i + b_i)
        = 0.
    \]
\end{proof}

\begin{instance}
    Define $0$ in the real numbers to be the real version of the rational $0$,
    that is, the equivalence class of the constant sequence $0$.
\end{instance}

\begin{instance}
    $0$ is an additive identity in $\R$.
\end{instance}
\begin{proof}
    Let $a_n$ be a Cauchy sequence.  Using Lemma \ref{real_eq_zero},
    \[
        (0 + a_i) - a_i
        = a_i - a_i
        = 0.
    \]
\end{proof}

\begin{definition}
    Define an operation $\ominus$ on Cauchy sequences that takes a Cauchy
    sequence $a_n$ and produces a new sequence given by $-a_i$.
\end{definition}

\begin{lemma}
    For all Cauchy sequences $a_n$, $\ominus a_n$ is a Cauchy sequence.
\end{lemma}
\begin{proof}
    Let $\varepsilon > 0$.  Because $a_n$ is Cauchy, there exists an $N$ such
    that $|a_i - a_j| < \varepsilon$ for all $i \geq N$ and $j \geq N$.  Then
    \[
        |(\ominus a_n)_i - (\ominus a_n)_j|
        = |{-}a_i - -a_j|
        = |{-}(a_i - a_j)|
        = |a_i - a_j|
        < \varepsilon.
    \]
\end{proof}

\begin{lemma}
    The operation $\ominus$ is well-defined under the equivalence relation
    $\sim$.
\end{lemma}
\begin{proof}
    Let $a_n$ and $b_n$ be Cauchy sequences with $a_n \sim b_n$.  We must prove
    that $\ominus a_n \sim \ominus b_n$.  Let $\varepsilon > 0$.  Then because
    $a_n \sim b_n$, there exists an $N$ such that $|a_i - b_i| < \varepsilon$
    for all $i \geq N$.  Then
    \[
        |(\ominus a_n)_i - (\ominus b_n)_i|
        = |{-}a_i - -b_i|
        = |{-}(a_i - b_i)|
        = |a_i - b_i|
        < \varepsilon.
    \]
\end{proof}

\begin{instance}
    Define additive inverses in the real numbers as the operation given by
    Theorem \ref{unary_op_ex} and the previous lemma.
\end{instance}

\begin{instance}
    Negation is an additive inverse in $\R$.
\end{instance}
\begin{proof}
    Let $a_n$ be a Cauchy sequence.  Using Lemma \ref{real_eq_zero},
    \[
        (-a_i + a_i) - 0 = 0 - 0 = 0.
    \]
\end{proof}

\section{Multiplication}

Before moving on to multiplication itself, we need a few lemmas.

\begin{lemma} \label{real_cauchy_bounded}
    For all Cauchy sequences $a_n$, there exists a rational number $M$ such that
    $|a_i| < M$ for all $i : \N$.
\end{lemma}
\begin{proof}
    By $a_n$ being Cauchy, since $1 > 0$, we have a natural number $N$ such that
    for all $i \geq N$ and $j \geq N$, we have $|a_i - a_j| < 1$.  Let $S$ be
    the set $\{q : \Q \mid \exists i < N, |a_i| = q\}$.  We will prove that $S$
    has an upper bound.

    First, if $N = 0$, $S$ is empty, so $0$ works as an upper bound.  When $N
    \neq 0$, notice that $S$, being the image of a function from a finite set,
    is finite.  Furthermore, because $N \neq 0$, $|a_0| \in S$, so $s$ is
    nonempty.  Thus, by Theorem \ref{simple_finite_max}, $S$ has a maximum
    element, which is an upper bound of $S$.

    Thus, $S$ has an upper bound $m$.  Let $M = \max(m + 1, |a_N| + 1)$.  We
    will show that $|a_i| < M$ for all $i : \N$.  There will be two cases: when
    $i < N$, and when $i \geq N$.  When $i < N$,
    \[
        |a_i| \leq m < m + 1 \leq \max(m + 1, |a_N| + 1).
    \]
    When $i \geq N$, we have $|a_i - a_N| < 1$ and by Theorem
    \ref{abs_reverse_tri2}, we have $|a_i| - |a_N| < 1$, so
    \[
        |a_i| < |a_N| + 1 \leq \max(m + 1, |a_N| + 1).
    \]
    Either way, $|a_i| < M$.
\end{proof}

\begin{lemma} \label{real_cauchy_nz}
    For all Cauchy sequences $a_n$, if $0 \neq a$, then there exists a rational
    number $\varepsilon > 0$ and a natural number $N$ such that for all $i \geq
    N$, we have $\varepsilon \leq |a_i|$.
\end{lemma}
\begin{proof}
    Because $0 \neq a_n$, we have some $\varepsilon > 0$ such that for all
    natural numbers $N$, there exists a $j \geq N$ with
    \begin{equation} \label{real_cauchy_nz_1}
        |a_j - 0| = |a_j| \geq \varepsilon.
    \end{equation}
    Because $a_n$ is Cauchy and $\frac{\varepsilon}{2} > 0$, there exists a
    natural number $N$ such that for all $i \geq N$ and $j \geq N$, we have
    \begin{equation} \label{real_cauchy_nz_2}
        |a_i - a_j| < \frac{\varepsilon}{2}.
    \end{equation}
    Let $j$ be the natural number greater than or equal to $N$ satisfying
    (\ref{real_cauchy_nz_1}).  Then for all $i \geq N$, by Theorem
    \ref{abs_reverse_tri2} and (\ref{real_cauchy_nz_2}) we have
    \begin{gather*}
        |a_j| - |a_i| \leq |a_j - a_i| < \frac{\varepsilon}{2} \\
        \implies |a_j| < \frac{\varepsilon}{2} + |a_i|,
    \end{gather*}
    and by transitivity with (\ref{real_cauchy_nz_1}) we get
    \begin{align*}
        \varepsilon &< \frac{\varepsilon}{2} + |a_i| \\
        \frac{\varepsilon}{2} &< |a_i|,
    \end{align*}
    showing that for all $i \geq N$, we have $\frac{\varepsilon}{2} < |a_i|$ as
    required.
\end{proof}

\begin{definition}
    Define a binary operation $\otimes$ on Cauchy sequences that takes two
    Cauchy sequences $a_n$ and $b_n$ and produces a new sequence given by $a_i
    b_i$.
\end{definition}

\begin{lemma}
    If $a_n$ and $b_n$ are Cauchy sequences, then $a_n \otimes b_n$ is as well.
\end{lemma}
\begin{proof}
    By Lemma \ref{real_cauchy_bounded}, we have rational numbers $M_1$ and $M_2$
    with $|a_i| < M_1$ and $|b_i| < M_2$ for all $i : \N$.  Since the absolute
    value is always positive, $M_1$ and $M_2$ are always greater than zero.  Now
    let $\varepsilon > 0$.  Then $\frac{\varepsilon}{2M_2} > 0$, so by $a_n$
    being Cauchy we have a natural number $N_1$ such that for all $i \geq N_1$
    and $j \geq N_1$, we have $|a_i - a_j| < \frac{\varepsilon}{2M_2}$.
    Similarly, because $\frac{\varepsilon}{2M_1} > 0$, we have a natural number
    $N_2$ such that for all $i \geq N_2$ and $j \geq N_2$, we have $|b_i - b_j|
    < \frac{\varepsilon}{2M_1}$.  Now define $N = \max(N_1, N_2)$.  Then for all
    $i \geq N$ and $j \geq N$,
    \begin{align*}
        |a_i b_i - a_j b_j|
        &= |a_i b_i - a_i b_j + a_i b_j - a_j b_j| \\
        &= |a_i (b_i - b_j) + b_j (a_i - a_j)| \\
        &\leq |a_i (b_i - b_j)| + |b_j (a_i - a_j)| \\
        &= |a_i|\, |b_i - b_j| + |b_j|\, |a_i - a_j| \\
        &< M_1 |b_i - b_j| + M_2 |b_i - b_j| \\
        &< M_1 \frac{\varepsilon}{2M_1} + M_2 \frac{\varepsilon}{2M_2} \\
        &= \varepsilon.
    \end{align*}
\end{proof}

\begin{lemma}
    The operation $\otimes$ is well-defined under the equivalence relation
    $\sim$.
\end{lemma}
\begin{proof}
    Let $a_n$, $b_n$, $c_n$, and $d_n$ be Cauchy sequences with $a_n \sim b_n$
    and $c_n \sim d_n$.  We must prove that $a_n \otimes c_n \sim b_n \otimes
    d_n$.  By Lemma \ref{real_cauchy_bounded}, we have rational numbers $M_1$
    and $M_2$ with $|b_i| < M_1$ and $|c_i| < M_2$ for all $i : \N$.  Since the
    absolute value is always positive, $M_1$ and $M_2$ are always greater than
    zero.  Now let $\varepsilon > 0$.  Then $\frac{\varepsilon}{2M_2} > 0$, so
    by $a_n \sim b_n$ we have a natural number $N_1$ such that for all $i \geq
    N_1$, we have $|a_i - b_i| < \frac{\varepsilon}{2M_2}$.  Similarly,
    $\frac{\varepsilon}{2M_1} > 0$, so by $c_n \sim d_n$ we have a natural
    number $N_2$ such that for all $i \geq N_2$, we have $|c_i - d_i| <
    \frac{\varepsilon}{2M_1}$.  Set $N = \max(N_1, N_2)$.  Then for all $i \geq
    N$,
    \begin{align*}
        |a_i c_i - b_i d_i|
        &= |a_i c_i - b_i c_i + b_i c_i - b_i d_i| \\
        &= |c_i (a_i - b_i) + b_i (c_i - d_i)| \\
        &\leq |c_i (a_i - b_i)| + |b_i (c_i - d_i)| \\
        &= |c_i|\, |a_i - b_i| + |b_i|\, |c_i - d_i| \\
        &< M_2 |a_i - b_i| + M_1 |c_i - d_i| \\
        &< M_2 \frac{\varepsilon}{2M_2} + M_1 \frac{\varepsilon}{2M_1} \\
        &= \varepsilon.
    \end{align*}
\end{proof}

\begin{instance}
    Define multiplication of real numbers as the binary operation given by
    Theorem \ref{binary_op_ex} and the previous lemma.
\end{instance}

\begin{instance}
    Multiplication of real numbers is distributive over addition.
\end{instance}
\begin{proof}
    Let $a_n$, $b_n$, and $c_n$ be Cauchy sequences.  Using Lemma
    \ref{real_eq_zero},
    \[
        (a_i(b_i + c_i)) - (a_ib_i + a_ic_i)
        = (a_ib_i + a_ic_i) - (a_ib_i + a_ic_i)
        = 0.
    \]
\end{proof}

\begin{instance}
    Multiplication of real numbers is commutative.
\end{instance}
\begin{proof}
    Let $a_n$ and $b_n$ be Cauchy sequences.  Using Lemma \ref{real_eq_zero},
    \[
        (a_i b_i) - (b_i a_i)
        = (a_i b_i) - (a_i b_i)
        = 0.
    \]
\end{proof}

\begin{instance}
    Multiplication of real numbers is associative.
\end{instance}
\begin{proof}
    Let $a_n$, $b_n$, and $c_n$ be Cauchy sequences.  Using Lemma
    \ref{real_eq_zero},
    \[
        (a_i (b_i c_i)) - ((a_i b_i) c_i)
        = ((a_i b_i) c_i) - ((a_i b_i) c_i)
        = 0.
    \]
\end{proof}

\begin{instance}
    Define $1$ in the real numbers to be the real version of the rational $1$,
    that is, the equivalence class of the constant sequence $1$.
\end{instance}

\begin{instance}
    $1$ is a multiplicative identity in $\R$.
\end{instance}
\begin{proof}
    Let $a_n$ be a Cauchy sequence.  Using Lemma \ref{real_eq_zero},
    \[
        (1 a_i) - a_i
        = a_i - a_i
        = 0.
    \]
\end{proof}

\begin{definition}
    Let $a_n$ be a Cauchy sequence.  Then we will define a new sequence $\oslash
    a_n$ by saying that $\oslash a_i = 0$ if $a = 0$ as a real number, and
    $\oslash a_i = a_i^{-1}$ otherwise.
\end{definition}

\begin{lemma}
    For all Cauchy sequences $a_n$, $\oslash a_n$ is also a Cauchy sequence.
\end{lemma}
\begin{proof}
    If $a = 0$, then the result follows from Lemma \ref{rat_to_real_cauchy}.  So
    consider the case $a \neq 0$.  By Lemma \ref{real_cauchy_nz}, we have a
    rational $\varepsilon' > 0$ and a natural number $N_1$ such that for all $i
    \geq N_1$, we have $\varepsilon' \leq |a_i|$.  Now let $\varepsilon > 0$.
    Then $\varepsilon \varepsilon'^2 > 0$, so by $a_n$ being Cauchy, there
    exists a natural number $N_2$ such that for all $i \geq N_2$ and $j \geq
    N_2$, we have $|a_i - a_j| < \varepsilon\varepsilon'^2$.  Now set $N =
    \max(N_1, N_2)$.  For all $i \geq N$ and $j \geq N$, we have $a_i \neq 0$
    because $|a_i| \geq \varepsilon' > 0$, and by the same argument we have $a_j
    \neq 0$.  Thus,
    \begin{align*}
        \left|\frac{1}{a_i} - \frac{1}{a_j}\right|
        &= \left|\frac{a_j}{a_ia_j} - \frac{a_i}{a_ia_j}\right| \\
        &= \left|\frac{a_j - a_i}{a_ia_j}\right| \\
        &= \frac{|a_j - a_i|}{|a_i|\,|a_j|}.
    \end{align*}
    Now by $a_n$ being Cauchy we have
    \[
        |a_j - a_i| < \varepsilon \varepsilon'^2
        \leq \varepsilon |a_i|\,|a_j|,
    \]
    so
    \[
        \left|\frac{1}{a_i} - \frac{1}{a_j}\right|
        = \frac{|a_j - a_i|}{|a_i|\,|a_j|}
        < \varepsilon.
    \]
\end{proof}

\begin{lemma}
    The operation $\oslash$ is well-defined under the equivalence relation
    $\sim$.
\end{lemma}
\begin{proof}
    Let $a_n$ and $b_n$ be Cauchy sequences with $a_n \sim b_n$.  Since $a = b$,
    it's impossible to have $a = 0$ and $b \neq 0$ and vice versa, so there will
    be two cases: when both $a$ and $b$ are equal to zero, and when neither are
    equal to zero.  When $a = 0$ and $b = 0$, $\oslash a_i = 0$ and $\oslash b_i
    = 0$, so $\oslash a_n = \oslash b_n$.  So now consider the case $a \neq 0$
    and $b \neq 0$.

    By $a \neq 0$, by Lemma \ref{real_cauchy_nz} we have a rational
    $\varepsilon_1 > 0$ and a natural number $N_1$ such that $\varepsilon_1 \leq
    |a_i|$ for all $i \geq N_1$, and similarly, we have a rational
    $\varepsilon_2 > 0$ and a natural number $N_2$ such that $\varepsilon_2 \leq
    |b_i|$ for all $i \geq N_2$.  Now let $\varepsilon > 0$.  Because $a_n \sim
    b_n$, there exists a natural number $N_3$ such that for all $i \geq N_3$, we
    have $|a_i - b_i| < \varepsilon \varepsilon_1 \varepsilon_2$.  Now set $N =
    \max(N_1, N_2, N_3)$.  Then for all $i \geq N$, we have $a_i \neq 0$ because
    $|a_i| \geq \varepsilon_1 > 0$, and by the same argument we have $b_i \neq
    0$.  Thus,
    \begin{align*}
        \left|\frac{1}{a_i} - \frac{1}{b_i}\right|
        &= \left|\frac{b_i}{a_ib_i} - \frac{a_i}{a_ib_i}\right| \\
        &= \left|\frac{b_i - a_i}{a_ib_i}\right| \\
        &= \frac{|b_i - a_i|}{|a_i|\,|b_i|}.
    \end{align*}
    Now by $a_n \sim b_n$ we have
    \[
        |b_i - a_i| < \varepsilon \varepsilon_1 \varepsilon_2
        \leq \varepsilon |a_i|\,|b_i|,
    \]
    so
    \[
        \left|\frac{1}{a_i} - \frac{1}{b_i}\right|
        = \frac{|b_i - a_i|}{|a_i|\,|b_i|}
        < \varepsilon.
    \]
\end{proof}

\begin{instance}
    Define multiplicative inverses as the operation produced by Theorem
    \ref{unary_op_ex} and the previous lemma.
\end{instance}

\begin{instance}
    The reciprocal in $\R$ is a multiplicative inverse.
\end{instance}
\begin{proof}
    Let $a_n$ be a Cauchy sequence with $a \neq 0$.  Now we can't use Lemma
    \ref{real_eq_zero} here because even though $a \neq 0$, there may exist some
    elements of the sequence that are equal to zero.  Instead, let $\varepsilon
    > 0$.  Because $a \neq 0$, by Lemma \ref{real_cauchy_nz}, there exists a
    $\varepsilon' > 0$ and a natural number $n$ such that for all $i \geq N$, we
    have $\varepsilon' \leq |a_i|$.  This means that for all $i \geq N$, $|a_i|
    \geq \varepsilon > 0$, so $a_i \neq 0$.  Thus, for all $i \geq N$, we have
    $|a_i^{-1} a_i - 1| = |1 - 1| = 0 < \varepsilon$.
\end{proof}

\begin{instance}
    $\R$ is not trivial.
\end{instance}
\begin{proof}
    Assume that $0 \sim 1$.  Because $0 < 1$, there exists an $N$ such that for
    all $i \geq N$, we have $|0 - 1| < 1$.  $i = N$ works, so we have $|0 - 1| =
    1 < 1$, which is impossible.
\end{proof}

\section{Order}

Ordering the real numbers is annoying because Cauchy sequences can approach the
same number from two different directions, making any definition of order
require two cases.  We will develop the order of the real numbers using the
theory of cones.

\begin{definition}
    Let $a_n$ be a Cauchy sequence.  We will say that $a_n$ is positive if
    either $0 = a$ as a real number, or if there exists a natural number $N$
    such that for all $i \geq N$, we have $0 \leq a_i$.
\end{definition}

\begin{lemma}
    Cauchy sequences being positive is well-defined under the equivalence
    relation $\sim$.
\end{lemma}
\begin{proof}
    Let $a_n$ and $b_n$ be Cauchy sequences with $a_n \sim b_n$, and assume that
    $a_n$ is positive.  We must prove that $b_n$ is positive.  If $a = 0$, we
    would have $b = 0$ because $a = b$.  So consider the case where there exists
    an $N_3 : \N$ such that for all $i \geq N_3$, we have $0 \leq a_i$.  We must
    prove that either $b = 0$, or that there exists an $N : \N$ such that for
    all $i \geq N$, we have $0 \leq b_i$.  We will prove the second while
    assuming that $b \neq 0$.  Since $b \neq 0$, we also have $a \neq 0$ by $a_n
    \sim b_n$.  By Lemma \ref{real_cauchy_nz}, we have an $\varepsilon > 0$ and
    $N_1 : \N$ such that for all $i \geq N_1$, we have $\varepsilon \leq |a_i|$.
    And because $a_n \sim b_n$, we have a natural number $N_2$ such that for all
    $i \geq N_2$, we have $|a_i - b_i| < \varepsilon$.  Now let $N = \max(N_1,
    N_2, N_3)$.  Then for all $i \geq N$, we have
    \begin{equation} \label{real_pos_wd_1}
        0 \leq a_i,
    \end{equation}
    \begin{equation} \label{real_pos_wd_2}
        \varepsilon \leq |a_i|
    \end{equation}
    and
    \begin{equation} \label{real_pos_wd_3}
        |a_i - b_i| < \varepsilon.
    \end{equation}
    From (\ref{real_pos_wd_1}) and (\ref{real_pos_wd_2}), we have
    \begin{equation} \label{real_pos_wd_4}
        \varepsilon \leq a_i.
    \end{equation}
    By Theorem \ref{abs_le_pos} and (\ref{real_pos_wd_3}), we have
    \begin{align}
        a_i - b_i &< \varepsilon \nonumber \\
        a_i &< \varepsilon + b_i. \label{real_pos_wd_5}
    \end{align}
    Then by (\ref{real_pos_wd_4}) and (\ref{real_pos_wd_5}), we have
    \begin{align*}
        \varepsilon &\leq \varepsilon + b_i \\
        0 &\leq b_i.
    \end{align*}
\end{proof}

We can now talk about a real number being positive by using Theorem
\ref{unary_op_ex}.

\begin{instance}
    Real numbers being positive form a cone.
\end{instance}
\begin{proof}
    We must prove the five conditions for being a cone.

    \textit{If $a$ and $b$ are positive, $a + b$ is positive.}  If $a = 0$, then
    $a + b = b$ which is already positive, and similarly for $b = 0$.  So
    consider when $a \neq 0$ and $b \neq 0$.  Then because $a$ and $b$ are
    positive, we have natural numbers $N_1$ and $N_2$ such that for all $i \geq
    N_1$, we have $0 \leq a_i$, and for all $i \geq N_2$, we have $0 \leq b_i$.
    Let $N = \max(N_1, N_2)$.  Then for all $i \geq N$, we have $0 \leq a_i$ and
    $0 \leq b_i$, and adding the inequalities, we get $0 \leq a_i + b_i$.


    \textit{If $a$ and $b$ are positive, $ab$ is positive.}  If $a = 0$, then
    $ab = 0$ which is already positive, and similarly for $b = 0$.  So consider
    when $a \neq 0$ and $b \neq 0$.  Then because $a$ and $b$ are positive, we
    have natural numbers $N_1$ and $N_2$ such that for all $i \geq N_1$, we have
    $0 \leq a_i$, and for all $i \geq N_2$, we have $0 \leq b_i$.  Let $N =
    \max(N_1, N_2)$.  Then for all $i \geq N$, we have $0 \leq a_i$ and $0 \leq
    b_i$, and multiplying the inequalities, we get $0 \leq a_i b_i$.

    \textit{For all $a$, $a^2$ is positive.}  Defining $N = 0$, for all $i \geq
    N$, we have $0 \leq a_i^2$.

    \textit{-1 is not positive.}  We must prove that both $-1 \neq 0$, and that
    for all $N : \N$, there exists an $i \geq N$ such that $-1 < 0$.  For $-1
    \neq 0$, if $-1 = 0$, we would have $1 = 0$, which is impossible.  For the
    second condition, we have $N \geq N$, and $-1 < 0$ is true in the rationals.

    \textit{For all $a$, either $a$ is positive or $-a$ is positive.}  Assume
    that $a_n$ is not positive.  Then $a \neq 0$ and for all $N : \N$, we have
    a $j \geq N$ with $a_j < 0$.  Because $a \neq 0$, by Lemma
    \ref{real_cauchy_nz}, we have a $\varepsilon > 0$ and an $N_1 : \N$ such
    that for all $i \geq N_1$, we have $\varepsilon \leq |a_i|$.  Because $a_n$
    is Cauchy, we have an $N_2 : \N$ such that for all $i \geq N_2$ and $j \geq
    N_2$, we have $|a_i - a_j| < \varepsilon$.  Set $N = \max(N_1, N_2)$.  Then
    we have some $j \geq N$ with
    \begin{equation} \label{real_cone_1}
        a_j < 0.
    \end{equation}
    Because $a \neq 0$, we also have
    \begin{equation} \label{real_cone_2}
        \varepsilon \leq |a_j|.
    \end{equation}
    Then for any $i \geq N$, we have
    \begin{equation} \label{real_cone_3}
        |a_i - a_j| < \varepsilon.
    \end{equation}
    From (\ref{real_cone_1}) and (\ref{real_cone_2}),
    \begin{equation} \label{real_cone_4}
        \varepsilon \leq -a_j.
    \end{equation}
    By Theorem \ref{abs_le_pos} and (\ref{real_cone_3}),
    \begin{equation} \label{real_cone_5}
        a_i - a_j < \varepsilon.
    \end{equation}
    Then by (\ref{real_cone_4}) and (\ref{real_cone_5}),
    \begin{align*}
        a_i - a_j &\leq -a_j \\
        0 &\leq -a_i,
    \end{align*}
    showing that $-a_n$ is positive.
\end{proof}

Because real numbers being positive forms a cone, by the development of cones in
chapter \ref{chap_ealgebra}, we can order the real numbers by saying that $a
\leq b$ if and only if $b - a$ is positive, and this order turns the real
numbers into an ordered field.

There are a couple of useful lemma about the order of the real numbers that will
be useful in the next section.

\begin{lemma} \label{real_lt}
    For all Cauchy sequences $a_n$ and $b_n$, if $a < b$, then there exists an
    $\varepsilon > 0$ and an $N$ such that for all $i \geq N$, we have $a_i +
    \varepsilon < b_i$.
\end{lemma}
\begin{proof}
    Because $a < b$, we have $a \leq b$ and $a \neq b$.  Because $a \neq b$, we
    have $0 \neq b - a$.  Then, because $a \leq b$ and $0 \neq b - a$, we have
    an $N_1 : \N$ such that for all $i \geq N_1$, we have $0 \leq b_i - a_i$.
    Because $0 \neq b - a$, by Lemma \ref{real_cauchy_nz} we have a $\varepsilon
    > 0$ and an $N_2 : \N$ such that for all $i \geq N_2$, we have $\varepsilon
    \leq |b_i - a_i|$.  Set $N = \max(N_1, N_2)$.  We will prove that for all $i
    \geq N$, we have $a_i + \frac{\varepsilon}{2} < b_i$.  For such an $i$, we
    already have
    \[
        0 \leq b_i - a_i
    \]
    and
    \[
        \varepsilon \leq |b_i - a_i|.
    \]
    Combining them, we get
    \begin{align*}
        \varepsilon &\leq b_i - a_i \\
        a_i + \varepsilon &\leq b_i.
    \end{align*}
    And since $\frac{\varepsilon}{2} < \varepsilon$, we have $a_i +
    \frac{\varepsilon}{2} < b_i$ as required.
\end{proof}

\begin{lemma} \label{real_leq_seq}
    For all real numbers $x$ and Cauchy sequences $y_n$, if $x \leq y_i$ for all
    $i : \N$, then $x \leq y$.
\end{lemma}
\begin{proof}
    By definition, we will have $x \leq y$ if $x = y$ or if for all $i$ greater
    than some $N$, we have $x_i \leq y_i$.  We will prove the latter while
    assuming the negation of the former.  If $x \neq y$, we have $0 \neq y - x$,
    so by Lemma \ref{real_cauchy_nz}, we have a $\varepsilon > 0$ and a natural
    number $N_1$ such that for all $i \geq N_1$, we have $\varepsilon \leq |y_i
    - x_i|$.  Because $y$ is Cauchy and $\frac{\varepsilon}{2} > 0$, we have a
    natural number $N_2$ such that for all $i \geq N_2$ and $j \geq N_2$, we
    have $|y_i - y_j| < \frac{\varepsilon}{2}$.  Let $N' = \max(N_1, N_2)$.  By
    assumption, we have $x \leq y_{N'}$.  By definition, this means that either
    $x = y_{N'}$, or there exists some $N_3$ with $x_i \leq y_{N'}$ for all $i
    \geq N_3$.

    For the case $x = y_{N'}$, we have an $N_3 : \N$ such that for all $i \geq
    N_3$, we have $|y_{N'} - x_i| < \frac{\varepsilon}{2}$.  Define $i =
    \max(N', N_3)$.  Then we have
    \begin{equation} \label{real_leq_seq_1}
        \varepsilon \leq |y_i - x_i|,
    \end{equation}
    \begin{equation} \label{real_leq_seq_2}
        |y_i - y_{N'}| < \frac{\varepsilon}{2},
    \end{equation}
    and
    \begin{equation} \label{real_leq_seq_3}
        |y_{N'} - x_i| < \frac{\varepsilon}{2}.
    \end{equation}
    Adding (\ref{real_leq_seq_2}) and (\ref{real_leq_seq_3}), we get
    \begin{align*}
        |y_i - y_{N'}| + |y_{N'} - x_i|
            &< \frac{\varepsilon}{2} + \frac{\varepsilon}{2} \\
        |y_i - y_{N'} + y_{N'} - x_i| < \varepsilon \\
        |y_i - x_i| < \varepsilon,
    \end{align*}
    contradicting (\ref{real_leq_seq_1}).  Thus, this case is impossible.

    For the case where we have some $N_3$ with $x_i \leq y_{N'}$ for all $i
    \geq N_3$, set $N = \max(N', N_3)$.  Then for all $i \geq N$, we have
    \begin{equation} \label{real_leq_seq_4}
        \varepsilon \leq |y_i - x_i|,
    \end{equation}
    \begin{equation} \label{real_leq_seq_5}
        |y_{N'} - y_i| < \frac{\varepsilon}{2} < \varepsilon,
    \end{equation}
    and
    \begin{equation} \label{real_leq_seq_6}
        x_i \leq y_{N'}.
    \end{equation}
    We must prove that $x_i \leq y_i$.  For a contradiction, assume that $y_i -
    x_i < 0$.  Then $|y_i - x_i| = x_i - y_i$, so by (\ref{real_leq_seq_4}) we
    have
    \begin{equation} \label{real_leq_seq_7}
        \varepsilon \leq x_i - y_i.
    \end{equation}
    By Theorem \ref{abs_le_pos} in (\ref{real_leq_seq_5}), we have
    \begin{equation} \label{real_leq_seq_8}
        y_{N'} - y_i < \varepsilon.
    \end{equation}
    Combining (\ref{real_leq_seq_7}) and (\ref{real_leq_seq_8}), we get
    \begin{align*}
        y_{N'} - y_i &< x_i - y_i \\
        y_{N'} &< x_i,
    \end{align*}
    contradicting (\ref{real_leq_seq_6}).  Thus, $x_i \leq y_i$.
\end{proof}

\section{Completeness}

This whole section will be proving the following instance:

\begin{instance}
    The real numbers are supremum complete.
\end{instance}

The proof will be done throughout this whole section, with many definitions and
lemmas that will only be used in this section.  Let $S$ be a nonempty set of
real numbers that has an upper bound.

\begin{definition}
    Define a set $S' : \Q \to \Prop$ where $q \in S'$ means that $q$ is an upper
    bound of $S$.
\end{definition}

\begin{lemma}
    $S'$ is nonempty.
\end{lemma}
\begin{proof}
    By assumption, $S$ has a real upper bound $u$.  By Lemma
    \ref{real_cauchy_bounded}, there exists a rational $M$ such that $|u_i| < M$
    for all $i$.  We will prove that $M$ is an upper bound of $S$.  Let $x \in
    S$.  Since $u$ is an upper bound, we have $x \leq u$.  And since $u_i \leq
    |u_i| < M$ for all $i$, we have $u \leq M$.  Thus, $x \leq M$, showing that
    $M$ is an upper bound of $S$, so is in $S'$.
\end{proof}

\begin{lemma}
    The complement of $S'$ is nonempty.
\end{lemma}
\begin{proof}
    By assumption, there exists a real value in $S$.  Call that value $\ell$.
    By Lemma \ref{real_cauchy_bounded}, there exists a rational $M$ such that
    $|\ell_i| < M$ for all $i$.  By Theorem \ref{abs_le_neg}, we have
    \begin{equation} \label{real_sup_lower_ex_1}
        -\ell_i < M
    \end{equation}
    for all $i$.  We will show that $-M - 1 \notin S'$.  For a contradiction,
    assume that $-M - 1$ is an upper bound of $S$.  Then we would have $\ell
    \leq -M - 1$.  This means that either $\ell = -M - 1$, or there exists an
    $N$ such that $\ell_i \leq -M - 1$ for all $i \geq N$.

    When $\ell = -M - 1$, by the definition of real equality, there exists an
    $N$ such that for all $i \geq N$, we have $|\ell_i + M + 1| < 1$  By Theorem
    \ref{abs_le_pos}, we have
    \begin{align*}
        \ell_i + M + 1 &< 1
        M &< -\ell_i,
    \end{align*}
    contradicting (\ref{real_sup_lower_ex_1}).

    When there exists an $N$ such that $\ell_i \leq -M - 1$ for all $i \geq N$,
    we have
    \[
        M \leq -\ell_i - 1 < \ell_i,
    \]
    again contradicting (\ref{real_sup_lower_ex_1}).

    In both cases, we got a contradiction, so $-M - 1$ must not be in $S'$.
\end{proof}

We now know that both $S'$ and its complement are nonempty.  Let $u$ be a
rational value in $S'$ and let $\ell$ be a rational value not in $S'$.  Unlike
the original element of $S$ and upper bound that we had before, these two values
are rational, not real.

\begin{definition}
    Define a sequence of pairs of rational numbers $p_n$ recursively, with $a_i
    = P_1(p_i)$, $b_i = P_2(p_i)$, and $m_i = \frac{a_i + b_i}{2}$, by
    \[
        p_i =
        \begin{cases}
            (\ell, u) & \text{if $i = 0$} \\
            (a_j, m_j) & \text{if $i = j + 1$ and $m_j \in S'$} \\
            (m_j, b_j) & \text{if $i = j + 1$ and $m_j \notin S'$.}
        \end{cases}
    \]
    Furthermore, let $d_n$ be the sequence given by $d_i = b_i - a_i$.
\end{definition}

\begin{lemma} \label{real_sup_a_lower}
    For all $i$, $a_i \notin S'$.
\end{lemma}
\begin{proof}
    The proof will be by induction on $i$.  When $i = 0$, $a_0 = \ell$, which is
    not in $S'$ by definition.  Now assume that $a_i \notin S'$.  Then when $m_i
    \in S'$, $a_{i + 1} = a_i \notin S'$, and when $m_i \notin S'$, $a_{i + 1} =
    m_i \notin S'$.  Thus the inductive case is true.
\end{proof}

\begin{lemma} \label{real_sup_b_upper}
    For all $i$, $b_i \in S'$.
\end{lemma}
\begin{proof}
    The proof will be by induction on $i$.  When $i = 0$, $b_0 = u$, which is in
    $S'$ by definition.  Now assume that $b_i \in S'$.  Then when $m_i \in S'$,
    $b_{i + 1} = m_i \in S'$, and when $m_i \notin S'$, $b_{i + 1} = b_i \in
    S'$.  Thus the inductive case is true.
\end{proof}

\begin{lemma} \label{real_sup_d_suc}
    For all $i$, $d_{n+1} = \frac{d_i}{2}$.
\end{lemma}
\begin{proof}
    When $m_i \in S'$,
    \begin{align*}
        d_{i + 1}
        &= b_{i + 1} - a_{i + 1} \\
        &= m_i - a_i \\
        &= \frac{b_i + a_i}{2} - a_i \\
        &= \frac{b_i - a_i}{2} \\
        &= \frac{d_i}{2}.
    \end{align*}
    When $m_i \notin S'$,
    \begin{align*}
        d_{i + 1}
        &= b_{i + 1} - a_{i + 1} \\
        &= b_i - m_i \\
        &= b_i - \frac{b_i + a_i}{2} \\
        &= \frac{b_i - a_i}{2} \\
        &= \frac{d_i}{2}.
    \end{align*}
\end{proof}

\begin{lemma} \label{real_sup_d_eq}
    For all $i$, $d_i = \frac{d_0}{2^n}$.
\end{lemma}
\begin{proof}
    The proof will be by induction on $i$.  When $i = 0$,
    \[
        d_0 = \frac{d_0}{2^0},
    \]
    so the base case is true.  Now assume that $d_i = \frac{d_0}{2^i}$.  Then
    \[
        d_{i+1} = \frac{d_i}{2} = \frac{d_0}{2^i2} = \frac{d_0}{2^{i+1}},
    \]
    so the inductive case is true.
\end{proof}

\begin{lemma} \label{real_sup_d_pos}
    For all $i$, $0 < d_i$.
\end{lemma}
\begin{proof}
    Since $d_i = \frac{d_0}{2^i}$ and $\frac{1}{2^i}$ is positive, it suffices
    to prove that $d_0 = u - \ell$ is positive.  Since $u$ is an upper bound of
    $S'$ and $\ell$ is not, we have $\ell < u$.  Then by Lemma \ref{real_lt} we
    have a $\varepsilon > 0$ and an $N : \N$ such that for all $i \geq N$, we
    have $\ell + \varepsilon < u$.  $i = N$ suffices to show that $\ell +
    \varepsilon < u$.  Then $0 < \varepsilon < u - \ell$, so $u - \ell$ is
    positive.
\end{proof}

\begin{lemma} \label{real_sup_ab_cauchy}
    For all sequences $x$, if $|x_i - x_{i+1}| \leq d_{i+1}$ for all $i$, then
    $x$ is a Cauchy sequence.
\end{lemma}
\begin{proof}
    Let $\varepsilon > 0$.  Since $d_i > 0$, $\frac{\varepsilon}{d_0}$ is as
    well, so by Theorem \ref{arch_pow2}, there exists a natural number $N$ such
    that $\frac{1}{2^N} < \frac{\varepsilon}{d_0}$.  We will prove that for all
    $i \geq N$ and $j \geq N$, we have $|x_i - x_j| < \varepsilon$.  Without
    loss of generality, we may assume that $i \leq j$.  Because $i \leq j$, by
    Theorem \ref{nat_le_ex}, we have a natural number $c$ with $j = i + c$.  So
    we must prove that $|x_i - x_{i+c}| < \varepsilon$.  Because $\frac{1}{2^N}
    < \frac{\varepsilon}{d_0}$, we have $d_N = \frac{d_0}{2^N} < \varepsilon$.
    Thus, it suffices to prove that $|x_i - x_{i+c}| < d_N$.  We will do so by
    induction on $c$.

    When $c = 0$, $|x_i - x_{i + 0}| = 0 < d_N$ by Lemma \ref{real_sup_d_pos},
    so the base case is true.  Now assume that for all $N$ and $i$ with $N \leq
    i$, we have $|x_i - x_{i+c}| < d_N$.  Since $N \leq i$, we have $N + 1 \leq
    i + 1$, so
    \begin{equation} \label{real_sup_ab_cauchy_1}
        |x_{i+1} - x_{i+1+c}| < d_{N+1}.
    \end{equation}
    By assumption, we have
    \begin{equation} \label{real_sup_ab_cauchy_2}
        |x_i - x_{i + 1}| \leq d_{i+1}.
    \end{equation}
    Because $N + 1 \leq i + 1$, we have
    \begin{align*}
        \frac{1}{2^{i + 1}} &\leq \frac{1}{2^{N + 1}} \\
        \frac{d_0}{2^{i + 1}} &\leq \frac{d_0}{2^{N + 1}} \\
        d_{i+1} &\leq d_{N+1},
    \end{align*}
    and combining this with (\ref{real_sup_ab_cauchy_2}) we get
    \begin{equation} \label{real_sup_ab_cauchy_3}
        |x_i - x_{i + 1}| \leq d_{N+1}.
    \end{equation}
    Adding (\ref{real_sup_ab_cauchy_1}) and (\ref{real_sup_ab_cauchy_3}), we get
    \begin{align*}
        |x_i - x_{i + 1}| + |x_{i + 1} - x_{i + 1 + c}| &< d_{N+1} + d_{N + 1} \\
        |x_i - x_{i + 1} + x_{i + 1} - x_{i + 1 + c}|
            &< \frac{d_N}{2} + \frac{d_N}{2} \\
        |x_i - x_{i + (c + 1)}| &< d_N,
    \end{align*}
    showing that the inductive case is true.  Thus, $|x_i - x_{i+c}| < d_N$ by
    induction.
\end{proof}

\begin{lemma} \label{real_sup_a_cauchy}
    $a_n$ is a Cauchy sequence.
\end{lemma}
\begin{proof}
    The proof will be by Lemma \ref{real_sup_ab_cauchy}.  When $m_i \in S'$,
    \[
        |a_i - a_{i+1}| = |a_i - a_i| = 0 \leq d_{i+1}.
    \]
    When $m_i \notin S'$,
    \begin{align*}
        |a_i - a_{i+1}|
        &= |a_i - m_i| \\
        &= \left|a_i - \frac{b_i - a_i}{2}\right| \\
        &= \left|\frac{a_i - b_i}{2}\right| \\
        &= \left|\frac{b_i - a_i}{2}\right| \\
        &= \left|\frac{d_i}{2}\right| \\
        &= |d_{i+1}| \\
        &= d_{i+1}.
    \end{align*}
\end{proof}

\begin{lemma} \label{real_sup_b_cauchy}
    $b_n$ is a Cauchy sequence.
\end{lemma}
\begin{proof}
    The proof will be by Lemma \ref{real_sup_ab_cauchy}.  When $m_i \in S'$,
    \begin{align*}
        |b_i - b_{i+1}|
        &= |b_i - m_i| \\
        &= \left|b_i - \frac{b_i - a_i}{2}\right| \\
        &= \left|\frac{b_i - a_i}{2}\right| \\
        &= \left|\frac{d_i}{2}\right| \\
        &= |d_{i+1}| \\
        &= d_{i+1}.
    \end{align*}
    When $m_i \notin S'$,
    \[
        |b_i - b_{i+1}| = |b_i - b_i| = 0 \leq d_{i+1}.
    \]
\end{proof}

Because $a_n$ and $b_n$ are Cauchy sequences, we can now think of the real
numbers $a$ and $b$.

\begin{lemma}
    $a = b$.
\end{lemma}
\begin{proof}
    Let $\varepsilon > 0$.  Then $\frac{\varepsilon}{d_0} > 0$, so by Theorem
    \ref{arch_pow2}, we have a natural number $N$ such that $\frac{1}{2^N} <
    \frac{\varepsilon}{d_0}$.  This means that $\frac{d_0}{2^N} <
    \varepsilon$.  Then for all $i \geq N$, we have
    \[
        |b_i - a_i| = |d_i| = d_i = \frac{d_0}{2^i} \leq \frac{d_0}{2^N} <
        \varepsilon.
    \]
\end{proof}

\begin{theorem}
    $a$ is the supremum of $S$.
\end{theorem}
\begin{proof}
    Let $x \in S$.  Since $b_i$ is an upper bound of $S$ for all $i$, we have $x
    \leq b_i$ for all $i$.  Thus, by Lemma \ref{real_leq_seq}, we have $x \leq b
    = a$, proving that $a$ is an upper bound of $S$.

    Let $y$ be an upper bound of $S$.  Then because $a_i$ is not an upper bound
    of $S$ for all $i$, we have $a_i \leq y$ for all $i$.  This means that $-y
    \leq -a_i$ for all $i$, so by Lemma \ref{real_leq_seq}, we have $-y \leq
    -a$.  Thus, $a \leq y$, proving that $a$ is the least upper bound of $S$.
\end{proof}

\end{document}
