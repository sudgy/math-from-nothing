\documentclass[../../math.tex]{subfiles}
\externaldocument{../../math.tex}
\externaldocument{../basics/elementary_algebra}

\begin{document}

\setcounter{chapter}{10}

\chapter{The Real Numbers}

Constructing the real numbers is much more complicated than it initially seems.
I have gone through several constructions in the Coq code trying to find the
simplest one.  Of all of the constructions I have tried, the simplest one was
the one using Cauchy sequences of rationals, so that is the one I will be
presenting in this chapter.  The other constructions I have done will be covered
in a future chapter.  Because I have swapped constructions before and am open to
doing it again, after this chapter the particular construction used will not be
mentioned outside of this chapter.  The only important things will be developed
in the last section.

One may wonder, why are we using Cauchy sequences when we haven't done any
analysis yet?  Shouldn't we do analysis first to get the definition of Cauchy
sequences?  The issue here is that the definition of a metric space depends on
the real numbers.  One could define metric spaces where the metric has values in
an arbitrary ordered field, but this approach isn't really that useful and ends
up introducing a great deal of complexity.  Thus, all of the ideas in analysis
that we will need for this chapter will be introduced here using the rational
numbers and then never used again outside of this chapter.  Until the last
section, none of definitions or theorems in this chapter will be used anywhere
else, and some future definitions will even have the same names as in this
chapter despite being distinct.

Throughout this chapter, we will be using sequences of rational numbers
extensively.  Recall that a sequence of rational numbers is function $a$ from
$\N$ to $\Q$.  Instead of writing $a(i)$ to denote the $i$th element of the
sequence, we will use $a_i$.  We will often refer to the sequence as $a_n$
rather than just $a$, to distinguish real numbers from sequences.

\section{Basic Construction}

\begin{definition}
    Let $a_n$ be a sequence of rational numbers.  Then we say that $a_n$ is a
    Cauchy sequence if for all $\varepsilon > 0$, there exists a natural number
    $N$ such that for all $i$ and $j$ with $N \leq i$ and $N \leq j$, we have
    $|a_i - a_j| < \varepsilon$.
\end{definition}

\begin{definition}
    Define an equivalence relation $\sim$ on Cauchy sequences by saying that
    $a_n \sim b_n$ if for all $\varepsilon > 0$, there exists a natural number
    $N$ such that for all $i \geq N$, we have $|a_i - b_i| < \varepsilon$.
\end{definition}

\begin{lemma}
    The relation $\sim$ is an equivalence relation on Cauchy sequences.
\end{lemma}
\begin{proof}
    \textit{Reflexivity}.  Let $a_n$ be a Cauchy sequence, and let $\varepsilon
    > 0$.  Then setting $N = 0$, for all $i \geq N$, we have $|a_i - a_i| = 0 <
    \varepsilon$, showing that $\sim$ is reflexive.

    \textit{Symmetry}.  Let $a_n$ and $b_n$ be Cauchy sequences with $a_n \sim
    b_n$.  Let $\varepsilon > 0$.  Because $a_n \sim b_n$, we have an $N$ such
    that for $i \geq N$, we have $|a_i - b_i| < \varepsilon$.  By Theorem
    \ref{abs_minus}, this means that for all $i \geq N$, we have $|b_i - a_i| <
    \varepsilon$, showing that $b_n \sim a_n$.

    \textit{Transitivity}.  Let $a_n$, $b_n$, and $c_n$ be Cauchy sequences with
    $a_n \sim b_n$ and $b_n \sim c_n$.  Let $\varepsilon > 0$.  Then
    $\frac{\varepsilon}{2} > 0$, so by $a_n \sim b_n$ and $b_n \sim c_n$ we have
    a natural number $N_1$ such that for all $i \geq N_1$ we have $|a_i - b_i| <
    \frac{\varepsilon}{2}$ and a natural number $N_2$ such that for all $i \geq
    N_2$ we have $|b_i - c_i| < \frac{\varepsilon}{2}$.  Setting $N = \max(N_1,
    N_2)$, for all $i \geq N$, we have both inequalities.  Then we have
    \begin{align*}
        |a_i - c_i|
        &= |(a_i - b_i) + (b_i - c_i)| && \\
        &\leq |a_i - b_i| + |b_i - c_i| && \text{by Theorem \ref{abs_tri}} \\
        &< \frac{\varepsilon}{2} + \frac{\varepsilon}{2} && \\
        &= \varepsilon. &&
    \end{align*}
\end{proof}

\begin{definition}
    Define the type $\R$ of real numbers to be the quotient of Cauchy sequences
    by $\sim$.  Given a Cauchy sequence $a_n$, we will often use just $a$ to
    represent the equivalence class of $a_n$ under this quotient.
\end{definition}

\begin{lemma}
    Constant sequences are Cauchy.
\end{lemma}
\begin{proof}
    For some constant rational number $q$, let $q_n = q$ for all $n : \N$.  Then
    for all $\varepsilon > 0$, setting $N = 0$, for all $i \geq 0$ and $j \geq
    0$ we have $|q_i - q_j| = |q - q| = 0 < \varepsilon$, so $a_n$ is Cauchy.
\end{proof}

\begin{definition}
    Given a rational number $q$, we can consider it to be a real number by
    turning it into the constant sequence $q_n$ (which we know is Cauchy by the
    previous lemma) and then taking the equivalence class of it.  We will often
    refer to the real number as $q$ as well.
\end{definition}

This induces a function from the rationals to the reals that is in fact equal to
$\iQ$.  However, by the time we can apply $\iQ$ to $\R$, we will be so far along
in the construction that we will never need to prove this.  Again, after this
chapter, this construction won't be used anymore, and with it this way of
turning rational numbers to real numbers.

\end{document}
