\documentclass[../../math.tex]{subfiles}
\externaldocument{../../math.tex}
\externaldocument{../basics/foundations}
\externaldocument{../basics/set}
\externaldocument{../basics/elementary_algebra}

\begin{document}

\setcounter{chapter}{8}

\chapter{The Ordinals}

The ordinals are a way of studying well-ordered sets, by basically considering
all of them at once.  They were initially described as equivalence classes of
well-ordered sets, but there were issues with this initial conception and modern
set theories tend to use other constructions.  However, in the foundations used
here, we won't have those issues, so we can describe ordinals using their
traditional interpretation as equivalence classes of well-ordered sets.

\section{Basic Construction}

\begin{definition}
    We will say that an order type is a type $\U$ that has a well order.
\end{definition}

In general, people often use ``order type'' to refer to more than just well
orders, but the concept is mainly useful for well orders, so for the sake of
simplicity we will consider order types to only be well orders.  Remember that
to prove that a type is well-ordered, it suffices to prove that the relation is
antisymmetric and that every nonempty subset has a least element.

\begin{definition}
    Define a relation $\sim$ on order types where $\A \sim \B$ is defined to
    mean that there exists an order isomorphism between $\A$ and $\B$, that is,
    a function $f$ that is bijective and preserves inequalities in one
    direction.  Note that these conditions also imply that $f$ preserves both
    inequalities and strict inequalities in both directions.
\end{definition}

\begin{lemma}
    The relation $\sim$ is an equivalence relation.
\end{lemma}
\begin{proof}
    \textit{Reflexivitity.}  The identity function is an order isomorphism.

    \textit{Symmetry.}  Assume that there is an order isomorphism $f$ from $\A$
    to $\B$.  Then its inverse $f^{-1}$ is bijective, and when $a \leq b$, we
    have $f(f^{-1}(a)) \leq f(f^{-1}(b))$, and because $f$ is an order
    isomorphism, $f^{-1}(a) \leq f^{-1}(b)$, so $f^{-1}$ is an order isomorphism
    from $\B$ to $\A$.

    \textit{Transitivity.}  Given three order types $\A$, $\B$, and $\C$ with
    order isomorphisms $f : \A \to \B$ and $g : \B \to \C$, $g \circ f$ is an
    order isomorphism from $\A$ to $\C$.
\end{proof}

\begin{definition}
    Define the type $\Ord$ to be the quotient of all order types by $\sim$.
\end{definition}

\section{Order}

It turns out that the ordinals themselves are well ordered.  This ordering is so
important that unlike the previous types that have been developed, this ordering
will be developed before the other operations.  Most sources describe the order
on ordinals using strict inequalities, but describing them using non-strict
inequalities can be useful at times too and is the default way to describe
orderings in the Coq code, so I will first describe the order on ordinals using
non-strict inequalities.

\begin{definition}
    Let $\A$ and $\B$ be order types.  Then we say that $f : \A \to \B$ is an
    ordinal inequality function if it is injective, preserves inequalities, and
    if for all $b : \B$ with $f(a) \neq b$ for all $a : \A$, we have $f(a) < b$
    for all $a : \A$.
\end{definition}

\begin{definition}
    Given two order types $\A$ and $\B$, we will say that $\A \leqq \B$ if there
    exists an ordinal inequality function from $\A$ to $\B$.
\end{definition}

\begin{lemma}
    The relation $\leqq$ is well-defined under the relation $\sim$.
\end{lemma}
\begin{proof}
    Let $A$, $B$, $C$, and $D$ be order types with order isomorphisms $f : A \to
    B$ and $g : C \to D$, and let $h$ be an ordinal inequality function from $A$
    to $C$.  We must prove that there exists an ordinal inequality function from
    $B$ to $D$.  We will prove that $g \circ h \circ f^{-1}$ is such a function.

    \noindent \textit{Injectivity.}
    $g$, $h$, and $f^{-1}$ are all injective, so by Instance \ref{inj_comp}, $g
    \circ h \circ f^{-1}$ is as well.

    \noindent \textit{Preserving inequalities.}
    Let $a \leq b$.  Then $f(f^{-1}(a)) \leq f(f^{-1}(b))$, and because $f$
    preserves inequalities, $f^{-1}(a) \leq f^{-1}(b)$.  Then because $g$ and
    $h$ preserve inequalities, $g(h(f^{-1}(a))) \leq g(h(f^{-1}(b)))$, showing
    that $g \circ h \circ f^{-1}$ preserves inequalities.

    \noindent \textit{Ordinal inequality condition.}
    Let $b : D$ be such that there is no $a : B$ such that $g(h(f^{-1}(a))) =
    b$.  This means that there is no $x : A$ such that $g(h(f^{-1}(f(x)))) =
    g(h(x)) = b = g(g^{-1}(b))$, so by the injectivity of $g$ there is no $x :
    A$ such that $h(x) = g^{-1}(b)$.  Because $h$ is an ordinal inequality
    function, this means that for all $x : A$, $h(x) < g^{-1}(b)$.  Thus, for
    any $a : B$, we have $h(f^{-1}(a)) < g^{-1}(b)$, and applying $g$ to both
    sides gives $g(h(f^{-1}(a))) < b$.
\end{proof}

\begin{instance}
    Define order in the ordinals as the operation obtained by applying Theorem
    \ref{binary_op_ex} to the previous lemma.
\end{instance}

As the following theorem shows, this order is equivalent to the usual definition
of the order on ordinals.

\begin{theorem} \label{ord_lt_simpl}
    For two order types $\A$ and $\B$, $[\A] < [\B]$ iff there exists a $x : \B$
    such that $\A \sim \B_x$, the initial segment of $x$ in $\B$.
\end{theorem}
\begin{proof}
    First assume that $[\A] < [\B]$.  This means that there exists an ordinal
    inequality function $f : \A \to \B$, but no order isomorphism from $\A$ to
    $\B$.  First, there exists an $x' : \B$ such that $f(a) \neq x'$ for all $a
    : \A$.  For if there wasn't such an $x'$, $f$ would be surjective, meaning
    that it is an order isomorphism from $\A$ to $\B$, which is impossible.  Now
    define $x$ to be the least value in $B$ such that $f(a) \neq x$ for all $a :
    \A$.  Because $f$ is an ordinal inequality function, this means that $f(a) <
    x$ for all $a : \A$.  Thus, $f$ can be considered to be a function from $\A$
    to $\B_x$.  $f$ is surjective, because for any $y < x$, by the minimality of
    $x$, there must be an $a : \A$ such that $f(y) = a$.  Thus, $f$ is an order
    isomorphism from $\A$ to $\B_x$.

    Now assume that there exists an $x : \B$ and an order isomorphism $f : \A
    \to \B_x$.  We must prove that $[\A] \leq [\B]$, and that $[\A] \neq [\B]$.

    For $[\A] \leq [\B]$, $f$ itself is already injective and preserves
    inequalities, so all that's left to prove is that for all $b : \B$, if $f(a)
    \neq b$ for all $a$, then $f(a) < b$ for all $a$.  So let $b$ be such that
    $f(a) \neq b$ for all $a$, and let $a : \A$.  If $b \leq f(a)$, because
    $f(a) < x$ by definition of $f$, we have $b < x$.  Thus, $b$ is in $\B_x$,
    so by surjectivity of $f$ there is some $z$ such that $f(z) = b$.  This
    however contradicts the definition of $b$.  Thus, we must have $f(a) < b$,
    showing that $f$ is an ordinal inequality function.

    For $[\A] \neq [\B]$, assume for a contradiction that $\B \sim \A$.  Then
    there is an order isomorphism $g : B \to A$.  Then we can prove by
    transfinite induction that for all $a : \B$, $a \leq f(g(a))$.  The
    inductive hypothesis states that for all $b < a$, $b \leq f(g(b))$.  Now if
    $f(g(a)) < a$, we would have $f(g(a)) < f(g(f(g(a))))$, and because $f$ and
    $g$ preserve inequalities, we have $a < f(g(a))$.  This contradicts $f(g(a))
    < a$, so we must have $a \leq f(g(a))$ for all $a : \B$.  This implies that
    $x \leq f(g(x))$, but by the definition of $f$ we must also have $f(g(x)) <
    x$, which is a contradiction.  Thus, we must have $[\A] \neq [\B]$.
\end{proof}

Even though we can normally get away with only proving antisymmetry and the
well-ordering property to prove that a type is well ordered, the proof that the
ordinals are well-ordered here uses the fact that the order is transitive and
connex, so those will have to be proven on their own anyway.

\begin{instance}
    The order on ordinals is connex.
\end{instance}

\begin{proof}
    The proof of this instance is rather involved and will be split up into
    several sub-definitions and lemmas.

    It suffices to prove that for all ordinals $\alpha$ and $\beta$, if $\alpha
    \nless \beta$, then $\beta \leq \alpha$.  Let $\A$ be an order type of
    $\alpha$ and $\B$ be an order type of $\beta$.  Then we know that $\A$ is
    not order isomorphic to an initial segment of $\B$.
    \begin{definition}
        Define a function $f : (p : \B) \to (\B_p \to \A) \to \A$ with $f(p,
        g)$ given by:
        \begin{itemize}
            \item If there exists an $a$ such that $g(x) < a$ for all $x < p$,
                define $f(p, g)$ to be the least such $a$.
            \item Otherwise, $f(p, g)$ will be a random throwaway value in $\A$.
                Such a value is guaranteed to exist since in this context, $p$ is a
                value in $\B$, so if $\A$ was empty, we would have $[\A] < [\B]$,
                which is impossible by assumption.
        \end{itemize}
        By transfinite recursion, there exists a function $g : \B \to \A$ such
        that for all $n : \B$, we have $g(n) = f(n, g \uparrow n)$.
    \end{definition}
    We will prove that $g$ is an ordinal inequality function, proving that
    $\beta \leq \alpha$.

    \begin{definition}
        Given an $n : \B$, define $G_n$ to be the set $\{z : \A \mid \forall x <
        n, g(x) < z\}$.
    \end{definition}

    \begin{definition}
        Say that a value $n : \B$ is \textit{good} if $G_n$ is nonempty.
    \end{definition}
    We will now prove several properties of $g$ and good values:

    \begin{lemma} \label{ord_connex_least}
        For all good $n$, $g(n)$ is the least element of $G_n$.
    \end{lemma}
    \begin{subproof}
        Because $n$ is good, $g(n) = f(n, g \uparrow n)$ takes the first branch
        in the definition of $f$, which produces exactly the least element of
        $G_n$.
    \end{subproof}

    \begin{lemma} \label{ord_connex_preserve}
        $g$ preserves strict inequalities for good values.
    \end{lemma}
    \begin{subproof}
        This follows directly from the previous lemma.
    \end{subproof}

    \begin{lemma} \label{ord_connex_inj}
        $g$ is injective for good values.
    \end{lemma}
    \begin{subproof}
        Let $a$ and $b$ be good values such that $g(a) = g(b)$.  For a
        contradiction, assume that $a \neq b$, and without loss of generality,
        assume that $a < b$.  Because $g$ preserves strict inequalities for good
        values, we have $g(a) < g(b)$, contradicting $g(a) = g(b)$.  Thus, $a =
        b$.
    \end{subproof}

    \begin{lemma} \label{ord_connex_all_good}
        All $n : \B$ are good.
    \end{lemma}
    \begin{subproof}
        The proof will be by transfinite induction on $n$.  The inductive
        hypothesis states that for all $b < n$, $b$ is good, and we must prove
        that $n$ is good.  For a contradiction, assume that $n$ is not good.  We
        will prove that $\A$ is isomorphic to $\B_n$, contradicting the
        hypothesis that $[\A] \nless [\B]$.  The isomorphism from $\B_n$ to $\A$
        will be $g$ itself.

        $g$ is injective for all $x < n$ because by the inductive hypothesis,
        all such $x$ are good, and $g$ is injective on good values by Lemma
        \ref{ord_connex_inj}.

        For surjectivity, let $x : \A$.  For a contradiction, assume that there
        does not exist any $y : \B_n$ such that $g(y) = x$.  We will prove that
        under this assumption, $n$ is in fact good, contradicting our previous
        assumption.  To prove that $n$ is good, we will prove that for all $a <
        n$, we have $g(a) < x$.  This will be proven using a second transfinite
        induction, this time on $a$: the second inductive hypothesis states that
        for all $b < a$ such that $b < n$ (note that this second condition is
        redundant since $a < n$), we have $g(b) < x$.  This is equivalent to
        saying that $x \in G_a$.  We assumed earlier that $g(y) \neq x$ for all
        $y$, so we only need to prove that $g(a) \leq x$.  By the first
        inductive hypothesis, $a$ is good, so by Lemma \ref{ord_connex_least},
        $g(a)$ is the least element of $G_a$.  Since $x \in G_a$, we have $g(a)
        \leq x$, as required.

        $g$ preserves inequalities for all $x < n$ because by the inductive
        hypothesis, all such $x$ are good, and $g$ preserves inequalities on
        good values by Lemma \ref{ord_connex_preserve}.

        Thus, under the assumption that $n$ is not good, $g$ is an order
        isomorphism from $\B_n$ to $\A$, which is impossible, so all $n : \B$
        are good.
    \end{subproof}
    At this point, most parts of the proof are complete.  Because all values are
    good, $g$ is an injective order-preserving function from $\B$ to $\A$.  We
    only need to prove the final condition to show that $g$ is an ordinal
    inequality function.  Let $b : \A$ with $g(a) \neq b$ for all $a : \B$, and
    let $a : \B$.  We must prove that $g(a) < b$.  The proof will be by
    transfinite induction on $a$.  The inductive hypothesis states that for all
    $c < a$, we have $g(c) < b$.  This means that $b \in G_a$.  Now since $g(a)
    \neq b$, we only need to prove that $g(a) \leq b$.  By Lemma
    \ref{ord_connex_least}, $g(a)$ is the least element of $G_a$, and since $b
    \in G_a$, $g(a) \leq b$ as required.
\end{proof}

\begin{lemma} \label{ord_le_part}
    Let $f$ be an ordinal inequality functions from $\A$ to $\B$.  For all $x$
    and $y$ such that all $a < x$ implies $f(a) \neq y$, we have $f(x) \leq y$.
\end{lemma}
\begin{proof}
    Assume for a contradiction that $y < f(x)$.  For an $a : \A$, if $a < x$, we
    have $f(a) \neq y$ by assumption, and if $x \leq a$, we have $y < f(x) \leq
    f(a)$, so $y \neq f(a)$.  Thus, for all $a : \A$, we have $f(a) \neq y$, so
    because $f$ is an ordinal inequality function, we have $f(x) < y$.  This
    contradicts $y < f(x)$, so it must be the case that $f(x) \leq y$ as
    required.
\end{proof}

\begin{instance}
    The order on ordinals is antisymmetric.
\end{instance}
\begin{proof}
    Let $\A$ and $\B$ be order types with ordinal inequality functions $f : \A
    \to \B$ and $g : \B \to \A$.  We will prove that $f$ itself is an order
    isomorphism.  It already preserves inequalities and is injective, so we only
    need to prove that it is surjective.  Let $x : \B$.  We will prove that
    $f(g(x)) = x$.  The proof will be by transfinite induction on $x$.  The
    inductive hypothesis states that for all $b < x$, $f(g(b)) = b$.

    We will now use Lemma \ref{ord_le_part} to prove that $f(g(x)) \leq x$.  So
    let $a < g(x)$.  We must prove that $f(a) \neq x$.  For a contradiction,
    assume that $f(a) = x$.  If $g(x) \leq a$, we would have a contradiction, so
    let's again use Lemma \ref{ord_le_part} to prove this.  So let $b < x$, and
    again for a contradiction assume that $g(b) = a$.  Then from $f(a) = x$ and
    $g(b) = a$, we have $f(g(b)) = x$.  $f(g(b)) = b$ by the inductive
    hypothesis, so $b = x$, contradicting $b < x$.

    All of these contradictions end up showing that, in fact, $f(g(x)) \leq x$.
    Again, we need to prove that $f(g(x)) = x$.  If $f(g(x)) \neq x$, we would
    have $f(g(x)) < x$, so by the inductive hypothesis we would have
    $f(g(f(g(x)))) = f(g(x))$, and since $f$ and $g$ are injective we have
    $f(g(x)) = x$ as required.
\end{proof}

\begin{instance}
    The order on ordinals is transitive.
\end{instance}
\begin{proof}
    Let $\A$, $\B$, and $\C$ be order types with ordinal inequality functions $f
    : \A \to \B$ and $g : \B \to \C$.  We need to find an ordinal inequality
    function from $\A$ to $\C$.  We will prove that $g \circ f$ is such a
    function.  Injectivity and preserving inequalities are already satisfied by
    instances \ref{inj_comp} and \ref{homo_le_compose}, so we only need to prove
    the final condition.

    Let $c : \C$ be such that for all $x : \A$, we have $g(f(x) \neq c$, and let
    $a : \A$ be an arbitrary value.  We must prove that $g(f(a)) < c$.  If $g(b)
    \neq c$ for all $b : \B$, we would have $g(f(a)) < c$ directly by $g$ being
    an ordinal inequality function, so now consider the case when there is a $b
    : \B$ such that $g(b) = c$.  Then what we need to prove is $g(f(a)) < g(b)$,
    which reduces to $f(a) < c$.  Because $g(f(x)) \neq c = g(b)$ for all $x$,
    there must be no $x$ such that $f(x) = b$.  From this and $f$ being an
    ordinal inequality function, we have $f(a) < c$ as required.
\end{proof}

\begin{lemma} \label{ords_lt_wo}
    For all ordinals $\alpha$, the initial segment of $\alpha$ is well-ordered.
\end{lemma}
\begin{proof}
    Let $\A$ be an order type, let $\beta < [\A]$, and let $S$ be a set on
    $\Ord_{[\A]}$, the initial segment of $[\A]$, with $\beta \in S$.  Now for
    all $a : \A$, the initial segment $\A_a$ is itself well-ordered.
    Furthermore, for all $a$, we have $[\A_a] < [\A]$ since $\A_a$ is itself an
    initial segment of $\A$.  Thus, we can define a function $f : \A \to
    \Ord_{[\A]}$ given by $f(a) = [\A_a]$.

    We will now prove that $f$ is surjective.  Let $[\Y]$ be an ordinal less
    than $[\A]$.  This means that there exists some $x : \A$ such that $\Y$ is
    order isomorphic to $\A_x$.  Then $f(x) = [\A_x] = [\Y]$ as required.

    We will now prove that $f$ preserves inequalities in one direction.  Let $a$
    and $b$ be values in $\A$ such that $a \leq b$.  Then we must prove that
    $f(a) \leq f(b)$.  Because $a \leq b$, everything in $\A_a$ is in $\A_b$, so
    we will prove that the identity function is an ordinal inequality function.
    Injectivity and preserving equalities is trivial.  For the final condition,
    we must prove that if $y \in \A_b$ is such that for all $m \in \A_a$, $m
    \neq y$, we have $x < y$ for all $x \in \A_a$.  For a contradiction, assume
    that $y \leq x$.  Then $y \leq x < a$, so $y \in \A_a$, meaning that $y \neq
    y$, which is a contradiction.  Thus, $x < y$, and the identity function is
    an ordinal inequality function.

    Now that we have a well-behaved function $f : \A \to \Ord_{[\A]}$, define a
    new set $S' = \{a \A \mid f(a) \in S\}$.  Because $f$ is surjective, there
    exists some $x : \A$ such that $f(a) = \beta$.  Since $\beta \in S$, we have
    $x \in S'$.  Thus, because $\A$ is well-ordered, $S'$ has a least element
    $m$.  We will prove that $f(m)$ is the least element of $S$.  $f(m) \in S$
    since $m \in S'$.  To prove that it's the least element in $S$, let $\gamma$
    be an ordinal less than $[\A]$ that is in $S$.  Since $f$ is surjective,
    there exists some $z$ such that $f(z) = \gamma$.  Then $z \in S'$, so $m
    \leq z$ because $m$ is the least element of $S'$.  Because $f$ preserves
    inequalities, $f(m) \leq f(z) = \gamma$, showing that $f(m)$ is the least
    element of $S$.
\end{proof}

\begin{instance}
    The order on ordinals is a well order.
\end{instance}
\begin{proof}
    Let $S$ be a set of ordinals, and let $\alpha \in S$.  If $\alpha$ is the
    least element of $S$ then we're done, so assume that $\alpha$ is not the
    least element of $S$.  Then this means that there exists some ordinal $\beta
    < \alpha$ such that $\beta \in S$.  Now define a set $S' = \{\gamma \in
    \Ord_\alpha \mid \gamma \in S\}$.  Since $\beta \in S'$, by Lemma
    \ref{ords_lt_wo} $S'$ has a least element $\mu < \alpha$.  We will prove
    that $\mu$ is the least element of $S$ as well.  $\mu$ is already an element
    of $S$ by virtue of being in $S'$, so we just need to prove that it's the
    least element of $S$.  So let $y \in S$.  If $y < \alpha$, Then $y \in S'$,
    and since $\mu$ is the least element of $S'$, $\mu \leq S'$.  If $y \geq
    \alpha$, then since $\mu < \alpha$, we have $\mu < y$.  Either way, $\mu
    \leq y$, showing that $\mu$ is the least element of $S$.
\end{proof}

\end{document}
