\documentclass[../../math.tex]{subfiles}
\externaldocument{../../math.tex}
\externaldocument{../basics/foundations}
\externaldocument{../basics/set}
\externaldocument{../basics/elementary_algebra}
\externaldocument{../basics/natural}

\begin{document}

\setcounter{chapter}{8}

\chapter{The Ordinals}

The cardinals and ordinals are concepts that extend the natural numbers to
infinite values in two different ways.  Traditionally, the cardinals were
conceived of as equivalence classes of types under the existence of bijections,
and the ordinals as equivalence classes of well-ordered types under the
existence of order isomorphisms.  However, under naive set theory, these
definitions are inconsistent due to problems with things like the set of all
ordinals or the set of all cardinals.

In modern set theories such as ZFC, ordinals are instead taken to be particular
sets, with cardinals being defined as particular ordinals.  In ZFC, there is no
``set of all ordinals'' or ``set of all cardinals'', which removes the
inconsistencies of the original definition.

However, in the foundations used here, there is another way around the
inconsistencies which allows us to define cardinals and ordinals using the
original definitions.  The exact way around the inconsistencies involves the
universe hierarchy, and some of the theorems here will need to be explicit about
what universe certain types are in.

While this chapter is mostly about the ordinals, a few aspects of the theory are
simpler with some of the very basics of the cardinals.  Thus, the basics of the
cardinals will be presented in this chapter as well.  They will be developed
further in the next chapter.

The arithmetic operations on ordinals can be defined in two different ways.  One
is to define them constructive using sum, product, and (subsets of) function
types.  This is useful because it will align with the cardinal operations.
However, these definitions can be rather cumbersome, especially for
exponentiation.  If these cumbersome definitions are not desirable, there also
exists a different way to construct the arithmetic operations using transfinite
recursion.  While using transfinite recursion makes the connection with cardinal
arithmetic more opaque, it ends up being simpler to define, so the definitions
using transfinite recursion are used here.  Unlike the cardinals, the ordinals
are described here mainly for their own sake, so the connection between the
arithmetic of the cardinals and ordinals is not really needed here.

\section{Construction of the Ordinals}

\begin{definition}
    Define a relation $\sim$ on well-ordered types where $\A \sim \B$ is defined
    to mean that there exists an order isomorphism between $\A$ and $\B$, that
    is, a function $f$ that is bijective and orderly.  Note that these
    conditions also imply that $f$ is equivalently orderly and strictly
    equivalently orderly.
\end{definition}

\begin{lemma}
    The relation $\sim$ is an equivalence relation.
\end{lemma}
\begin{proof}
    \textit{Reflexivitity.}  The identity function is an order isomorphism.

    \textit{Symmetry.}  Assume that there is an order isomorphism $f$ from $\A$
    to $\B$.  Then its inverse $f^{-1}$ is bijective, and when $a \leq b$, we
    have $f(f^{-1}(a)) \leq f(f^{-1}(b))$, and because $f$ is an order
    isomorphism, $f^{-1}(a) \leq f^{-1}(b)$, so $f^{-1}$ is an order isomorphism
    from $\B$ to $\A$.

    \textit{Transitivity.}  Given three well-ordered types $\A$, $\B$, and $\C$
    with order isomorphisms $f : \A \to \B$ and $g : \B \to \C$, $g \circ f$ is
    an order isomorphism from $\A$ to $\C$.
\end{proof}

\begin{definition}
    Define the type $\Ord$ to be the quotient of all well-ordered types by
    $\sim$.  Given a well-ordered type $\A$, we will call $[\A]$ the order type
    of $\A$.
\end{definition}

Note that because $\Ord$ is a type that contains other types, $\Ord$ must be
higher on the universe hierarchy than the well-ordered types are.  In
particular, if the well-ordered types live in $\Type_n$, then $\Ord$ lives in
$\Type_{n+1}$.  This means that it is impossible to build an ordinal out of a
set of ordinals.  So while there is a type of all ordinals, it lives in a
different universe than the well-ordered types used to make the ordinals,
removing the inconsistencies of naive set theory.  However, being unable to
create ordinals out of sets of ordinals entirely seems to be a major restriction
on the theory.  We will find a way around this restriction in a few sections
after developing the theory of cardinals for a bit.

\section{Ordering Ordinals}

It turns out that the ordinals themselves are well ordered.  This ordering is so
important that unlike the previous types that have been developed, this ordering
will be developed before the other operations.  Most sources describe the order
on ordinals using strict inequalities, but describing them using non-strict
inequalities can be useful at times too and is the default way to describe
orderings in the Rocq code, so I will first describe the order on ordinals using
non-strict inequalities.

\begin{definition}
    Let $\A$ and $\B$ be well-ordered types.  Then we say that $f : \A \to \B$
    is an ordinal inequality function if it is injective, orderly, and if for
    all $b : \B$ with $f(a) \neq b$ for all $a : \A$, we have $f(a) < b$ for all
    $a : \A$.
\end{definition}

\begin{definition}
    Given two well-ordered types $\A$ and $\B$, we will say that $\A \leqq \B$
    if there exists an ordinal inequality function from $\A$ to $\B$.
\end{definition}

\begin{lemma}
    The relation $\leqq$ is well-defined under the relation $\sim$.
\end{lemma}
\begin{proof}
    Let $A$, $B$, $C$, and $D$ be well-ordered types with order isomorphisms $f
    : A \to B$ and $g : C \to D$, and let $h$ be an ordinal inequality function
    from $A$ to $C$.  We must prove that there exists an ordinal inequality
    function from $B$ to $D$.  We will prove that $g \circ h \circ f^{-1}$ is
    such a function.

    \noindent \textit{Injectivity.}
    $g$, $h$, and $f^{-1}$ are all injective, so by Instance \ref{inj_comp}, $g
    \circ h \circ f^{-1}$ is as well.

    \noindent \textit{Orderly.}
    Let $a \leq b$.  Then $f(f^{-1}(a)) \leq f(f^{-1}(b))$, and because $f$ is
    orderly, $f^{-1}(a) \leq f^{-1}(b)$.  Then because $g$ and $h$ are orderly,
    $g(h(f^{-1}(a))) \leq g(h(f^{-1}(b)))$, showing that $g \circ h \circ
    f^{-1}$ is orderly.

    \noindent \textit{Ordinal inequality condition.}
    Let $b : D$ be such that there is no $a : B$ such that $g(h(f^{-1}(a))) =
    b$.  This means that there is no $x : A$ such that $g(h(f^{-1}(f(x)))) =
    g(h(x)) = b = g(g^{-1}(b))$, so by the injectivity of $g$ there is no $x :
    A$ such that $h(x) = g^{-1}(b)$.  Because $h$ is an ordinal inequality
    function, this means that for all $x : A$, $h(x) < g^{-1}(b)$.  Thus, for
    any $a : B$, we have $h(f^{-1}(a)) < g^{-1}(b)$, and applying $g$ to both
    sides gives $g(h(f^{-1}(a))) < b$.
\end{proof}

\begin{instance}
    Define order in the ordinals as the operation obtained by applying Theorem
    \ref{binary_op_ex} to the previous lemma.
\end{instance}

As the following theorem shows, this order is equivalent to the usual definition
of the order on ordinals.

\begin{theorem} \label{ord_lt_simpl}
    For two well-ordered $\A$ and $\B$, $[\A] < [\B]$ iff there exists a $x :
    \B$ such that $\A \sim \B_x$, the initial segment of $x$ in $\B$.
\end{theorem}
\begin{proof}
    First assume that $[\A] < [\B]$.  This means that there exists an ordinal
    inequality function $f : \A \to \B$, but no order isomorphism from $\A$ to
    $\B$.  First, there exists an $x' : \B$ such that $f(a) \neq x'$ for all $a
    : \A$.  For if there wasn't such an $x'$, $f$ would be surjective, meaning
    that it is an order isomorphism from $\A$ to $\B$, which is impossible.  Now
    define $x$ to be the least value in $B$ such that $f(a) \neq x$ for all $a :
    \A$.  Because $f$ is an ordinal inequality function, this means that $f(a) <
    x$ for all $a : \A$.  Thus, $f$ can be considered to be a function from $\A$
    to $\B_x$.  $f$ is surjective, because for any $y < x$, by the minimality of
    $x$, there must be an $a : \A$ such that $f(y) = a$.  Thus, $f$ is an order
    isomorphism from $\A$ to $\B_x$.

    Now assume that there exists an $x : \B$ and an order isomorphism $f : \A
    \to \B_x$.  We must prove that $[\A] \leq [\B]$, and that $[\A] \neq [\B]$.

    For $[\A] \leq [\B]$, $f$ itself is already injective and orderly, so all
    that's left to prove is that for all $b : \B$, if $f(a) \neq b$ for all $a$,
    then $f(a) < b$ for all $a$.  So let $b$ be such that $f(a) \neq b$ for all
    $a$, and let $a : \A$.  If $b \leq f(a)$, because $f(a) < x$ by definition
    of $f$, we have $b < x$.  Thus, $b$ is in $\B_x$, so by surjectivity of $f$
    there is some $z$ such that $f(z) = b$.  This however contradicts the
    definition of $b$.  Thus, we must have $f(a) < b$, showing that $f$ is an
    ordinal inequality function.

    For $[\A] \neq [\B]$, assume for a contradiction that $\B \sim \A$.  Then
    there is an order isomorphism $g : B \to A$.  Then we can prove by
    transfinite induction that for all $a : \B$, $a \leq f(g(a))$.  The
    inductive hypothesis states that for all $b < a$, $b \leq f(g(b))$.  Now if
    $f(g(a)) < a$, we would have $f(g(a)) < f(g(f(g(a))))$, and because $f$ and
    $g$ are orderly, we have $a < f(g(a))$.  This contradicts $f(g(a)) < a$, so
    we must have $a \leq f(g(a))$ for all $a : \B$.  This implies that $x \leq
    f(g(x))$, but by the definition of $f$ we must also have $f(g(x)) < x$,
    which is a contradiction.  Thus, we must have $[\A] \neq [\B]$.
\end{proof}

Even though we can normally get away with only proving antisymmetry and the
well-ordering property to prove that a type is well ordered, the proof that the
ordinals are well-ordered here uses the fact that the order is transitive and
connex, so those will have to be proven on their own anyway.

\begin{instance}
    The order on ordinals is connex.
\end{instance}
\begin{proof}
    The proof of this instance is rather involved and will be split up into
    several sub-definitions and lemmas.

    It suffices to prove that for all ordinals $\alpha$ and $\beta$, if $\alpha
    \nless \beta$, then $\beta \leq \alpha$.  Let $\A$ be a well-ordered type
    with order type $\alpha$ and $\B$ be a well-ordered type of order type
    $\beta$.  Then we know that $\A$ is not order isomorphic to an initial
    segment of $\B$.
    \begin{definition}
        Define a function $f : (p : \B) \to (\B_p \to \A) \to \A$ with $f(p,
        g)$ given by:
        \begin{itemize}
            \item If there exists an $a$ such that $g(x) < a$ for all $x < p$,
                define $f(p, g)$ to be the least such $a$.
            \item Otherwise, $f(p, g)$ will be a random throwaway value in $\A$.
                Such a value is guaranteed to exist since in this context, $p$ is a
                value in $\B$, so if $\A$ was empty, we would have $[\A] < [\B]$,
                which is impossible by assumption.
        \end{itemize}
        By transfinite recursion, there exists a function $g : \B \to \A$ such
        that for all $n : \B$, we have $g(n) = f(n, g \uparrow n)$.
    \end{definition}
    We will prove that $g$ is an ordinal inequality function, proving that
    $\beta \leq \alpha$.

    \begin{definition}
        Given an $n : \B$, define $G_n$ to be the set $\{z : \A \mid \forall x <
        n, g(x) < z\}$.
    \end{definition}

    \begin{definition}
        Say that a value $n : \B$ is \textit{good} if $G_n$ is nonempty.
    \end{definition}
    We will now prove several properties of $g$ and good values:

    \begin{lemma} \label{ord_connex_least}
        For all good $n$, $g(n)$ is the least element of $G_n$.
    \end{lemma}
    \begin{subproof}
        Because $n$ is good, $g(n) = f(n, g \uparrow n)$ takes the first branch
        in the definition of $f$, which produces exactly the least element of
        $G_n$.
    \end{subproof}

    \begin{lemma} \label{ord_connex_preserve}
        $g$ is strictly orderly for good values.
    \end{lemma}
    \begin{subproof}
        This follows directly from the previous lemma.
    \end{subproof}

    \begin{lemma} \label{ord_connex_inj}
        $g$ is injective for good values.
    \end{lemma}
    \begin{subproof}
        Let $a$ and $b$ be good values such that $g(a) = g(b)$.  For a
        contradiction, assume that $a \neq b$, and without loss of generality,
        assume that $a < b$.  Because $g$ is stirctly orderly for good values,
        we have $g(a) < g(b)$, contradicting $g(a) = g(b)$.  Thus, $a = b$.
    \end{subproof}

    \begin{lemma} \label{ord_connex_all_good}
        All $n : \B$ are good.
    \end{lemma}
    \begin{subproof}
        The proof will be by transfinite induction on $n$.  The inductive
        hypothesis states that for all $b < n$, $b$ is good, and we must prove
        that $n$ is good.  For a contradiction, assume that $n$ is not good.  We
        will prove that $\A$ is isomorphic to $\B_n$, contradicting the
        hypothesis that $[\A] \nless [\B]$.  The isomorphism from $\B_n$ to $\A$
        will be $g$ itself.

        $g$ is injective for all $x < n$ because by the inductive hypothesis,
        all such $x$ are good, and $g$ is injective on good values by Lemma
        \ref{ord_connex_inj}.

        For surjectivity, let $x : \A$.  For a contradiction, assume that there
        does not exist any $y : \B_n$ such that $g(y) = x$.  We will prove that
        under this assumption, $n$ is in fact good, contradicting our previous
        assumption.  To prove that $n$ is good, we will prove that for all $a <
        n$, we have $g(a) < x$.  This will be proven using a second transfinite
        induction, this time on $a$: the second inductive hypothesis states that
        for all $b < a$ such that $b < n$ (note that this second condition is
        redundant since $a < n$), we have $g(b) < x$.  This is equivalent to
        saying that $x \in G_a$.  We assumed earlier that $g(y) \neq x$ for all
        $y$, so we only need to prove that $g(a) \leq x$.  By the first
        inductive hypothesis, $a$ is good, so by Lemma \ref{ord_connex_least},
        $g(a)$ is the least element of $G_a$.  Since $x \in G_a$, we have $g(a)
        \leq x$, as required.

        $g$ is orderly for all $x < n$ because by the inductive hypothesis, all
        such $x$ are good, and $g$ is orderly on good values by Lemma
        \ref{ord_connex_preserve}.

        Thus, under the assumption that $n$ is not good, $g$ is an order
        isomorphism from $\B_n$ to $\A$, which is impossible, so all $n : \B$
        are good.
    \end{subproof}
    At this point, most parts of the proof are complete.  Because all values are
    good, $g$ is an injective orderly function from $\B$ to $\A$.  We only need
    to prove the final condition to show that $g$ is an ordinal inequality
    function.  Let $b : \A$ with $g(a) \neq b$ for all $a : \B$, and let $a :
    \B$.  We must prove that $g(a) < b$.  The proof will be by transfinite
    induction on $a$.  The inductive hypothesis states that for all $c < a$, we
    have $g(c) < b$.  This means that $b \in G_a$.  Now since $g(a) \neq b$, we
    only need to prove that $g(a) \leq b$.  By Lemma \ref{ord_connex_least},
    $g(a)$ is the least element of $G_a$, and since $b \in G_a$, $g(a) \leq b$
    as required.
\end{proof}

\begin{lemma} \label{ord_le_part}
    Let $f$ be an ordinal inequality functions from $\A$ to $\B$.  For all $x$
    and $y$ such that all $a < x$ implies $f(a) \neq y$, we have $f(x) \leq y$.
\end{lemma}
\begin{proof}
    Assume for a contradiction that $y < f(x)$.  For an $a : \A$, if $a < x$, we
    have $f(a) \neq y$ by assumption, and if $x \leq a$, we have $y < f(x) \leq
    f(a)$, so $y \neq f(a)$.  Thus, for all $a : \A$, we have $f(a) \neq y$, so
    because $f$ is an ordinal inequality function, we have $f(x) < y$.  This
    contradicts $y < f(x)$, so it must be the case that $f(x) \leq y$ as
    required.
\end{proof}

\begin{instance}
    The order on ordinals is antisymmetric.
\end{instance}
\begin{proof}
    Let $\A$ and $\B$ be well-ordered types with ordinal inequality functions $f
    : \A \to \B$ and $g : \B \to \A$.  We will prove that $f$ itself is an order
    isomorphism.  It is already orderly and injective, so we only need to prove
    that it is surjective.  Let $x : \B$.  We will prove that $f(g(x)) = x$.
    The proof will be by transfinite induction on $x$.  The inductive hypothesis
    states that for all $b < x$, $f(g(b)) = b$.

    We will now use Lemma \ref{ord_le_part} to prove that $f(g(x)) \leq x$.  So
    let $a < g(x)$.  We must prove that $f(a) \neq x$.  For a contradiction,
    assume that $f(a) = x$.  If $g(x) \leq a$, we would have a contradiction, so
    let's again use Lemma \ref{ord_le_part} to prove this.  So let $b < x$, and
    again for a contradiction assume that $g(b) = a$.  Then from $f(a) = x$ and
    $g(b) = a$, we have $f(g(b)) = x$.  $f(g(b)) = b$ by the inductive
    hypothesis, so $b = x$, contradicting $b < x$.

    All of these contradictions end up showing that, in fact, $f(g(x)) \leq x$.
    Again, we need to prove that $f(g(x)) = x$.  If $f(g(x)) \neq x$, we would
    have $f(g(x)) < x$, so by the inductive hypothesis we would have
    $f(g(f(g(x)))) = f(g(x))$, and since $f$ and $g$ are injective we have
    $f(g(x)) = x$ as required.
\end{proof}

We can now prove a simpler form of ordinal inequalities.

\begin{lemma} \label{ord_le_simpl}
    For all well-ordered types $\A$ and $\B$, if there exists an injective
    orderly function from $\A$ to $\B$, then $[\A] \leq [\B]$.
\end{lemma}
\begin{proof}
    Let $f$ be an injective orderly function from $\A$ to $\B$, and assume for a
    contradiction that $[\B] < [\A]$.  This means that there exists $x : \A$ and
    an order isomorphism $g : \B \to \A_x$.  Defining $h = g \circ f$, we have
    an injective orderly function from $\A$ to $\A_x$.

    Let $a$ be an arbitrary value in $\A$.  We will prove by transfinite
    induction that $a \leq h(a)$.  The inductive hypothesis states that for all
    $b < a$, we have $b \leq h(b)$.  Now if $h(a) < a$, we would have $h(a) \leq
    h(h(a))$ by the inductive hypothesis, so $a \leq h(a)$ by $h$ being
    equivalently orderly.  $h(a) < a$ contradicts $a \leq h(a)$, proving that $a
    \leq h(a)$ for all $a$.

    By the previous paragraph, we have $x \leq h(x)$.  But by definition, $h(x)
    < x$.  This is a contradiction, showing that our original assumption that
    $[\B] < [\A]$ is wrong.  Thus, $[\A] \leq [\B]$.
\end{proof}

With this lemma, we no longer need to prove that the ordinal inequality
condition holds when trying to prove that one ordinal is less than or equal to
another.

\begin{instance}
    The order on ordinals is transitive.
\end{instance}
\begin{proof}
    Let $\A$, $\B$, and $\C$ be well-ordered types with ordinal inequality
    functions $f : \A \to \B$ and $g : \B \to \C$.  By Lemma \ref{ord_le_simpl},
    We need to find an injective orderly function from $\A$ to $\C$.  By
    instances \ref{inj_comp} and \ref{homo_le_compose}, $g \circ f$ is such a
    function.
\end{proof}

Because we can't have $\Type_n : \Type_n$, it is impossible to make an ordinal
out of ordinals.  However, in many of the cases where we would want to make an
ordinal out of ordinals, using the following function works well enough.

\begin{definition} \label{ord_type_init_ord}
    Given a well-ordered type $\A$, define a function $\mathcal O_\A : \A \to
    \Ord_{[\A]}$ that takes $a : \A$ to $[\A_a]$.  This is well-defined by
    Theorem \ref{ord_lt_simpl}.
\end{definition}

\begin{theorem}
    For all order types $\A$, $\mathcal O_\A$ is orderly.
\end{theorem}
\begin{proof}
    Let $a$ and $b$ be values in $\A$ such that $a \leq b$.  Then we must prove
    that $\mathcal O_\A(a) \leq \mathcal O_\A(b)$.  By Lemma \ref{ord_le_simpl},
    we need to find an injective orderly function from $\A_a$ to $\A_b$.
    Because $a \leq b$, everything in $\A_a$ is in $\A_b$, so the identity
    function is such a function.
\end{proof}

\begin{theorem}
    For all well-ordered types $\A$, $\mathcal O_\A$ is bijective.
\end{theorem}
\begin{proof}
    For injectivity, let $a$ and $b$ be such that $\mathcal O_\A(a) = \mathcal O_\A(b)$.  We will
    prove by antisymmetry that $a = b$.  By symmetry, we only need to prove that
    $a \leq b$.  Because $\mathcal O_\A(a) = \mathcal O_\A(b)$, we have an order
    isomorphism $g$ from $\A_a$ to $\A_b$.  We will prove that for all $x < a$,
    $x \leq g(x)$.  By transfinite induction we can assume that for all $c < x$,
    we have $c \leq g(c)$.  Then if $g(x) < x$, we have $g(x) < a$, so by the
    inductive hypothesis we have $g(x) \leq g(g(x))$.  Because $g$ is
    equivalently orderly, we have $x \leq g(x)$, contradicting $g(x) < x$.
    Thus, we must have $x \leq g(x)$ for all $x < a$.

    Now assume that $b < a$.  Then we have $b \leq g(b)$ by the previous
    paragraph, but $g(b) < b$ by definition of $g$.  Thus, we must have $a \leq
    b$, proving that $\mathcal O_\A$ is injective.

    For surjectivity, let $[\Y]$ be an ordinal less than $[\A]$.  This means
    that there exists some $x : \A$ such that $\Y$ is order isomorphic to
    $\A_x$.  Then $\mathcal O_\A(x) = [\A_x] = [\Y]$ as required.
\end{proof}

\begin{lemma} \label{ords_lt_wo}
    For all ordinals $\alpha$, the initial segment of $\alpha$ is well-ordered.
\end{lemma}
\begin{proof}
    Let $\A$ be a well-ordered type, let $\beta < [\A]$, and let $S$ be a set on
    $\Ord_{[\A]}$, the initial segment of $[\A]$, with $\beta \in S$.  Define a
    new set $S' = \{a \A \mid \mathcal O_\A(a) \in S\}$.  Because $\mathcal
    O_\A$ is surjective, there exists some $x : \A$ such that $\mathcal O_\A(a)
    = \beta$.  Since $\beta \in S$, we have $x \in S'$.  Thus, because $\A$ is
    well-ordered, $S'$ has a least element $m$.  We will prove that $\mathcal
    O_\A(m)$ is the least element of $S$.  $\mathcal O_\A(m) \in S$ since $m \in
    S'$.  To prove that it's the least element in $S$, let $\gamma$ be an
    ordinal less than $[\A]$ that is in $S$.  Since $\mathcal O_\A$ is
    surjective, there exists some $z$ such that $\mathcal O_\A(z) = \gamma$.
    Then $z \in S'$, so $m \leq z$ because $m$ is the least element of $S'$.
    Because $\mathcal O_\A$ is orderly, $\mathcal O_\A(m) \leq \mathcal O_\A(z)
    = \gamma$, showing that $\mathcal O_\A(m)$ is the least element of $S$.
\end{proof}

\begin{instance}
    The order on ordinals is a well order.
\end{instance}
\begin{proof}
    Let $S$ be a set of ordinals, and let $\alpha \in S$.  If $\alpha$ is the
    least element of $S$ then we're done, so assume that $\alpha$ is not the
    least element of $S$.  Then this means that there exists some ordinal $\beta
    < \alpha$ such that $\beta \in S$.  Now define a set $S' = \{\gamma \in
    \Ord_\alpha \mid \gamma \in S\}$.  Since $\beta \in S'$, by Lemma
    \ref{ords_lt_wo} $S'$ has a least element $\mu < \alpha$.  We will prove
    that $\mu$ is the least element of $S$ as well.  $\mu$ is already an element
    of $S$ by virtue of being in $S'$, so we just need to prove that it's the
    least element of $S$.  So let $y \in S$.  If $y < \alpha$, Then $y \in S'$,
    and since $\mu$ is the least element of $S'$, $\mu \leq S'$.  If $y \geq
    \alpha$, then since $\mu < \alpha$, we have $\mu < y$.  Either way, $\mu
    \leq y$, showing that $\mu$ is the least element of $S$.
\end{proof}

Thus, we can now perform transfinite induction and recursion on the ordinals.
However, given that not all sets of ordinals have a supremum, the use of
transfinite recursion seems to be severely limited.  However, if we first
develop some of the theory of cardinals, we will find a way around this issue,
making transfinite recursion of ordinals as powerful as it is in other
foundations.

\section{Construction of the Cardinals} \label{card_base}

The construction of the cardinals is similar to the construction of the
ordinals.

\begin{definition}
    Define a relation $\sim$ on types where $\A \sim \B$ is defined to mean that
    there exists a bijective function from $\A$ to $\B$.
\end{definition}

\begin{lemma}
    The relation $\sim$ is an equivalence relation.
\end{lemma}
\begin{proof}
    For reflexivity, the identity function is bijective, for symmetry, the
    inverse of a bijective function is bijective, and for transitivity, the
    composition of two bijective functions is bijective.
\end{proof}

\begin{definition}
    Define the type $\Card$ to be the quotient of all types by $\sim$.
\end{definition}

Like the ordinals, the cardinals are a type that contains other types, so if the
types used to build the cardinals are of type $\Type_n$, the cardinals are of
type $\Type_{n+1}$.  This solves the naive set theory paradoxes in the same way
as with the ordinals, and also introduces the same issues as before.  Again, we
will solve these issues a little later.

\begin{definition}
    Given a type $\A$, we will use the notation $|\A|$ to mean the cardinality
    of $\A$, that is, the equivalence class of $\A$.
\end{definition}

We can also relate the cardinals to the ordinals in a few ways.

\begin{lemma}
    For all order types $\A$ and $\B$, if $\A \sim \B$ in the ordinal sense,
    then $|\A| = |\B|$.
\end{lemma}
\begin{proof}
    Because $\A \sim \B$, we have an order isomorphism between $\A$ and $\B$.
    Order isomorphisms are bijective, meaning that $|\A| = |\B|$.
\end{proof}

\begin{definition}
    Define a function from the ordinals to the cardinals given by Theorem
    \ref{unary_op_ex} and the previous lemma.  At the possible risk of abusing
    notation, denote this function with $|\alpha|$ for a given ordinal $\alpha$.
\end{definition}

\begin{lemma}
    For all cardinals $\kappa$, there exists an ordinal $\alpha$ such that
    $|\alpha| = \kappa$.
\end{lemma}
\begin{proof}
    Let $\A$ be a type.  By the well-ordering theorem, $\A$ can be well-ordered.
    Then the ordinal $[\A]$ is such that $|[\A]| = |\A|$ as required.
\end{proof}

\begin{definition}
    Define a function from the cardinals to the ordinals given by the least
    ordinal satisfying the result of the previous theorem.  Denote this function
    $\lfloor \kappa \rfloor$ for a given cardinal $\kappa$.
\end{definition}

\begin{theorem} \label{card_to_initial_ord_to_card_eq}
    For all cardinals $\kappa$, we have $|\lfloor \kappa \rfloor| = \kappa$.
\end{theorem}
\begin{proof}
    This is true by definition.
\end{proof}

\begin{instance}
    The function $\lfloor \kappa \rfloor$ is injective.
\end{instance}
\begin{proof}
    Let $\kappa$ and $\mu$ be cardinals such that $\lfloor \kappa \rfloor =
    \lfloor \mu \rfloor$.  Then we can take the cardinality of both sides to get
    $|\lfloor \kappa \rfloor| = |\lfloor \mu \rfloor|$.  By Theorem
    \ref{card_to_initial_ord_to_card_eq}, we have $\kappa = \mu$.
\end{proof}

\begin{theorem} \label{ord_to_card_to_initial_ord_le}
    For all ordinals $\alpha$, we have $\lfloor|\alpha|\rfloor \leq \alpha$.
\end{theorem}
\begin{proof}
    This follows directly from $\lfloor|\alpha|\rfloor$ being the smallest
    ordinal equivalent to $|\alpha|$.
\end{proof}

\begin{theorem} \label{ord_to_card_eq} \label{ord_to_card_eq1}
    \label{ord_to_card_eq2}
    For all ordinals $\alpha$ and types $\B$, to prove that $|\alpha| = |\B|$,
    it suffices to prove that there exists a bijective function between
    $\Ord_\alpha$ and $\B$.
\end{theorem}
\begin{proof}
    Let $\A$ be an order type for $\alpha$, and let $g$ be a bijective function
    between $\Ord_\alpha$ and $\B$.  Let $f_\A$ be the function given in
    Definition \ref{ord_type_init_ord}.  Then $g \circ f_\A$ is a bijective
    function from $\A$ to $\B$, proving that $|\alpha| = |\B|$.
\end{proof}

\begin{theorem} \label{ord_to_card_init}
    With $f$ being the function given by Definition \ref{ord_type_init_ord}, for
    all well-ordered types $\A$ and values $a : \A$, $|f_\A(a)| = |\A_a|$.
\end{theorem}
\begin{proof}
    By the previous theorem, it suffices to prove that there exists a bijective
    function from the initial segment of $a$ to the initial segment of
    $f_\A(a)$.  $f_\A$ itself is already a function from $\A$ to the ordinals,
    and for any $x < a$, we have $f_\A(x) < f_\A(a)$, so $f_\A$ is a function
    from the initial segment of $a$ to the initial segment of $f_\A$.  It is
    injective because $f_\A$ as a whole is injective.  To prove that it is
    surjective, let $\gamma < f_\A(a)$.  Because $f_\A(a) < [\A]$, we have
    $\gamma < [\A]$, so by the surjectivity of $f_\A$ we have some $x$ such that
    $f_\A(x) = \gamma$.  Then $f_\A(x) < f_\A(a)$, so $x < a$, meaning that $x$
    is something in the initial segment of $a$ with $f_\A(x) = \gamma$, proving
    that $f_\A$ is surjective on these initial segments.
\end{proof}

\section{Ordering Cardinals}

\begin{definition}
    Let $\A$ and $\B$ be types.  Then we say that $\A \leqq \B$ if there exists
    an injective function from $\A$ to $\B$.
\end{definition}

\begin{lemma}
    The relation $\leqq$ is well-defined.
\end{lemma}
\begin{proof}
    Let $\A$, $\B$, $\C$, and $\D$ be types with bijective functions $f : \A \to
    \B$ and $g : \C \to \D$, and an injective function $h : \A \to \C$.  Then
    because the composition of injective functions is injective, the function $g
    \circ h \circ f^{-1}$ is an injective function from $\B$ to $\D$.
\end{proof}

\begin{instance}
    Define order in the cardinals as the operation obtained by applying Theorem
    \ref{binary_op_ex} to the previous lemma.
\end{instance}

\begin{instance}[Schr\"oder-Bernstein]
    The order on the cardinals is antisymmetric.
\end{instance}
\begin{proof}
    Let $\A$ and $\B$ be types with injective functions $f : \A \to \B$ and $g :
    \B \to \A$.  We will now make several definitions.  First, we will say that
    a $b : \B$ is \textit{lonely} if $f(a) \neq b$ for all $a$.  We say that a
    value $b_0 : \B$ is a descendant of a value $b_1 : \B$ if there exists an
    $n$ such that $b_1 = (f \circ g)^n(b_0)$.  We will say that a value $a : \A$
    has an ancestor if there exists a lonely $b$ such that $b$ is a descendant
    of $f(a)$.

    First, we will prove that for every $a$ that has an ancestor, there exists
    some $c : \B$ such that $g(c) = a$.  Because $a$ has an ancestor, there
    exsits some lonely $b : \B$ and natural number $n$ with $f(a) = (f \circ
    g)^n(b)$.  If $n = 0$, we would have $f(a) = b$, contradicting $b$ being
    lonely, so this case is impossible.  If $n \neq 0$, we have $f(a) = f(g((f
    \circ g)^{n-1}(b)))$, and by injectivity of $f$ we have $a = g((f \circ
    g)^{n-1}(b))$, showing that $(f \circ g)^{n-1}(b)$ is the value we are
    looking for.  Either way, we have proven that for every $a$ that has an
    ancestor, there exists some $c : \B$ such that $g(c) = a$.

    Define a new function $p$ that takes in an $a : \A$ that has an ancestor and
    produces the parent of $a$ as guaranteed by the previous paragraph.  Now
    define a function $h : \A \to \B$ given by
    \[
        h(a) = \begin{cases}
            p(a) &\text{if $a$ has an ancestor} \\
            f(a) &\text{otherwise.}
        \end{cases}
    \]
    We will now prove that $h$ is bijective, proving the theorem.

    For injectivity, let $a_1$ and $a_2$ be such that $h(a_1) = h(a_2)$.  We
    need to prove that $a_1 = a_2$.  We have three cases: when $a_1$ and $a_2$
    both have ancestors, when neither have an ancestor, and when one has an
    ancestor and the other doesn't.
    \begin{case} \textit{$a_1$ and $a_2$ both have ancestors.}
        We now have $p(a_1) = p(a_2)$.  Applying $g$ to both sides, we get
        $g(p(a_1)) = g(p(a_2))$.  By the definition of $p$, we get $a_1 = a_2$
        as required.
    \end{case}
    \begin{case} \textit{One of $a_1$ and $a_2$ have an ancestor, and the other
        doesn't.}
        Without loss of generality, assume that $a_1$ has an ancester and $a_2$
        does not.  Then we have $p(a_1) = f(a_2)$.  Applying $g$ to both sides
        and using the definition of $p$, we get $a_1 = g(f(a_2))$.  Because
        $a_1$ has an ancestor and $a_1 = g(f(a_2))$, there exists a lonely $b$
        and a natural number $n$ such that $f(g(f(a_2))) = (f \circ g)^n(b)$.
        Now if $n = 0$, we would have $f(g(f(a_2))) = b$, contradicting $b$
        being lonely.  If $n \neq 0$, then we would have $f(g(f(a_2))) = f(g((f
        \circ g)^{n-1}(b)))$, which by injectivity means that $f(a_2) = (f \circ
        g)^{n-1}(b)$.  This means that $a_2$ has an ancestor, showing that this
        case is actually impossible.
    \end{case}
    \begin{case} \textit{Neither $a_1$ or $a_2$ have an ancestor.}
        We now have $f(a_1) = f(a_2)$, so by injectivity we have $a_1 = a_2$ as
        required.
    \end{case}

    For surjectivity, let $y : \B$.  There will be two cases: when there exists
    a lonely $b$ such that $y$ is a descendant of $b$, or when no such $b$
    exists.
    \setcounter{case}{0}
    \begin{case} \textit{There exists a lonely $b$ such that $y$ is a descendant
        of $b$.}
        Because $y$ is a descendant of $b$, there exists an $n$ such that $y =
        (f \circ g)^n(b)$.  Then $f(g(y)) = (f \circ g)^{n+1}(b)$, showing that
        $g(y)$ has an ancestor.  Thus, $h(g(y)) = p(g(y)) = y$ as required.
    \end{case}
    \begin{case} \text{There does not exist a lonely $b$ such that $y$ is a
        descendent of $b$.}
        In particular, since $y$ is trivially a descendent of itself, $y$ must
        be not lonely.  This implies that there exists an $a$ with $f(a) = y$.
        Then by the assumption of this case we know that there does not exist a
        lonely $b$ such that $f(a)$ is a descendent of $b$, which is precisely
        the statement that $a$ has no ancestor.  Thus $h(a) = f(a) = y$ as
        required.
    \end{case}
\end{proof}

The order of the cardinals and the order of the ordinals are related in many
ways.

\begin{theorem} \label{ord_to_card_le}
    For all ordinals $\alpha$ and $\beta$, if $\alpha \leq \beta$, then
    $|\alpha| \leq |\beta|$.
\end{theorem}
\begin{proof}
    Let $\A$ and $\B$ be well-ordered types representing $\alpha$ and $\beta$
    respectively.  By $\alpha \leq \beta$, we have an ordinal inequality
    function from $\A$ to $\B$.  Ordinal inequality functions are injective,
    proving that $|\alpha| \leq |\beta|$ as well.
\end{proof}

\begin{theorem} \label{card_to_initial_ord_le}
    For all cardinals $\kappa$ and $\mu$, if $\lfloor \kappa \rfloor \leq
    \lfloor \mu \rfloor$, then $\kappa \leq \mu$.
\end{theorem}
\begin{proof}
    From the previous theorem, we have $|\lfloor \kappa \rfloor| \leq |\lfloor
    \mu \rfloor|$, and the result follows directly from Theorem
    \ref{card_to_initial_ord_to_card_eq}.
\end{proof}

We can now prove the last of the fundamental properties of the order on
cardinals.

\begin{instance}
    The order on the cardinals is a well order.
\end{instance}
\begin{proof}
    Let $S$ be a nonempty set of cardinals.  Define a set $S'$ on the ordinals
    given by $\{\alpha : \Ord \mid \exists \kappa \in S, \lfloor \kappa \rfloor
    = \alpha\}$.  Then $S'$ is nonempty because $S$ in nonempty.  Thus, by the
    well-ordering of the ordinals, $S'$ has a least element $\alpha$, so there
    is some $\kappa \in S$ with $\lfloor \kappa \rfloor = \alpha$.  We will
    prove that $\kappa$ is the least element of $S$.  Let $\mu \in S$.  Then
    $\lfloor \mu \rfloor \in S'$, so by $\alpha$ being the least element of
    $S'$, we have $\lfloor \kappa \rfloor \leq \lfloor \mu \rfloor$.  Then by
    Theorem \ref{card_to_initial_ord_le}, we get $\kappa \leq \mu$, proving that
    $\kappa$ is the least element of $S$.
\end{proof}

Recall that antisymmetry and the well ordering property imply connexivity and
transitivity.  Thus, the cardinals are a well-ordered type.

\begin{instance}
    The function $\lfloor \kappa \rfloor$ is orderly.
\end{instance}
\begin{proof}
    Assume that we have cardinals $\kappa$ and $\mu$ such that $\kappa \leq
    \mu$.  For a contradiction, assume that $\lfloor \mu \rfloor < \lfloor
    \kappa \rfloor$.  Then by Theorem \ref{card_to_initial_ord_le}, we have $\mu
    \leq \kappa$, and by antisymmetry we have $\kappa = \mu$.  However, this
    contradicts $\lfloor \mu \rfloor < \lfloor \kappa \rfloor$, so we must have
    $\lfloor \kappa \rfloor \leq \lfloor \mu \rfloor$ as required.
\end{proof}

Because $\lfloor \kappa \rfloor$ is injective, this means that it also is
equivalently orderly and strictly orderly.

\begin{theorem} \label{ord_to_card_lt}
    For all ordinals $\alpha$ and $\beta$, if $|\alpha| < |\beta|$, then $\alpha
    < \beta$.
\end{theorem}
\begin{proof}
    This is just the contrapositive of Theorem \ref{ord_to_card_le}.
\end{proof}

\begin{theorem} \label{card_to_initial_ord_other_eq}
    For all ordinals $\alpha$, if for all ordinals $\beta < \alpha$ we have
    $|\beta| < |\alpha|$, then $\lfloor | \alpha | \rfloor = \alpha$.
\end{theorem}
\begin{proof}
    By antisymmetry and Theorem \ref{ord_to_card_to_initial_ord_le}, we just
    need to prove that $\alpha \leq \lfloor | \alpha | \rfloor$.  To do so,
    assume that $\lfloor | \alpha | \rfloor < \alpha$.  Then by hypothesis, we
    have
    \[
        |\lfloor|\alpha|\rfloor| < |\alpha|,
    \]
    and by Theorem \ref{card_to_initial_ord_to_card_eq} we have
    \[
        |\alpha| < |\alpha|,
    \]
    which is impossible.
\end{proof}

\begin{theorem} \label{card_lt_ex}
    For all types $\U$ and $\V$, if $|\U| < |\V|$, then for all functions $f :
    \U \to \V$, there exists a $y$ such that for all $x$, $f(x) \neq y$.
\end{theorem}
\begin{proof}
    Assume the opposite, which is precisely the statement that an surjective
    function $f : \U \to \V$ exists.  Then by the partition principle (Theorem
    \ref{partition_principle}), we have an injective function from $\V$ to $\U$,
    contradicting $|\U| < |\V|$.
\end{proof}

\begin{theorem}[Cantor's Theorem] \label{power_set_bigger}
    For all types $\A$, we have $|\A| < |\A \to \Prop|$.
\end{theorem}
\begin{proof}
    First, to prove that $|\A| \leq |\A \to \Prop|$, let $f : \A \to (\A \to
    \Prop)$ be defined by $f(a) = \{a\}$.  This function is injective because
    $\{a\} = \{b\}$ implies that $a = b$.  To prove that $|\A| \neq |\A \to
    \Prop|$, assume for a contradiction that some bijective function $f : \A \to
    (\A \to \Prop)$ exists.  Define a set $B : \A \to \Prop$ given by $\{x : \A,
    x \notin f(x)\}$.  Because $f$ is surjective, there exists some $x$ such
    that $f(x) = B$.  This means that $f(x)(x) = B(x)$, or more simply, that $x
    \in f(x)$ if and only if $x \notin f(x)$.  This is a contradiction, showing
    that no such bijection can exist.
\end{proof}

\begin{theorem} \label{card_unbounded}
    The cardinals are unbounded, that is, for any cardinal $\kappa$, there
    exists a cardinal $\mu$ such that $\kappa < \mu$.
\end{theorem}
\begin{proof}
    This follows directly from the previous theorem.
\end{proof}

\section{Small and Large Sets}

As mentioned above, it is impossible to make cardinals out of other cardinals or
ordinals out of other ordinals.  However, there are many situations where doing
this is desirable.  In this section, we will find ways to work around this
issue.  Care needs to be taken when talking about types in this section, since
the problem we are trying to work around involves universes.  Thus, in this
section, universes will be explicitly given for types.

Let $\Type_n$ be the universe that the types and well-ordered types that we use
to build ordinals and cardinals are in.  Then, ordinals and cardinals are in
$\Type_{n + 1}$.

\begin{definition}
    Let $\U$ be any type in $\Type_{n + 1}$ and let $S$ be a set on $\U$.  Then
    we will say that the set $S$ is small if there exists a type $X : \Type_n$
    and a surjective function $f : X \to \T(S)$.
\end{definition}

\begin{theorem}
    For all types $\U : \Type_{n + 1}$, the empty set is small.
\end{theorem}
\begin{proof}
    The empty type can be defined in $\Type_n$, and then the empty function is
    surjective.
\end{proof}

\begin{theorem}
    For all types $\U : \Type_{n + 1}$ and values $x : \U$, the set $\{x\}$ is
    small.
\end{theorem}
\begin{proof}
    The singleton type can be defined in $\Type_n$, at which point the function
    $f : \S \to \{x\}$ defined by $f(I) = x$ is surjective.
\end{proof}

\begin{theorem} \label{small_image_under} \label{small_image}
    Let $\A$ and $\B$ be types in $\Type_{n + 1}$ and let $f$ be a function from
    $\A$ to $\B$.  Then the image of all small sets under $f$ is small.
\end{theorem}
\begin{proof}
    Let $S$ be a small set, so that there exists a type $X : \Type_n$ and a
    surjective function $g : X \to \T(S)$.  Define $h = f \circ g$.  This is a
    function from $X$ to $f(S)$.  Let $y \in f(S)$.  Then there exists a $y' \in
    S$ such that $f(y') = y$.  Then because $g$ is surjective, we have an $x$
    such that $g(x) = y'$.  Then $h(x) = f(g(x)) = f(y') = y$, showing that $h$
    is surjective.
\end{proof}

\begin{theorem} \label{small_bij_ex}
    Let $\U : \Type_{n + 1}$ and let $S$ be a small set.  Then there exists a
    type $X$ and a bijective function $f : X \to \T(S)$.
\end{theorem}
\begin{proof}
    Because $S$ is small, we have a type $Y$ and a surjective function $f : Y
    \to \T(S)$.  We can define a function $f^{-1} : \T(S) \to Y$ where
    $f^{-1}(y)$ is defined such that $f(f^{-1}(y)) = y$.  Then define the type
    $X$ to be the image of $f^{-1}$.  Then we can take $f$ and restrict its
    domain to $X$, producing a function from $X$ to $\T(S)$.  Given any $y :
    \T(S)$, we have $f(f^{-1}(y)) = y$, showing that $f$ is surjective.  To
    prove that $f$ is injective, let $f(a') = f(b')$ with $a'$ and $b'$ in $X$.
    This means that we have an $a$ with $a' = f^{-1}(a)$ and a $b$ with $b' =
    f^{-1}(b)$.  Then $f(f^{-1}(a)) = f(f^{-1}(b))$, so $a = b$.  This means
    that $a' = b'$ as well, showing that $f$ is injective.  Because $f$ is
    injective and surjective, it is bijective.
\end{proof}

\begin{definition}
    Given a type $\U : \Type_{n + 1}$ and a small set $S$, we can define the
    cardinality of $S$ as the cardinality of a type $X$ that is in bijection
    with $S$, as given by the previous theorem.  We will write this as $|S|$.
\end{definition}

We also know that any type that is in bijection with a small set has the same
cardinality.

\begin{theorem} \label{card_small_bounded}
    Let $S$ be a small set of cardinals.  Then there exists a cardinal $\kappa$
    that is greater than all cardinals in $S$.
\end{theorem}
\begin{proof}
    Let $X$ and $f$ be the type and surjective function given by $S$ being
    small.  Define a new dependent type $\C$ that has a single constructor that
    takes in a value in $x : \X$ and a value in some type that has cardinality
    equal to $f(x)$.  $\C : \Type_n$, because $\X : \Type_n$ and the other value
    is in a type that we make cardinals from, which is also $\Type_n$.  Thus,
    $|\C|$ is a well-formed cardinal.  Let $\A$ be a type such that $|\A| \in
    S$.  We will prove that $|\A| \leq |\C|$, which by Theorem
    \ref{power_set_bigger} implies that $|\A| < |\C \to \Prop|$, which will show
    that $|\C \to \Prop|$ is the cardinal we are looking for.

    To prove that $|\A| \leq |\C|$, we must find an injective function from
    $\A$ to $\C$.  Because $f$ is surjective, we have some $x : \X$ such that
    $f(x) = |\A|$.  Let $\B$ be a type that has cardinality $f(x)$.  Then $|\A|
    = |\B|$ so there exists a bijective function $g : \A \to \B$.  Define the
    function $h : \A \to \C$ that takes $a : \A$ to the dependent pair $(x,
    g(a)) : \C$.  We will prove that $h$ is injective.  Let $a$ and $b$ be such
    that $h(a) = h(b)$.  Then $(x, g(a)) = (x, g(b))$, so $g(a) = g(b)$.  By the
    injectivity of $g$, we have $a = b$ as required.
\end{proof}

\begin{theorem} \label{ord_small_bounded}
    Let $S$ be a small set of ordinals.  Then there exists an ordinal $\gamma$
    that is greater than all ordinals in $S$.
\end{theorem}
\begin{proof}
    Consider the image of $S$ under the function that converts ordinals to
    cardinals.  This set of cardinals is small by Theorem
    \ref{small_image_under}.  Thus, by Theorem \ref{card_small_bounded}, it has
    a strict upper bound $\kappa$.  Then for any $\alpha \in S$, we have
    $|\alpha| < \kappa = |\lfloor \kappa \rfloor |$, so by Theorem
    \ref{ord_to_card_lt} we have $\alpha < \lfloor \kappa \rfloor$, showing that
    $\lfloor \kappa \rfloor$ is a strict upper bound of $S$.
\end{proof}

\begin{theorem} \label{ord_initial_small}
    All initial segments in the ordinals is small.
\end{theorem}
\begin{proof}
    Let $[\B]$ be an ordinal.  The function $\mathcal O_\B$ given by Definition
    \ref{ord_type_init_ord} is a surjective function from $\B$, which is in
    $\Type_n$, to the initial segment of $[\B]$.
\end{proof}

The previous theorem is the most useful out of all the theorems proved in this
section.  It theorem means that when doing transfinite recursion on the
ordinals, one can always use suprema.  As such, it is basically required for
doing much more with the ordinals.

As an example of the application of these theorems, we will define something
related to the $\aleph$ function.  Note that this isn't quite $\aleph$ itself,
so we will call it $\aleph'$.  The definition of $\aleph$ itself will be given
later.

\begin{definition}
    Consider a function $a : (\beta : \Ord) \to (g : \Ord_\beta \to \Card) \to
    \Card$ that takes in such a $\beta$ and $g$ and produces the least cardinal
    greater than the image of $g$.  Then define $\aleph'$ to be the function
    given by transfinite recursion using $a$.  Instead of writing the
    application of this function as $\aleph'(\alpha)$, we will write
    $\aleph'_\alpha$.
\end{definition}

\begin{theorem} \label{aleph'_gt}
    For all ordinals $\alpha$ and $\beta$ with $\alpha < \beta$, we have
    $\aleph'_\alpha < \aleph'_\beta$.
\end{theorem}
\begin{proof}
    This follows directly from the definition of $\aleph'$.
\end{proof}

\begin{theorem} \label{aleph'_least}
    For all ordinals $\alpha$ and cardinals $\mu$, if $\aleph'_\beta < \mu$ for
    all $\beta < \alpha$, then $\aleph'_\alpha \leq \mu$.
\end{theorem}
\begin{proof}
    This follows directly from the definition of $\aleph'$.
\end{proof}

\begin{instance}
    $\aleph'$ is injective.
\end{instance}
\begin{proof}
    Let $\aleph'_\alpha = \aleph'_\beta$ and assume for a contradiction that
    $\alpha \neq \beta$.  Without loss of generality, assume that $\alpha <
    \beta$.  Then by Theorem \ref{aleph'_gt}, we have $\aleph'_\alpha <
    \aleph'_\beta$, which is impossible.  Thus, we must have $\alpha = \beta$.
\end{proof}

\begin{instance}
    $\aleph'$ is orderly.
\end{instance}
\begin{proof}
    Let $\alpha \leq \beta$.  If $\alpha = \beta$, we would immediately have
    $\aleph'_\alpha \leq \aleph'_\beta$, and if $\alpha \neq \beta$, we would
    have $\alpha < \beta$, so by Theorem \ref{aleph'_gt} we would have
    $\aleph'_\alpha < \aleph'_\beta$.  Either way, $\aleph'_\alpha \leq
    \aleph'_\beta$.
\end{proof}

\begin{lemma}
    For all cardinals $\mu$, there exists an ordinal $\alpha$ with $\mu \leq
    \aleph'_\alpha$.
\end{lemma}
\begin{proof}
    We will prove that $\mu \leq \aleph'_{\lfloor \mu \rfloor}$ by transfinite
    induction.  The inductive hypothesis states that for all cardinals $\kappa <
    \mu$, we have $\kappa \leq \aleph'_{\lfloor \mu \rfloor}$.  Then if we had
    $\aleph'_{\lfloor \mu \rfloor} < \mu$, by the inductive hypothesis we would
    have $\aleph'_{\lfloor \mu \rfloor} \leq \aleph'_{\lfloor \aleph_{\lfloor
    \mu \rfloor} \rfloor}$.  Because $\aleph'$ and $\lfloor x \rfloor$ are
    equivalently orderly, this reduces to $\mu \leq \aleph'_{\lfloor \mu
    \rfloor}$.  But we assumed that $\aleph'_{\lfloor \mu \rfloor} < \mu$, so
    this is a contradiction.  Thus, we must have $\mu \leq \aleph'_{\lfloor \mu
    \rfloor}$.
\end{proof}

\begin{instance}
    $\aleph'$ is surjective.
\end{instance}
\begin{proof}
    Let $\kappa$ be a cardinal.  Let $\alpha$ be the least ordinal such that
    $\kappa \leq \aleph'_\alpha$.  We will prove that $\kappa =
    \aleph'_\alpha$.  This will be done by antisymmetry.  We already have
    $\kappa \leq \aleph'_\alpha$, so we only need to prove that $\aleph'_\alpha
    \leq \kappa$.  To do so, we will use Theorem \ref{aleph'_least}.  So let
    $\beta < \alpha$.  We must prove that $\aleph'_\beta < \kappa$.  If $\kappa
    \leq \aleph'_\beta$, we would have $\alpha \leq \beta$ by the minimality of
    $\alpha$.  This contradicts $\beta < \alpha$, so we must have $\aleph'_\beta
    < \kappa$, showing that $\aleph'_\alpha \leq \kappa$.
\end{proof}

Thus, $\aleph'$ is a bijective order-preserving function, so the cardinals and
ordinals are order-isomorphic to each other!  However, it is still helpful to
distinguish between the two because they are used in very different ways.

\section{Suprema, Successors, and Limits}

Given that every small set of ordinals is bounded, and since the ordinals are
well ordered, there a helpful definition we can make:

\begin{definition}
    Let $S$ be a small set of ordinals.  Then define $\sup S$ to be the least
    upper bound of $S$.
\end{definition}

Recall that the image of a small set is a small set, and that all initial
segments are small.  Thus, given any ordinal function $f$ and ordinal $\alpha$,
the set $\{f(\delta) \mid \delta < \alpha\}$ is small as well.  We will mostly
be using $\sup$ on sets of that form.  In fact, it's so common that in the Rocq
code there is a special definition just to make suprema of that form more
streamlined!  But we won't really need to do that here.

Supremahave many properties, and for convenience, these will be stated below.
No proofs will be given because they follow directly from the definitions.

\begin{theorem} \label{ord_sup_ge}
    For all $\alpha$, $\beta$, and $f$ with $\alpha < \beta$, we have $f(\alpha)
    \leq \sup\{f(\delta), \delta < \beta\}$.
\end{theorem}

\begin{theorem} \label{ord_sup_least}
    For all $\beta$, $\gamma$, and $f$, if for all $\alpha < \beta$ we have
    $f(\alpha) \leq \gamma$, we have $\sup\{f(\delta), \delta < \beta\} \leq
    \gamma$.
\end{theorem}

\begin{theorem} \label{ord_sup_other_leq}
    For all $\beta$, $\gamma$, and $f$, if $\gamma$ has the same property as
    $\sup\{f(\delta), \delta < \alpha\}$ in the previous theorem, then $\gamma
    \leq \sup\{f(\delta), \delta < \beta\}$.
\end{theorem}

\begin{theorem} \label{ord_sup_eq}
    $\sup\{f(\delta), \delta < \beta\}$ is uniquely described by the previous
    two theorems.
\end{theorem}

There are also a few not quite trivial facts about these functions:

\begin{theorem} \label{ord_sup_in}
    For all small sets $S$ and ordinals $\alpha$, if $\alpha < \sup S$, then
    there exists a $\gamma \in S$ with $\alpha < \gamma$.
\end{theorem}
\begin{proof}
    Assume that no such $\gamma$ exists.  Then we have $\beta \leq \alpha$ for
    all $\beta \in S$, so by Theorem \ref{ord_sup_least} we have $\sup S \leq
    \alpha$, contradicting $\alpha < \sup S$.
\end{proof}

\begin{theorem} \label{ord_sup_leq_sup}
    For all small sets $S$ and $T$, if for all $\alpha \in S$, there exists a
    $\beta \in T$ with $\alpha \leq \beta$, then $\sup S \leq \sup T$.
\end{theorem}
\begin{proof}
    By Theorem \ref{ord_sup_least}, it suffices to prove that $\alpha \leq \sup
    T$ for all $\alpha \in S$.  Then by Theorem \ref{ord_sup_other_leq} it
    suffices to prove that for all upper bounds of $T$ $\varepsilon$, we have
    $\alpha \leq \varepsilon$.  Because $\alpha \in S$, by assumption we have a
    $\beta \in T$ with $\alpha \leq \beta$.  Then $\beta \leq \varepsilon$, so
    $\alpha \leq \varepsilon$ by transitivity.
\end{proof}

\begin{theorem} \label{ord_sup_union}
    For all small sets $S$, let $\mathcal S$ be the collection of initial
    segments of $S$.  Then
    \[
        \Ord_{\sup S} = \bigcup \mathcal S.
    \]
\end{theorem}
\begin{proof}
    The proof will be by antisymmetry.  First, let $\beta \in \Ord_{\sup S}$,
    meaning that $\beta < \sup S$.  Then by Theorem \ref{ord_sup_in}, we have a
    $\gamma \in S$ such that $\beta < \gamma$.  Then $\beta \in \Ord_\gamma$,
    meaning that $\beta \in \bigcup \mathcal S$.  Thus, $\Ord_{\sup S} \subseteq
    \bigcup \mathcal S$.

    Now let $\beta \in \bigcup \mathcal S$.  This means that there exists an
    ordinal $\gamma \in S$ such that $\beta < \gamma$.  By Theorem
    \ref{ord_sup_ge}, we have $\gamma \leq \sup S$, so by transitivity we have
    $\beta < \sup S$.  Thus, $\beta \in \Ord_{\sup S}$, so $\bigcup \mathcal S
    \subseteq \Ord_{\sup S}$.
\end{proof}

Given that every small set of ordinals has a strict upper bound, we know that
there is no greatest ordinal.  Thus, we can find the smallest ordinal greater
than any given ordinal.

\begin{definition}
    Let $\alpha$ be an ordinal.  Then we define $S(\alpha)$ to be the smallest
    ordinal that is greater than $\alpha$.
\end{definition}

\begin{theorem} \label{ord_le_suc_lt}
    For all ordinals $\alpha$ and $\beta$, $S(\alpha) \leq \beta$ if and only if
    $\alpha < \beta$.
\end{theorem}
\begin{proof}
    The reverse direction follows directly from the definition of $S$.  For the
    forward direction, assume that $S(\alpha) \leq \beta$.  Along with $\alpha <
    S(\alpha$, we get $\alpha < \beta$ by transitivity.
\end{proof}

\begin{theorem} \label{ord_sucs_lt}
    For all ordinals $\alpha$ and $\beta$, $S(\alpha) < S(\beta)$ if and only if
    $\alpha < \beta$.
\end{theorem}
\begin{proof}
    First, assume that $S(\alpha) < S(\beta)$.  To prove $\alpha < \beta$, it
    suffices to prove $S(\alpha) \leq \beta$.  For a contradiction, assume that
    $\beta < S(\alpha)$.  That itself is equivalent to $S(\beta) \leq
    S(\alpha)$, which contradicts $S(\alpha) < S(\beta)$.  Thus, we must have
    $\alpha < \beta$.

    Now assume that $\alpha < \beta$.  This is equivalent to $S(\alpha) \leq
    \beta$.  For a contradiction, assume that $S(\beta) \leq S(\alpha)$.  This
    is equivalent to $\beta < S(\alpha)$.  This contradicts $S(\alpha) \leq
    \beta$, so we must have $S(\alpha) < S(\beta)$.
\end{proof}

\begin{theorem} \label{ord_sucs_le}
    For all ordinals $\alpha$ and $\beta$, $S(\alpha) \leq S(\beta)$ if and only
    if $\alpha \leq \beta$.
\end{theorem}
\begin{proof}
    \[
        S(\alpha) \leq S(\beta) \leftrightarrow
        \neg(S(\beta) < S(\alpha)) \leftrightarrow
        \neg(\beta < \alpha) \leftrightarrow
        \alpha \leq \beta.
    \]
\end{proof}

\begin{theorem}
    The ordinal function $S$ is injective.
\end{theorem}
\begin{proof}
    Let $S(\alpha) = S(\beta)$.  For a contradiction, assume that $\alpha \neq
    \beta$, and without loss of generality, assume that $\alpha < \beta$.  Then
    $S(\alpha) < S(\beta)$, contradicting $S(\alpha) = S(\beta)$.
\end{proof}

\begin{theorem} \label{ord_lt_suc_le}
    For all ordinals $\alpha$ and $\beta$, $\alpha < S(\beta)$ if and only if
    $\alpha \leq \beta$.
\end{theorem}
\begin{proof}
    \[
        \alpha < S(\beta) \leftrightarrow
        S(\alpha) \leq S(\beta) \leftrightarrow
        \alpha \leq \beta.
    \]
\end{proof}

Given the $S$ function, we can partition the ordinals into three categories:
The first ordinal, those that are the successor of another ordinal, and those
that are neither the first ordinal nor the successor of another ordinal.  We
will call the first ordinal $0$.

\begin{instance}
    Define $0$ in the ordinals to be the order type of the empty set.
\end{instance}

\begin{definition}
    We will say that an ordinal $\alpha$ is a successor ordinal if there exists
    an ordinal $\beta$ such that $S(\beta) = \alpha$.
\end{definition}

\begin{definition}
    We will say that an ordinal $\gamma$ is a limit ordinal if it is not zero
    and is not a successor ordinal.
\end{definition}

\begin{theorem} \label{ord_false_0}
    For all well-ordered types $\A$, if $\A \to \False$, we have $0 = [\A]$.
\end{theorem}
\begin{proof}
    The empty function is vacuously injective and orderly, and it is vacuously
    surjective because $\A \to \False$.
\end{proof}

\begin{instance}
    All ordinals are positive.
\end{instance}
\begin{proof}
    The empty function is vacuously injective and orderly, so the result follows
    by Theorem \ref{ord_le_simpl}.
\end{proof}

\begin{theorem} \label{ord_zero_suc}
    For all ordinals $\alpha$, we have $0 \neq S(\alpha)$.
\end{theorem}
\begin{proof}
    We have $0 \leq \alpha$ and $\alpha < S(\alpha)$, so $0 < S(\alpha)$, which
    implies $0 \neq S(\alpha)$.
\end{proof}

\begin{theorem} \label{ord_sup_suc}
    For all small sets $T$, if $\sup T$ is a successor ordinal, then $\sup T \in
    T$.
\end{theorem}
\begin{proof}
    Assume that $\sup T = S(\gamma)$ for some ordinal $\gamma$.  Assume for a
    contradiction that $\sup T \notin T$.  We will prove that $\sup T = \gamma$,
    which will contradict $\sup T = S(\gamma)$.  To prove that $\gamma$ is an
    upper bound, let $\delta \in T$.  To prove that $\delta \leq \gamma$, we can
    instead prove $\delta < S(\gamma)$.  We have $\delta \leq S(\gamma) = \sup
    T$ already, and we can't have $\delta = S(\gamma)$ because $\delta \in T$
    but $S(\gamma) \notin T$.  Thus, $\delta < S(\gamma)$, so $\gamma$ is an
    upper bound of $T$.  To prove that $\gamma$ is the least upper bound, let
    $\varepsilon$ be another upper bound of $T$.  Then we have $S(\gamma) \leq
    \varepsilon$ because $S(\gamma)$ is the least upper bound of $T$, so we have
    $\gamma \leq S(\gamma) \leq \varepsilon$ as required.  Thus, $\gamma$ is the
    least upper bound of $T$, contradicting $S(\gamma)$ being the least upper
    bound of $T$.
\end{proof}

\begin{theorem} \label{ord_lim_suc}
    For all ordinals $\alpha$ and $\gamma$, if $\gamma$ is a limit ordinal and
    $\alpha < \gamma$, then $S(\alpha) < \gamma$ as well.
\end{theorem}
\begin{proof}
    We already have $S(\alpha) \leq \gamma$ by Theorem \ref{ord_le_suc_lt}.  We
    can't have $S(\alpha) = \gamma$ because $\gamma$ isn't a successor ordinal,
    so we have $S(\alpha) < \gamma$.
\end{proof}

\begin{theorem} \label{ord_sup_zero}
    For all small sets $S$, if $\sup S = 0$, then $\alpha = 0$ for all $\alpha
    \in S$.
\end{theorem}
\begin{proof}
    Because $\sup S = 0$, we have $\alpha \leq 0$ for all $\alpha \in S$.
    Because all ordinals are positive, this means that $\alpha = 0$.
\end{proof}

\begin{theorem} \label{ord_sup_singleton}
    For all ordinals $\alpha$, $\sup \{\alpha\} = \alpha$.
\end{theorem}
\begin{proof}
    $\alpha$ is an upper bound because $\alpha \leq \alpha$, and for any other
    upper bound $\varepsilon$ we have $\alpha \leq \varepsilon$, showing that
    $\alpha$ is the least upper bound.
\end{proof}

\begin{theorem} \label{ord_lim_gt}
    For all limit ordinals $\alpha$, we have $S(0) < \alpha$.
\end{theorem}
\begin{proof}
    By Theorem \ref{ord_lim_suc}, it suffices to prove $0 < \alpha$.  This
    follows from $\alpha$ being a limit ordinal, since a part of the definition
    is that $0 \neq \alpha$.
\end{proof}

In the ordinals, there is a common variant of transfinite induction/recursion
that is often used:

\begin{theorem}[Ordinal Induction]
    For all sets of ordinals $S$, if $0 \in S$, if $\alpha \in S$ implies
    $S(\alpha) \in S$, and if for all limit ordinals $\gamma$, $\beta \in S$ for
    all $\beta < \gamma$ implies $\gamma \in S$, then all ordinals are in $S$.
\end{theorem}
\begin{proof}
    Let $\alpha$ be an arbitrary ordinal.  The proof will be by transfinite
    induction on $\alpha$.  The inductive hypothesis states that for all $\beta
    < \alpha$, we have $\beta \in S$.  There will now be three cases for when
    $\alpha = 0$, when $\alpha$ is a successor ordinal, and when $\alpha$ is a
    limit ordinal.  When $\alpha = 0$, we have $0 \in S$ by assumption.  When
    $\alpha = S(\beta)$ for some ordinal $\beta$, because $\beta < \alpha$, we
    have $\beta \in S$ by the inductive hypothesis, so by assumption $S(\beta) =
    \alpha \in S$.  When $\alpha$ is a limit ordinal, the inductive hypothesis
    matches the third assumption to show that $\alpha \in S$.
\end{proof}

\begin{theorem}[Ordinal Recursion]
    For all types $\X$, values $f_0 : \X$, functions $f_S : \Ord \to \X \to \X$,
    and functions $f_\infty : \forall \alpha, (\T(\Ord_\alpha) \to \X) \to \X$,
    there exists a function $g : \Ord \to \X$ such that
    \begin{align*}
        g(0) &= f_0 \\
        g(S(\alpha)) &= f_S(\alpha, g(\alpha)) \\
        g(\gamma) &= f_\infty(\alpha, g \uparrow \alpha)
            \text{\quad when $\gamma$ is a limit ordinal.}
    \end{align*}
\end{theorem}
\begin{proof}
    Define a function $f : \forall \alpha, (\T(\Ord_\alpha) \to \X) \to \X$
    given by
    \begin{align*}
        f(0, h) &= f_0 \\
        f(S(\alpha), h) &= f_S(\alpha, h(\alpha)) \\
        f(\gamma, h) &= f_\infty(\gamma, h)
            \text{\quad when $\gamma$ is a limit ordinal.}
    \end{align*}
    Then by transfinite recursion, there exists a $g : \Ord \to \X$ such that
    $g(\alpha) = f(\alpha, g \uparrow \alpha)$ for all ordinals $\alpha$.  This
    function $g$ satisfies all of the conditions necessary.
\end{proof}

To distinguish between the different forms of transfinite induction and
recursion, these versions will be called ordinal induction and recursion
instead.

\begin{theorem} \label{ord_near_lim}
    For all ordinals $\alpha$, there exists a non-successor ordinal $\beta$ and
    natural number $n$ such that $S^n(\beta) = \alpha$.
\end{theorem}
\begin{proof}
    The proof will be by ordinal induction on $\alpha$.  The non-successor cases
    follow directly with $\beta = \alpha$ and $n = 0$, so the only interesting
    case is the successor case.  So assume that there exists some $\beta$ and
    $n$ such that $\beta$ isn't a successor ordinal and that $S^n(\beta) =
    \alpha$.  Then $S^{S(n)}(\beta) = S(\alpha)$.
\end{proof}

\newcommand{\prelim}[1]{{{}_\infty {#1}}}
\begin{definition}
    Given an ordinal $\alpha$, call its preceding non-successor ordinal as
    given by the previous theorem $\prelim{\alpha}$.
\end{definition}

\begin{theorem} \label{ord_near_lim_other}
    Let $\alpha$ and $\beta$ be ordinals such that $\prelim \beta \leq \alpha$
    and $\alpha < \beta$.  Then there exists a natural number $n$ such that
    $S^n(\alpha) = \beta$.
\end{theorem}
\begin{proof}
    Let $m$ be such that $S^m(\prelim \beta) = \beta$.  Define the set $T =
    \{n : \N \mid \beta \leq S^n(\alpha)\}$.  This set is nonempty because
    $\beta = S^m(\prelim \beta) \leq S^m(\alpha)$.  Thus, $T$ has a least
    element we'll call $n$.  We will prove that $S^n(\alpha) = \beta$.  We
    already have $S^n(\alpha) \geq \beta$, so by antisymmetry it suffices to
    prove $S^n(\alpha) \leq \beta$.  For a contradiction, assume that $\beta <
    S^n(\alpha)$.  When $n = 0$, we would have $\beta < \alpha$, contradicting
    $\alpha < \beta$.  When $n = S(n')$, we have $\beta < S(S^{n'}(\alpha))$, so
    $\beta \leq S^{n'}(\alpha)$, contradicting the minimality of $n$.  Either
    way, we got a contradiction, so we must have $S^n(\alpha) \leq \beta$.
\end{proof}

\begin{theorem} \label{ord_near_lim_lt}
    For all ordinals $\alpha$ and $\beta$ such that $\alpha < \beta$ and
    $S^n(\alpha) \neq \beta$ for all natural numbers $n$, we have $\alpha <
    \prelim \beta$.
\end{theorem}
\begin{proof}
    Assume otherwise.  Then by Theorem \ref{ord_near_lim_other}, we have an $n$
    such that $S^n(\alpha) = \beta$, contradicting one of this theorem's
    assumptions.
\end{proof}

\begin{theorem} \label{ord_near_lim_lim}
    For all ordinals $\alpha$ and $\beta$ such that $\alpha < \beta$ and
    $S^n(\alpha) \neq \beta$ for all natural numbers $n$, $\prelim \beta$ is a
    limit ordinal.
\end{theorem}
\begin{proof}
    Because $\prelim \beta$ is a non-successor ordinal, all we need to do is
    show that it can't be zero.  So assume that $\prelim \beta = 0$.  Then we
    have $\prelim \beta \leq \alpha$ and $\alpha < \beta$, so by Theorem
    \ref{ord_near_lim_other}, we have an $n$ such that $S^n(\alpha) = \beta$,
    contradicting one of this theorem's assumptions.
\end{proof}

\begin{theorem} \label{ord_sup_lim_eq}
    For all limit ordinals $\gamma$, $\sup\{\delta \mid \delta < \gamma\} =
    \gamma$.
\end{theorem}
\begin{proof}
    $\gamma$ is directly an upper bound, so we only need to prove that it's the
    least upper bound.  So let $\varepsilon$ be another upper bound.  For a
    contradiction, assume that $\varepsilon < \gamma$.  Then by Theorem
    \ref{ord_lim_suc}, we have $S(\varepsilon) < \gamma$.  Because $\varepsilon$
    is an upper bound, we have $S(\varepsilon) \leq \varepsilon$, which is
    impossible.  Thus, $\gamma$ is the least upper bound.
\end{proof}

Up to this point, we have mostly been working with successors in an abstract
way.  But ordinals are equivalence classes of well-ordered types.  What is a
well-ordered type that corresponds to a successor?

\begin{definition}
    Let $\A$ be a well-ordered type.  Then we will order the type $\A + \S$ as
    follows:
    \begin{align*}
        (\iota_1(a) \leq \iota_1(b)) &= (a \leq b) \\
        (\iota_2(a) \leq \iota_1(b)) &= \False \\
        (a \leq \iota_2(b)) &= \True.
    \end{align*}
\end{definition}

\begin{theorem}
    When $\A$ is well ordered, the order on $\A + \S$ is a well order.
\end{theorem}
\begin{proof}
    For antisymmetry, let $a$ and $b$ be values in $\A + \S$ such that $a \leq
    b$ and $b \leq a$.  $a$ and $b$ have to either both come from $\A$ or both
    come from $\S$ because it's impossible for something in $\A$ to be greater
    than or equal to something in $\S$.  When they both come from $\A$, the
    result follows from the antisymmetry of the order in $\A$.  When they both
    come from $\S$, they are equal because $\S$ only has one value.

    For being well-ordered, let $S$ be a nonempty set in $\A + \S$.  There will
    be two cases: for when there exists an $a : \A$ such that $\iota_1(a) \in
    S$, or when no such $a$ exists.  When such an $a$ exists, define the set $S'
    = \{a : \A \mid \iota_1(a) \in S\}$.  This is a nonempty set in $\A$, and
    because $\A$ is well ordered, it has a least element $m$.  $m$ is the least
    element of $S$ as well.  When no $\iota_1(a) \in S$ exists, then $I$, the
    value in the singleton type, must be the least element of $S$ instead.
\end{proof}

\begin{theorem} \label{ord_suc_type}
    For all well-ordered types $\A$, we have $S([\A]) = [\A + \S]$.
\end{theorem}
\begin{proof}
    The proof will be by antisymmetry.  To prove that $S([\A]) \leq [\A + \S]$,
    it suffices to prove that $[\A] < [\A + \S]$.  $\A$ is order isomorphic to
    the initial segment of $\iota_2(I)$ in $\A + \S$, so the result follows by
    Theorem \ref{ord_lt_simpl}.

    Proving that $[\A + \S] \leq S([\A])$ is much more difficult.  We know that
    $[\A] < S([\A])$.  Let $\B$ be a well-ordered type with order type
    $S([\A])$.  So we have $[\A] < [\B]$ and must prove that $[\A + \S] \leq
    [\B]$.  By Theorem \ref{ord_lt_simpl} we have some $b : \B$ such that $\A$
    is order isomorphic to $\B_b$.  Let $f$ be that order isomorphism.  Then
    define a function $g : \A + \S \to \B$ given by
    \begin{align*}
        g(\iota_1(a)) &= f(a) \\
        g(\iota_2(I)) &= b.
    \end{align*}
    This function is injective and orderly, so by Theorem \ref{ord_le_simpl} we
    have $[\A + \S] \leq [\B]$.
\end{proof}

\section{Normal Functions}

\begin{class}
    Let $f$ be an ordinal function.  Then we say that $f$ satisfies the normal
    limit condition if for all limit ordinals $\gamma$, we have
    \[
        f(\gamma) = \sup\{f(\delta) \mid \delta < \gamma\}.
    \]
\end{class}

\begin{definition}
    We call an ordinal function $f$ a normal function if it is orderly,
    injective, and satisfies the normal limit condition.  Equivalently, $f$ is
    normal if it is strictly orderly and satisfies the normal limit condition.
\end{definition}

We commonly make normal functions through ordinal recursion, so here is a
specialized recursion scheme for making normal functions:

\begin{theorem}[Normal Recursion]
    Let $f_0$ be an ordinal and let $f_S : \Ord \to \Ord \to \Ord$.  Then there
    exists a function $g : \Ord \to \Ord$ such that
    \begin{align*}
        g(0) &= f_0 \\
        g(S(\alpha)) &= f_S(\alpha, g(\alpha)) \\
        g(\gamma) &= \sup\{g(\delta) \mid \delta < \gamma\}
            \text{\quad when $\gamma$ is a limit ordinal.}
    \end{align*}
\end{theorem}
\begin{proof}
    The result follows directly from ordinal recursion with $f_0$, $f_S$, and
    the function $f_\infty(\gamma, h) = \sup\{h(\delta) \mid \delta < \gamma\}$.
\end{proof}

To distinguish this type of recursion from transfinite recursion and ordinal
recursion, we will call this normal recursion.  Note that by definition, all
functions made with normal recursion satisfy the normal limit condition.

\begin{theorem} \label{make_ord_normal_le}
    If an ordinal function $f$ was defined with normal recursion and if
    $f(\alpha) \leq f(S(\alpha))$ for all $\alpha$, then $f$ is orderly.
\end{theorem}
\begin{proof}
    Let $\alpha \leq \beta$.  if $\alpha = \beta$, we would be done, so assume
    that $\alpha < \beta$.  The proof will be by ordinal induction on $\beta$.

    We can't have $\beta = 0$ because $\alpha < \beta$.

    Assume that if $\alpha < \beta$, then $f(\alpha) \leq f(\beta)$, and
    assume $\alpha < S(\beta)$.  This means that $\alpha \leq \beta$.  If
    $\alpha = \beta$, we would have $f(\alpha) \leq f(S(\alpha))$ by assumption.
    If $\alpha \neq \beta$, we would have $\alpha < \beta$, so by the inductive
    hypothesis we have $f(\alpha) \leq f(\beta)$, and $f(\beta) \leq
    f(S(\beta))$, so $f(\alpha) \leq f(S(\beta))$.  Either way, $f(\alpha) \leq
    f(S(\beta))$.

    When $\beta$ is a limit ordinal, we won't need the inductive hypothesis.  By
    the normal limit condition, we have $f(\beta) = \sup\{f(\delta) \mid \delta
    < \beta\}$, and since $\alpha < \beta$, we have $f(\alpha) \leq f(\beta)$.
\end{proof}

\begin{theorem} \label{make_ord_normal_inj}
    If an ordinal function $f$ was defined with normal recursion and if
    $f(\alpha) < f(S(\alpha))$ for all $\alpha$, then $f$ is injective.
\end{theorem}
\begin{proof}
    Let $f(\alpha) = f(\beta)$, and assume for a contradiction without loss of
    generality that $\alpha < \beta$.  We know that $f(\beta) = f(\alpha) <
    f(S(\alpha))$.  From $\alpha < \beta$, we have $S(\alpha) \leq \beta$, and
    because $f$ is orderly, we have $f(S(\alpha)) \leq f(\beta)$, contradicting
    $f(\beta) < f(S(\alpha))$.  Thus, we must have $\alpha = \beta$.
\end{proof}

\begin{theorem} \label{ord_normal_le}
    For all normal functions $f$, for all ordinals $\alpha$ we have $\alpha \leq
    f(\alpha)$.
\end{theorem}
\begin{proof}
    The proof will be by transfinite induction.  The inductive hypothesis states
    that for all $\beta < \alpha$, we have $\beta \leq f(\beta)$.  For a
    contradiction, assume that $f(\alpha) < \alpha$.  Then by the inductive
    hypothesis we have $f(\alpha) \leq f(f(\alpha))$, and because $f$ is
    equivalently orderly we have $\alpha \leq f(\alpha)$.
\end{proof}

\begin{theorem} \label{ord_normal_sup}
    For all nonempty small sets $T$ and functions $f$ that are orderly and
    satisfy the normal limit condition, we have
    \[
        f(\sup T) = \sup f(T).
    \]
\end{theorem}
\begin{proof}
    The proof will be by antisymmetry.  For $\sup f(T) \leq f(\sup T)$, we need
    to prove that for any $\alpha \in T$, we have $f(\alpha) \leq f(\sup T)$.
    By $f$ being orderly, it suffices to prove that $\alpha \leq \sup T$, which
    follows directly from $\alpha \in T$.  $f(\sup T) \leq \sup f(T)$ is a
    little more difficult.  We will split it into the three cases for when $\sup
    T = 0$, when it is a successor ordinal, or when it is a limit ordinal.

    When $\sup T = 0$, we must prove that $f(0) \leq \sup f(T)$.  Because $T$ is
    nonempty, it must have a value in it, and because $\sup T = 0$, by Theorem
    \ref{ord_sup_zero} we must have $0 \in T$.  Thus, $f(0) \in f(T)$, so $f(0)
    \leq \sup f(T)$.

    For the case when $\sup T$ is a successor ordinal, let $\sup T = S(\gamma)$.
    By Theorem \ref{ord_sup_suc}, $S(\gamma) \in T$.  Thus, $f(S(\gamma)) \in
    f(T)$, so $f(S(\gamma)) \leq \sup f(T)$.

    For the case when $\sup T$ is a limit ordinal, we have $f(\sup T) =
    \sup\{f(\delta) \mid \delta\}$.  We must prove that
    \[
        \sup\{f(\delta) \mid \delta\} \leq \sup f(T).
    \]
    By Theorem \ref{ord_sup_leq_sup}, it suffices to prove that for all $\beta <
    \sup T$, we have some $\zeta \in T$ such that $f(\beta) \leq f(\zeta)$.
    Because $\sup T$ is a limit ordinal, by Theorem \ref{ord_sup_in} we have a
    $\zeta \in T$ with $\beta \leq \zeta$.  Then because $f$ is orderly, we have
    $f(\beta) \leq f(\zeta)$ as required.
\end{proof}

Note that we don't need injectivity in the previous theorem.

\begin{theorem} \label{ord_normal_lim_ord}
    For all normal functions $f$, if $\gamma$ is a limit ordinal, then
    $f(\gamma)$ is a limit ordinal.
\end{theorem}
\begin{proof}
    We know that $f(\gamma) = \sup\{f(\delta) \mid \delta < \gamma\}$.  We must
    prove that this is not equal to zero and not a successor ordinal.

    Assume that $\sup\{f(\delta) \mid \delta < \gamma\} = 0$.  Because $\gamma$
    is a limit ordinal, we know that $S(0) < \gamma$, so $f(S(0)) \leq 0$.  But
    $f(S(0)) > f(0) \geq 0$, which is impossible.  Thus, $\sup\{f(\delta) \mid
    \delta < \gamma\} \neq 0$.

    Assume that $\sup\{f(\delta) \mid \delta < \gamma\} = S(\beta)$ for some
    ordinal $\beta$.  Then by Theorem \ref{ord_sup_suc}, there exists an ordinal
    $\delta < \gamma$ such that $f(\delta) = S(\beta)$.  Because $\gamma$ is a
    limit ordinal, $S(\delta) < \gamma$ as well, so
    \[
        f(S(\delta)) \leq \sup\{f(\delta) \mid \delta < \gamma\} = S(\beta)
        = f(\delta).
    \]
    This contradicts $f(\delta) < f(S(\delta))$, so $\sup\{f(\delta) \mid
    \delta < \gamma\}$ can't be a successor ordinal.
\end{proof}

\begin{definition}
    Let $S$ be a small set of any type, and let $f_i$ be a normal function for
    all $i \in S$.  We will call $f_i$ a small family of normal functions.
\end{definition}

\begin{definition} \label{ord_normal_family_fixed_set}
    Let $f_i$ be a small family of normal functions, and let $\alpha$ be an
    ordinal.  Define the set $F(\alpha) : \Ord \to \Prop$ defined by saying that
    $\beta \in F(\alpha)$ if there exists a list $[f_0, f_1, \ldots, f_n]$ of
    functions in $f_i$ such that $\beta = f_0(f_1(\cdots(f_n(\alpha))))$.
\end{definition}

\begin{theorem}
    For all small families of normal functions $f_i$ and ordinals $\alpha$, the
    set $F(\alpha)$ is small.
\end{theorem}
\begin{proof}
    Because $f_i$ is small, we have a type $\X : \Type_n$ and a surjective
    function $f : \X \to f_i$.  We can defined a surjection from $\L(\X)$ to
    $F(\alpha)$ by taking a list of values in $\X$, applying $f$ to them, and
    then applying the functions in sequence to $\alpha$.  We have $\L(\X) :
    \Type_n$, so $F(\alpha)$ is small.
\end{proof}

\begin{theorem} \label{ord_normal_family_fixed_eq}
    For all small families of normal functions $f_i$ and ordinals $\alpha$, we
    have $f_x(\sup F(\alpha)) = \sup F(\alpha)$ for all $x \in S$.
\end{theorem}
\begin{proof}
    We know that $F(\alpha)$ is nonempty because $\alpha \in F(\alpha)$, so we
    have
    \[
        f_x(\sup F(\alpha)) = \sup f_x(F(\alpha)).
    \]
    We must prove
    \[
        \sup f_x(F(\alpha)) = \sup F(\alpha),
    \]
    which we will do by antisymmetry and Theorem \ref{ord_sup_leq_sup}.

    Let $\beta \in F(\alpha)$.  We must prove that there exists an ordinal
    $\gamma \in F(\alpha)$ such that $f_x(\beta) \leq \gamma$.  $f_x(\beta)$ is
    precisely such a $\gamma$.

    Let $\beta \in F(\alpha)$.  We must prove that there exists an ordinal
    $\gamma \in f_x(F(\alpha))$ such that $\beta \leq \gamma$.  Because $\beta
    \in F(\alpha)$, $f_x(\beta) \in f_x(F(\alpha))$.  Because $f_x$ is normal,
    we have $\beta \leq f_x(\beta)$, showing that $\gamma = f_x(\beta)$
    suffices.
\end{proof}

\begin{theorem} \label{ord_normal_family_fixed_leq}
    For all small families of normal functions $f_i$ and ordinals $\alpha$, we
    have $\alpha \leq \sup F(\alpha)$.
\end{theorem}
\begin{proof}
    We have $\alpha \in F(\alpha)$ because applying the empty list of functions
    to $\alpha$ produces $\alpha$.
\end{proof}

\begin{theorem} \label{ord_normal_family_fixed_least}
    For all small families of normal functions $f_i$ and ordinals $\alpha$,
    $\sup F(\alpha)$ is the smallest ordinal greater than or equal to $\alpha$
    that is a fixed point of all $f_i$.
\end{theorem}
\begin{proof}
    Let $\beta$ be a fixed point of all $f_i$ with $\alpha \leq \beta$.  We will
    prove that $\beta$ is an upper bound of $F(\alpha)$, showing that $\sup
    F(\alpha) \leq \beta$.  Let $\gamma \in F(\alpha)$.  Then there exists a
    list $a$ of functions $f_i$ such that $\gamma =
    f_0(f_1(\cdots(f_n(\alpha))))$.  We will prove that $\gamma \leq \beta$ by
    induction on $a$.  When $a = []$, we have $\gamma = \alpha$, and we have
    $\alpha \leq \beta$ by assumption.  Now assume that
    $f_1(f_2(\cdots(f_n(\alpha)))) \leq \beta$.  Then
    \[
        f_0(f_1(f_2(\cdots(f_n(\alpha))))) \leq f_0(\beta) = \beta.
    \]
\end{proof}

Given that singleton sets are small, we can also find the fixed points of a
single normal function.

\section{Addition of Ordinals}

\begin{instance}
    Let $\alpha$ and $\beta$ be ordinals.  Define the sum $\alpha + \beta$
    recursively on $\beta$ using normal recursion such that
    \begin{align*}
        \alpha + 0 &= \alpha \\
        \alpha + S(\beta) &= S(\alpha + \beta).
    \end{align*}
\end{instance}

Addition of ordinals is quite strange in that it is noncommutative, and some
standard identities only hold on one side.  Care must be taken to know exactly
which identities hold and which do not.

\begin{instance}
    $0$ is a right additive identity in $\Ord$.
\end{instance}
\begin{proof}
    This follows directly from the definition.
\end{proof}

\begin{lemma} \label{ord_plus_lt_suc}
    For all ordinals $\alpha$ and $\beta$, we have $\alpha + \beta < \alpha +
    S(\beta)$.
\end{lemma}
\begin{proof}
    \[
        \alpha + \beta < S(\alpha + \beta) = \alpha + \S(\beta).
    \]
\end{proof}

\begin{instance}
    For all ordinals $\alpha$, the function $f_\alpha(\beta) = \alpha + \beta$
    satisfies the normal limit condition.
\end{instance}
\begin{proof}
    This comes directly from the definition using normal recursion.
\end{proof}

\begin{instance}
    For all ordinals $\alpha$, the function $f_\alpha(\beta) = \alpha + \beta$
    is orderly.
\end{instance}
\begin{proof}
    This is a direct application of Theorem \ref{make_ord_normal_le} and Lemma
    \ref{ord_plus_lt_suc}.
\end{proof}

\begin{instance}
    For all ordinals $\alpha$, the function $f_\alpha(\beta) = \alpha + \beta$
    is injective.
\end{instance}
\begin{proof}
    This is a direct application of Theorem \ref{make_ord_normal_inj} and Lemma
    \ref{ord_plus_lt_suc}.
\end{proof}

Thus, ordinal addition is a normal function.

\begin{theorem} \label{ord_plus_sup}
    For all ordinals $\alpha$ and small sets $S$, we have
    \[
        \alpha + \sup S = \sup\{\alpha + \delta \mid \delta \in S\}.
    \]
\end{theorem}
\begin{proof}
    This follows directly from Theorem \ref{ord_normal_sup}.
\end{proof}

\begin{instance}
    $0$ is a left additive identity in $\Ord$.
\end{instance}
\begin{proof}
    Let $\alpha$ be an ordinal.  We will prove that $0 + \alpha = \alpha$ by
    ordinal induction.  When $\alpha = 0$, we have $0 + 0 = 0$.  For the
    successor case, assume that $0 + \alpha = \alpha$.  Then
    \[
        0 + S(\alpha) = S(0 + \alpha) = S(\alpha).
    \]
    For the limit case, assume that $\alpha$ is a limit ordinal with $0 + \delta
    = \delta$ for all $\delta < \alpha$.  Then
    \[
        0 + \alpha
        = \sup\{0 + \delta \mid \delta < \alpha\}
        = \sup\{\delta \mid \delta < \alpha\}
        = \alpha.
    \]
\end{proof}

Because normal functions are equivalently strictly orderly and injective, we can
add and cancel addition on the left in equations, inequalities, and strict
inequalities.

\begin{theorem} \label{ord_le_self_rplus}
    For all ordinals $\alpha$ and $\beta$, we have $\alpha \leq \alpha + \beta$.
\end{theorem}
\begin{proof}
    \begin{align*}
        0 & \leq \beta \\
        \alpha + 0 &\leq \alpha + \beta \\
        \alpha &\leq \alpha + \beta.
    \end{align*}
\end{proof}

\begin{theorem} \label{ord_le_self_lplus}
    For all ordinals $\alpha$ and $\beta$, we have $\alpha \leq \beta + \alpha$.
\end{theorem}
\begin{proof}
    This follows from Theorem \ref{ord_normal_le}.
\end{proof}

\begin{theorem} \label{ord_le_ex}
    For all ordinals $\alpha$ and $\beta$, if $\alpha \leq \beta$, then there
    exists an ordinal $\gamma$ such that $\alpha + \gamma = \beta$.
\end{theorem}
\begin{proof}
    Let $C$ be the set $\{\varepsilon : \Ord \mid \beta
    \leq \alpha + \varepsilon\}$.  By Theorem \ref{ord_le_self_lplus}, this set
    is nonempty, so it has a least element $\gamma$.  Because $\beta \leq \alpha
    + \gamma$, all we need to prove is $\alpha + \gamma \leq \beta$.  We will
    look at the three cases for when $\gamma$ is zero, a successor ordinal, or a
    limit ordinal.

    If $\gamma = 0$, we have $\alpha + 0 = \alpha \leq \beta$ by assumption.

    If $\gamma = S(\gamma')$, assume for a contradiction that $\beta < \alpha +
    \gamma$.  Then $\beta < \alpha + S(\gamma') = S(\alpha + \gamma')$, so
    $\beta \leq \alpha + \gamma'$.  This means that $\gamma' \in T$.  But then
    $\gamma \leq \gamma' = S(\gamma)$, which is impossible.  Thus, $\alpha +
    \gamma \leq \beta$.

    If $\gamma$ is a limit ordinal, we have $\alpha + \gamma = \sup\{\alpha +
    \delta \mid \delta < \gamma\}$.  To prove that $\alpha + \gamma \leq \beta$,
    it suffices to show that $\beta$ is an upper bound of this set.  So let
    $\delta < \gamma$.  We must show that $\alpha + \delta \leq \beta$.  If we
    had $\beta < \alpha + \delta$, we would have $\delta \in T$, meaning that
    $\gamma \leq \delta$, contradicting $\delta < \gamma$.  Thus, $\alpha +
    \gamma \leq \beta$.
\end{proof}

\begin{theorem} \label{ord_lt_ex}
    For all ordinals $\alpha$ and $\beta$ with $\alpha < \beta$, there exists a
    $\gamma \neq 0$ such that $\alpha + \gamma = \beta$.
\end{theorem}
\begin{proof}
    By Theorem \ref{ord_le_ex}, we already have a $\gamma$ with $\alpha + \gamma
    = \beta$, so we just need to prove that $\gamma \neq 0$.  If $\gamma = 0$,
    we would gave $\alpha = \beta$, contradicting $\alpha < \beta$, so we must
    have $\gamma \neq 0$.
\end{proof}

\begin{theorem} \label{ord_nz_rplus}
    For all ordinals $\alpha$ and $\beta$, if $0 \neq \beta$, then $0 \neq
    \alpha + \beta$.
\end{theorem}
\begin{proof}
    By Theorem \ref{ord_le_self_lplus}, we have $\beta \leq \alpha + \beta$.
    If $0 = \alpha + \beta$, we would have $\beta \leq 0$, so $\beta = 0$,
    contradictiong $\beta \neq 0$.  Thus, $0 \neq \alpha + \beta$.
\end{proof}

\begin{theorem} \label{ord_nz_lplus}
    For all ordinals $\alpha$ and $\beta$, if $0 \neq \alpha$, then $0 \neq
    \alpha + \beta$.
\end{theorem}
\begin{proof}
    If $\beta = 0$, the result follows directly, and if $\beta \neq 0$, the
    result follows from the previous theorem.
\end{proof}

\begin{instance}
    Addition of ordinals is associative.
\end{instance}
\begin{proof}
    Let $\alpha$, $\beta$, and $\gamma$ be ordinals.  We will prove that $\alpha
    + (\beta + \gamma) = (\alpha + \beta) + \gamma$ by ordinal induction on
    $\gamma$.

    When $\gamma = 0$, we have
    \[
        \alpha + (\beta + 0) = \alpha + \beta = (\alpha + \beta) + 0.
    \]

    For the successor case, assume that $\alpha + (\beta + \gamma) = (\alpha +
    \beta) + \gamma$.  Then
    \begin{align*}
        \alpha + (\beta + S(\gamma))
        &= \alpha + S(\beta + \gamma) \\
        &= S(\alpha + (\beta + \gamma)) \\
        &= S((\alpha + \beta) + \gamma) \\
        &= (\alpha + \beta) + S(\gamma).
    \end{align*}

    For the limit case, assume that $\gamma$ is a limit ordinal and that for all
    $\delta < \gamma$, we have $\alpha + (\beta + \delta) = (\alpha + \beta) +
    \delta$.  Then
    \begin{align*}
        \alpha + (\beta + \gamma)
        &= \alpha + \sup\{\beta + \delta \mid \delta < \gamma\} \\
        &= \sup\{\alpha + (\beta + \delta)\} \\
        &= \sup\{(\alpha + \beta) + \delta\} \\
        &= (\alpha + \beta) + \gamma.
    \end{align*}
\end{proof}

\begin{instance}
    The order on ordinals is right additive.
\end{instance}
\begin{proof}
    Let $\alpha \leq \beta$.  We must prove that for all ordinals $\gamma$, we
    have $\alpha + \gamma \leq \beta + \gamma$.  By Theorem \ref{ord_le_ex}, we
    have a $\delta$ such that $\alpha + \delta = \beta$.  Then by Theorem
    \ref{ord_le_self_lplus},
    \begin{align*}
        \gamma &\leq \delta + \gamma \\
        \alpha + \gamma &\leq \alpha + \delta + \gamma \\
        \alpha + \gamma &\leq \beta + \gamma
    \end{align*}
\end{proof}

By Instance \ref{lt_plus_lcancel2}, the order is also strictly right addition
cancellative.

\begin{theorem} \label{ord_lt_self_rplus}
    For all ordinals $\alpha$ and $\beta$, if $0 \neq \beta$, we have $\alpha <
    \alpha + \beta$.
\end{theorem}
\begin{proof}
    We already know that $\alpha \leq \alpha + \beta$ by Theorem
    \ref{ord_le_self_rplus}, so it remains to prove that $\alpha \neq \alpha +
    \beta$.  If $\alpha = \alpha + \beta$, by cancelling $\alpha$ we would get
    $0 = \beta$, contradicting $0 \neq \beta$.
\end{proof}

\begin{theorem} \label{ord_plus_zero}
    For all ordinals $\alpha$ and $\beta$, if $0 = \alpha + \beta$, we have
    $\alpha = 0$ and $\beta = 0$.
\end{theorem}
\begin{proof}
    Each thing to prove can be seen as the contrapositive of Theorems
    \ref{ord_nz_lplus} and \ref{ord_nz_rplus}.
\end{proof}

\begin{theorem} \label{ord_plus_lim_lim}
    For all ordinals $\alpha$ and $\beta$, if $\beta$ is a limit ordinal, then
    $\alpha + \beta$ is a limit ordinal.
\end{theorem}
\begin{proof}
    This follows from addition being a normal function and Theorem
    \ref{ord_normal_lim_ord}.
\end{proof}

\section{Multiplication of Ordinals}

\begin{instance}
    Let $\alpha$ and $\beta$ be ordinals.  Define the product $\alpha \beta$
    recursively on $\beta$ using normal recursion such that
    \begin{align*}
        \alpha 0 &= 0 \\
        \alpha S(\beta) &= \alpha \beta + \alpha.
    \end{align*}
\end{instance}

\begin{instance}
    Zero is a right annihilator in the ordinals.
\end{instance}
\begin{proof}
    This follows directly from the definition.
\end{proof}

\begin{lemma} \label{ord_mult_le_suc}
    For all ordinals $\alpha$ and $\beta$, we have $\alpha \beta \leq \alpha
    S(\beta)$.
\end{lemma}
\begin{proof}
    \[
        \alpha \beta \leq \alpha \beta + \alpha = \alpha S(\beta).
    \]
\end{proof}

\begin{lemma} \label{ord_mult_lt_suc}
    For all ordinals $\alpha$ and $\beta$ with $\alpha \neq 0$, we have $\alpha
    \beta < \alpha S(\beta)$.
\end{lemma}
\begin{proof}
    \[
        \alpha \beta < \alpha \beta + \alpha = \alpha S(\beta).
    \]
\end{proof}

\begin{instance} \label{ord_mult_normal}
    For all ordinals $\alpha$, the function $f_\alpha(\beta) = \alpha \beta$
    satisfies the normal limit condition.
\end{instance}
\begin{proof}
    This follows directly from the definition.
\end{proof}

\begin{theorem} \label{ord_mult_homo_le}
    For all ordinals $\alpha$, the function $f_\alpha(\beta) = \alpha \beta$ is
    orderly.
\end{theorem}
\begin{proof}
    This is a direct application of Theorem \ref{make_ord_normal_le} and Lemma
    \ref{ord_mult_le_suc}.
\end{proof}

\begin{theorem} \label{ord_mult_homo_inj}
    For all ordinals $\alpha \neq 0$, the function $f_\alpha(\beta) = \alpha
    \beta$ is injective.
\end{theorem}
\begin{proof}
    This is a direct application of Theorem \ref{make_ord_normal_inj} and Lemma
    \ref{ord_mult_lt_suc}.
\end{proof}

Note that this requires $\alpha \neq 0$!  Thus, any time we want to use
multiplication as a normal function, we need to ensure that $\alpha \neq 0$.
However, some things that we normally do with normal functions still works when
$\alpha = 0$, like the following theorem.

\begin{theorem} \label{ord_mult_sup}
    For all ordinals $\alpha$ and small sets $S$, we have
    \[
        \alpha \sup S = \sup\{\alpha + \delta \mid \delta \in S\}.
    \]
\end{theorem}
\begin{proof}
    This follows directly from Theorem \ref{ord_normal_sup}.  Note that we don't
    need $\alpha \neq 0$ because the proof of Theorem \ref{ord_normal_sup}
    doesn't require the function to be strictly increasing.
\end{proof}

\begin{instance}
    Zero is a left annihilator in the ordinals.
\end{instance}
\begin{proof}
    Let $\alpha$ be an ordinal.  We will prove that $0\alpha = 0$ by ordinal
    induction.  When $\alpha = 0$, we have $0(0) = 0$.  When $0\alpha = 0$, we
    have $0S(\alpha) = 0\alpha + 0 = 0 + 0 = 0$.  When $\alpha$ is a limit
    ordinal with $0\delta = 0$ for all $\delta < \alpha$, we have
    \[
        0\alpha = \sup\{0\delta \mid \delta < \alpha\} = \sup\{0\} = 0.
    \]
\end{proof}

\begin{instance}
    Define the ordinal 1 to be the order type of the singleton type.
\end{instance}

\begin{instance} \label{ord_not_trivial}
    The ordinals are not trivial.
\end{instance}
\begin{proof}
    If $0 = 1$, we would have an order isomorphism $f$ from the singleton type
    to the empty type.  Then $f(I)$ is a value of the empty type, which is
    impossible.  Thus, we must have $0 \neq 1$.
\end{proof}

\begin{theorem} \label{ord_one_pos}
    In the ordinals, $0 < 1$.
\end{theorem}
\begin{proof}
    This follows directly from Theorem \ref{all_pos2} and the previous proof.
\end{proof}

\begin{theorem} \label{ord_lt_one_eq}
    For all ordinals $\alpha$, if $\alpha < 1$, then $0 = \alpha$.
\end{theorem}
\begin{proof}
    Let $\A$ be an order type with $[\A] = \alpha$.  Then by $\alpha < 1$ and
    Theorem \ref{ord_lt_simpl}, we know that there is an order isomorphism $f$
    from $\A$ to $\S_x$, an initial segment of the singleton type.  If we had a
    value in $a : \A$, we would have $f(a) < x$, but all values of the singleton
    type are equal, so this is impossible.  Thus, by Theorem \ref{ord_false_0},
    we have $0 = \alpha$.
\end{proof}

\begin{theorem} \label{ord_pos_one}
    For all ordinals $\alpha$, $0 \neq \alpha$ if and only if $1 \leq \alpha$.
\end{theorem}
\begin{proof}
    First assume that $0 \neq \alpha$.  If $\alpha < 1$, by the last theorem we
    have $0 = \alpha$, a contradiction.

    Now assume that $1 \leq \alpha$.  If $0 = \alpha$, we would have $1 \leq 0$,
    a contradiction.
\end{proof}

\begin{theorem} \label{ord_suc_zero_one}
    In the ordinals, $S(0) = 1$.
\end{theorem}
\begin{proof}
    We know that $0 < 1$, so by Theorem \ref{ord_le_suc_lt} we have $S(0) \leq
    1$.  Furthermore, because $0 \neq S(0)$, by Theorem \ref{ord_pos_one} we
    have $1 \leq S(0)$.  By antisymmetry, we have $S(0) = 1$.
\end{proof}

\begin{theorem} \label{ord_suc_plus_one}
    For all ordinals $\alpha$, we have $S(\alpha) = \alpha + 1$.
\end{theorem}
\begin{proof}
    \[
        S(\alpha)
        = S(\alpha + 0)
        = \alpha + S(0)
        = \alpha + 1.
    \]
\end{proof}

\begin{instance}
    One in the ordinals is a left multiplicative identity.
\end{instance}
\begin{proof}
    Let $\alpha$ be an ordinal.  We will prove that $1\alpha = \alpha$ by
    ordinal induction.  When $\alpha = 0$, we have $1(0) = 0$.  When $1\alpha =
    \alpha$, we have $1S(\alpha) = 1\alpha + 1 = \alpha + 1 = S(\alpha)$.  When
    $\alpha$ is a limit ordinal with $1\delta = \delta$ for all $\delta <
    \alpha$,
    \[
        1\alpha
        = \sup\{1\delta \mid \delta < \alpha\}
        = \sup\{\delta \mid \delta < \alpha\}
        = \alpha.
    \]
\end{proof}

\begin{instance}
    One in the ordinals is a right multiplicative identity.
\end{instance}
\begin{proof}
    Let $\alpha$ be an ordinal.  Then
    \[
        \alpha 1 = \alpha S(0) = \alpha 0 + \alpha = 0 + \alpha = \alpha.
    \]
\end{proof}

\begin{instance}
    Multiplication in the ordinals has the zero property.
\end{instance}
\begin{proof}
    Let $\alpha$ and $\beta$ be ordinals such that $0 = \alpha \beta$.  If
    $\beta = 0$, we are done, so consider the case when $\beta \neq 0$.  We must
    prove that $\alpha = 0$.  If $\beta = S(\beta')$ for some ordinal $\beta'$,
    we have $0 = \alpha S(\beta') = \alpha \beta' + \alpha$.  By Theorem
    \ref{ord_plus_zero}, we must have $0 = \alpha$.  The only case remaining is
    when $\beta$ is a limit ordinal.  In this case, we have $0 =
    \sup\{\alpha\delta \mid \delta < \beta\}$.  Because $\beta$ is a limit
    ordinal, we have $1 < \beta$, so $\alpha 1 \leq 0$.  This implies that
    $\alpha = 0$ as required.
\end{proof}

By the properties of normal functions, we can multiply and cancel on the left in
equations, inequalities, and strict inequalities.  Note that in some of these
cases, for multiplication to be normal, we need the first argument to be
nonzero.  However, those are precisely the cases where the corresponding
multiplicative property requires the first argument to be nonzero anyway.

\begin{instance}
    Multiplication of ordinals is left distributive.
\end{instance}
\begin{proof}
    Let $\alpha$, $\beta$, and $\gamma$ be ordinals.  We will prove that
    $\alpha(\beta + \gamma) = \alpha \beta + \alpha \gamma$ by ordinal induction
    on $\gamma$.  When $\gamma = 0$, we have
    \[
        \alpha(\beta + 0) = \alpha \beta = \alpha \beta + 0 = \alpha \beta +
        \alpha 0.
    \]
    When $\alpha(\beta + \gamma) = \alpha\beta + \alpha \gamma$,
    \[
        \alpha(\beta + S(\gamma))
        = \alpha S(\beta + \gamma)
        = \alpha (\beta + \gamma) + \alpha
        = \alpha \beta + \alpha \gamma + \alpha
        = \alpha \beta + \alpha S(\gamma).
    \]
    When $\gamma$ is a limit ordinal with $\alpha (\beta + \delta) = \alpha
    \beta + \alpha \delta$ for all $\delta < \gamma$,
    \begin{align*}
        \alpha(\beta + \gamma)
        &= \alpha \sup\{\beta + \delta \mid \delta < \gamma\} \\
        &= \sup\{\alpha (\beta + \delta) \mid \delta < \gamma\} \\
        &= \sup\{\alpha \beta + \alpha \delta \mid \delta < \gamma\} \\
        &= \alpha \beta + \sup\{\alpha \delta \mid \delta < \gamma\} \\
        &= \alpha \beta + \alpha \gamma.
    \end{align*}
\end{proof}

\begin{instance}
    Multiplication of ordinals is associative.
\end{instance}
\begin{proof}
    Let $\alpha$, $\beta$, and $\gamma$ be ordinals.  We will prove that
    $\alpha (\beta \gamma) = (\alpha \beta) \gamma$ by ordinal
    induction on $\gamma$.  When $\gamma = 0$,
    \[
        \alpha (\beta 0) = \alpha 0 = 0 = (\alpha \beta) 0.
    \]
    When $\alpha (\beta \gamma) = (\alpha \beta) \gamma$,
    \[
        \alpha (\beta S(\gamma))
        = \alpha (\beta \gamma + \beta)
        = \alpha (\beta \gamma) + \alpha \beta
        = (\alpha \beta) \gamma + \alpha \beta
        = (\alpha \beta) S(\gamma).
    \]
    When $\gamma$ is a limit ordinal and $\alpha (\beta \delta) = (\alpha \beta)
    \delta$ for all $\delta < \gamma$,
    \begin{align*}
        \alpha (\beta \gamma)
        &= \alpha \sup\{\beta \delta \mid \delta < \gamma\} \\
        &= \sup\{\alpha (\beta \delta) \mid \delta < \gamma\} \\
        &= \sup\{(\alpha \beta) \delta \mid \delta < \gamma\} \\
        &= (\alpha \beta) \gamma.
    \end{align*}
\end{proof}

\begin{instance}
    The order of the ordinals is right multiplicative.
\end{instance}
\begin{proof}
    Let $\alpha$, $\beta$, and $\gamma$ be such that $\alpha \leq \beta$.  We
    will prove that $\alpha \gamma \leq \beta \gamma$ by ordinal induction on
    $\gamma$.  When $\gamma = 0$, we have $\alpha 0 = 0 = \beta 0$, so $\alpha 0
    \leq \beta 0$.  When $\alpha \gamma \leq \beta \gamma$, adding $\alpha \leq
    \beta$ to that inequality produces
    \begin{align*}
        \alpha \gamma + \alpha &\leq \beta \gamma + \beta \\
        \alpha S(\gamma) &\leq \beta S(\gamma).
    \end{align*}
    When $\gamma$ is a limit ordinal such that $\alpha \delta \leq \beta \delta$
    for all $\delta < \gamma$, we must prove
    \begin{align*}
        \alpha \gamma &\leq \beta \gamma \\
        \sup\{\alpha \delta \mid \delta < \gamma\}
        &\leq
        \sup\{\beta \delta \mid \delta < \gamma\}.
    \end{align*}
    The inductive hypothesis tells us that everything in the first set is less
    than or equal to its corresponding element in the second set, so the result
    follows by Theorem \ref{ord_sup_leq_sup}.
\end{proof}

\begin{theorem} \label{ord_le_self_lmult}
    For all ordinals $\alpha$ and $\beta$, if $0 \neq \beta$, then $\alpha \leq
    \beta \alpha$.
\end{theorem}
\begin{proof}
    This follows from multiplication being normal.
\end{proof}

\begin{theorem} \label{ord_le_self_rmult}
    For all ordinals $\alpha$ and $\beta$, if $0 \neq \beta$, then $\alpha \leq
    \alpha \beta$.
\end{theorem}
\begin{proof}
    By Theorem \ref{ord_pos_one}, we have $1 \leq \beta$.  Then multiplying by
    $\alpha$ we get $\alpha \leq \alpha \beta$.
\end{proof}

\begin{theorem} \label{ord_div}
    For all ordinals $\alpha$ and $\beta$, if $0 \neq \beta$, then there exist
    ordinals $\gamma$ and $\delta$ such that $\alpha = \beta \gamma + \delta$
    and $\delta < \beta$.
\end{theorem}
\begin{proof}
    Let $D$ be the set $\{\eta : \Ord \mid \alpha < \beta \eta\}$.  Because $0
    \neq \beta$, by Theorem \ref{ord_le_self_lmult} we have
    \begin{align*}
        \alpha + 1 &\leq \beta(\alpha + 1) \\
        \alpha &< \beta(\alpha + 1),
    \end{align*}
    so $D$ is nonempty.  Let $\varepsilon$ be the least element of $D$.
    We will now look at the cases when $\varepsilon = 0$, when $\varepsilon$ is
    a successor ordinal, and when $\varepsilon$ is a limit ordinal.

    The case $\varepsilon = 0$ is impossible because we would then have $\alpha
    < \beta 0 = 0$.

    We will look at the limit case next, because it turns out to be impossible
    too.  We will prove that when $\varepsilon$ is a limit ordinal, we have
    $\beta \varepsilon \leq \alpha$, which is a contradiction because
    $\varepsilon \in D$.  Because $\varepsilon$ is a limit ordinal, we have
    $\beta \varepsilon = \sup\{\beta \delta \mid \delta < \varepsilon\}$.  To
    prove that this is less than or equal to $\alpha$, it suffices to prove that
    $\alpha$ is an upper bound of the set.  So let $\zeta < \varepsilon$.  We
    must prove that $\beta \zeta \leq \alpha$.  Assume for another contradiction
    that $\alpha < \beta \zeta$.  Then by the minimality of $\varepsilon$, we
    have $\varepsilon \leq \zeta$.  But $\zeta < \varepsilon$, which is a
    contradiction.

    With the two impossible cases out of the way, now consider the case when
    there exists an ordinal $\gamma$ with $\varepsilon = S(\gamma)$.  Because
    $\gamma < \varepsilon$, we must have $\beta \varepsilon \leq \alpha$ by the
    minimality of $\varepsilon$.  By Theorem \ref{ord_le_ex}, we have an ordinal
    $\delta$ such that $\beta \gamma + \delta = \alpha$.  This is precisely the
    equation we are looking for, and all that remains is to prove that $\delta <
    \beta$.  Because $\varepsilon \in D$, we have $\alpha < \beta \varepsilon =
    \beta S(\gamma) = \beta \gamma + \beta$, and since $\alpha = \beta \gamma +
    \delta$ we have
    \[
        \beta \gamma + \delta < \beta \gamma + \beta.
    \]
    Cancelling $\beta \gamma$ we get $\delta < \beta$ as required.
\end{proof}

\begin{theorem} \label{ord_mult_lim_lim}
    For all ordinals $\alpha \neq 0$ and $\beta$, if $\beta$ is a limit ordinal,
    then $\alpha \beta$ is a limit ordinal.
\end{theorem}
\begin{proof}
    This follows from addition being a normal function and Theorem
    \ref{ord_normal_lim_ord}.
\end{proof}

\begin{theorem} \label{ord_lim_mult_lim}
    For all ordinals $\alpha$ and $\beta \neq 0$, if $\alpha$ is a limit
    ordinal, then $\alpha \beta$ is a limit ordinal.
\end{theorem}
\begin{proof}
    If $\beta$ is a successor ordinal with $\beta = S(\beta')$, then
    \[
        \alpha \beta = \alpha S(\beta') = \alpha \beta' + \alpha,
    \]
    and then the result follows from Theorem \ref{ord_plus_lim_lim}.  And if
    $\beta$ is a limit ordinal, then the result follows from the previous
    theorem.
\end{proof}

\section{Ordinal Exponentiation}

Ordinal exponentiation is slightly more tricky to define than addition and
multiplication.  Because $0^0 = 1$ but $0^\omega = 0$, we don't have the normal
limit condition in all cases, so we can't define it with normal recursion.
However, the theory as built up so far relies on the normal limit condition

\begin{definition}
    Let $\alpha$ and $\beta$ be ordinals.  We will define the power $\alpha ^
    \beta$ piecewise.  When $\alpha = 0$ and $\beta = 0$, we define
    $\alpha^\beta = 1$.  When $\alpha = 0$ and $\beta \neq 0$, we define
    $\alpha^\beta = 0$.  Otherwise, we define $\alpha^\beta$ by normal recursion
    on $\beta$ such that
    \begin{align*}
        \alpha^0 &= 1 \\
        \alpha^{S(\beta)} &= \alpha^\beta \beta.
    \end{align*}
\end{definition}

\begin{theorem} \label{zero_ord_pow}
    For all ordinals $\alpha$, if $0 \neq \alpha$, we have $0 ^ \alpha = 0$.
\end{theorem}
\begin{proof}
    This is true by definition.
\end{proof}

\begin{theorem} \label{ord_pow_zero}
    For all ordinals $\alpha$, we have $\alpha^0 = 1$.
\end{theorem}
\begin{proof}
    We technically need to check the cases $\alpha = 0$ and $\alpha \neq 0$, but
    in both cases this is simply true by definition.
\end{proof}

\begin{instance} \label{ord_pow_lim}
    When $\alpha \neq 0$ and $\beta$ is a limit ordinal, the function
    $f_\alpha(\beta) = \alpha^\beta$ satisfies the normal limit condition.
\end{instance}
\begin{proof}
    When $\alpha \neq 0$, $\alpha^\beta$ is defined by normal recursion, so it
    satisfies the normal limit condition.
\end{proof}

\begin{lemma} \label{ord_pow_le_suc}
    For all ordinals $\alpha \neq 0$ and $\beta$, if $\alpha \neq 0$, then
    $\alpha ^ \beta \leq \alpha ^ {S(\beta)}$.
\end{lemma}
\begin{proof}
    Because $\alpha^{S(\beta)} = \alpha^\beta \alpha$, the result follows
    directly from Theorem \ref{ord_le_self_rmult}.
\end{proof}

\begin{theorem} \label{ord_pow_homo_le}
    For all $\alpha \neq 0$, the function $f_\alpha(\beta) = \alpha^\beta$ is
    orderly.
\end{theorem}
\begin{proof}
    This is a direct application of Theorem \ref{make_ord_normal_le} and Lemma
    \ref{ord_pow_le_suc}.
\end{proof}

\begin{theorem} \label{ord_pow_nz}
    For all ordinals $\alpha$ and $\beta$, if $0 \neq \alpha$, then $0 \neq
    \alpha^\beta$.
\end{theorem}
\begin{proof}
    Because ordinal exponentiation is orderly, from $0 \leq \beta$ we have
    $\alpha^0 \leq \alpha^\beta$.  Since $0 < 1 = \alpha^0 \leq \alpha^\beta$,
    we have $0 \neq \alpha^\beta$.
\end{proof}

\begin{lemma} \label{ord_pow_lt_suc}
    For all ordinals $\alpha$ and $\beta$, if $1 < \alpha$, then $\alpha ^ \beta
    < \alpha ^ {S(\beta)}$.
\end{lemma}
\begin{proof}
    Because $\alpha \neq 0$, we have $\alpha^\beta \neq 0$ by Theorem
    \ref{ord_pow_nz}.  Thus, we can multiply $1 < \alpha$ by $\alpha^\beta$ to
    get $\alpha^\beta < \alpha^\beta \alpha = \alpha^{S(\beta)}$.
\end{proof}

\begin{theorem} \label{ord_pow_homo_inj}
    For all $\alpha > 1$, the function $f_\alpha(\beta) = \alpha^\beta$ is
    injective.
\end{theorem}
\begin{proof}
    This is a direct application of Theorem \ref{make_ord_normal_inj} and Lemma
    \ref{ord_pow_lt_suc}.
\end{proof}

Thus, ordinal exponentiation is only normal when $\alpha > 1$.  However, it
still satisfies the normal limit condition and being orderly even when $\alpha =
1$, but not when $\alpha = 0$.

\begin{theorem} \label{ord_pow_sup}
    For all $\alpha \neq 0$ and small sets $S$, we have
    \[
        \alpha^{\sup S} = \sup\{\alpha^\delta \mid \delta \in S\}.
    \]
\end{theorem}
\begin{proof}
    This follows directly from Theorem \ref{ord_normal_sup}.
\end{proof}

\begin{theorem} \label{ord_pow_one}
    For all ordinals $\alpha$, $\alpha^1 = \alpha$.
\end{theorem}
\begin{proof}
    \[
        \alpha ^ 1 = \alpha ^{S(0)} = \alpha^0 \alpha = 1 \alpha = \alpha.
    \]
\end{proof}

\begin{theorem} \label{one_ord_pow}
    For all ordinals $\alpha$, $1^\alpha = 1$.
\end{theorem}
\begin{proof}
    The proof will be by ordinal induction.  When $\alpha = 0$, we have $1^0 =
    1$.  When $1^\alpha = 1$, we have $1^{S(\alpha)} = 1^\alpha 1 = 1^\alpha =
    1$.  When $\alpha$ is a limit ordinal such that $1^\delta = 1$ for all
    $\delta < \alpha$, we have
    \[
        1^\alpha
        = \sup\{1^\delta \mid \delta < \alpha\}
        = \sup\{1 \mid \delta < \alpha\}
        = 1.
    \]
\end{proof}

\begin{theorem} \label{ord_pow_le}
    For all ordinals $\alpha$, $\beta$, and $\gamma$, if $0 \neq \alpha$ and
    $\beta \leq \gamma$, then $\alpha ^ \beta \leq \alpha ^ \gamma$.
\end{theorem}
\begin{proof}
    This is just a rephrasing of ordinal exponentiation being orderly.
\end{proof}

\begin{theorem} \label{ord_pow_plus}
    For all ordinals $\alpha$, $\beta$, and $\gamma$, we have $\alpha^{\beta +
    \gamma} = \alpha ^ \beta \alpha ^ \gamma$.
\end{theorem}
\begin{proof}
    The proof will be by ordinal induction on $\gamma$.  When $\gamma = 0$, we
    have
    \[
        \alpha^{\beta + 0} = \alpha^\beta = \alpha^\beta 1 = \alpha^\beta
        \alpha^0.
    \]
    When $\alpha^{\beta + \gamma} = \alpha ^ \beta \alpha ^ \gamma$,
    \[
        \alpha^{\beta + S(\gamma)}
        = \alpha^{S(\beta + \gamma)}
        = \alpha^{\beta + \gamma} \alpha
        = \alpha^\beta \alpha^\gamma \alpha
        = \alpha^\beta \alpha^{S(\gamma)}.
    \]
    The limit case will be a bit more complicated.  Assume that $\gamma$ is a
    limit ordinal such that $\alpha^{\beta + \delta} = \alpha^\beta
    \alpha^\delta$ for all $\delta < \gamma$.  Then when $\alpha = 0$, we have
    $\beta + \gamma \neq 0$ as well from $\gamma$ being a limit ordinal, so
    \[
        0^{\beta + \gamma} = 0 = 0^\beta 0 = 0^\beta 0^\gamma.
    \]
    Then when $\alpha \neq 0$,
    \begin{align*}
        \alpha ^ {\beta + \gamma}
        &= \alpha ^ {\sup\{\beta + \delta \mid \delta < \gamma\}} \\
        &= \sup\{\alpha^{\beta + \delta} \mid \delta < \gamma\} \\
        &= \sup\{\alpha^\beta \alpha^\delta \mid \delta < \gamma\} \\
        &= \alpha^\beta \sup\{\alpha^\delta \mid \delta < \gamma\} \\
        &= \alpha^\beta \alpha^\gamma.
    \end{align*}
\end{proof}

\begin{theorem} \label{ord_pow_pow}
    For all ordinals $\alpha$, $\beta$, and $\gamma$, we have $(\alpha ^ \beta)
    ^ \gamma = \alpha ^ {\beta \gamma}$.
\end{theorem}
\begin{proof}
    The proof will be by ordinal induction on $\gamma$.  When $\gamma = 0$,
    \[
        (\alpha ^ \beta) ^ 0 = 1 = \alpha^0 = \alpha^{\beta 0}.
    \]
    When $(\alpha ^ \beta)^\gamma = \alpha ^ {\beta \gamma}$,
    \[
        (\alpha ^ \beta)^{S(\gamma)}
        = (\alpha ^ \beta)^\gamma \alpha ^ \beta
        = \alpha^{\beta \gamma} \alpha^\beta
        = \alpha^{\beta \gamma + \beta}
        = \alpha^{\beta S(\gamma)}.
    \]
    The limit case is a little more complicated.  Assume that $\gamma$ is a
    limit ordinal with $(\alpha ^ \beta) ^ \delta = \alpha ^ {\beta \delta}$ for
    all $\delta < \gamma$.  When $\beta = 0$,
    \[
        (\alpha ^ 0) ^ \gamma = 1 ^ \gamma = 1 = \alpha^0 = \alpha^{0 \gamma}.
    \]
    When $\beta \neq 0$ and $\alpha = 0$,
    \[
        (0 ^ \beta) ^ \gamma = 0 ^ \gamma = 0 = 0 ^ {\beta \gamma}.
    \]
    Finally, when $\beta \neq 0$ and $\alpha \neq 0$,
    \begin{align*}
        (\alpha^\beta)^\gamma
        &= \sup\{(\alpha^\beta)^\delta \mid \delta < \gamma\} \\
        &= \sup\{\alpha^{\beta\delta} \mid \delta < \gamma\} \\
        &= \alpha^{\sup\{\beta\delta \mid \delta < \gamma\}} \\
        &= \alpha^{\beta \gamma}.
    \end{align*}
\end{proof}

\begin{theorem} \label{ord_pow_lt}
    For all ordinals $\alpha$, $\beta$, and $\gamma$, if $1 < \alpha$ and $\beta
    < \gamma$, we have $\alpha ^ \beta < \alpha ^ \gamma$.
\end{theorem}
\begin{proof}
    This is a rephrasing of ordinal exponentiation being strictly orderly.
\end{proof}

\begin{theorem} \label{ord_pow_lcancel}
    For all ordinals $\alpha$, $\beta$, and $\gamma$, if $1 < \alpha$ and
    $\alpha ^ \beta = \alpha ^ \gamma$, then $\beta = \gamma$.
\end{theorem}
\begin{proof}
    This is a rephrasing of ordinal exponentiation being injective.
\end{proof}

\begin{theorem} \label{ord_le_rpow}
    For all ordinals $\alpha$, $\beta$, and $\gamma$, if $\alpha \leq \beta$,
    then $\alpha ^ \gamma \leq \beta ^ \gamma$.
\end{theorem}
\begin{proof}
    The proof will be by ordinal induction on $\gamma$.  When $\gamma = 0$, we
    have $\alpha^0 \leq \beta^0$.  When $\alpha ^ \gamma \leq \beta ^ \gamma$,
    we can multiply this by $\alpha \leq \beta$ to get
    \begin{align*}
        \alpha ^ \gamma \alpha &\leq \beta ^ \gamma \beta \\
        \alpha ^ {S(\gamma)} &\leq \beta ^ {S(\gamma)}.
    \end{align*}
    The limit case is more complicated.  Assume that $\gamma$ is a limit ordinal
    such that for all $\delta < \gamma$, we have $\alpha ^ \delta \leq \beta ^
    \delta$.  Then when $\alpha = 0$, we have
    \[
        0^\gamma = 0 \leq \beta^\gamma,
    \]
    and when $\beta = 0$, we get $\alpha = 0$ as well from $\alpha \leq \beta$,
    so this is just the $\alpha = 0$ case again.  When $\alpha \neq 0$ and
    $\beta \neq 0$, we must prove
    \begin{align*}
        \alpha ^ \gamma &\leq \beta ^ \gamma \\
        \sup\{\alpha ^ \delta \mid \delta < \gamma\}
        &\leq
        \sup\{\beta ^ \delta \mid \delta < \gamma\}.
    \end{align*}
    The inductive hypothesis tells us that everything in the first set is less
    than or equal to its corresponding element in the second set, so the result
    follows by Theorem \ref{ord_sup_leq_sup}.
\end{proof}

\begin{theorem} \label{ord_pow_le_pow}
    For all ordinals $\alpha$ and $\beta$ with $1 < \beta$, we have $\alpha
    \leq \beta ^ \alpha$.
\end{theorem}
\begin{proof}
    This follows directly from Theorem \ref{ord_normal_le}.
\end{proof}

\begin{theorem} \label{ord_pow_lim_lim}
    For all $\alpha > 1$ and limit ordinals $\beta$, $\alpha^\beta$ is a limit
    ordinal.
\end{theorem}
\begin{proof}
    This follows from Theorem \ref{ord_normal_lim_ord}.
\end{proof}

\begin{theorem} \label{ord_pow_self_le}
    For all ordinals $\alpha$ and $\beta \neq 0$, we have $\alpha \leq \alpha ^
    \beta$.
\end{theorem}
\begin{proof}
    When $\alpha = 0$, the result is trivial.  When $\alpha \neq 0$, we can
    apply Theorem \ref{ord_pow_le} to $1 \leq \beta$ to get $\alpha ^ 1 = \alpha
    \leq \alpha ^ \beta$.
\end{proof}

\section{Relationship Between the Ordinals and Natural Numbers}

The natural number inclusion function $\iN$ given by Definition \ref{from_nat}
can be applied to the ordinals too.  This section will be looking at some of the
consequences of thinking of natural numbers as ordinals.  However, the
definition of $\iN$ is not too useful here, so we will come up with an
alternative way of thinking of natural numbers in the ordinals.

\begin{theorem} \label{nat_ord_suc}
    For all natural numbers $n$, $\iN(S(n)) = S(\iN(n))$.
\end{theorem}
\begin{proof}
    \[
        \iN(S(n)) = \iN(n + 1) = \iN(n) + 1 = S(\iN(n)).
    \]
\end{proof}

\begin{theorem} \label{from_nat_ord}
    For all natural numbers $n$, $\iN(n) = [\N_n]$.
\end{theorem}
\begin{proof}
    The proof will be by induction on $n$.  When $n = 0$, we must find an order
    isomorphism between $\N_0$ and the empty type.  $\N_0$ is empty, so the
    empty function suffices.  Now assume that $\iN(n) = [\N_n]$.  Then
    \[
        \iN(S(n)) = S(\iN(n)) = S([\N_n]).
    \]
    By Theorem \ref{ord_suc_type}, this is equal to $[\N_n + \S]$.  We must
    prove that
    \[
        [\N_n + \S] = [\N_{S(n)}].
    \]
    Define a function $f : \N_n + \S \to \N_{S(n)}$ given by
    \begin{align*}
        f(\iota_1(m)) &= m \\
        f(\iota_2(I)) &= n.
    \end{align*}
    $f$ is an order isomorphism.
\end{proof}

\begin{instance}
    The ordinals are characteristic zero.
\end{instance}
\begin{proof}
    If $0 = S(n)$ for some $n$, we would have $0 = S(\iN(n))$ in the ordinals,
    which is impossible.
\end{proof}

Most proofs of the properties of $\iN$ work just fine for the ordinals, with the
exception of it being orderly.  It is still orderly for the ordinals however.

\begin{instance}
    $\iN$ is orderly in the ordinals.
\end{instance}
\begin{proof}
    Let $a \leq b$ in the natural numbers.  By Theorem \ref{nat_le_ex}, we have
    a natural number $c$ such that $a + c = b$.  Then adding $\iN(a)$ to both
    sides of $0 \leq \iN(c)$, we get $\iN(a) \leq \iN(a) + \iN(c) = \iN(a + c) =
    \iN(b)$.
\end{proof}

\begin{theorem} \label{from_nat_ord_pow}
    For all natural numbers $a$ and $b$, we have $\iN(a^b) = \iN(a)^{\iN(b)}$.
\end{theorem}
\begin{proof}
    The proof will be by induction on $b$.  When $b = 0$,
    \[
        \iN(a^0) = \iN(1) = 1 = \iN(a)^0 = \iN(a)^{\iN(0)},
    \]
    so the base case is true.  Now assume that $\iN(a^b) = \iN(a)^{\iN(b)}$.
    Then
    \[
        \iN(a^{S(b)})
        = \iN(a^ba)
        = \iN(a^b)\iN(a)
        = \iN(a)^{\iN(b)} \iN(a)
        = \iN(a)^{\iN(b) + 1}
        = \iN(a)^{\iN(S(b))}.
    \]
    Thus, the theorem is true by induction.
\end{proof}

\begin{definition}
    Define the ordinal $\omega$ to be the ordinal corresponding to the natural
    numbers.
\end{definition}

\begin{theorem} \label{nat_lt_omega}
    For all natural numbers $n$, $n < \omega$.
\end{theorem}
\begin{proof}
    Since $n = [\N_n]$ and $\omega = [\N]$, we have $[\N_n] < [\N]$ by a direct
    application of Theorem \ref{ord_lt_simpl}.
\end{proof}

\begin{theorem} \label{ord_lt_omega}
    For all ordinals $\alpha$, if $\alpha < \omega$, then there exists a natural
    number $n$ with $n = \alpha$.
\end{theorem}
\begin{proof}
    By Theorem \ref{ord_lt_simpl}, we know that any order type of $\alpha$ is
    isomorphic to an initial segment of the natural numbers, which is precisely
    what it means for there to be a natural number $n$ with $n = \alpha$.
\end{proof}

\begin{theorem} \label{omega_nz}
    $0 \neq \omega$.
\end{theorem}
\begin{proof}
    By Theorem \ref{nat_lt_omega}, we know that $0 < \omega$, which implies that
    $0 \neq \omega$.
\end{proof}

\begin{theorem} \label{omega_lim}
    $\omega$ is a limit ordinal.
\end{theorem}
\begin{proof}
    We just proved that $\omega \neq 0$, so we only need to prove that $\omega$
    is not a successor ordinal.  If $\omega = S(n)$ for some ordinal $n$, we
    would have $n < \omega$.  Thus, by Theorem \ref{ord_lt_omega}, $n$ is a
    natural number.  Then $S(n) = \omega$ should be a natural number as well,
    which is impossible.
\end{proof}

\begin{theorem} \label{ord_lim_omega}
    $\omega$ is the smallest limit ordinal, that is, for all limit ordinals
    $\alpha$, we have $\omega \leq \alpha$.
\end{theorem}
\begin{proof}
    Assume that $\alpha < \omega$.  Then $\alpha$ is a natural number, and all
    natural numbers are either $0$ or a successor, contradicting $\alpha$ being
    a limit ordinal.
\end{proof}

\begin{theorem} \label{ord_lim_pow_lim}
    For all ordinals $\alpha$ and $\beta \neq 0$, if $\alpha$ is a limit
    ordinal, then $\alpha^\beta$ is a limit ordinal.
\end{theorem}
\begin{proof}
    If $\beta$ is a successor ordinal with $\beta = S(\beta')$, we would have
    $\alpha^\beta = \alpha^{\beta'} \alpha$, and the result follows from Theorem
    \ref{ord_mult_lim_lim}.  If $\beta$ is a limit ordinal, we know that $1 <
    \omega \leq \alpha$, so the result follows from Theorem
    \ref{ord_pow_lim_lim}.
\end{proof}

\begin{theorem} \label{nat_plus_omega}
    For all natural numbers $n$, $n + \omega = \omega$.
\end{theorem}
\begin{proof}
    The proof will be by antisymmetry.  $\omega \leq n + \omega$ follows
    directly from Theorem \ref{ord_le_self_lplus}.  To prove that $n + \omega
    \leq \omega$, by Theorem \ref{ord_sup_least} it suffices to prove that $n +
    \delta \leq \omega$ for all $\delta < \omega$.  By Theorem
    \ref{ord_lt_omega}, there is some natural number $m = \delta$.  Then $n + m$
    is a natural number, which is less than $\omega$ by Theorem
    \ref{nat_lt_omega}.
\end{proof}

\begin{theorem} \label{ord_plus_eat}
    For all ordinals $\alpha$ and $\beta$ such that $\alpha \omega \leq \beta$,
    we have $\alpha + \beta = \beta$.
\end{theorem}
\begin{proof}
    By Theorem \ref{ord_le_ex}, we have an ordinal $\gamma$ such that $\beta =
    \alpha \omega + \gamma$.  Then
    \[
        \alpha + \beta
        = \alpha + \alpha \omega + \gamma
        = \alpha(1 + \omega) + \gamma
        = \alpha \omega + \gamma
        = \beta.
    \]
\end{proof}

\begin{theorem} \label{ord_plus_no_eat}
    For all ordinals $\alpha$ and $\beta$, if $\beta < \alpha\omega$, then
    $\beta < \alpha + \beta$.
\end{theorem}
\begin{proof}
    Because $\beta < \alpha \omega$, by Theorem \ref{ord_sup_in} we have a
    natural number $m'$ such that $\beta < \alpha m'$.  Let $m$ be the least
    such natural number.  We can't have $m = 0$ because that would imply $\beta
    < \alpha 0 = 0$.  So $m = S(n)$ for some natural number $n$.  Then we must
    have $\alpha n \leq \beta$ by the minimality of $m$.  Adding $\alpha$ to
    both sides we get
    \[
        \alpha + \alpha n \leq \alpha + \beta,
    \]
    and along with
    \[
        \beta < \alpha m = \alpha S(n) = \alpha(1 + n) = \alpha + \alpha n,
    \]
    we get $\beta < \alpha + \beta$ by transitivity.
\end{proof}

\begin{theorem} \label{nat_mult_omega}
    For all natural numbers $n \neq 0$, $n\omega = \omega$.
\end{theorem}
\begin{proof}
    The proof will be by antisymmetry.  $\omega \leq n \omega$ follows directly
    from Theorem \ref{ord_le_self_lmult}.  To prove that $n\omega \leq \omega$,
    by Theorem \ref{ord_sup_least} it suffices to prove that $n \delta + n \leq
    \omega$ for all $\delta < \omega$.  By Theorem \ref{ord_lt_omega}, there is
    some natural number $m = \delta$.  Then $nm + n$ is a natural number, which
    is less than $\omega$ by Theorem \ref{nat_lt_omega}.
\end{proof}

\begin{theorem} \label{nat_pow_omega}
    For all natural numbers $n > 1$, $n^\omega = \omega$.
\end{theorem}
\begin{proof}
    The proof will be by antisymmetry.  $\omega \leq n^\omega$ follows directly
    from Theorem \ref{ord_pow_le_pow}.  To prove that $n^\omega \leq \omega$,
    by Theorem \ref{ord_sup_least} it suffices to prove that $n^\delta n \leq
    \omega$ for all $\delta < \omega$.  By Theorem \ref{ord_lt_omega}, there is
    some natural number $m = \delta$.  Then $n^m n$ is a natural number, which
    is less than $\omega$ by Theorem \ref{nat_lt_omega}.
\end{proof}

\begin{theorem} \label{ord_nat_plus_limit}
    For all natural numbers $n$ and limit ordinals $\alpha$, we have $n + \alpha
    = \alpha$.
\end{theorem}
\begin{proof}
    When $n = 0$, the result is trivial.  When $n \neq 0$, we have $n \omega =
    \omega \leq \alpha$, so the result follows by Theorem \ref{ord_plus_eat}.
\end{proof}

\begin{theorem} \label{ord_iterate_suc}
    For all ordinals $\alpha$ and natural numbers $n$, we have $S^n(\alpha) =
    \alpha + n$.
\end{theorem}
\begin{proof}
    The proof will be by induction on $n$.  When $n = 0$, both sides are equal
    to $\alpha$.  When $S^n(\alpha) = \alpha + n$,
    \[
        S^{S(n)}(\alpha) = S(S^n(\alpha)) = S(\alpha + n) = \alpha + S(n).
    \]
\end{proof}

\begin{theorem} \label{ord_lim_omega_times}
    For all limit ordinals $\alpha$, there exists an ordinal $\beta$ such that
    $\alpha = \omega \beta$.
\end{theorem}
\begin{proof}
    By ordinal division, we have ordinals $\beta$ and $n < \omega$ such
    that $\alpha = \omega \beta + n$.  Now if $n = S(n')$, we would have $\alpha
    = \omega \beta + S(n') = S(\omega \beta + n')$, contradicting $\alpha$ being
    a limit ordinal.  Thus, $n = 0$, so $\alpha = \omega \beta$.
\end{proof}

\begin{theorem} \label{ord_nat_mult_limit}
    For all natural numbers $n \neq 0$ and limit ordinals $\alpha$, we have
    $n\alpha = \alpha$.
\end{theorem}
\begin{proof}
    By Theorem \ref{ord_lim_omega_times}, we have an ordinal $\beta$ such that
    $\alpha = \omega \beta$.  Then
    \[
        n \alpha = n \omega \beta = \omega \beta = \alpha.
    \]
\end{proof}

\begin{theorem} \label{ord_nz_one_plus}
    For all ordinals $\alpha \neq 0$, there exists an ordinal $\beta$ such that
    $\alpha = 1 + \beta$.
\end{theorem}
\begin{proof}
    If $\alpha < \omega$, because $\alpha \neq 0$, $\alpha = S(n)$ for some
    natural number $n$, and $\alpha = 1 + n$.  If $\alpha \geq \omega$, then we
    have $1 + \alpha = \alpha$ by Theorem \ref{ord_plus_eat}, so in this case
    $\beta = \alpha$ suffices.
\end{proof}

\begin{theorem} \label{nat_mult_omega_pow}
    For all natural numbers $n$ and nonzero ordinals $\alpha$,
    \[
        S(n) \omega^\alpha = \omega^\alpha.
    \]
\end{theorem}
\begin{proof}
    By Theorem \ref{ord_nz_one_plus}, we have an ordinal $\beta$ such that
    $\alpha = 1 + \beta$.  Then
    \[
        S(n) \omega^\alpha = S(n) \omega^{1 + \beta} = S(n) \omega \omega^\beta
        = \omega \omega^\beta
        = \omega^{1 + \beta}
        = \omega^\alpha.
    \]
\end{proof}

\begin{theorem} \label{ord_plus_nat_mult_omega}
    For all nonzero ordinals $\alpha$ and all natural numbers $n$, we have
    $(\alpha + n)\omega = \alpha \omega$.
\end{theorem}
\begin{proof}
    If $\alpha < \omega$, then $\alpha$ is a nonzero natural number, so $(\alpha
    + n)\omega = \omega = \alpha \omega$.  If $\alpha \geq \omega$,
    \[
        (\alpha + n)\omega
        \leq (\alpha + \alpha) \omega
        = \alpha 2 \omega
        = \alpha \omega
        \leq (\alpha + n)\omega.
    \]
\end{proof}

\begin{theorem} \label{ord_plus_nat_mult_lim}
    For all nonzero ordinals $\alpha$, natural numbers $n$, and limit ordinals
    $\beta$, we have $(\alpha + n) \beta = \alpha \beta$.
\end{theorem}
\begin{proof}
    By Theorem \ref{ord_lim_omega_times}, we have an ordinal $\gamma$ such that
    $\beta = \omega \gamma$.  Then
    \[
        (\alpha + n) \beta
        = (\alpha + n) \omega \gamma
        = \alpha \omega \gamma
        = \alpha \beta.
    \]
\end{proof}

Now that we have gotten some of the basic properties of $\omega$, we can finally
define the true $\aleph$ function.

\begin{definition}
    Define $\aleph : \Ord \to \Card$ by $\aleph_\alpha = \aleph'_{\omega +
    \alpha}$.
\end{definition}

\begin{theorem}
    For all ordinals $\alpha \geq \omega^2$, $\aleph_\alpha = \aleph'_\alpha$.
\end{theorem}
\begin{proof}
    We get $\omega + \alpha = \alpha$ directly from Theorem \ref{ord_plus_eat}.
\end{proof}

\begin{instance}
    $\aleph$ is injective.
\end{instance}
\begin{proof}
    Let $\aleph_\alpha = \aleph_\beta$.  This means that $\aleph'_{\omega +
    \alpha} = \aleph'_{\omega + \beta}$  Then by the injectivity of $\aleph'$,
    we get $\omega + \alpha = \omega + \beta$, and cancelling $\omega$ we get
    $\alpha = \beta$.
\end{proof}

\begin{instance}
    $\aleph$ is orderly.
\end{instance}
\begin{proof}
    Let $\alpha \leq \beta$.  Then $\omega + \alpha \leq \omega + \beta$, and
    because $\aleph'$ is orderly, we have $\aleph'_{\omega + \alpha} \leq
    \aleph_{\omega + \beta}$, so $\aleph_\alpha \leq \aleph_\beta$.
\end{proof}

\begin{theorem} \label{aleph_sur}
    For all cardinals $\mu \geq \aleph_0$, there exists an ordinal $\alpha$ with
    $\aleph_\alpha = \mu$.
\end{theorem}
\begin{proof}
    By the surjectivity of $\aleph'$, there exists some ordinal $\beta$ with
    with $\aleph'_\beta = \mu$.  Then by assumption we have $\aleph'_\beta \geq
    \aleph'_\omega$, and since $\aleph'$ is equivalently orderly we have $\beta
    \geq \omega$.  Then by Theorem \ref{ord_le_ex} we have an ordinal $\alpha$
    such that $\beta = \omega + \alpha$.  Then
    \[
        \mu = \aleph'_\beta = \aleph'_{\omega + \alpha} = \aleph_\alpha.
    \]
\end{proof}

\begin{theorem} \label{aleph_least}
    For all ordinals $\alpha \neq 0$ and cardinals $\mu$, if for all $\beta <
    \alpha$ we have $\aleph_\beta < \mu$, then $\aleph_\alpha \leq \mu$.
\end{theorem}
\begin{proof}
    First, because $0 < \alpha$, we have $\aleph_0 < \mu$.
    We will use the corresponding theorem for $\aleph'$, Theorem
    \ref{aleph'_least}.  We must prove that for all $\beta < \omega + \alpha$,
    we have $\aleph'_\beta < \mu$.  Now if $\beta < \omega$, $\beta$ is a
    natural number.  Then $\aleph'_\beta < \aleph_0 < \mu$, so this case is
    true.  Now if $\omega \leq \beta$, by Theorem \ref{ord_le_ex} we have a
    $\gamma$ with $\beta = \omega + \gamma$.  Then $\gamma < \alpha$, so by
    assumption we have $\aleph_\gamma < \mu$, and $\aleph_\gamma =
    \aleph'_{\omega + \gamma} = \aleph'_\beta$, so this case is also true.
\end{proof}

Arithmetic of ordinals can often be confusing because it is necessary to
remember which of the basic algebraic properties we are used to are true in the
ordinals.  The basic properties that are true were listed in the previous
sections.  We will now prove that the ordinals do not have any of the other
basic algebraic properties that we like types to have.

\begin{theorem}
    Addition of ordinals is not commutative.
\end{theorem}
\begin{proof}
    If $1 + \omega = \omega + 1$, we would have $\omega = \omega + 1$, and
    cancelling $\omega$ we get $0 = 1$, which is false.
\end{proof}

\begin{theorem}
    Multiplication of ordinals is not commutative.
\end{theorem}
\begin{proof}
    If $2 \omega = \omega 2$, we would have $\omega = \omega 2$, and
    cancelling $\omega$ we get $1 = 2$, which is false.
\end{proof}

\begin{theorem}
    Multiplication of ordinals is not right distributive over addition.
\end{theorem}
\begin{proof}
    We have $(1 + 1)\omega = \omega$, but $1 \omega + 1\omega = \omega +
    \omega$.
\end{proof}

\begin{theorem}
    Addition of ordinals is not right cancellative.
\end{theorem}
\begin{proof}
    We have $0 + \omega = 1 + \omega$, but not $0 = 1$.
\end{proof}

\begin{theorem}
    Multiplication of ordinals is not right cancellative.
\end{theorem}
\begin{proof}
    We have $1\omega = 2\omega$ but not $1 = 2$.
\end{proof}

\begin{theorem}
    The order of ordinals is not strictly right additive.
\end{theorem}
\begin{proof}
    We have $0 < 1$ but not $0 + \omega < 1 + \omega$.
\end{proof}

\begin{theorem}
    The order of ordinals is not right addition cancellative.
\end{theorem}
\begin{proof}
    We have $1 + \omega \leq 0 + \omega$ but not $1 \leq 0$.
\end{proof}

\begin{theorem}
    The order of ordinals is not strictly right multiplicative.
\end{theorem}
\begin{proof}
    We have $1 < 2$ but not $1 \omega < 2 \omega$.
\end{proof}

\begin{theorem}
    The order of ordinals is not right multiplication cancellative.
\end{theorem}
\begin{proof}
    We have $2 \omega \leq 1 \omega$ but not $2 \leq 1$.
\end{proof}

\section{Derivatives}

\begin{definition}
    Let $f_i$ be a small family of normal functions.  We will define the
    derivative $f'$ of the family by normal recursion where $f'(0)$ is defined
    to be the first common fixed point of all the $f_i$, and $f'(S(\alpha))$ is
    defined to be the first common fixed point greater than or equal to
    $S(f'(\alpha))$.
\end{definition}

\begin{lemma} \label{ord_family_derivative_base_increasing}
    For all small families of normal functions $f_i$ and for all $\alpha$, we
    have $f'(\alpha) < f'(S(\alpha))$.
\end{lemma}
\begin{proof}
    $f'(S(\alpha))$ is defined to be the first common fixed point greater than
    or equal to $S(f'(\alpha))$.  Thus,
    \[
        f'(\alpha) < S(f'(\alpha)) \leq f'(S(\alpha)).
    \]
\end{proof}

\begin{lemma} \label{ord_family_derivative_le}
    For all small families of normal functions $f_i$, $f'$ is orderly.
\end{lemma}
\begin{proof}
    This follows from Theorem \ref{make_ord_normal_le} and Lemma
    \ref{ord_family_derivative_base_increasing}.
\end{proof}

\begin{lemma} \label{ord_family_derivative_inj}
    For all small families of normal functions $f_i$, $f'$ is injective.
\end{lemma}
\begin{proof}
    This follows from Theorem \ref{make_ord_normal_inj} and Lemma
    \ref{ord_family_derivative_base_increasing}.
\end{proof}

Thus, the derivative of a family of normal functions is itself normal.

\begin{theorem} \label{ord_family_derivative_fixed}
    For all small families of normal functions $f_i$ with $i : \X$, all values
    $x : \X$, and ordinals $\alpha$, we have $f_x(f'(\alpha)) = f'(\alpha)$.
\end{theorem}
\begin{proof}
    The proof will be by ordinal induction on $\alpha$.  $f'$ was defined to
    make this true when $\alpha$ is $0$ or a successor ordinal, so the only
    interesting case is when $\alpha$ is a limit ordinal.  The inductive
    hypothesis says that for all $\delta < \alpha$, we have $f_x(f'(\delta)) =
    f'(\delta)$.  By definition, we have
    \[
        f'(\alpha) = \sup\{f'(\delta) \mid \delta < \alpha\}.
    \]
    Since $0 < \alpha$, we have
    \begin{align*}
        f(f'(\alpha))
        &= f(\sup\{f'(\delta) \mid \delta < \alpha\}) \\
        &= \sup\{f(f'(\delta)) \mid \delta < \alpha\} \\
        &= \sup\{f'(\delta) \mid \delta < \alpha\} \\
        &= f'(\alpha).
    \end{align*}
\end{proof}

\begin{theorem} \label{ord_family_derivative_lim_eq}
    For all small families of normal functions $f_i$ and all limit ordinals
    $\gamma$, we have $f'(\gamma) = \sup F(f'(\gamma))$, where $F(\gamma)$ is the
    set given by Definition \ref{ord_normal_family_fixed_set}.
\end{theorem}
\begin{proof}
    The proof will be by antisymmetry.  To prove that $f'(\gamma) \leq \sup
    F(f'(\gamma))$, by $f'$ being normal, it suffices to prove
    \[
        \sup\{f'(\delta) \mid \delta < \gamma\} \leq \sup F(f'(\gamma)),
    \]
    which we can do by showing that $\sup F(f'(\gamma))$ is an upper bound of
    the set.  So let $\delta < \gamma$.  We must show that $f'(\delta) \leq \sup
    F(f'(\gamma))$.  By Theorem \ref{ord_normal_family_fixed_leq} we have
    $f'(\gamma) \leq \sup F(f'(\gamma))$, so by transitivity it suffices to
    prove that $f'(\delta) \leq f'(\gamma)$.  We get this from the fact that
    $f'$ is orderly and that $\delta < \gamma$.

    For the other direction, $\sup F(f'(\gamma)) \leq f'(\gamma)$ comes directly
    from Theorems \ref{ord_normal_family_fixed_least} and
    \ref{ord_family_derivative_fixed}.
\end{proof}

Like when we were finding the fixed points of a family of normal functions, we
can define the derivative of a single normal function as well.  Given a normal
function $f$, we will call its derivative $f'$.

\section{Cantor Normal Form}

Despite what this section is called, I won't be proving the Cantor normal form
theorem here!  I was just wanting to learn how exactly to do calculations with
ordinals in Cantor normal form, so I proved a bunch of theorems relating to how
those calculations are done.

\newcommand{\CNF}{\mathrm{CNF}}
\begin{definition}
    Call $\Ord \times \N$ the CNF type.  Given a pair $(\alpha, n)$ in the CNF
    type, define a function that takes $(\alpha, n)$ to $\omega ^ \alpha S(n)$.
    Then, given a list $a : \L(\Ord \times \N)$, define the CNF evaluation
    function $\CNF : \L(\Ord \times \N) \to \Ord$ to be the sum of this function
    on each element of the list.
\end{definition}

\begin{theorem} \label{cnf_eval_plus}
    For all lists $a$ and $b$ of the CNF type, $\CNF(a + b) = \CNF(a) +
    \CNF(b)$.
\end{theorem}
\begin{proof}
    The proof will be by induction on $a$.  When $a = []$, we have
    \[
        \CNF([] + b) = \CNF(b) = 0 + \CNF(b) = \CNF([]) + \CNF(b).
    \]
    When $\CNF(a + b) = \CNF(a) + \CNF(b)$, then for all pairs $(\alpha, n)$,
    \[
        \CNF((\alpha, n) : a + b)
        = \omega^\alpha S(n) + \CNF(a + b)
        = \omega^\alpha S(n) + \CNF(a) + \CNF(b)
        = \CNF((\alpha, n) : a) + \CNF(b).
    \]
\end{proof}

\begin{theorem} \label{cnf_lt_omega_le}
    For all ordinals $\alpha$ and $\beta$ with $\alpha < \beta$ and all natural
    numbers $m$ and $n$,
    \[
        \omega^\alpha S(m) \omega \leq \omega^\beta S(n).
    \]
\end{theorem}
\begin{proof}
    We immediately get
    \[
        \omega^\alpha S(m) \omega = \omega^\alpha \omega = \omega^{S(\alpha)}.
    \]
    From $\alpha < \beta$, we get $S(\alpha) \leq \beta$, so $\omega^{S(\alpha)}
    \leq \omega^\beta$.  Because $1 \leq S(n)$, we can multiply these two
    inequalities to get
    \[
        \omega^{S(\alpha)} \leq \omega^\beta S(n).
    \]
\end{proof}

By Theorem \ref{ord_plus_eat}, this implies that when $\alpha < \beta$, we have
$\omega ^ \alpha S(m) + \omega ^ \beta S(n) = \omega ^ \beta S(n)$.

\begin{theorem} \label{cnf_plus_base}
    For all lists $a$ of the CNF type and ordinals $\alpha$, $\CNF(a) +
    \omega^\alpha = \CNF(a') + \omega^\alpha$, where $a'$ is the list $a$ with all
    of the entries $(\beta, n)$ where $\beta < \alpha$ removed.
\end{theorem}
\begin{proof}
    The proof will be by backwards induction on $a$.  When $a = []$, we have $b
    = []$ as well, so the base case is trivial.  Now assume that
    \[
        \CNF(a) + \omega ^ \alpha = \CNF(a') + \omega^\alpha.
    \]
    Then for all pairs $(\beta, m)$,
    \[
        \CNF(a + [(\beta, m)]) + \omega ^ \alpha
        = \CNF(a) + \omega^\beta S(m) + \omega ^ \alpha.
    \]
    The proof will now have three cases: when $\alpha < \beta$, when $\alpha =
    \beta$, and when $\alpha > \beta$.

    When $\alpha < \beta$, we have
    \begin{align*}
        \CNF(a) + \omega^\beta S(m) + \omega ^ \alpha
        &= \CNF(a) + \omega^\alpha + \omega^\beta S(m) + \omega ^ \alpha \\
        &= \CNF(a') + \omega^\alpha + \omega^\beta S(m) + \omega ^ \alpha \\
        &= \CNF(a') + \omega^\beta S(m) + \omega ^ \alpha \\
        &= \CNF((a + [(\beta, m)])') + \omega ^ \alpha.
    \end{align*}

    When $\alpha = \beta$, we have
    \begin{align*}
        \CNF(a) + \omega^\beta S(m) + \omega ^ \alpha
        &= \CNF(a) + \omega^\alpha S(m) + \omega ^ \alpha \\
        &= \CNF(a) + \omega^\alpha (1 + m) + \omega ^ \alpha \\
        &= \CNF(a) + \omega^\alpha + \omega^\alpha m + \omega ^ \alpha \\
        &= \CNF(a') + \omega^\alpha + \omega^\alpha m + \omega ^ \alpha \\
        &= \CNF(a') + \omega^\beta S(m) + \omega ^ \alpha \\
        &= \CNF((a + [(\beta, m)])') + \omega ^ \alpha.
    \end{align*}

    When $\alpha > \beta$, we have
    \begin{align*}
        \CNF(a) + \omega^\beta S(m) + \omega ^ \alpha
        &= \CNF(a) + \omega ^ \alpha \\
        &= \CNF(a') + \omega ^ \alpha \\
        &= \CNF((a + [(\beta, m)])') + \omega ^ \alpha \\
    \end{align*}
\end{proof}

\begin{theorem} \label{cnf_plus}
    For all lists $a$ of the CNF type, ordinals $\alpha$, and natural numbers
    $n$, $\CNF(a) + \omega^\alpha S(n) = \CNF(a') + \omega^\alpha S(n)$, where
    $a'$ is the list $a$ with all of the entries $(\beta, n)$ where $\beta <
    \alpha$ removed.
\end{theorem}
\begin{proof}
    \begin{align*}
        \CNF(a) + \omega^\alpha S(n)
        &= \CNF(a) + \omega^\alpha + \omega^\alpha n \\
        &= \CNF(a') + \omega^\alpha + \omega^\alpha n \\
        &= \CNF(a') + \omega^\alpha S(n).
    \end{align*}
\end{proof}

\begin{lemma} \label{lt_omega_pow_two}
    For all ordinals $\delta$ and $\beta$ with $\delta < \omega ^\beta$, we have
    $2\delta < \omega ^ \beta$.
\end{lemma}
\begin{proof}
    If $\beta = 0$, we have $\delta < \omega ^ 0 = 1$, so $\delta = 0$ as well.
    Then $2\delta = 0 < 1 = \omega^0$.  When $\beta \neq 0$, by Theorem
    \ref{ord_nz_one_plus} we have an ordinal $\beta'$ such that $\beta = 1 +
    \beta'$.  Then we can multiply by $2$ on the left of $\delta < \omega^\beta$
    to get
    \[
        2\delta < 2\omega^\beta = 2\omega^{1 + \beta'} = 2\omega \omega^{\beta'}
        = \omega \omega^{\beta'}
        = \omega^\beta.
    \]
\end{proof}

\begin{lemma} \label{cnf_eval_le1}
    For all lists $a$ of the CNF type, if for a given ordinal $\alpha$, we have
    $\beta < \alpha$ for all pairs $(\beta, n)$ in $a$, then there exists a
    natural number $N$ such that $\CNF(a) \leq \omega^\alpha S(N)$.
\end{lemma}
\begin{proof}
    The proof will be by induction on $a$.  When $a = []$, we have $\CNF(a) = 0
    \leq \omega^\alpha S(0)$.  Now assume that there exists an $N$ such that
    $\CNF(a) \leq \omega^\alpha S(N)$, and let $\beta < \alpha$ and let $n$ be a
    natural number.  We will prove that $\CNF((\beta, n) : a) \leq \omega^\alpha
    S(S(n) + N)$.
    \begin{align*}
        \CNF((\beta, n) : a)
        &= \omega^\beta S(n) + \CNF(a) \\
        &\leq \omega^\alpha S(n) + \omega^\alpha S(N) \\
        &= \omega^\alpha (S(n) + S(N)) \\
        &= \omega^\alpha S(S(n) + N).
    \end{align*}
\end{proof}

\begin{lemma} \label{cnf_eval_le2}
    For all lists $a$ of the CNF type, if for a given ordinal $\alpha$, we have
    $\beta < \alpha$ for all pairs $(\beta, n)$ in $a$, then $\CNF(a) \leq
    \omega^\alpha \omega$.
\end{lemma}
\begin{proof}
    By the previous lemma, we have an $N$ such that $\CNF(a) \leq \omega ^
    \alpha S(n)$.  Then we also have $\omega ^ \alpha S(n) \leq \omega ^ \alpha
    \omega$.
\end{proof}

\begin{theorem} \label{cnf_mult_omega}
    For all ordinals $\alpha$ and $\beta \neq 0$, natural numbers $n$, and lists
    $a$ such that $\delta < \alpha$ for all pairs $(\delta, m)$ in $a$,
    \[
        \CNF((\alpha, n) : a) \omega^\beta = \omega^{\alpha + \beta}.
    \]
\end{theorem}
\begin{proof}
    By Lemma \ref{cnf_eval_le1}, we have some $N$ such that $\CNF(a) \leq
    \omega^\alpha S(N)$.  Then
    \begin{align*}
        \CNF((\alpha, n) : a) \omega^\beta
        &= (\omega^\alpha S(n) + \CNF(a) )\omega^\beta \\
        &\leq (\omega^\alpha S(n) + \omega^\alpha S(N) )\omega^\beta \\
        &= \omega^\alpha (S(n) + S(N)) \omega^\beta \\
        &= \omega^\alpha \omega^\beta \\
        &= \omega^{\alpha + \beta} \\
        &= \omega^\alpha \omega^\beta \\
        &= \omega^\alpha S(n) \omega^\beta \\
        &\leq (\omega^\alpha S(n) + \CNF(a)) \omega^\beta \\
        &= \CNF((\alpha, n) : a) \omega^\beta.
    \end{align*}
\end{proof}

\begin{theorem} \label{cnf_mult_nat}
    For all ordinals $\alpha$ and $\beta \neq 0$, natural numbers $n$ and $m$,
    and lists $a$ such that $\delta < \alpha$ for all pairs $(\delta, m)$ in
    $a$,
    \[
        \CNF((\alpha, n) : a) S(m) = \CNF((\alpha, nS(m) + m) : a).
    \]
\end{theorem}
\begin{proof}
    The proof will be by induction on $m$.  When $m = 0$,
    \[
        \CNF((\alpha, n) : a) S(0)
        = \CNF((\alpha, n S(0) + 0) : a).
    \]
    When
    \[
        \CNF((\alpha, n) : a) S(m) = \CNF((\alpha, nS(m) + m) : a),
    \]
    we have
    \begin{align*}
        \CNF((\alpha, n) : a) S(S(m))
        &= \CNF((\alpha, n) : a) + \CNF((\alpha, n) : a) S(m) \\
        &= \CNF((\alpha, n) : a) + \CNF((\alpha, nS(m) + m) : a) \\
        &= \omega^\alpha S(n) + \CNF(a) + \omega^\alpha S(nS(m) + m) + \CNF(a)\\
        &= \omega^\alpha S(n) + \omega^\alpha S(nS(m) + m) + \CNF(a)\\
        &= \omega^\alpha (S(n) + S(nS(m) + m)) + \CNF(a)\\
        &= \omega^\alpha S(n + nS(m) + S(m)) + \CNF(a)\\
        &= \omega^\alpha S((nS(S(m)) + S(m)) + \CNF(a)\\
        &= \CNF((\alpha, nS(S(m)) + S(m)) : a).
    \end{align*}
\end{proof}

\begin{theorem} \label{cnf_pow_omega}
    For all ordinals $\alpha \neq 0$ and $\beta \neq 0$, natural numbers $n$,
    and lists $a$ such that $\delta < \alpha$ for all pairs $(\delta, m)$ in
    $a$,
    \[
        \CNF((\alpha, n) : a)^{\omega^\beta} = \omega^{\alpha \omega^\beta}.
    \]
\end{theorem}
\begin{proof}
    By Lemma \ref{cnf_eval_le2}, we have $\CNF(a) \leq \omega^\alpha \omega$.
    Then
    \begin{align*}
        \CNF((\alpha, n) : a)^{\omega^\beta}
        &= (\omega^\alpha S(n) + \CNF(a))^{\omega^\beta} \\
        &\leq (\omega^\alpha S(n) + \omega^\alpha \omega)^{\omega^\beta} \\
        &= (\omega^\alpha (S(n) + \omega))^{\omega^\beta} \\
        &= (\omega^\alpha \omega)^{\omega^\beta} \\
        &= (\omega^{\alpha + 1})^{\omega^\beta} \\
        &= \omega^{(\alpha + 1) \omega^\beta} \\
        &= \omega^{\alpha \omega^\beta} \\
        &= (\omega^{\alpha})^{\omega^\beta} \\
        &\leq (\omega^{\alpha}S(n))^{\omega^\beta} \\
        &\leq (\omega^{\alpha}S(n) + \CNF(a))^{\omega^\beta} \\
        &= \CNF((\alpha, n) : a)^{\omega^\beta}.
    \end{align*}
\end{proof}

\end{document}
