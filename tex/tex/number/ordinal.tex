\documentclass[../../math.tex]{subfiles}
\externaldocument{../../math.tex}

\begin{document}

\setcounter{chapter}{8}

\chapter{The Ordinals}

The ordinals are a way of studying well-ordered sets, by basically considering
all of them at once.  They were initially described as equivalence classes of
well-ordered sets, but there were issues with this initial conception and modern
set theories tend to use other constructions.  However, in the foundations used
here, we won't have those issues, so we can describe ordinals using their
traditional interpretation as equivalence classes of well-ordered sets.

\section{Basic Construction}

\begin{definition}
    We will say that an order type is a type $\U$ that has a well order.
\end{definition}

In general, people often use ``order type'' to refer to more than just well
orders, but the concept is mainly useful for well orders, so for the sake of
simplicity we will consider order types to only be well orders.  Remember that
to prove that a type is well-ordered, it suffices to prove that the relation is
antisymmetric and that every nonempty subset has a least element.

\begin{definition}
    Define a relation $\sim$ on order types where $\A \sim \B$ is defined to
    mean that there exists an order isomorphism between $\A$ and $\B$, that is,
    a function $f$ that is bijective and preserves inequalities in one
    direction.  Note that these conditions also imply that $f$ preserves both
    inequalities and strict inequalities in both directions.
\end{definition}

\begin{lemma}
    The relation $\sim$ is an equivalence relation.
\end{lemma}
\begin{proof}
    \textit{Reflexivitity.}  The identity function is an order isomorphism.

    \textit{Symmetry.}  Assume that there is an order isomorphism $f$ from $\A$
    to $\B$.  Then its inverse $f^{-1}$ is bijective, and when $a \leq b$, we
    have $f(f^{-1}(a)) = f(f^{-1}(b))$, and because $f$ is an order isomorphism,
    $f^{-1}(a) = f^{-1}(b)$, so $f^{-1}$ is an order isomorphism from $\B$ to
    $\A$.

    \textit{Transitivity.}  Given three order types $\A$, $\B$, and $\C$ with
    order isomorphisms $f : \A \to \B$ and $g : \B \to \C$, $g \circ f$ is an
    order isomorphism from $\A$ to $\C$.
\end{proof}

\begin{definition}
    Define the type $\Ord$ to be the quotient of all order types by $\sim$.
\end{definition}


\end{document}
