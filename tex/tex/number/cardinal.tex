\documentclass[../../math.tex]{subfiles}
\externaldocument{../../math.tex}
\externaldocument{../basics/foundations}
\externaldocument{../basics/set}
\externaldocument{../basics/elementary_algebra}
\externaldocument{../basics/natural}
\externaldocument{ordinal}

\begin{document}

\setcounter{chapter}{9}

\chapter{The Cardinals}

The cardinals were already introduced in the previous chapter as part of the
development of the theory of ordinals.  Cardinals represent the size of types,
and are defined here as equivalence classes of types under the existence of
bijections.  If you want to see the definition of the cardinals, the
development of their order, and the definition of the function $\aleph$, see the
previous chapter.  In this chapter, we will describe cardinal arithmetic,
describe infinite cardinals, and prove many useful theorems to make working with
cardinals easier.

\section{Addition of Cardinals}

Remember that for two types $\A$ and $\B$, the type $\A + \B$ is the sum of $\A$
and $\B$, also called their disjoint union.  Given an $a : \A$, we call the
corresponding value in $\A + \B$ $a|$, and given a $b : \B$, we call the
corresponding value in $\A + \B$ $|b$.  To make definitions and proofs simpler
here, we will introduce a new notation.  When working with an arbitrary value in
$\A + \B$, we will often use the notation $a|b$ to represent a value made with
either an $a : \A$ or a $b : \B$.  We can manipulate functions to/from sum types
using this notation, and doing so can compactly represent multiple cases of a
definition or proof at once.

\begin{lemma}
    The sum type operation is well-defined under the equivalence relation $\sim$
    given in Section \ref{card_base}.
\end{lemma}
\begin{proof}
    Let $\A \sim \B$ and $\C \sim \D$.  This means that we have bijections $f :
    \A \to \B$ and $g : \C \to \D$.  We must find a bijection between $\A + \C$
    and $\B + \D$.  Define $h : \A + \C \to \B + \D$ by $h(a|c) = f(a)|g(c)$.
    We can also define $h^{-1} : \B + \D \to \A + \C$ by $h(b|d) =
    f^{-1}(b)|g^{-1}(d)$.  Then
    \[
        h(h^{-1}(b|d))
        = h(f^{-1}(b)|g^{-1}(d))
        = f(f^{-1}(b))|g(g^{-1}(d))
        = b|d
    \]
    and
    \[
        h^{-1}(h(a|b))
        = h^{-1}(f(a)|g(b))
        = f^{-1}(f(a))|g^{-1}(g(b)) =
        a|b,
    \]
    so $h$ is bijective by Theorem \ref{inverse_ex_bijective}.
\end{proof}

\begin{instance}
    Define addition in the cardinals as the binary operation given by Theorem
    \ref{binary_op_ex} and the previous lemma.
\end{instance}

\begin{instance}
    Addition of cardinals is associative.
\end{instance}
\begin{proof}
    Let $\A$, $\B$, and $\C$ be types.  Define the functions
    \[
        f : \A + (\B + \C) \to (\A + \B) + \C
    \]
    and
    \[
        g : (\A + \B) + \C \to \A + (\B + \C)
    \]
    by
    \[
        f(a|(b|c)) = (a|b)|c
    \]
    and
    \[
        g((a|b)|c) = a|(b|c).
    \]
    Then
    \[
        f(g((a|b)|c)) = f(a|(b|c)) = (a|b)|c
    \]
    and
    \[
        g(f(a|(b|c))) = g((a|b)|c) = a|(b|c).
    \]
\end{proof}

\begin{instance}
    Addition of cardinals is commutative.
\end{instance}
\begin{proof}
    Let $\A$ and $\B$ be types.  Define functions
    \[
        f : \A + \B \to \B + \A
    \]
    and
    \[
        g : \B + \A \to \A + \B
    \]
    by
    \[
        f(a|b) = b|a
    \]
    and
    \[
        g(b|a) = a|b.
    \]
    Then
    \[
        f(g(b|a)) = f(a|b) = b|a
    \]
    and
    \[
        g(f(a|b)) = g(b|a) = a|b.
    \]
\end{proof}

\begin{instance}
    Define zero in the cardinals to be $|\E|$, the cardinality of the empty
    type.
\end{instance}

\begin{instance}
    Zero is a left additive identity in the cardinals.
\end{instance}
\begin{proof}
    Let $\A$ be a type.  Define the functions $f : \A + \E \to \A$ and $g : \A
    \to \A + \E$ by $f(_|a) = a$ and $g(a) = |a$.  Note that the first case of
    the definition of $f$ is unnecessary because no values of the empty type
    exist.  Then
    \[
        f(g(a)) = f(|a) = a
    \]
    and
    \[
        g(f(|a)) = g(a) = |a.
    \]
\end{proof}

\begin{instance}
    The order of the cardinals is left additive.
\end{instance}
\begin{proof}
    Let $\A$, $\B$, and $\C$ be types such that there exists an injective
    function $f : \A \to \B$.  Then define a function $g : \C + \A \to \C + \B$
    given by
    \[
        g(c|a) = c|f(a).
    \]
    Now assume that $g(c_1|a_1) = g(c_2|a_2)$.  Then $c_1|f(a_1) = c_2|f(a_2)$.
    If it's the first branch, we have $c_1 = c_2$, and if it's the second
    branch, by the injectivity of $f$, we have $a_1 = a_2$.  Thus, we have
    $c_1|a_1 = c_2|a_2$ as required.
\end{proof}

\begin{instance}
    All cardinals are positive.
\end{instance}
\begin{proof}
    The empty function is vacuously injective to any type.
\end{proof}

\begin{theorem} \label{card_false_0}
    For all types $\A$, if $\A \to \False$, then $0 = |\A|$.
\end{theorem}
\begin{proof}
    The empty function is vacuously injective from $\E$ to $\A$, and since $\A
    \to \False$, the empty function is also vacuously surjective.
\end{proof}

\begin{theorem} \label{card_nz_ex}
    For all types $\U$, if $0 \neq |\U|$, then a value exists in $\U$.
\end{theorem}
\begin{proof}
    This is the contrapositive of the previous theorem.
\end{proof}

\section{Multiplication of Cardinals}

Remember that for two types $\A$ and $\B$, the type $\A \times \B$ is the
product of $\A$ and $\B$, also called their Cartesian product.  Given an $a :
\A$ and a $b : \B$, we call the corresponding value in $\A \times \B$ $(a, b)$.

\begin{lemma}
    The product type operation is well-defined under the equivalence relation
    $\sim$ given in Section \ref{card_base}.
\end{lemma}
\begin{proof}
    Let $\A \sim \B$ and $\C \sim \D$.  This means that we have bijections $f :
    \A \to \B$ and $g : \C \to \D$.  We must find a bijection between $\A \times
    \C$ and $\B \times \D$.  Define $h : \A \times \C \to \B \times \D$ by $h(a,
    c) = (f(a), g(c))$.  We can also define $h^{-1} : \B \times \D \to \A \times
    \C$ by $h^{-1}(b, d) = (f^{-1}(b), g^{-1}(d))$.  Then
    \[
        h(h^{-1}(b, d))
        = h(f^{-1}(b), g^{-1}(d))
        = (f(f^{-1}(b)), g(g^{-1}(d)))
        = (b, d)
    \]
    and
    \[
        h^{-1}(h(a, c))
        = h^{-1}(f(a), g(c))
        = (f^{-1}(f(a)), g^{-1}(g(c)))
        = (a, c),
    \]
    so $h$ is bijective by Theorem \ref{inverse_ex_bijective}.
\end{proof}

\begin{instance}
    Define multiplication in the cardinals as the binary operation given by
    Theorem \ref{binary_op_ex} and the previous lemma.
\end{instance}

\begin{instance}
    Multiplication of cardinals is multiplicative.
\end{instance}
\begin{proof}
    Let $\A$, $\B$, and $\C$ be types.  Define
    \[
        f : \A \times (\B \times \C) \to (\A \times \B) \times \C
    \]
    and
    \[
        g : (\A \times \B) \times \C \to \A \times (\B \times \C)
    \]
    by
    \[
        f(a, (b, c)) = ((a, b), c)
    \]
    and
    \[
        g((a, b), c) = (a, (b, c)).
    \]
    Then
    \[
        f(g((a, b), c)) = f(a, (b, c)) = ((a, b), c)
    \]
    and
    \[
        g(f(a, (b, c))) = g((a, b), c) = (a, (b, c)).
    \]
\end{proof}

\begin{instance}
    Multiplication of cardinals is commutative.
\end{instance}
\begin{proof}
    Let $\A$ and $\B$ be types.  Define functions
    \[
        f : \A \times \B \to \B \times \A
    \]
    and
    \[
        g : \B \times \A \to \A \times \B
    \]
    given by
    \[
        f(a, b) = (b, a)
    \]
    and
    \[
        g(b, a) = (a, b).
    \]
    Then
    \[
        f(g(b, a)) = f(a, b) = (b, a)
    \]
    and
    \[
        g(f(a, b)) = g(b, a) = (a, b).
    \]
\end{proof}

\begin{instance}
    Zero is an annihilator in the cardinals.
\end{instance}
\begin{proof}
    We must find a bijective function from $\A \times \E$ to $\E$ for any type
    $\A$.  Any value in $\A \times \E$ induces a value in $\E$, which is
    impossible, so $\A \times \E$ can't have any values.  Thus, the empty
    function works, and is surjective because the codomain is empty as well.
\end{proof}

\begin{instance}
    Define one in the cardinals to be $|\S|$, the cardinality of the singleton
    type.
\end{instance}

\begin{instance}
    One is a multiplicative identity in the cardinals.
\end{instance}
\begin{proof}
    Let $\A$ be a type.  Define $f : \A \times \S \to \A$ by $f(a, I) = a$ and
    $g : \A \to \A \times \S$ by $g(a) = (a, I)$.  Then
    \[
        f(g(a)) = f(a, I) = a
    \]
    and
    \[
        g(f(a, I)) = g(a) = (a, I).
    \]
\end{proof}

\begin{instance}
    Cardinal multiplication distributes over addition.
\end{instance}
\begin{proof}
    Let $\A$, $\B$, and $\C$ be types.  Define functions
    \[
        f : \A \times (\B + \C) \to (\A \times \B) + (\A \times \C)
    \]
    and
    \[
        g : (\A \times \B) + (\A \times \C) \to \A \times (\B + \C)
    \]
    by
    \[
        f(a, b|c) = (a, b)|(a, c)
    \]
    and
    \[
        g((a, b)|(a, c)) = (a, b|c).
    \]
    Then
    \[
        f(g((a, b)|(a, c))) = f(a, b|c) = (a, b)|(a, c)
    \]
    and
    \[
        g(f(a, b|c)) = g((a, b)|(a, c)) = (a, b|c).
    \]
\end{proof}

\begin{instance}
    Multiplication of cardinals has the zero property.
\end{instance}
\begin{proof}
    For a contradiction, assume that $0 = |\A \times \B|$ but $0 \neq |\A|$ and
    $0 \neq |\B|$.  By Theorem \ref{card_nz_ex}, we have values $a : \A$ and $b
    : \B$.  Because $0 = |\A \times \B|$, we have a bijection $f : \A \times \B
    \to \E$, meaning that $f(a, b)$ is a value in the empty type, which is
    impossible.
\end{proof}

\begin{instance}
    The order of cardinals is left multiplicative.
\end{instance}
\begin{proof}
    Let $\A$, $\B$, and $\C$ be types such that $|\A| \leq |\B|$.  We must prove
    that $|\C \times \A| \leq |\C \times \B|$.  Because $|\A| \leq |\B|$, there
    exists an injective function $f : \A \to \B$.  Define a new function $g : \C
    \times \A \to \C \times \B$ given by $g(c, a) = (c, f(a))$.  Then if
    \[
        g(c_1, a_1) = g(c_2, a_2),
    \]
    we have
    \[
        (c_1, f(a_1)) = (c_2, f(a_2)),
    \]
    from which we get $c_1 = c_2$ and $f(a_1) = f(a_2)$.  By the injectivity of
    $f$, we get $a_1 = a_2$.  Thus, $(c_1, a_1) = (c_2, a_2)$, showing that $g$
    is injective.
\end{proof}

\section{Cardinal Exponentiation}

Given two types $\A$ and $\B$, we can create the type of all functions $\A \to
\B$.  In this sense, we can think of $\to$ as being a binary operation on types.
This leads to cardinal exponentiation.  However, we want to define the
exponential $|\A|^{|\B|}$ to be the cardinality of $\B \to \A$, not $\A \to \B$.
Because of this, we can't quite just say to consider $\to$ as a binary
operation for the purpose of constructing cardinal exponentiation.

\begin{lemma}
    $\to$ is well-defined under the equivalence relation $\sim$ given in Section
    \ref{card_base}, in the reverse sense described above.  More explicitly, if
    $\A \sim \B$ and$\C \sim \D$, then $(\C \to \A) \sim (\D \to \B)$.
\end{lemma}
\begin{proof}
    From $\A \sim \B$ and $\C \sim \D$, we have bijections $f : \A \to \B$ and
    $g : \C \to \D$.  Define
    \[
        h : (\C \to \A) \to (\D \to \B)
    \]
    and
    \[
        h^{-1} : (\D \to \B) \to (\C \to \A)
    \]
    by
    \[
        h(i)(d) = f(i(g^{-1}(d)))
    \]
    and
    \[
        h^{-1}(i)(c) = f^{-1}(i(g(c))).
    \]
    To prove that $h(h^{-1}(i)) = i$ and $h^{-1}(h(i)) = i$, by functional
    extensionality it suffices that
    \[
        h(h^{-1}(i))(d)
        = f(h^{-1}(i)(g^{-1}(d)))
        = f(f^{-1}(i(g(g^{-1}(d)))))
        = i(d)
    \]
    and
    \[
        h^{-1}(h(i))(c)
        = f^{-1}(h(i)(g(c)))
        = f^{-1}(f(i(g^{-1}(g(c)))))
        = i(c).
    \]
\end{proof}

\begin{definition}
    Define cardinal exponentiation $\mu^\kappa$ to be the binary operation given
    by Lemma \ref{binary_op_ex} and the previous lemma.  To be explicit, we have
    $|\A|^{|\B|} = |\B \to \A|$.
\end{definition}

\begin{theorem}
    For all cardinals $\kappa$, we have $\kappa^0 = 1$.
\end{theorem}
\begin{proof}
    Given a type $\A$, we can define a map $f$ from $\E \to \A$ to $\S$ that
    takes all functions to the single value in $\S$, and a map $g$ from $\S$ to
    $\E$ to $\A$ that takes in the value in $\S$ and produces the empty
    function $e$, which is the only possible function.  Then
    \[
        f(g(I)) = f(e) = I
    \]
    and
    \[
        g(f(e)) = g(I) = e.
    \]
\end{proof}

\begin{theorem}
    For all cardinals $\kappa \neq 0$, we have $0 ^ \kappa = 0$.
\end{theorem}
\begin{proof}
    Let $\A$ be a type of cardinality $\kappa$.  Since $0 \neq |\A|$, we have a
    value $a : \A$ by Theorem \ref{card_nz_ex}.  If we had a function $f : \A
    \to \E$, we would have a value in the empty type $f(a)$, which is
    impossible.  Thus, by Theorem \ref{card_false_0}, we have $0^\kappa = 0$.
\end{proof}

\begin{theorem}
    For all cardinals $\kappa$, we have $1^\kappa = 1$.
\end{theorem}
\begin{proof}
    Let $\A$ be a type.  Then define the functions $f : (\A \to \S) \to \S$ and
    $g : \S \to (\A \to \S)$ by $f(h) = I$ and $g(I)(x) = I$.  Then
    \[
        f(g(I)) = f(I)
    \]
    and
    \[
        g(f(h)) = g(I) = h.
    \]
\end{proof}

\begin{theorem}
    For all cardinals $\kappa$, we have $\kappa ^ 1 = \kappa$.
\end{theorem}
\begin{proof}
    Let $\A$ be a type.  Define the functions $f : (\S \to \A) \to \A$ and $g :
    \A \to (\S \to \A)$ by $f(h) = h(I)$ and $g(a)(I) = a$.  Then
    \[
        f(g(a)) = g(a)(I) = a
    \]
    and
    \[
        g(f(h))(I) = g(h(I))(I) = h(I).
    \]
\end{proof}

\begin{theorem}
    For all cardinals $\kappa$, $\mu$, and $\nu$, we have $\kappa ^ {\mu + \nu}
    = \kappa ^ \mu \kappa ^ \nu$.
\end{theorem}
\begin{proof}
    Let $\A$, $\B$, and $\C$ be types.  Then define the functions
    \[
        f : ((\B + \C) \to \A) \to ((\B \to \A) \times (\C \to \A))
    \]
    and
    \[
        g : ((\B \to \A) \times (\C \to \A)) \to ((\B + \C) \to \A)
    \]
    by
    \[
        f(h) = ((\lambda b, h(b|)), (\lambda c, h(|c)))
    \]
    and
    \[
        g(h_1, h_2)(b|c) = h_1(b)|h_2(c).
    \]
    Then
    \[
        f(g(h_1, h_2))
        = ((\lambda b, g(h_1, h_2)(b|)), (\lambda c, g(h_1, h_2)(|c)))
        = (\lambda b, h_1(b), \lambda c, h_2(c))
        = (h_1, h_2)
    \]
    and
    \[
        g(f(h))(b|c)
        = g((\lambda b, h(b|)), (\lambda c, h(|c)))(b|c)
        = h(b|)|h(|c)
        = h(b|c).
    \]
\end{proof}

\begin{theorem}
    For all cardinals $\kappa$, $\mu$ and $\nu$, we have $(\kappa ^ \mu) ^ \nu =
    \kappa ^ {\mu \nu}$.
\end{theorem}
\begin{proof}
    Let $\A$, $\B$, and $\C$ be types.  Define the functions
    \[
        f : (\C \to (\B \to \A)) \to ((\B \times \C) \to \A)
    \]
    and
    \[
        g : ((\B \times \C) \to \A) \to (\C \to (\B \to \A))
    \]
    by
    \[
        f(h)(b, c) = h(c)(b)
    \]
    and
    \[
        g(h)(c)(b) = h(b, c).
    \]
    Then
    \[
        f(g(h))(b, c) = g(h)(c)(b) = h(b, c)
    \]
    and
    \[
        g(f(h))(c)(b) = f(h)(b, c) = h(c)(b).
    \]
\end{proof}

\begin{theorem}
    For all cardinals $\kappa$, $\mu$, and $\nu$, we have $(\kappa \mu)^\nu =
    \kappa ^ \nu \mu ^ \nu$.
\end{theorem}
\begin{proof}
    Let $\A$, $\B$, and $C$ be types.  Then define functions
    \[
        f : (\C \to (\A \times \B)) \to ((\C \to \A) \times (\C \to \B))
    \]
    and
    \[
        g : ((\C \to \A) \times (\C \to \B)) \to (\C \to (\A \times \B))
    \]
    by
    \[
        f(h) = ((\lambda c, P_1(h(c))), (\lambda c, P_2(h(c))))
    \]
    and
    \[
        g(h_1, h_2)(c) = (h_1(c), h_2(c)).
    \]
    Then
    \begin{align*}
        f(g(h_1, h_2))
        &= ((\lambda c, P_1(g(h_1, h_2))(c)), (\lambda c, P_2(g(h_1, h_2))(c)))
            \\
        &= ((\lambda c, P_1(h_1, h_2)(c)), (\lambda c, P_2(h_1, h_2)(c))) \\
        &= ((\lambda c, h_1(c)), (\lambda c, h_2(c))) \\
        &= (h_1, h_2)
    \end{align*}
    and
    \[
        g(f(h))(c)
        = g((\lambda c', P_1(h(c'))), (\lambda c', P_2(h(c'))))(c)
        = (P_1(h(c)), P_2(h(c)))
        = h(c).
    \]
\end{proof}

\begin{theorem}
    $|\Prop| = 2$.
\end{theorem}
\begin{proof}
    Recall that $2$ is equal to $|\S| + |\S|$.  Define functions $f : \Prop \to
    \S + \S$ and $g : \S + \S \to \Prop$ given by
    \[
        f(P) = \begin{cases}
            I| & \text{if $P$} \\
            |I & \text{if $\neg P$}
        \end{cases}
    \]
    and
    \[
        g(I|I) = \True|\False.
    \]
    Then
    \[
        f(g(I|I)) = f(\True|\False) = I|I
    \]
    and
    \[
        g(f(P)) = g(I|I) = P.
    \]
\end{proof}

Notice that because of this, $|\A \to \Prop|$ is equal to $2^{|\A|}$, and more
importantly, by Theorem \ref{power_set_bigger}, we have $|\A| < 2^{|\A|}$ for
all cardinals $\A$.

\end{document}
