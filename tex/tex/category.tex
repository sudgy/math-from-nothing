\documentclass[../math.tex]{subfiles}
\externaldocument{../math.tex}

\begin{document}

\setcounter{chapter}{5}

\chapter{Category Theory}

Category theory is the general study of mathematical structures.  It can provide
a unified way of describing many seemingly different aspects of math, and using
it can often simplify many definitions and theorems.  However, I don't actually
know much category theory, so I haven't developed much of it here.  I've mostly
just defined categories, functors, and natural transformations and a few basic
facts about them, and haven't even proven the Yoneda lemma.  I know that there
are many things later on that could be simplified by the use of category theory,
but the fact is I just haven't studied it enough yet.

Another thing that I should point out is that in the Coq code, I am unable to
make arguments using commutative diagrams.  Because this document is supposed to
mirror the Coq code, I will not be using commutative diagrams here either.  This
is not to say that I prefer not using commutative diagrams.  It's just that not
using them makes this document a better representation of the Coq code.

A quick note on the order here: In the Coq code, functors and natural
transformations are defined the moment after categories are.  That's because
it's helpful to have categories be objects in the category of categories, to
have functors be both the morphisms there and the objects in the category of
functors, and to have natural transformations be the morphisms in the category
of functors.  By defining these notions first, they can be used as the
definition of categories, functors, and natural transformations, allowing for
similar notation to be used for all of them.  However, here we don't have to
worry about making sure Coq will know what categories are objects are in, so
I'll present things in a different order here.  Instead, I will define
categories and several of the notions that are definable with just categories,
then functors and their notions after that, and then natural transformations.

\section{Categories}

\begin{definition}
    A Category is a combination of the following things:
    \begin{itemize}
        \item A $\Type$, which we call the objects of the category.  Given a
            particular category $C$, we abuse the notation $A : C$ to mean that
            $A$ is an object in $C$.
        \item A function from objects $A$ and $B$ to a $\Type$ that we call $A
            \to B$, the values of which are called the morphisms from $A$ to
            $B$.  Given a morphism $A \to B$, we call $A$ the domain and $B$ the
            codomain.
        \item A function that takes in objects $A$, $B$, and
            $C$ and morphisms $f : B \to C$ and $g : A \to B$, and produces a
            new morphism $f \circ g : A \to C$, called the composition of $f$
            and $g$.
        \item A function that takes in an object $A$ and produces a morphism
            $\mathds 1 : A \to A$ called the identity morphism.  If the object
            $A$ needs to be explicit we will write $\mathds 1_A$.
        \item A proof that composition of morphisms is associative, that is, for
            all objects $A$, $B$, $C$, and $D$ and all morphisms $f : C \to D$,
            $g : B \to C$, and $h : A \to B$, we have $f \circ (g \circ h) = (f
            \circ g) \circ h$.
        \item A proof that every identity morphism is a left identity, that is,
            for all objects $A$ and $B$ and morphisms $f : A \to B$, we have
            $\mathds 1_B \circ f = f$.
        \item A proof that every identity morphism is a right identity, that is,
            for all objects $A$ and $B$ and morphisms $f : A \to B$, we have
            $f \circ \mathds 1_A= f$.
    \end{itemize}
\end{definition}

\section{Functors}

\begin{definition}
    Given two categories $C_1$ and $C_2$, a functor $F$ from $C_1$ to $C_2$ is a
    combination of the following things:
    \begin{itemize}
        \item A function $C_1 \to C_2$, which will be denoted by treating the
            functor $F$ itself as a function.  For example, if $A : C_1$, then
            $F(A) : C_2$.
        \item For all objects $A$ and $B$ in $C_1$, a function from $A \to B$ to
            $F(A) \to F(B)$.  Again, this will be denoted by treating $F$ as a
            function, so if we have a morphism $f : A \to B$, then $F(f) : F(A)
            \to F(B)$.  Thus, functors will be used to represent two different
            functions that must be determined from context.
        \item A proof that for all objects $A$, $B$, and $C$ in $C_1$ and
            morphisms $f : B \to C$ and $g : A \to B$, we have $F(f \circ g) =
            F(f) \circ F(g)$.
        \item A proof that for all objects $A : C_1$, we have $F(\mathds 1_A) =
            \mathds 1_{F(A)}$.
    \end{itemize}
\end{definition}

\begin{theorem}
    Given a category $C$, the identity map on both objects and morphisms is a
    functor that we call the identity functor.
\end{theorem}
\begin{proof}
    Trivial.
\end{proof}

\begin{theorem}
    Given three categories $C_1$, $C_2$, and $C_3$ and two functors $F : C_2 \to
    C_3$ and $G : C_1 \to C_2$, the functions $F(G(A))$ and $F(G(f))$ are a
    functor that we call the composition of $F$ and $G$.
\end{theorem}
\begin{proof}
    As for composition,
    \[
        F(G(f \circ g)) = F(G(f) \circ G(g)) = F(G(f)) \circ F(G(g)).
    \]
    As for the identity,
    \[
        F(G(\mathds 1)) = F(\mathds 1) = \mathds 1.
    \]
\end{proof}

\begin{theorem}
    Categories form a category \Cat where the objects are categories, the
    morphisms are functors between them, composition of functors is as defined
    above, and the identity functor is as defined above.
\end{theorem}
\begin{proof}
    Composition of functors is associative by the associativity of function
    composition.  The identity functor is an identity because it is defined to
    be the identity function.
\end{proof}

Thus, without confusion we may use the notation $\mathds 1$ to represent the
identity functor and $F \circ G$ to represent the composition of two functors.

\section{Natural Transformations}

\begin{definition}
    Given two categories $C_1$ and $C_2$ and two functors $F$ and $G$ from $C_1$
    to $C_2$, a natural transformation $\alpha : F \to G$ is a combination of
    the following things:
    \begin{itemize}
        \item A function from objects $A : C_1$ to morphisms $F(A) \to G(A)$,
            which will be denoted $\alpha(A)$.
        \item A proof that for all objects $A$ and $B$ in $C_1$ and morphisms $f
            : A \to B$, we have $\alpha(B) \circ F(f) = G(f) \circ \alpha(A)$.
    \end{itemize}
\end{definition}

\begin{theorem}
    Given categories $C_1$ and $C_2$ and a functor $F : C_1 \to C_2$, the
    function that takes objects $A : C_1$ to $\mathds 1_{F(A)}$ is a natural
    transformation from $F$ to $F$ called the identity natural transformation.
\end{theorem}
\begin{proof}
    We mult prove that for all objects $A$ and $B$ in $C_1$ and morphisms $f : A
    \to B$, we have $\mathds 1 \circ F(f) = F(f) \circ \mathds 1$, which follows
    from $\mathds 1$ being an identity.
\end{proof}

\begin{theorem}
    Given categories $C_1$ and $C_2$, functors $F$, $G$, and $H$ from $C_1$
    to $C_2$, and natural transformations $\alpha : G \to H$ and $\beta : F \to
    G$, the function taking objects $A : C_1$ to $\alpha(A) \circ \beta(A)$ is a
    natural transformation from $F$ to $H$ that we call the vertical composition
    of $\alpha$ and $\beta$.
\end{theorem}
\begin{proof}
    \[
        \alpha(B) \circ \beta(B) \circ F(f) =
        \alpha(B) \circ G(f) \circ \beta(A) =
        H(f) \circ \alpha(A) \circ \beta(A).
    \]
\end{proof}

\begin{theorem}
    Given two categories $C_1$ and $C_2$, functors form a category $\Fct(C_1,
    C_2)$ where the objects are functors from $C_1$ to $C_2$, the morphisms are
    natural transformations between them, composition of natural transformations
    is as defined above, and the identity natural transformation is as defined
    above.
\end{theorem}
\begin{proof}
    Every property needed follows from the corresponding property in $C_2$.
\end{proof}

Thus, like with functors, without confusion we may use the notation $\mathds 1$
to represent the identity natural transformation and $\alpha \circ \beta$ to
represent the vertical composition of two natural transformations.

\end{document}
